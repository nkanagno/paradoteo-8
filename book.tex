% Options for packages loaded elsewhere
\PassOptionsToPackage{unicode}{hyperref}
\PassOptionsToPackage{hyphens}{url}
%
\documentclass[
]{article}
\usepackage{amsmath,amssymb}
\usepackage{lmodern}
\usepackage{iftex}
\ifPDFTeX
  \usepackage[T1]{fontenc}
  \usepackage[utf8]{inputenc}
  \usepackage{textcomp} % provide euro and other symbols
\else % if luatex or xetex
  \usepackage{unicode-math}
  \defaultfontfeatures{Scale=MatchLowercase}
  \defaultfontfeatures[\rmfamily]{Ligatures=TeX,Scale=1}
\fi
% Use upquote if available, for straight quotes in verbatim environments
\IfFileExists{upquote.sty}{\usepackage{upquote}}{}
\IfFileExists{microtype.sty}{% use microtype if available
  \usepackage[]{microtype}
  \UseMicrotypeSet[protrusion]{basicmath} % disable protrusion for tt fonts
}{}
\makeatletter
\@ifundefined{KOMAClassName}{% if non-KOMA class
  \IfFileExists{parskip.sty}{%
    \usepackage{parskip}
  }{% else
    \setlength{\parindent}{0pt}
    \setlength{\parskip}{6pt plus 2pt minus 1pt}}
}{% if KOMA class
  \KOMAoptions{parskip=half}}
\makeatother
\usepackage{xcolor}
\usepackage{graphicx}
\makeatletter
\def\maxwidth{\ifdim\Gin@nat@width>\linewidth\linewidth\else\Gin@nat@width\fi}
\def\maxheight{\ifdim\Gin@nat@height>\textheight\textheight\else\Gin@nat@height\fi}
\makeatother
% Scale images if necessary, so that they will not overflow the page
% margins by default, and it is still possible to overwrite the defaults
% using explicit options in \includegraphics[width, height, ...]{}
\setkeys{Gin}{width=\maxwidth,height=\maxheight,keepaspectratio}
% Set default figure placement to htbp
\makeatletter
\def\fps@figure{htbp}
\makeatother
\setlength{\emergencystretch}{3em} % prevent overfull lines
\providecommand{\tightlist}{%
  \setlength{\itemsep}{0pt}\setlength{\parskip}{0pt}}
\setcounter{secnumdepth}{-\maxdimen} % remove section numbering
\ifLuaTeX
  \usepackage{selnolig}  % disable illegal ligatures
\fi
\IfFileExists{bookmark.sty}{\usepackage{bookmark}}{\usepackage{hyperref}}
\IfFileExists{xurl.sty}{\usepackage{xurl}}{} % add URL line breaks if available
\urlstyle{same} % disable monospaced font for URLs
\hypersetup{
  pdftitle={Microsoft Word - game\_programming\_lecture\_note\_2\_2009.doc},
  pdfauthor={Marisa Drexel},
  hidelinks,
  pdfcreator={LaTeX via pandoc}}

\title{Microsoft Word - game\_programming\_lecture\_note\_2\_2009.doc}
\author{Marisa Drexel}
\date{2009-09-08T17:26:21+00:00}

\begin{document}
\maketitle

\includegraphics{cover_image.jpg}

\protect\hypertarget{titlepage.xhtml}{}{}

\protect\hypertarget{index_split_000.html}{}{}

\protect\hypertarget{index_split_000.htmlux5cux23p1}{}{}

Lecture \#1

Basic Skills

\protect\hypertarget{index_split_001.html}{}{}

\hypertarget{index_split_001.htmlux5cux23calibre_pb_0}{%
\subsection{Introduction}\label{index_split_001.htmlux5cux23calibre_pb_0}}

The Game Industry is a multi-billion dollar industry and still growing.
Years ago, the technology forced games to have simple designs. Programs
could often be developed by a small team of pure hackers with the major
requirement being a good idea. Back then, formal training and education
was often looked down on. Now, it is almost essential.

Nowadays, games are so complex they require large teams of programmers,
designers, artists, testers, advertisers, and producers to organize and
develop them. Games are now as complex as, if not more than, the latest
blockbuster film. It is interactive entertainment, pure and simple.

The tools and training needed for game development are enormous, and the
complexities warrant academic education beyond a single course. In fact,
an entire undergraduate curriculum could be (and in some places is)
based on game development.

This lecture note is designed to help students to learn fundamental
principles that apply to game programming regardless of the language
they use to create the game. These principles include gathering input
from users, processing game data, and rendering game objects to the
screen. This lecture shows you how to use JavaScript to program a simple
game. Anyway, \textbf{game} \textbf{programming is a habit, a skill, and
an art.}

Important note

It is the instructor's intention to use \textbf{HTML},
\textbf{JavaScript}, \textbf{DHTML}, and \textbf{CSS} to explain the
game programming concepts. As an art, game programming is independent of
language, meaning you can choose any language you feel comfortable to
work with the write games in any platform as long as you understand the
concepts of game programming. However, the instructor will also use
other programming languages (such as Java) to explain game programming
skills whenever it is deemed necessary.

Language is just a tool that facilitates the project you are working on.
The instructor chooses the following combination, so the students do not
need to install any language, SDKs (system development kit), or IDEs.
You can begin writing games simply using applications that come with
Windows XP/2000/2003.

Once you learn the concept of programming games, you can apply the
skills and knowledge to a platform-sensitive language (or scripting
language), such as Visual Basic, C++, C\#, Java, and so on. Each
language, certainly, has its very own supports of game programming. You
just need to be familiar with these language-specific supports to
develop games using these platform-sensitive languages, because the
concepts you learn in this class are fundamental and universal to all
languages.

What is a

A satisfying game can be played over and over again and there are
different strategies that lead to game?

success. So what is a (computer) game then? Here is an old-fashion
definition:

"A computer game is a software program in which one or more players make
decisions through the control of game objects and

resources, in pursuit of a goal."

Note that the definition does not talk about multimedia supports, such
as graphics, or sound effect, or in-game movies. Multimedia obviously
plays a important role in making nice, appealing games, so a newer
definition is:

"A computer game is a game composed of a computer-controlled virtual
universe with multimedia supports that players may

Game Programming -- Penn Wu

1

\protect\hypertarget{index_split_001.htmlux5cux23p2}{}{}interact with in
order to achieve a goal or set of goals."

Object-oriented Most game programming languages adopt the
object-oriented model, which is a programming programming

paradigm that uses ``objects'' to design game programs. The term
``object'' refers to any identifiable item in a game. For example, in a
game that has a ball moving from the left to right, the \textbf{ball} is
an object.

An object has many \textbf{properties} (also known as
\textbf{attributes}) that determine the appearance of the object, the
place the object should be, and so on. For example, a ball has size,
color, file name, ID

(identification), etc.. Each is an individual property of the ball
object, but altogether they decide what and how the ball should be.

An object also has functions, such as start, stop, turn left, hide,
show, etc.. Many programming languages provide \textbf{methods}
(functions that are part of the programming language) for programmers to
use. The game programmer's job is to associate appropriate methods with
the object, or ``to build the use-defined functions into the objects''.

When the player play games, he/she will have much action, such as
clicking a mouse button, pressing a key, etc.. These user actions are
known as \textbf{events}. Most languages provide \textbf{event}
\textbf{handlers} for game programmers to respond to the events. For
example, Although object-oriented programming is a complicated program
paradigm, you only need to learn two ``magic'' formulas and some basic
concepts. These two formulas have two forms: JavaScript and DHTML forms.
They will be used throughout the entire class.

In JavaScript, they are:

objectID.propertyName = "value";

and

objectID.MethodName();

For example, the following code block use the
\textless img\textgreater{} HTML tags to load an image file
\textbf{cat.gif}, which is certainly an object, because you can see it
(identify it) on the body area of the browser once the HTML page is
loaded.

\textless html\textgreater\textless body\textgreater{}

\textless img id="m1" src="cat.gif"

onClick="m1.src=\textquotesingle dog.gif\textquotesingle"\textgreater{}

\textless/body\textgreater\textless/html\textgreater{}

The following section assigns an ID (which is m1) to the image file.

id="m1"

In the following section, \textbf{src} is the name of a property
(meaning the source). It uses the \textbf{propertyName = ``value''}
format to assign cat.gif as the value of \textless img\textgreater{}
tag.

src="cat.gif"

The following section uses the entire \textbf{objectID.propertyName =
"value";} format to tell the computer ``when the player click on the
image, change the value of the \textbf{m1} object's \textbf{src}
property to \textbf{dog.gif}.

onClick=" \textbf{m1.src=\textquotesingle dog.gif\textquotesingle{}} "

JavaScript

JavaScript is the premier client-side scripting language used today on
the Web. It's widely used in Game Programming -- Penn Wu

2

\protect\hypertarget{index_split_001.htmlux5cux23p3}{}{}Basics

tasks ranging from the validation of form data to the creation of
complex user interfaces. Yet the language has capabilities that many of
its users have yet to discover.

JavaScript can be used to manipulate the very markup in the documents in
which it is contained.

As more developers discover its true power, JavaScript is becoming a
first class client-side Web technology, ranking alongside (X)HTML, CSS,
and XML. As such, it will be a language that any Web designer would be
remiss not to master. This chapter serves as a brief introduction to the
language and how it is included in Web pages.

JavaScript is not Java language, and does have direct relation to Java
language. Java was developed and created by Sun Micro, while JavaScript
is solely developed and created by Netscape.

JavaScript code must be embedded into an HTML document using the
\textless script\textgreater{} and \textless/script\textgreater{} tags.
The format:

\textless script language="JavaScript"\textgreater{}

JavaScript code fragments

\textless/script\textgreater{}

In object oriented languages, object variables are called "properties"
and object functions are called "methods." For example,

\textless html\textgreater\textless script
language=JavaScript\textgreater{}

window.alert("This is a warning message")

\textless/script\textgreater\textless/html\textgreater{}

The alert() method of the window (the browser window) object displays a
warning message on the Web page.

JavaScript use the following syntax:

• \textbf{objectID.method()}: for example, window.prompt(). The prompt
method of the window object creates a pop-up prompt box.

• \textbf{objectID.property}: for example, navigator.appName. Identify
the appName property of the navigator object.

• \textbf{object.method(object.property)}: for example,
document.write(navigator.appName).

This displays the value of the appName property on the Web page.

When a web page is loaded, the browser\textquotesingle s JavaScript
interpreter automatically creates objects for most of the components
contained in the document. This includes -\/- but is not limited to -\/-
objects for the document\textquotesingle s forms, images, and
hyperlinks. According to Netscape, the developers of JavaScript, every
page has the following objects:

• \textbf{navigator}: has properties for the name and version of the
Navigator being used, for the MIME

types supported by the client, and for the plug-ins installed on the
client.

• \textbf{window}: the top-level object; has properties that apply to
the entire window. There is also a window object for each "child window"
in a frames document.

• \textbf{document}: contains properties based on the content of the
document, such as title, background color, links, and forms.

• \textbf{location}: has properties based on the current URL.

• \textbf{history}: contains properties representing URLs the client has
previously requested.

Other built-in JavaScript Objects include:

• \textbf{Array}: an Array object is used to store a set of values in a
single variable name.

• \textbf{Date}: The Date object is used to work with dates and times.

• \textbf{String}: The String object is used to work with text.

• \textbf{Math}: The built-in Math object includes mathematical
constants and functions.

For example,

Game Programming -- Penn Wu

3

\protect\hypertarget{index_split_001.htmlux5cux23p4}{}{}

\textless script language="JavaScript"\textgreater{}

document. \textbf{write}(document. \textbf{lastModified});

\textless/script\textgreater{}

Declaring

When a car runs, the speed changes all the time. The meter does the
reading of speed, and it gives variables

you different values every time when the speed changes. Consider the
following table: time

1

2

3

4

5

6

7

8

9

10

11

12

13

14

15

16

17

18

19

20

speed

65

68

67

66

65

67

68

69

70

71

70

68

67

68

67

66

65

64

66

64

During the last 20 minutes, the value of speed varies up and down. The
value is never constant; it varies, so it is a variable.

In the world of game programming, when a programmer needs to handle a
property with its value changing all the time, he/she creates (or
declare) a ``variable''. Variables hold the value when the users enter
them. Variables hold the values till the computer notify them to stop
holding them.

You can think of variables as "holding places" where values are stored
and retrieved. Each holding place has a unique name, which is any
combination of letters and numbers that does not begin with a number. In
JavaScript, the characters a-z, A-Z, and the underscore character "\_"
are all considered to be unique letters.

JavaScript was loosely designed, so it does not require you to state the
kind of data type that will be associated with a variable. The syntax
JavaScript use to declare a variable is: var VariableName

The ``var'' keyword is not strictly required

For example,

var mileage = 10987; // number

var CompanyName = "Orange Inc. "; // string

var Readable = "true" // logical / boolean

var score;

var style = null;

Any kind of data type can be stored in any variable. Common types of
values include:

•

Numbers --- any number, (ex. 5 or 3.14567)

•

Strings or Text --- any strings enclosed in double quotation marks, (ex.
``\$12.34'', ``Hello!'')

•

Logical (Boolean) --- either true or false

•

null --- a special value that refers to nothing

Basic rules:

•

Case Sensitive Very important... JavaScript is case sensitive, which
means, upper case letters and lower case letters are interpreted as
being different. For example, the variables ``apple''

and ``Apple'', ``ApPle'', ``APPLE'' are considered four individual
variable names.

•

Do not use the following words as variable names:

abstract delete

goto

null

this

boolean

do

if

package

throw

break

double

implements private

throws

byte

else

import

protected

transient

case

extends

in

public

true

catch

false

instanceof return

try

char

final

int

short

typeof

class

finally

interface

static

var

const

float

long

super

void

continue for

native

switch

while

default

function new

synchronized with

Game Programming -- Penn Wu

4

\protect\hypertarget{index_split_001.htmlux5cux23p5}{}{}

• Do not use such symbols as 8 \^{} \$@+()
\{\}{[}{]}\textbackslash\textbar\textasciitilde!?/`''\,':; as part of
variable name.

In the following example, ``current\_time'' is the name of variable that
represents a certain registers on the computer memory (DRAMs). This
variable holds the readings of data and time, and set the readings ready
for the user to use.

With var keyword

Without var keyword

\textless html\textgreater\textless script\textgreater{}

\textless html\textgreater\textless script\textgreater{}

function st() \{

function st() \{

\textbf{var} current\_time=new Date()

current\_time=new Date()

alert(current\_time);

alert(current\_time);

\}

\}

\textless/script\textgreater{}

\textless/script\textgreater{}

\textless button

\textless button

onClick=st()\textgreater Click\textless/button\textgreater{}

onClick=st()\textgreater Click\textless/button\textgreater{}

\textless html\textgreater{}

\textless html\textgreater{}

The user clicks the button to call the st() function, and the st()
function in turns force the alert() method to display the most current
readings on a warning message box.

Create user-

When creating games, you need to use computer statements to instruct the
computer what to do.

defined

Each statement is considered an individual instruction. However, it is
very inconvenient to give functions

out instruction one by one, because the computer responds to each
instruction one by one.

Now, try another approach. Pack all the statements together into a
batch, and then name the batch with a uniquely identifiable ID. You can
then execute all the statements packed in that batch by simply calling
the ID. In JavaScript, such batch is known as a user-defined function.

A JavaScript function is a collection of instructions (i.e., statements
or commands) whose purpose is to accomplish a well-defined task, based
on the implementation of existing JavaScript objects, methods, and
properties. When the browser executes a function, it is very similar to
a VCR

executes the pre-programmed action. In JavaScript, a
function\textquotesingle s input is called its parameters; its output is
called the return value. There are two types of functions: built-in
functions and custom functions: built-In and custom functions.

In JavaScript, the keyword ``function'' creates a custom function. A
function is defined inside the

\textless SCRIPT\textgreater{} and \textless/SCRIPT\textgreater{} tags;
it consists of the keyword function, followed by the following things:

• A name for the function;

• A list of arguments for the function (enclosed in parentheses and
separated by commas); and

• JavaScript statements defining the function (enclosed in curly braces
\{ \} ).

The syntax:

function functionName(arguments) \{

JavaScript statements

\}

Additionally, once a function has been defined, it must be called in
order to execute it. For example, a JavaScript function whose purpose is
to compute the square of a number might be defined as follows:

\textless html\textgreater\textless script\textgreater{}

function sqr(n) \{

document.write(n*n);

\}

Game Programming -- Penn Wu

5

\protect\hypertarget{index_split_001.htmlux5cux23p6}{}{}\textless/script\textgreater{}

\textless input type=text name=t1 onClick=sqr(t1.value)\textgreater{}

\textless/html\textgreater{}

In this "sqr" function, sqr(n), the parenthesized variable "n" is called
a parameter. It represents the input entered to the text box. The output
is equal to number squared. Most of all, the user must click the text
box to execute the sqr() function.

CSS basics

Cascading Style Sheets (CSS) is a way to control the appearance and
dynamics of objects in a HTML document. It allows you to convert static
HTML objects into motion-enabled objects in an organized and efficient
manner.

You can use any existing HTML tag that formats objects in the browser's
body area as a

``selector'' by following the syntax:

selector \{ PropertyName \emph{1}:value \emph{1}; PropertyName
\emph{2}:value \emph{2}\ldots\ldots\}

For example, the pair of \textless i\textgreater{} and
\textless/i\textgreater{} are HTML tags that italicizes contents in
between them.

However, it cannot specify color and font size. Compare the following
codes and their outputs.

Without CSS

With CSS

\textless html\textgreater\textless body\textgreater{}

\textless html\textgreater{}

\textless i\textgreater This is a sentence.\textless/i\textgreater{}

\textless head\textgreater{}

\textless/body\textgreater\textless/html\textgreater{}

\textbf{\textless style\textgreater{}}

i \{color: red; font-size:24px\}

\textbf{\textless/style\textgreater{}}

\textless/head\textgreater{}

\textless body\textgreater{}

\textless i\textgreater This is a sentence.\textless/i\textgreater{}

\textless/body\textgreater{}

\textless/html\textgreater{}

\emph{This is a sentence.}

\emph{This is a sentence.}

Notice that you must define your CSS property sets in between
\textless style\textgreater{} and \textless/style\textgreater. And you
are supposed to place your CSS definitions between
\textless head\textgreater{} and \textless/head\textgreater. Never place
them between

\textless body\textgreater{} and \textless/body\textgreater.

Pseudo-classes are also used in CSS to add different effects to some
selectors, or to a part of some selectors. When you want one tag but
different styles. In a web page you want different type style for
different paragraph. Pseudo Class can be defined in embedded style or an
external style.

The syntax for setting the style is:

.classID \{property: value\}

The syntax for calling the style is:

\textless selector
class="classID"\textgreater...\textless/selector\textgreater{}

Consider the following example,

\textless html\textgreater\textless style\textgreater{}

.p1 \{color:red;font-family:arial\}

.p2 \{color:green;font-family:times new roman\}

\textless/style\textgreater{}

\textless p class=p1\textgreater This is red.\textless/p\textgreater{}

\textless p class=p2\textgreater This is green.\textless/p\textgreater{}

\textless/html\textgreater{}

Game Programming -- Penn Wu

6

\protect\hypertarget{index_split_001.htmlux5cux23p7}{}{}

There are two pseudo classes: p1 and p2. They are pseudo because they
are not an existing HTML

tag (like \textless p\textgreater{} tag). Now that they are pseudo
classes to ``p'', they will automatically inherit the functions the
\textless p\textgreater{} tag has---creating a paragraph. The p1 class,
however, is assigned two styles:

\{color:red;font-family:arial\}

The p2 class is assigned two styles:

\{color:green;font-family:times new roman\}

To call the two classes, use

\textless p class=p1\textgreater..\textless/p\textgreater{}

and

\textless p class=p2\textgreater..\textless/p\textgreater.

Integrate CSS

In most cases, CSS properties are converted to JavaScript properties by
removing the data from with JavaScript the CSS property name and
capitalizing any words following the dash. For example, the CSS

property \textbf{background-color} becomes \textbf{backgroundColor} in
JavaScript.

\textless html\textgreater{}

\textless u id=u1
onClick=u1.style.visibility="hidden"\textgreater Hi!\textless/u\textgreater\textless br\textgreater{}
This is the line 2.

\textless/html\textgreater{}

In addition to identify each style, the keyword ``this'' is a good
alternative.

\textless html\textgreater{}

\textless u
onClick=this.style.display="none"\textgreater Hi!\textless/u\textgreater\textless br\textgreater{}
This is the line 2.

\textless/html\textgreater{}

With new Internet Explorer, even the following works!

\textless html\textgreater{}

\textless u
onClick=style.display="none"\textgreater Hi!\textless/u\textgreater\textless br\textgreater{}
This is the line 2.

\textless/html\textgreater{}

When click the word ``Hi!'', the first line disappears, but the line
remains.

DHTML basics DHTML is Dynamic HyperText Markup Language. As the name
conveys, \textquotesingle dynamic\textquotesingle{} is something that
changes at any given time, but the fact is that, every detail has to be
pre-coded. DHTML is a combination of HTML, CSS and JavaScript. DHTML is
a feature of Netscape Communicator 4.0

and Microsoft Internet Explorer 4.0 and 5.0 and is entirely a
"client-side" technology.

Many features of DHTML can be duplicated with either Flash or Java,
DHTML provides an alternative that does not require plug-ins and embeds
seamlessly into a web page.

DHTML is useful for creating an advanced functionality web page. It can
be used to create animations, games, applications, provide new ways of
navigating through web sites, and create out-of-this world layout that
simply aren\textquotesingle t possible with just HTML.

The underlying technologies of DHTML (HTML,CSS, JavaScript) are
standardized, the manner in which Netscape and Microsoft have
implemented them differ dramatically. For this reason, writing DHTML
pages that work in both browsers can be a very complex issue.

Game Programming -- Penn Wu

7

\protect\hypertarget{index_split_001.htmlux5cux23p8}{}{}\includegraphics{index-8_1.png}

A simply definition of DHTML is: Dynamic HTML is simply HTML that can
change even after a page has been loaded into a browser.

DHTML consist of:

•

DOM (Document Object Model): Core of DHTML

•

CSS: Style provider of DHTML

•

Scripting Language: JavaScript (or Jscript) as function provider of
DHTML

To sum all this up: CSS and "old HTML" are what you change, the DOM is
what makes it changeable, and client-side scripting is what actually
changes it. And that\textquotesingle s dynamic HTML.

Additionally, Netscape and Internet Explorer use fundamentally different
approaches to create DHTML pages. For example, Netscape use the
\textless layer\textgreater{} tag to display a Web page inside a Web
page, while Internet Explorer uses \textless iframe\textgreater{} to do
so. It is the intention of this lecture to focus on Internet Explorer.

Consider the following DHTML code example,

\textless html\textgreater{}

\textless u style="cursor:hand"
onClick=alert("Hi!")\textgreater Click\textless/u\textgreater{}

\textless/html\textgreater{}

This code is a combination of HTML, CSS, and JavaScript.
\textless html\textgreater{} and \textless u\textgreater{} are generic
HTML

tags; style="cursor:hand" is a CSS style definition; onClick is a
JavaScript event handler and alert("Hi!") is a JavaScript method.

Consider the document.body.clientWidth and document.body.clientHeight
properties, and you will understand why DHTML can be used to create
games.

DHTML can be used to position objects based on the location and
measurement of other objects.

The clientWidth and clientHeight properties retrieve the width of the
object. The offsetWidth and offsetHeight properties retrieve the width
and height of the object relative to the layout or coordinate parent, as
specified by the offsetParent property.

Consider the following example,

\textless html\textgreater\textless script\textgreater{}

function cc() \{

alert(document.body.clientWidth+"x"+document.body.clientHeight)

\}

\textless/script\textgreater{}

\textless button
onClick=cc()\textgreater Click\textless/button\textgreater{}

\textless/html\textgreater{}

When the user click the button, a warning message pops up telling the
user width and height of the currently opened window, as shown below:

In the following example, we will force the screen size to change when
the user click the button.

\textless html\textgreater\textless script\textgreater{}

function shrinkIt() \{

Game Programming -- Penn Wu

8

\protect\hypertarget{index_split_001.htmlux5cux23p9}{}{}\includegraphics{index-9_1.png}

x=document.body.clientWidth-100;

y=document.body.clientHeight-100;

window.resizeTo (x,y)

\}

\textless/script\textgreater{}

\textless button
onClick=shrinkIt()\textgreater Click\textless/button\textgreater{}

\textless/html\textgreater{}

First of all, we declare a variable x and assignment the value
document.body.clientWidth-100, which means ``detect the current window
width and subtract 100 from it.'' Second, we declare the y variable and
assign the value of y=document.body.clientHeight-100 to it, which means
``detect the current window height and subtract 100 from it.'' Finally,
we use the resizeTo() method to change the size of window.

In the following example, the Web page (document) is the parent object,
the overflow area is the offset object. When the user click the button,
a warning message of ``184:200'' appears.

\textless html\textgreater{}

\textless p id=dt STYLE="overflow:scroll; width:200;
height:100"\textgreater Add a long paragraph
here.....\textless/p\textgreater{}

\textless button

onclick=alert(dt.clientWidth+":"+dt.offsetWidth)\textgreater Click\textless/button\textgreater{}

\textless/html\textgreater{}

DHTML is ``dynamics'', so you can use innerText, outerText, innerHTML,
and outerHTML

properties to dynamically inserting texts or codes.

The syntax is:

objectID.innerText="strings"; \textbf{}

objectID.innerHTML="strings";

objectID.outerText="strings";

objectID.outerHTML="strings";

The innerText property sets or retrieves the text between the start and
end tags of the object. It is valid for block elements only. By
definition, elements that do not have both an opening and closing tag
cannot have an innerText property. When the innerText property is set,
the given string completely replaces the existing content of the object.

For example,

\textless html\textgreater You can order \textless b
id=dt\textgreater Pizza\textless/b\textgreater{}
here.\textless br\textgreater{}

\textless button
onclick="dt.innerText=\textquotesingle Hamburg\textquotesingle"\textgreater{}
Hamburg\textless/button\textgreater{}

\textless/html\textgreater{}

Click the ``Hamburg'' button to replace the word ``Pizza'' with
``Hamburg''.

Consider this example,

\textless P
onMouseOver="bt.innerHTML=\textquotesingle\textless b\textgreater bold\textless/b\textgreater\textquotesingle"

onMouseOut="bt.innerHTML=\textquotesingle\textless i\textgreater italic\textless/i\textgreater\textquotesingle"\textgreater{}
It\textquotesingle s \textless i
id=bt\textgreater italic\textless/i\textgreater{} now.

\textless/P\textgreater{}

Game Programming -- Penn Wu

9

\protect\hypertarget{index_split_001.htmlux5cux23p10}{}{}The inserted
string will be formatted by the HTML tag. Also, test the following code
to view the differences.

\textless div id=dt\textgreater\textless b\textgreater Dynamic
HTML!\textless/b\textgreater\textless/div\textgreater{}

\textless script\textgreater{}

alert("innerText: "+dt.innerText)

alert("innerHTML: "+dt.innerHTML)

alert("outerText: "+dt.outerText)

alert("outerHTML: "+dt.outerHTML)

\textless/script\textgreater{}

However, the output of innerText and outerText look the same.

Using

CSS styles can integrate with JavaScript using DHTML format. One of the
beauties of using JavaScript

JavaScript is its capability to pack several commands into one function.
It is important to learn Functions in

how to use JavaScript function in DHTML.

DHTML

Consider the following example,

\textless html\textgreater\textless script
language=JavaScript\textgreater{}

function moving() \{

dt.style.left="250"; dt.style.top="200";

\}

\textless/script\textgreater{}

\textless p id=dt
style="position:absolute;left:25;top:50"\textgreater This is a
sentence.\textless/p\textgreater{}

\textless button
onClick=moving()\textgreater Click\textless/button\textgreater{}

\textless/html\textgreater{}

The sentence was originally display at the position (25, 50). When the
user click the button, the onClick event handler calls the moving()
JavaScript function, which in turn moves the sentence to (250, 200).

Consider another example,

\textless html\textgreater\textless script
language="JavaScript"\textgreater{}

function coloring() \{

dt.style.color="blue"; dt.style.backgroundColor="yellow";

\}

\textless/script\textgreater{}

\textless p id=dt style="color:red;background-color:blue"

onClick=coloring()\textgreater This is a
sentence.\textless/p\textgreater{}

\textless/html\textgreater{}

When the user clicks any place on the sentence, the onClick event
handler calls the coloring() function, so the colors changes.

Consider the third example,

\textless html\textgreater{}

\textless img src="http://www.geocities.com/cistcom/bs.jpg"

style="width:100"

onClick="this.style.width=\textquotesingle200\textquotesingle"\textgreater{}

\textless/html\textgreater{}

We can use JavaScript function to add more options of image size.

\textless html\textgreater{}

\textless script language=JavaScript\textgreater{}

function size1() \{img1.style.width="200"\}

Game Programming -- Penn Wu

10

\protect\hypertarget{index_split_001.htmlux5cux23p11}{}{}function
size2() \{img1.style.width="300"\}

function size3() \{img1.style.width="400"\}

\textless/script\textgreater{}

\textless img id=img1 src=http://www.geocities.com/cistcom/bs.jpg

style="width:100"\textgreater{}

\textless button onClick=size1()\textgreater Size
1\textless/button\textgreater{}

\textless button onClick=size2()\textgreater Size
2\textless/button\textgreater{}

\textless button onClick=size3()\textgreater Size
3\textless/button\textgreater{}

\textless/html\textgreater{}

Game

This tutorial should have give n you a rough idea of the things that
matter when trying to create a programming

good computer game. But in the end the best way to learn is to do it
yourself. Although game is just the

programming may sound a big project to you at the current stage, it
really is not that far from matter doing it! what you have learned in
any programming course. It takes only a new way of thinking, and most of
all, a willingness to try. Game programming is just the matter doing it!

There are few simple games created by the instructor as demonstrations.
It is time to try them. Do not worry if you don't understand the code.
Details will be described in later lectures. For now, just have fun!

Review

1. In a game that has a pitcher who pitches a baseball to a baseball
hitter, and a catcher who Questions

attempts to catch the baseball. Which is an object?

A. pitcher

B. hitter

C. baseball

D. all of the above.

2. Given a statement "A red baseball moves from left to right". Which is
the property of the baseball?

A. left to right

B. red

C. move

D. the word "baseball"

3. Given the following code, which is the property of the image file?

\textless img id="m1" src="cat.gif"\textgreater{}

A. img

B. src

C. cat.gif

D. m1

4. Given the following code, which is the event handler?

\textless img id="m1" src="cat.gif"
onClick=\textquotesingle this.src=\textquotesingle dog.gif\textquotesingle"\textgreater{}
A. id

B. src

C. onClick

D. img

5. Given the following code, which is the method?

window.alert("Welcome!")

A. window

B. alert()

Game Programming -- Penn Wu

11

\protect\hypertarget{index_split_001.htmlux5cux23p12}{}{}C. Welcome!

D. All of the above

6. Given the following code segment, which is a boolean variable?

var selected = "true";

var Saturday = "Yes";

var vitamine = "good";

var monitor = "white";

A. selected

B. Saturday

C. vitamine

D. monitor

7. Given the following code segment, which is a user-defined function?

function sqr(n) \{

document.write(n*n);

window.alert(n+1);

n++;

\}

A. sqr();

B. document.write()

C. window.alert()

D. n++

8. Given the following CSS code segment, which is a property?

\textless style\textgreater{} p \{border: solid 1
black;\}\textless/style\textgreater{}

A. \textless style\textgreater{}

B. p

C. border

D. solid 1 black

9. Given the following CSS code segment, which is a class?

\textless style\textgreater{}

p1 \{border: solid 1 black;\}

p2 \{border: solid 1 blue;\}

p3 \{border: solid 1 red;\}

.p4 \{border: solid 1 blue;\}

\textless/style\textgreater{}

A. p1

B. p2

C. p3

D. .p4

10. Given the following code, which statement is correct?

\textless html\textgreater\textless script
language="JavaScript"\textgreater{}

function coloring() \{

dt.style.color="blue"; dt.style.backgroundColor="yellow";

\}

\textless/script\textgreater{}

\textless p id=dt style="color:red;background-color:blue"

Game Programming -- Penn Wu

12

\protect\hypertarget{index_split_001.htmlux5cux23p13}{}{}onClick=coloring()\textgreater This
is a sentence.\textless/p\textgreater{}

\textless/html\textgreater{}

A. the onClick event handler calls the coloring() function.

B. when the coloring() function executes, the dt object turns blue.

C. when the coloring() function executes, the dt
object\textquotesingle s background color turns yellow.

D. All of the above

Game Programming -- Penn Wu

13

\protect\hypertarget{index_split_001.htmlux5cux23p14}{}{}\includegraphics{index-14_1.png}

\textbf{}

Lab \#1

\protect\hypertarget{index_split_002.html}{}{}

\hypertarget{index_split_002.htmlux5cux23calibre_pb_1}{%
\subsection{Introduction to Game
Programming}\label{index_split_002.htmlux5cux23calibre_pb_1}}

\textbf{}

\textbf{Preparation \#1: Download image files}

1. Create a new directory named \textbf{C:\textbackslash lab1}.

2. Use Internet Exploer to go to
\textbf{http://business.cypresscollege.edu/\textasciitilde pwu/cis261/download.htm}
to download lab1.zip (a zipped) file. Extract the files to
C:\textbackslash lab1 directory. Make sure the C:\textbackslash lab1
directory contains:

•

ani\_cat.gif

•

ball.gif

•

bat.gif

•

gopher1.gif

•

gopher2.gif

\textbf{Preparation \#2:} Sign up a free Web Hosting Service

1. Use Internet Explorer to go to \textbf{http://www.geocities.com} or
\textbf{http://www.tripod.com} to sign up a free web hosting service. Be
sure to learn to upload files and view your uploaded files through the
web.

\textbf{Learning Activity \#1: A dropping letter}

\textbf{}

Note: Do not worry if you don't understand the code. Details will be
described in later lectures. For now, just have fun!

\textbf{}

1. Change to the C:\textbackslash lab1 directory.

2. Use Notepad to create a new file named
\textbf{C:\textbackslash lab1\textbackslash lab1\_1.htm} with the
following contents:

\textless html\textgreater{}

\textless script\textgreater{}

function init() \{

if (obj.style.pixelTop \textgreater=
\textbf{document.body.clientHeight}) \{

obj.style.pixelTop = 0; \}

else \{obj.style.pixelTop++; \}

s1 = setTimeout("init()", 10);

\}

\textless/script\textgreater{}

\textless body onLoad=init()\textgreater{}

\textless p id="obj" style="position:absolute;
top:0"\textgreater o\textless/p\textgreater{}

\textless/body\textgreater{}

\textless/html\textgreater{}

3. Use Internet Explorer (do no use other browser) to execute the file.
\textbf{Be sure to have your speaker or earphone} \textbf{ready} to hear
the sound!

4. If the ``Dig you notice the Information Bar?'' warning box appears,
as shown below, click OK.

5. If the default information bar remains, right click the bar, and then
select ``Allow Blocked Content\ldots'', as shown below.

Game Programming -- Penn Wu

14

\protect\hypertarget{index_split_002.htmlux5cux23p15}{}{}\includegraphics{index-15_1.png}

\includegraphics{index-15_2.png}

\includegraphics{index-15_3.png}

6. You should now see a letter o dropping from top to bottom over and
over again.

\textbf{}

\textbf{}

\textbf{}

\textbf{Learning Activity \#2: Using animated Gif file as visual
effects}

\textbf{}

Note: Do not worry if you don't understand the code. Details will be
described in later lectures. For now, just have fun!

\textbf{}

1. Change to the C:\textbackslash lab1 directory.

2. Use Notepad to create a new file named
\textbf{C:\textbackslash lab1\textbackslash lab1\_2.htm} with the
following contents:

\textless html\textgreater{}

\textless body onLoad=run()\textgreater{}

\textless script\textgreater{}

var i=0;

function run() \{

\textbf{document.body.clientWidth} represents the current browser width
m1.style.left=i;

i+=5;

if (i\textgreater=\textbf{document.body.clientWidth} -
m1.style.pixelWidth) \{i=0;\}

setTimeout("run()",70);

\}

\textbf{m1.style.pixelWidth} is the width of ant\_cat.gif

\textless/script\textgreater{}

\textless img id="m1" src="ani\_cat.gif"

style="position:absolute;top:10;left:0"\textgreater{}

\textless/body\textgreater\textless/html\textgreater{}

3. Use Internet Explorer to execute the file. The output now looks: The
cat will run across the screen.

Game Programming -- Penn Wu

15

\protect\hypertarget{index_split_002.htmlux5cux23p16}{}{}\includegraphics{index-16_1.png}

\textbf{}

\textbf{Learning Activity \#3: Effects of distance}

\textbf{}

Note: Do not worry if you don't understand the code. Details will be
described in later lectures. For now, just have fun!

\textbf{}

1. Change to the C:\textbackslash lab1 directory.

2. Use Notepad to create a new file named
\textbf{C:\textbackslash lab1\textbackslash lab1\_3.htm} with the
following contents:

\textless html\textgreater{}

\textless script\textgreater{}

function cc() \{

m1.style.width = m1.style.pixelWidth + 10;

if (m1.style.pixelWidth \textless{} 400) \{

setTimeout("cc()", 50);

\}

else \{dd(); \}

\}

function dd() \{

m1.style.width = m1.style.pixelWidth - 10;

if (m1.style.pixelWidth \textgreater{} 10) \{

setTimeout("dd()", 50);

\}

else \{cc(); \}

\}

\textless/script\textgreater{}

\textless body onLoad=cc()\textgreater{}

\textless center\textgreater\textless img id=m1 src="bat.gif"
style="width:10"\textgreater\textless/center\textgreater{}

\textless/body\textgreater{}

\textless/html\textgreater{}

3. Use Internet Explorer to execute the file. The bat will fly toward
you and then backward.

\textbf{Learning Activity \#4: Your First Game}

Note: Do not worry if you don't understand the code. Details will be
described in later lectures. For now, just have fun!

1. Change to the C:\textbackslash lab1 directory.

2. Use Notepad to create a new file named
\textbf{C:\textbackslash lab1\textbackslash lab1\_4.htm} with the
following contents:

\textless html\textgreater{}

\textless script\textgreater{}

function init() \{

Game Programming -- Penn Wu

16

\protect\hypertarget{index_split_002.htmlux5cux23p17}{}{}\includegraphics{index-17_1.png}

ball.style.pixelLeft=Math.floor(Math.random()*250);

ball.style.display="inline";

fall();

\}

function fall() \{

if (ball.style.pixelTop\textgreater275) \{ball.style.pixelTop=0;\}

else \{ ball.style.pixelTop+=5;\}

if (ball.style.pixelLeft\textgreater=375) \{ball.style.pixelLeft=0;\}

else \{ ball.style.pixelLeft+=Math.floor(Math.random()*5);\}

if (ball.style.pixelLeft+15\textless=bar.style.pixelLeft+50 \&\&
ball.style.pixelLeft\textgreater=bar.style.pixelLeft \&\&

ball.style.pixelTop\textgreater=bar.style.pixelTop)

\{ clearTimeout(s1); ball.style.display="none";\}

else \{ s1=setTimeout("fall()",30);\}

\}

function move() \{

var e=event.keyCode;

switch (e) \{

case 37:

if(bar.style.pixelLeft \textless= 0) \{bar.style.pixelLeft = 0;\}

else \{ bar.style.pixelLeft -= 5;\}

break;

case 39:

if(bar.style.pixelLeft+50 \textgreater= 400) \{bar.style.pixelLeft =
350;\}

else \{ bar.style.pixelLeft += 5;\}

break;

\}

\}

\textless/script\textgreater{}

\textless body onLoad="init()" onKeyDown="move()"\textgreater{}

\textless div style="position:absolute; width:400; height:300;
background-color:green; border:solid 2 red; top:10;
left:10"\textgreater{}

\textless img id="ball" src="ball.gif" style="position:absolute;
display:none; width:15"\textgreater{}

\textless span id="bar" style="background-color:white;
width:50;height:20; position:absolute; top:277;left:165;
border-bottom:solid 2 red"\textgreater\textless/span\textgreater{}

\textless/div\textgreater{}

\textless/body\textgreater{}

\textless/html\textgreater{}

3. Use Internet Explorer to execute the file. To play the game, use Left
(←) and Right(→) arrow keys to move the bar. A sample output looks:

Game Programming -- Penn Wu

17

\protect\hypertarget{index_split_002.htmlux5cux23p18}{}{}\textbf{Learning
Activity \#5: Random gophers}

Note: Do not worry if you don't understand the code. Details will be
described in later lectures. For now, just have fun!

1. Change to the C:\textbackslash lab1 directory.

2. Use Notepad to create a new file named
\textbf{C:\textbackslash lab1\textbackslash lab1\_5.htm} with the
following contents:

\textless html\textgreater{}

\textless style\textgreater{}

img \{position:absolute;\}

\textless/style\textgreater{}

\textless script\textgreater{}

function gplace() \{

g1.style.pixelLeft =
Math.floor(Math.random()*(document.body.clientWidth-50));
g1.style.pixelTop =
Math.floor(Math.random()*(document.body.clientHeight-50));
g2.style.pixelLeft =
Math.floor(Math.random()*(document.body.clientWidth-50));
g2.style.pixelTop =
Math.floor(Math.random()*(document.body.clientHeight-50));
g3.style.pixelLeft =
Math.floor(Math.random()*(document.body.clientWidth-50));
g3.style.pixelTop =
Math.floor(Math.random()*(document.body.clientHeight-50)); gjump();

\}

function gjump() \{

var i = Math.floor(Math.random()*3)+1;

switch (i) \{

case 1:

g1.src="gopher2.gif";g2.src="gopher1.gif";g3.src="gopher1.gif"; break;

case 2:

g2.src="gopher2.gif";g1.src="gopher1.gif";g3.src="gopher1.gif"; break;

case 3:

g3.src="gopher2.gif";g1.src="gopher1.gif";g2.src="gopher1.gif"; break;

\}

setTimeout("gplace()", 800);

\}

\textless/script\textgreater{}

\textless body onLoad="gplace()"\textgreater{}

\textless img src="gopher1.gif" id="g1"\textgreater{}

\textless img src="gopher1.gif" id="g2"\textgreater{}

\textless img src="gopher1.gif" id="g3"\textgreater{}

\textless/body\textgreater{}

\textless/html\textgreater{}

3. Use Internet Explorer to execute the file. Each gopher will stretch
its head out of the hole randomly.

Game Programming -- Penn Wu

18

\protect\hypertarget{index_split_002.htmlux5cux23p19}{}{}\includegraphics{index-19_1.png}

\textbf{Submittal}

Upon completing all the learning activities,

1. Upload the following files to your remote web site (e.g.
Geocities.com)

•

ani\_cat.gif

•

ball.gif

•

bat.gif

•

gopher1.gif

•

gopher2.gif

•

lab1\_1.htm

•

lab1\_2.htm

•

lab1\_3.htm

•

lab1\_4.htm

•

lab1\_5.htm

2. Test your program remotely through the web. Make sure they function
correctly.

3. Log in to Blackboard, launch Assignment 01, and then scroll down to
question 11.

4. Copy and paste the URLs to your games to the textbox, such as:

•

http://www.geocities.com/cis261/lab1\_1.htm

•

http://www.geocities.com/cis261/lab1\_2.htm

•

http://www.geocities.com/cis261/lab1\_3.htm

•

http://www.geocities.com/cis261/lab1\_4.htm

•

http://www.geocities.com/cis261/lab1\_5.htm

Note: No credit is given to broken link(s).

Game Programming -- Penn Wu

19

\protect\hypertarget{index_split_002.htmlux5cux23p20}{}{}

Lecture \#2

Sprite Programming - Game graphics

\protect\hypertarget{index_split_003.html}{}{}

\hypertarget{index_split_003.htmlux5cux23calibre_pb_2}{%
\subsection{Introduction}\label{index_split_003.htmlux5cux23calibre_pb_2}}

The term ``graphics'' refers to any computer device or program that
makes a computer capable of displaying and manipulating pictures.
Graphics images are used extensively in games. You are surrounded with
graphics as a matter of fact.

Before jumping into the details of how graphics work and how they are
applied to games, it\textquotesingle s important to establish some
ground rules and gain an understanding of how computer graphics work in
general. More specifically, you need to have a solid grasp on what a
graphics coordinate system is, as well as how color is represented in
computer graphics.

The Graphics

All graphical computing systems use some sort of graphics coordinate
system to specify how Coordinate

points are arranged in a window or on the screen. Graphics coordinate
systems typically spell out System

the origin (0,0) of the system, as well as the axes and directions of
increasing value for each of the axes. If you\textquotesingle re not a
big math person, this simply means that a coordinate system describes
how to pinpoint any location on the screen as an X-Y value. The
traditional mathematical coordinate system familiar to most of us is
shown below:

\emph{y} axis

\emph{x} axis

(0, 0)

The Basics of

Before this lecture can move on and use some sample code to explain the
graphic programming Color

concepts, you need to have some background in computer and web color
theory.

The main function of color in a computer system is to accurately reflect
the physical nature of color within the confines of a computer, which
requires a computer color system to mix colors with accurate,
predictable results.

A color monitor has three electron guns: red, green, and blue, and they
are called the three color elements. The output from these three guns
converges on each pixel on the screen, exciting phosphors to produce the
appropriate color. The combined intensities of each gun determine the
resulting pixel color.

Technically speaking, Windows colors are represented by the combination
of the numeric intensities of the primary colors (red, green, and blue).
This color system is known as RGB (Red Green Blue) and is standard
across most graphical computer systems.

The following table shows the numeric values for the red, green, and
blue components of some basic colors. Notice that the intensities of
each color component range from 0 to 255 in value.

Table: Numeric RGB Color Component Values for Commonly Used Colors
\textbf{Color}

\textbf{Red}

\textbf{Green}

\textbf{Blue}

White

255

255

255

Black

0

0

0

Light Gray

192

192

192

Dark Gray

128

128

128

Red

255

0

0

Game Programming -- Penn Wu

20

\protect\hypertarget{index_split_003.htmlux5cux23p21}{}{}\includegraphics{index-21_1.png}

Green

0

255

0

Blue

0

0

255

Yellow

255

255

0

Purple

255

0

255

For example,

\textless html\textgreater{}

\textless style\textgreater{}

span \{width:20; height:70; border:solid 1 \textbf{black}\}

\textless/style\textgreater{}

\textless span id="bar1" style="background-color:
\textbf{red}"\textgreater\textless/span\textgreater{}

\textless span id="bar2" style="background-color:
\textbf{green}"\textgreater\textless/span\textgreater{}

\textless span id="bar2" style="background-color:
\textbf{blue}"\textgreater\textless/span\textgreater{}

\textless/html\textgreater{}

and the output looks:

Notice that the above colors are represented by using decimal values. In
CSS and DHTML, colors can be also defined as a hexadecimal notation for
the combination of Red, Green, and Blue color values (RGB). The lowest
value that can be given to one light source is 0 (hex \#00) and the
highest value is 255 (hex \#FF). For example,

Color

Color HEX

Color RGB

Black

\#000000

rgb(0,0,0)

Red

\#FF0000

rgb(255,0,0)

Lime

\#00FF00

rgb(0,255,0)

Blue

\#0000FF

rgb(0,0,255)

Yellow

\#FFFF00

rgb(255,255,0)

Aqua

\#00FFFF

rgb(0,255,255)

Purple

\#FF00FF

rgb(255,0,255)

White

\#FFFFFF

rgb(255,255,255)

Consider the following code, it uses a both hexadecimal and decimal
color values to specify color schemes.

\textless html\textgreater{}

\textless style\textgreater{}

span \{width:20; height:70; border:solid 1 \textbf{rgb(0,0,0)}\}

\textless/style\textgreater{}

\textless span id="bar1" style="background-color:
\textbf{rgb(255,0,0)}"\textgreater\textless/span\textgreater{}

\textless span id="bar2" style="background-color:
\textbf{\#00FF00}"\textgreater\textless/span\textgreater{}

\textless span id="bar2" style="background-color:
\textbf{\#0000FF}"\textgreater\textless/span\textgreater{}

\textless/html\textgreater{}

The output looks:

Game Programming -- Penn Wu

21

\protect\hypertarget{index_split_003.htmlux5cux23p22}{}{}\includegraphics{index-22_1.png}

Some years ago, when most computers only supported \textbf{256}
different colors, a list of \textbf{216} Web Safe Colors was suggested
as a Web standard. The reason for this was that the Microsoft and Mac
operating system used 40 different "reserved" fixed system colors (about
20 each).

We are not sure how important this is now, since most computers today
have the ability to display millions of different colors, but the choice
is left to you. The 216 cross-browser color palette was created to
ensure that all computers would display the colors correctly when
running a 256 color palette.

Sprite

A sprite is a general term for a graphic that can be moved independently
around the screen to programming

produce animated effects. Many sprites are programmed by the game
programmer and are created after the game starts, so they are also known
as movable object blocks. This type of sprites is actually a code blocks
that is temporarily stored in memory and is transferred to the screen.

Given the following text arts (also known as ASCII arts), each is an
individual character-formed graphic simulating a stage of a human's
dancing motion.

O

O

O

O

.\textbar-.\textquotesingle{}

..\textbar-\/-,

`.\textbar-

-.\textbar-\/-\textquotesingle{}

/ =

` =

= \textbackslash.

` =

\textbar{} \textbackslash{}

/ \textbar{}

/ \textbackslash{}

/ \textbackslash{}

/. \textbackslash.

/. /.

/. \textbar.

\textbar. \textbackslash.

By displaying one of them at a time in a sequence and then repeat the
sequence over and over again, you can create a sprite of dancing man. In
the following code, each ASCII art is placed in between
\textless pre\textgreater{} and \textless/pre\textgreater. The HTML
\textless pre\textgreater{} tag defines preformatted text. The text
enclosed in the pre element usually preserves spaces and line breaks.
The text renders in a fixed-pitch font.

\textless body onLoad=init()\textgreater{}

\textbf{\textless pre id="h1" style="display:none"\textgreater{}} O

.\textbar-.\textquotesingle{}

/ =

\textbar{} \textbackslash{}

/. \textbackslash.

\textless/pre\textgreater{}

\textbf{\textless pre id="h2" style="display:none"\textgreater{}} O

..\textbar-\/-,

` =

/ \textbar{}

/. /.

\textless/pre\textgreater{}

\textbf{\textless pre id="h3" style="display:none"\textgreater{}}
Ò.\textbar-

= \textbackslash.

/ \textbackslash{}

/. \textbar.

\textless/pre\textgreater{}

\textbf{\textless pre id="h4" style="display:none"\textgreater{}} O

Game Programming -- Penn Wu

22

\protect\hypertarget{index_split_003.htmlux5cux23p23}{}{}-.\textbar-\/-\textquotesingle{}

` =

/ \textbackslash{}

\textbar. \textbackslash.

\textless/pre\textgreater{}

\textless/body\textgreater{}

Each \textless pre\textgreater{} tag is assigned an ID-\/-h1, h2, h3,
h4, so they can be easily identified. They are also configured by
\textbf{display:none}, so they will not be displayed by default.

In order to dynamically control the sprite, you can add the following
JavaScript code with DHTML support.

\textless script\textgreater{}

var i=1;

function init() \{

if (i\textgreater4) \{ i=1; \}

else \{

clear();

switch (i) \{

case 1: h1.style.display=\textquotesingle inline\textquotesingle; break;

case 2: h2.style.display=\textquotesingle inline\textquotesingle; break;

case 3: h3.style.display=\textquotesingle inline\textquotesingle; break;

case 4: h4.style.display=\textquotesingle inline\textquotesingle; break;

\}

i++;

\}

s1 = setTimeout("init()", 100);

\}

function clear() \{

h1.style.display=\textquotesingle none\textquotesingle;

h2.style.display=\textquotesingle none\textquotesingle;

h3.style.display=\textquotesingle none\textquotesingle;

h4.style.display=\textquotesingle none\textquotesingle;

\}

\textless/script\textgreater{}

The \textbf{init()} function uses i++ to continuously adding 1 to the
value of \emph{i}. The following line forces the computer to set the
maximum to 4. Literally it means ``if the value of \emph{i} is greater
than 4, the value of \emph{i} must return to 1''.

if (i\textgreater4) \{ i=1; \}

The following code block forces the computer to apply codes based on the
value of \emph{i}. For example, when \emph{i} = 3, on the codes for case
3 are executed, the rest are ignored.

switch (i) \{

case 1: h1.style.display=\textquotesingle inline\textquotesingle; break;

case 2: h2.style.display=\textquotesingle inline\textquotesingle; break;

case 3: h3.style.display=\textquotesingle inline\textquotesingle; break;

case 4: h4.style.display=\textquotesingle inline\textquotesingle; break;

\}

The \textbf{setTimeout()} method forces the computer to execute the
init() function every 100

milliseconds.

The \textbf{clear()} function sets each ASCII art to be hidden again
(display='none'). This function clears the background, so you will not
see double-, triple-, or multiple-images.

Game Programming -- Penn Wu

23

\protect\hypertarget{index_split_003.htmlux5cux23p24}{}{}\includegraphics{index-24_1.png}

A sample output looks:

Some sprites are pre-created animated .gif files. This kind of sprites
is rectangular graphic created by combining multiple GIF images in one
file. In other words, they are a batch of images, displayed one after
another to give the appearance of movement.

Although graphics design is not a topic of this course, Appendix A
provides a short lecture on how you can create animated GIF files.

Create

New game programmers usually relate this ``term'' to a pre-designed
software-produced graphics, graphics

but tend to forget the fact that many computer languages are also
capable of producing graphics programmatica

programmatically.

lly

Consider the following code, it uses CSS (cascade style sheet) to
specify how an area defined by

\textless span\textgreater\ldots\textless/span\textgreater{} in a code
division (defined by
\textless div\textgreater\ldots\textless/div\textgreater) should look in
a Web browser's body area.

\textless html\textgreater{}

\textless style\textgreater{}

embedded CSS style sheet

.wKey \{

background-color:white;

border-top:solid 1 \#cdcdcd;

border-left:solid 1 \#cdcdcd;

border-bottom:solid 4 \#cdcdcd;

border-right:solid 1 black;

height:150px;

width:40px

\}

\textless/style\textgreater{}

\textless div\textgreater{}

\textless!-\/- white keys -\/-\textgreater{}

\textless span \textbf{class="wKey"}
id="midC"\textgreater\textless/span\textgreater{}

\textless/div\textgreater{}

\textless/html\textgreater{}

CSS uses \textless style\textgreater\ldots\textless/style\textgreater{}
to embed a style sheet onto an HTML code. In this example, wKey is the
name of style, which contains the following attributes and their
associated values in the format of \emph{atttributeName} : \emph{value}.

•

\textbf{background-color : white}. This set of attribute and value
simply defines the background of the object using this style as
\emph{white}.

•

\textbf{border-top : solid 1 \#cdcdcd}. This sets the top border to be a
solid line with size of 1px and color value of \#cdcdcd (in the RGB
format). This color combination produces the color of light gray.

•

\textbf{height : 150px}. This sets the height of the object to be 150px.

•

\textbf{width : 40px}. This sets the height of the object to be 40px.

Game Programming -- Penn Wu

24

\protect\hypertarget{index_split_003.htmlux5cux23p25}{}{}\includegraphics{index-25_1.png}

\includegraphics{index-25_2.png}

The following line associates an object with ID \textbf{midC} with the
\textbf{wKey} style, such that all the styles wKey has will apply to
this midC object.

\textless span \textbf{class="wKey"}
id="midC"\textgreater\textless/span\textgreater{} The output of this
code looks:

In order to add two more objects that are clones of the midC object,
simply add the following two lines. Just be sure to assign new IDs for
each of the two new objects!

\textless span class="wKey" id="midC"
style="left:50"\textgreater\textless/span\textgreater{}

\textless span class="wKey" \textbf{id="midD"}
style="left:91"\textgreater\textless/span\textgreater{}

\textless span class="wKey" \textbf{id="midF"}
style="left:132"\textgreater\textless/span\textgreater{} The output now
looks:

Take a close look at these three objects, they are placed closely
together, there's no space between any consecutive ones. In order to
dynamically place each object on the web browser's body area, you need
to use absolute positioning system. You do so by adding the following
bold line:

.wKey \{

\textbf{position:absolute;}

background-color:white;

border-top:solid 1 \#cdcdcd;

border-left:solid 1 \#cdcdcd;

border-bottom:solid 4 \#cdcdcd;

border-right:solid 1 black;

height:150px;

width:40px

\}

DHTML, with CSS, use two pre-defined variables -\/- \emph{left} and
\emph{top} -\/- to represent \emph{x}- and \emph{y}-

coordinate of the browser's body area. In fact, most graphics in a
Windows program are drawn to the client area of a window, which uses the
Windows graphics coordinate system.

The following figures explains how the coordinates of the client area
begin in the upper-left Game Programming -- Penn Wu

25

\protect\hypertarget{index_split_003.htmlux5cux23p26}{}{}\includegraphics{index-26_1.png}

\includegraphics{index-26_2.png}

\includegraphics{index-26_3.png}

corner of the window and increase down and to the right, as you learned
earlier in the hour. This coordinate system is very important because
most GDI graphics operations are based on them.

Consequently, by adding the following bold line, all the three objects'
y-coordinate are set to be 20px, which means they all are 20px below the
\emph{x}-axis.

.wKey \{

position:absolute;

\textbf{top:20;}

background-color:white;

border-top:solid 1 \#cdcdcd;

border-left:solid 1 \#cdcdcd;

border-bottom:solid 4 \#cdcdcd;

border-right:solid 1 black;

height:150px;

width:40px

\}

The output now looks weird. There are 3 objects, supposedly, but only
one is displayed. What's wrong?

The answer is, the absolute positioning system uses both \emph{left} and
\emph{top} to specify a given object of absolute position. All the three
objects' \emph{top} values are defined as 20px in the above code
segment.

It's time to assign each object's \emph{left} value individually.

\textless span class="wKey" id="midC" \textbf{style="left:50"}
\textgreater\textless/span\textgreater{}

\textless span class="wKey" id="midD" \textbf{style="left:91"}
\textgreater\textless/span\textgreater{}

\textless span class="wKey" id="midE" \textbf{style="left:132"}
\textgreater\textless/span\textgreater{} The above uses a technique
known as \textbf{inline style}, which means to add a style to the HTML
tag as an attribute of the tag. The syntax is:

style = "attributeName1:Value1;attributeName2:Value2;...;

attributeName \emph{n}:Value \emph{n}"

Game Programming -- Penn Wu

26

\protect\hypertarget{index_split_003.htmlux5cux23p27}{}{}\includegraphics{index-27_1.png}

The midC object is now assigned an initial point (20, 50), midD (20,
91), and midE (20, 132). The distance between 50 and 91 is 41. In this
code example, 41 is determined because each object is 40px wide (as
specified by width:40px), plus 1px as the space between two objects.

50 + 40 + 1 = 91

90 + 40 + 1 = 132

Take a good look at the output. There is a visible space between every
two objects.

(20, 91)

(20, 50)

(20, 132)

To make the above graphics looks like a section of piano keyboard, add
the following bold lines:

\textless html\textgreater{}

\textless style\textgreater{}

.wKey \{

position:absolute;

top:20;

background-color:white;

border-top:solid 1 \#cdcdcd;

border-left:solid 1 \#cdcdcd;

border-bottom:solid 4 \#cdcdcd;

border-right:solid 1 black;

height:150px;

width:40px

\}

A new CSS style

\textbf{.bKey \{}

\textbf{position:absolute;}

\textbf{top:20;}

\textbf{background-color:black;}

\textbf{border:solid 1 white;}

\textbf{height:70px;}

\textbf{width:36px}

\textbf{\}}

\textless/style\textgreater{}

\textless div\textgreater{}

\textless!-\/- white keys -\/-\textgreater{}

\textless span class="wKey" id="midC"
style="left:50"\textgreater\textless/span\textgreater{}

\textless span class="wKey" id="midD"
style="left:91"\textgreater\textless/span\textgreater{}

\textless span class="wKey" id="midE"
style="left:132"\textgreater\textless/span\textgreater{}

\textbf{\textless!-\/- black keys -\/-\textgreater{}}

\textbf{\textless span class="bKey" id="cSharp"
style="left:72"\textgreater\textless/span\textgreater{}} Game
Programming -- Penn Wu

27

\protect\hypertarget{index_split_003.htmlux5cux23p28}{}{}\includegraphics{index-28_1.png}

\textbf{\textless span class="bKey" id="dSharp"
style="left:114"\textgreater\textless/span\textgreater{}}

\textless/div\textgreater{}

\textless/html\textgreater{}

The first added code segment defines a new CSS style named
\textbf{bKey}, which specifies how a black key should look. The second
simply adds two black keys in the division ( as defined by
\textless div\textgreater{} \ldots{}

\textless/div\textgreater). Each black key has a unique ID.

The output now looks (Surprisingly, you create a graphic using DHTML
codes!): Creating

To add animated effects to the above graphics, you need to write
\textbf{user-defined functions}. Such animated

functions can perform visual effects. For example, you can add visual
effects to the keys, so they graphics

will have an effect of key-down-key-up.

programmatica

lly

Consider the following added bold codes:

\textless span class="wKey" id="midC" style="left:50"
\textbf{onMouseOver="CDown()"}

\textbf{onMouseOut="CUp()"} \textgreater\textless/span\textgreater{}

It uses two \textbf{event handlers:} \emph{onMouseOver} \textbf{} and
\emph{onMouseOut} to call two different user-defined functions:
\emph{CDown}() and \emph{Cup}(). Such programming techniques will be
discussed in a later lecture.

For now, you just have to know what they are.

Two functions are created by adding the following lines to the HTML
file:

\textless script\textgreater{}

\textless!-\/- for middle C -\/-\textgreater{}

function CDown() \{

midC.style.borderTop="solid 1 black";

midC.style.borderLeft="solid 1 black";

midC.style.borderBottom="solid 1 black";

midC.style.borderRight="solid 1 \#cdcdcd";

\}

function CUp() \{

midC.style.borderTop="solid 1 \#cdcdcd";

midC.style.borderLeft="solid 1 \#cdcdcd";

midC.style.borderBottom="solid 4 \#cdcdcd";

midC.style.borderRight="solid 1 black";

\}

\textless/script\textgreater{}

These two user-defined functions, CDown() and Cup(), are combinations of
DHTML statements.

The syntax is (notice that \textbf{style} is a keyword):

objectID. \textbf{style}.AttributeNme = "Value";

In CDown(), the first line changes the top border of the \textbf{midC}
object to ``solid 1 black'', the second changes the left border, the
third changes the bottom border, and the fourth changes the right border
to ``solid 1 \#cdcdcd''.

Game Programming -- Penn Wu

28

\protect\hypertarget{index_split_003.htmlux5cux23p29}{}{}Compare CDown()
and CUp() carefully and you will find out their color values in the
reversed order to each other. Also, be sure to know that the color
values of CUp() are exactly the same as those defined in wKey style.

function CDown() \{

midC.style.borderTop="solid 1 \textbf{black}";

midC.style.borderLeft="solid 1 \textbf{black}";

midC.style.borderBottom="solid 1 \textbf{black}";

midC.style.borderRight="solid 1 \textbf{\#cdcdcd}";

\}

function CUp() \{

midC.style.borderTop="solid 1 \textbf{\#cdcdcd}";

midC.style.borderLeft="solid 1 \textbf{\#cdcdcd}";

midC.style.borderBottom="solid 4 \textbf{\#cdcdcd}";

midC.style.borderRight="solid 1 \textbf{black}";

\}

By executing these two functions, the color of midC's top border changes
from \#cdcdcd to black, and then back to \#cdcdcd. This creates an
effect of key-down-key-up. A complete code is available in the learning
activity, please learning this technique for your own good.

The advantage of creating graphic programmatically is that you can
easily apply dynamic functions to the graphics.

Adding visual

A technique called \textbf{alpha} is greatly used today to create
visual. Alpha is a value representing a effects to

pixel's degree of transparency. The more transparent a pixel, the less
it hides the background graphics

against which the image is presented. In PNG, alpha is really the degree
of opacity: zero alpha programmatica

represents a completely transparent pixel, maximum alpha represents a
completely opaque pixel.

lly

But most people refer to alpha as providing transparency information,
not opacity information.

Many other techniques are available as explained in the following table,
which contains commonly used DHTML effects. The syntax of using filter
is:

filter:effectName(parameter)

Table: Common filter effects:

Effects

Description

Example

alpha()

Set a transparency level

\{filter:alpha(opacity=2)\}

blur()

Creates the impression of moving at \{filter:blur(strength=5)\}

high speed

chroma()

Makes a specific color transparent

\{filter:chroma(color=\#008855)\}

dropshadow() Creates an offset solid silhouette

\{filter:dropshadow(color=\#0066cc)\}

fliph

Creates a horizontal mirror image

\{filter:fliph()\}

flipv()

Creates a vertical mirror image

\{filter:flipv()\}

glow()

Adds radiance around the outside

\{filter:glow(color=\#ccffdd)\}

edges of the object

grayscale()

Drops color information from the

image

invert()

Reverses the hue, saturation, and

\{filter:invert()\}

brightness values

light()

Project light sources onto an object

mask()

Creates a transparent mask from an

object

shadow()

Creates a solid silhouette of the

\{filter:shadow(direction=10)\}

object

wave()

Create a sine wave distortion along

the x- and y-axis

Game Programming -- Penn Wu

29

\protect\hypertarget{index_split_003.htmlux5cux23p30}{}{}\includegraphics{index-30_1.png}

the x- and y-axis

xray()

Shows just the edges of the object

\{filter:xray()\}

For example, you can use some of these functions to create visual
effects of texts.

\textless html\textgreater\textless head\textgreater\textless style\textgreater{}

font \{font-family:arial;font-size:24px;width:100\%;\}

\textless/style\textgreater\textless/head\textgreater\textless body\textgreater{}

\textless font
style="filter:glow()"\textgreater Glow\textless/font\textgreater{}

\textless font
style="filter:blur()"\textgreater Blur\textless/font\textgreater{}

\textless font style="filter:fliph()"\textgreater Flip
Horontally\textless/font\textgreater{}

\textless font style="filter:flipv()"\textgreater Flip
Vertically\textless/font\textgreater{}

\textless font
style="filter:shadow()"\textgreater Shadow\textless/font\textgreater{}

\textless font style="filter:dropshadow()"\textgreater Drop
Shadow\textless/font\textgreater{}

\textless/body\textgreater\textless/html\textgreater{}

The output looks:

On the other hand, some of them apply to images and graphics, too.
Consider the following example, it demonstrates how the original images
are displayed using some of these functions:

\textless html\textgreater{}

\textless style\textgreater{}

img \{width:100\}

\textless/style\textgreater{}

\textless table\textgreater{}

\textless tr\textgreater\textless td\textgreater Original\textless/td\textgreater\textless td\textgreater Alpha:\textless/td\textgreater\textless td\textgreater Gray:\textless/td\textgreater{}

\textless td\textgreater Invert:\textless/td\textgreater\textless td\textgreater Xray:\textless/td\textgreater\textless/tr\textgreater{}

\textless tr\textgreater\textless td\textgreater{}

\textless img src="fujiwara.jpg"\textgreater{}

\textless/td\textgreater\textless td\textgreater{}

\textless img src="fujiwara.jpg"
style="filter:alpha(Opacity=20)"\textgreater{}

\textless/td\textgreater\textless td\textgreater{}

\textless img src="fujiwara.jpg" style="filter:gray()"\textgreater{}

\textless/td\textgreater\textless td\textgreater{}

\textless img src="fujiwara.jpg" style="filter:invert()"\textgreater{}

\textless/td\textgreater\textless td\textgreater{}

\textless img src="fujiwara.jpg" style="filter:xray()"\textgreater{}

\textless/td\textgreater\textless/tr\textgreater\textless/table\textgreater{}

\textless/html\textgreater{}

The output looks:

Game Programming -- Penn Wu

30

\protect\hypertarget{index_split_003.htmlux5cux23p31}{}{}\includegraphics{index-31_1.png}

Review

1. Which is the color combination of "red"?

Questions

A. \#0000FF

B. rgb(255,255,0)

C. \#00FF00

D. rgb(255,0,0)

2. Given the following code segment, what background color will the
object have?

style="background-color: \#00FF00"

A. red

B. green

C. blue

D. yellow

3. Given the following code segment, which is the closet name of the
color?

border-top:solid 1 \#cdcdcd;

A. lime

B. grey

C. orange

D. pink

4. Given the following code segment, which statement is correct?

\textless script\textgreater{}

img \{ width:35; height:40\}

\textless/script\textgreater{}

A. the image height is set to be 40 centimeters

B. the image width is set to be 35 pixels

C. the image height is set to be 40 dot-per-inch

D. the image width is set to be 35 inches

5. Given the following code segment, which is the correct coordinate
combination of the object

"dot"?

dot.style.right=50;

dot.style.left=200;

dot.style.top=150;

dot.style.b0ttom=100;

Game Programming -- Penn Wu

31

\protect\hypertarget{index_split_003.htmlux5cux23p32}{}{}A. (50, 200)

B. (50, 150)

C. (200, 150)

D. (150, 100)

6. Which is the correct way to create a glow effect?

A. \textless span
style="filter:glow()"\textgreater Glow\textless/span\textgreater{} B.
\textless span
style="glow()"\textgreater Glow\textless/span\textgreater{}

C. \textless span
style="filter:glow"\textgreater Glow\textless/span\textgreater{}

D. \textless span
style="filter(glow)"\textgreater Glow\textless/span\textgreater{} 7.
\_\_ is a value representing a pixel's degree of transparency.

A. Alpha

B. chroma

C. fliph

D. shadow

8. A(n) \_\_ is a general term for a graphic that can be moved
independently around the screen to produce animated effects.

A. image

B. graphic

C. sprite

D. gif

9. Given the following code, which statement is incorrect?

\textless pre id="h1" style="display:none"\textgreater{}

O

.\textbar-.\textquotesingle{}

/ =

\textbar{} \textbackslash{}

/. \textbackslash.

\textless/pre\textgreater{}

A. It is an object, and its name is h1.

B. It is defaulted to be displayed as the word "none".

C. using h1.style.display=\textquotesingle inline\textquotesingle{} can
make it visible.

D. It is defaulted to be invisible.

10. Given the following code segment, which statement is correct?

function init() \{

........

i++;

........

\}

A. It uses i++ to continuously adding 1 to the value of i.

B. It uses i++ to continuously adding + to the value of i.

C. It uses i++ to continuously adding ++ to the value of i.

D. It uses i++ to continuously adding i to the value of i.

Game Programming -- Penn Wu

32

\protect\hypertarget{index_split_003.htmlux5cux23p33}{}{}

Lab \#2

Game Graphics

\textbf{}

\textbf{Preparation \#1:}

1. Create a new directory named \textbf{C:\textbackslash games}.

2. Use Internt Explorer to go to
\textbf{http://business.cypresscollege.edu/\textasciitilde pwu/cis261/download.htm}
to download lab2.zip (a zipped) file. Extract the files to
C:\textbackslash games directory.

\textbf{}

\textbf{Learning Activity \#1: Basic Sprite Programming I}

1. Change to the C:\textbackslash games directory.

2. Use Notepad to create a new file named
\textbf{C:\textbackslash games\textbackslash lab2\_1.htm} with the
following contents:

\textless html\textgreater{}

\textless script\textgreater{}

var i=1;

function init() \{

if (i\textgreater4) \{ i=1; \}

else \{

clear();

switch (i) \{

case 1: h1.style.display=\textquotesingle inline\textquotesingle; break;

case 2: h2.style.display=\textquotesingle inline\textquotesingle; break;

case 3: h3.style.display=\textquotesingle inline\textquotesingle; break;

case 4: h4.style.display=\textquotesingle inline\textquotesingle; break;

\}

i++;

\}

s1 = setTimeout("init()", 100);

\}

function clear() \{

h1.style.display=\textquotesingle none\textquotesingle;

h2.style.display=\textquotesingle none\textquotesingle;

h3.style.display=\textquotesingle none\textquotesingle;

h4.style.display=\textquotesingle none\textquotesingle;

\}

\textless/script\textgreater{}

\textless body onLoad=init()\textgreater{}

\textless pre id="h1" style="display:none"\textgreater{}

O

.\textbar-.\textquotesingle{}

/ =

\textbar{} \textbackslash{}

/. \textbackslash.

\textless/pre\textgreater{}

\textless pre id="h2" style="display:none"\textgreater{}

O

..\textbar-\/-,

` =

/ \textbar{}

/. /.

\textless/pre\textgreater{}

\textless pre id="h3" style="display:none"\textgreater{}

Ò.\textbar-

= \textbackslash.

Game Programming -- Penn Wu

33

\protect\hypertarget{index_split_003.htmlux5cux23p34}{}{}\includegraphics{index-34_1.png}

/ \textbackslash{}

/. \textbar.

\textless/pre\textgreater{}

\textless pre id="h4" style="display:none"\textgreater{}

O

-.\textbar-\/-\textquotesingle{}

` =

/ \textbackslash{}

\textbar. \textbackslash.

\textless/pre\textgreater{}

\textless/body\textgreater{}

\textless/html\textgreater{}

3. Use Internet Explorer to execute the file. A sample output looks:

\textbf{}

\textbf{Learning Activity \#2: Basic Sprite Programming II}

\textbf{}

Note: Do not worry if you don't understand the code. Details will be
described in later lectures. For now, just have fun!

\textbf{}

1. Change to the C:\textbackslash games directory.

2. Use Notepad to create a new file named
\textbf{C:\textbackslash games\textbackslash lab2\_2.htm} with the
following contents:

\textless html\textgreater{}

\textless style\textgreater{}

.pm \{position: absolute;left=0; top:10 \}

\textless/style\textgreater{}

\textless script\textgreater{}

function init() \{

if (ball.style.pixelLeft \textgreater{} 60) \{ pacman.style.pixelLeft++;
\}

if (pacman.style.pixelLeft \textgreater{} 70) \{
ghost.style.pixelLeft++;\}

ball.style.pixelLeft++;

s1=setTimeout("init()", 10);

\}

\textless/script\textgreater{}

\textless body onLoad=init()\textgreater{}

\textless pre id="ball" class="pm"\textgreater{}

.-.

\textquotesingle-\textquotesingle{}

\textless/pre\textgreater{}

\textless pre id="pacman" class="pm"\textgreater{}

-\/-.

/ o.-\textquotesingle{}

Game Programming -- Penn Wu

34

\protect\hypertarget{index_split_003.htmlux5cux23p35}{}{}\includegraphics{index-35_1.png}

\textbackslash{} \textquotesingle-.

\textquotesingle-\/-\textquotesingle{}

\textless/pre\textgreater{}

\textless pre id="ghost" class="pm"\textgreater{}

.-.

\textbar{} OO\textbar{}

\textbar{} \textbar{}

\textquotesingle\^{}\^{}\^{}\textquotesingle{}

\textless/pre\textgreater{}

\textless/body\textgreater{}

\textless/html\textgreater{}

3. Use Internet Explorer to execute the file. A sample output looks:
\textbf{Learning Activity \#3: Creating graphics programmatically -- A
Chess Board} 1. Change to the C:\textbackslash games directory.

2. Use Notepad to create a new file named
\textbf{C:\textbackslash games\textbackslash lab2\_3.htm} with the
following contents:

\textless html\textgreater{}

\textless style\textgreater{}

.cell \{ width:30; height:30; border:solid 1 black; \}

\textless/style\textgreater{}

\textless script\textgreater{}

function init() \{

code1="";

for (i=1; i\textless=72; i++) \{

if (i\%9==0) \{ code1+="\textless br\textgreater"; \}

else if (i\%2==0) \{ code1+="\textless span id=c"+i+"
class=cell\textbackslash n";
code1+="style=\textquotesingle background-color:gray\textquotesingle\textgreater\textless/span\textgreater";
\}

else \{ code1+="\textless span id=c"+i+" class=cell\textbackslash n";
code1+="style=\textquotesingle background-color:white\textquotesingle\textgreater\textless/span\textgreater";
\}

\}

area1.innerHTML = code1;

\}

\textless/script\textgreater{}

\textless body onLoad=init()\textgreater{}

\textless div id="area1"\textgreater\textless/div\textgreater{}

\textless/body\textgreater{}

\textless/html\textgreater{}

3. Use Internet Explorer to execute the file. A sample output looks:

\textbf{}

Game Programming -- Penn Wu

35

\protect\hypertarget{index_split_003.htmlux5cux23p36}{}{}\includegraphics{index-36_1.png}

\textbf{}

\textbf{}

\textbf{Learning Activity \#4: Creating animated graphics
programmatically} 1. Change to the C:\textbackslash games directory.

2. Use Notepad to create a new file named
\textbf{C:\textbackslash games\textbackslash lab2\_4.htm} with the
following contents:

\textless html\textgreater{}

\textless style\textgreater{}

.wKey \{

position:absolute;

top:20;

background-color:white;

border-top:solid 1 \#cdcdcd;

border-left:solid 1 \#cdcdcd;

border-bottom:solid 4 \#cdcdcd;

border-right:solid 1 black;

height:150px;

width:40px

\}

.bKey \{

position:absolute;

top:20;

background-color:black;

border:solid 1 white;

height:70px;

width:36px

\}

\textless/style\textgreater{}

\textless script\textgreater{}

function Down() \{

var keyID = event.srcElement.id;

document.getElementById(keyID).style.borderTop="solid 1 black";
document.getElementById(keyID).style.borderLeft="solid 1 black";
document.getElementById(keyID).style.borderBottom="solid 1 black";
document.getElementById(keyID).style.borderRight="solid 1 \#cdcdcd";

\}

function Up() \{

var keyID = event.srcElement.id;

document.getElementById(keyID).style.borderTop="solid 1 \#cdcdcd";
document.getElementById(keyID).style.borderLeft="solid 1 \#cdcdcd";
document.getElementById(keyID).style.borderBottom="solid 4 \#cdcdcd";
document.getElementById(keyID).style.borderRight="solid 1 black";

\}

\textless/script\textgreater{}

\textless div\textgreater{}

\textless span class="wKey" id="midC" style="left:50"

onClick="Down()"
onMouseOut="Up()"\textgreater\textless/span\textgreater{} Game
Programming -- Penn Wu

36

\protect\hypertarget{index_split_003.htmlux5cux23p37}{}{}\includegraphics{index-37_1.png}

\textless span class="wKey" id="midD" style="left:91"

onClick="Down()"
onMouseOut="Up()"\textgreater\textless/span\textgreater{}

\textless span class="wKey" id="midE" style="left:132"

onClick="Down()"
onMouseOut="Up()"\textgreater\textless/span\textgreater{}

\textless span class="bKey" id="cSharp"
style="left:72"\textgreater\textless/span\textgreater{}

\textless span class="bKey" id="dSharp"
style="left:114"\textgreater\textless/span\textgreater{}

\textless/div\textgreater{}

\textless/html\textgreater{}

3. Use Internet Explorer to execute the file. Use the mouse cursor to
move among the three keys, you should the visual effects.

\textbf{Learning Activity \#5: Creating animation programmatically}

1. Change to the C:\textbackslash games directory.

2. Use Notepad to create a new file named
\textbf{C:\textbackslash games\textbackslash lab2\_5.htm} with the
following contents:

\textless html\textgreater{}

\textless script\textgreater{}

function rotate1() \{

m1.src=\textquotesingle star2.gif\textquotesingle;

m2.src=\textquotesingle star1.gif\textquotesingle;

m3.src=\textquotesingle star2.gif\textquotesingle;

m4.src=\textquotesingle star1.gif\textquotesingle;

m5.src=\textquotesingle star2.gif\textquotesingle;

setInterval("rotate2()",100);

\}

function rotate2() \{

m1.src=\textquotesingle star1.gif\textquotesingle;

m2.src=\textquotesingle star2.gif\textquotesingle;

m3.src=\textquotesingle star1.gif\textquotesingle;

m4.src=\textquotesingle star2.gif\textquotesingle;

m5.src=\textquotesingle star1.gif\textquotesingle;

setInterval("rotate1()",100);

\}

\textless/script\textgreater{}

\textless body
onKeyDown=\textquotesingle rotate1()\textquotesingle\textgreater{}

\textless img id=m1 src="star1.gif"\textgreater{}

\textless img id=m2 src="star2.gif"\textgreater{}

\textless img id=m3 src="star1.gif"\textgreater{}

\textless img id=m4 src="star2.gif"\textgreater{}

\textless img id=m5 src="star1.gif"\textgreater{}

\textless/body\textgreater{}

\textless/html\textgreater{}

3. Use Internet Explorer to execute the file. Press any key, so the
stars rotate. A sample output looks: Game Programming -- Penn Wu

37

\protect\hypertarget{index_split_003.htmlux5cux23p38}{}{}\includegraphics{index-38_1.png}

\textbf{}

\textbf{Submittal}

Upon completing all the learning activities,

1. Upload all files you created in this lab to your remote web server.

2. Log in to Blackboard, launch Assignment 02, and then scroll down to
question 11.

3. Copy and paste the URLs to the textbox. For example,

•

http://www.geocities.com/cis261/lab2\_1.gif

•

http://www.geocities.com/cis261/lab2\_2.gif

•

http://www.geocities.com/cis261/lab2\_3.htm

•

http://www.geocities.com/cis261/lab2\_4.htm

•

http://www.geocities.com/cis261/lab2\_5.htm

No credit is given to broken link(s).

Game Programming -- Penn Wu

38

\protect\hypertarget{index_split_003.htmlux5cux23p39}{}{}

Lecture \#3

Control of flows

Objective

Game programs are not linear-\/-each statement executes in order, from
top to bottom-\/-especially when the players have options. To create
interesting games, you need to write programs that execute (or skip)
sections of code based on some condition. This lecture discusses about
how you can control flows of a game program.

\protect\hypertarget{index_split_004.html}{}{}

\hypertarget{index_split_004.htmlux5cux23calibre_pb_3}{%
\subsection{Introduction}\label{index_split_004.htmlux5cux23calibre_pb_3}}

In computer science, control flow (or alternatively, flow of control)
refers to the order in which the individual statements or instructions
of an imperative program are performed or executed.

Within a programming language, a control flow statement is an
instruction that when executed can cause a change in the subsequent
control flow to differ from the natural sequential order in which the
instructions are listed. Discussion of control flow is almost always
restricted to a single thread of execution, as it depends upon a
definite sequence in which instructions are executed one at a time.

Consider the following example. The numbers at the beginning of each
indicates the line of the code. However, the sequence of the execution,
which is how the computer will read the statements are different.

Li

Code

ne

01 \textless html\textgreater\textless body onLoad=cc()\textgreater{}

02

03 \textless script\textgreater{}

04 var i=0;

05

06 function cc() \{

07 m1.style.left=i;

08 i+=5;

09

10 \textbf{if (i \textgreater= document.body.clientWidth)}

\textbf{}

11 \textbf{\{i=0;\}}

12

13 setTimeout("cc()",70);

14 \}

15 \textless/script\textgreater{}

16

17 \textless img id=m1 src="ani\_cat.gif"

18 style="position:absolute;top:10;left:0"\textgreater{}

19

20 \textless/body\textgreater\textless/html\textgreater{}

The first line, \textless html\textgreater\textless body
onLoad=cc()\textgreater, contains an event handler \textbf{onLoad} which
calls a user-defined function \textbf{cc()} to make decision about what
the computer should respond to the changing of the \emph{x}-coordinate
of an image file that has an ID \textbf{m1}. In other word, the
execution sequence is not necessary same as the line sequence.

Expressions,

An \textbf{expression} is a construct made up of variables, operators,
and method invocations, which are Statements,

constructed according to the syntax of the language, which evaluates to
a single value. You\textquotesingle ve and Blocks

already seen examples of expressions, illustrated in bold below:

•

var \textbf{i = Math.floor(Math.random()*10)};

•

p1.innerHTML = " \textbf{\textless img
src=\textquotesingle my.gif\textquotesingle\textgreater{}} ";

•

\textbf{i++};

•

for (\textbf{i=0}; \textbf{i\textless=10}; \textbf{i++)}

Game Programming -- Penn Wu

39

\protect\hypertarget{index_split_004.htmlux5cux23p40}{}{}\textbf{Statements}
are roughly equivalent to sentences in natural languages. A statement
forms a complete unit of execution. The following types of expressions
can be made into a statement by terminating the expression with a
semicolon (;).

•

Assignment expressions

•

Any use of ++ or -\/-

•

Method invocations

•

Object creation expressions

Such statements are called \textbf{expression statements}. Here are some
examples of expression statements.

x = 8933.234; // assignment statement

i++; // increment statement

document.write("Hello World!"); // method invocation statement Bicycle
myBike = new Bicycle(); // object creation statement

In addition to expression statements, there are two other kinds of
statements: declaration statements and control flow statements. A
declaration statement declares a variable. You have not seen any
examples of declaration statements, because JavaScript does not require
type declaration.

The following is an example used by other languages:

double aValue = 8933.234; //declaration statement

Finally, \textbf{control flow} statements regulate the order in which
statements get executed. Details of control flow statement will be
discussed in later section.

A \textbf{block} is a group of zero or more statements between balanced
braces and can be used anywhere a single statement is allowed.

for (i=1; i\textless=300; i++) \{

x = Math.round(Math.random()*screen.width);

y = Math.round(Math.random()*screen.height);

code1 = "\textless span id=dot"+i;

code1 += " style=\textquotesingle left:" + x + "; top:" + y
+"\textquotesingle\textgreater.\textless/span\textgreater";
document.write(code1)

\}

Noticeably, operators may be used in building expressions, which compute
values; expressions are the core components of statements; statements
may be grouped into blocks.

The if-then

The \textbf{if-then} statement is the most basic of all the control flow
statements. It tells your program to statement

execute a certain section of code only if a particular test evaluates to
true. In JavaScript, the syntax:

if (condition)

\{

code to be executed if condition is true

\}

Refer back to the above code, an \emph{if..then} statement is used in
the format of JavaScript to determine if the value of variable
\emph{\textbf{i}} is larger or equal to player's web browser width
(which is represented by \textbf{document.body.clientWidth}).

if (i\textgreater=document.body.clientWidth-20)

If the evaluation result is true, then the computer executes:

\{i=0;\}

Game Programming -- Penn Wu

40

\protect\hypertarget{index_split_004.htmlux5cux23p41}{}{}\includegraphics{index-41_1.png}

If this test evaluates to false (meaning that the value of \emph{i} is
less than that of document.body.clientWidth), control jumps to the end
of the if-then statement and continue with the following statement:

setTimeout("cc()",70);

\textbf{setTimeout} is a method that evaluates an expression after a
specified number of milliseconds has elapsed. The syntax is:

setTimeout(vCode, iMilliSeconds {[}, sLanguage{]})

The above statement uses the setTimeout method to evaluate the cc()
function after 70

milliseconds has elapsed. And, since it is place within the cc()
function, this setTimeout method forces the computer to execute cc()
every 70 milliseconds. In other words, over and over again.

Obviously, the first line \textless body onLoad=cc()\textgreater{} is
only executed once during the entire program life cycle. It's mission is
to start the cc() function. All the control flow arrangement in cc()
will determine what the computer should do afterwards. The output thus
looks: The \textbf{if..then} statement used in this code provides a
simple \textbf{decisive logic}, which means ``if the condition does not
make, forget the whole thing! You don't get to have the second
option!.'' Refer back to the following code block,

if (i \textgreater= document.body.clientWidth)

\{i=0;\}

The condition is the part, i \textgreater= document.body.clientWidth,
which can be either true or false. This statement read ``if \emph{i} is
greater than or equal to the value of the document body's width of the
client's browser, then \emph{i} equals one''. However, it does not
describe what to do if the above condition is false.

The

The \textbf{if-then-else} statement provides a secondary path of
execution when an "if" clause evaluates if..then..else

to false. In JavaScript, the syntax:

statement

if (condition)

\{

code to be executed if condition is true

\}

else

\{

code to be executed if condition is not true

\}

In the following example, there are two user defined functions
\textbf{cc()} and \textbf{dd()}. They both use the if..then..else
statement to take some action if the expected condition does not happen.
To be more specific, \textbf{m1} is the ID of an object (which is the
bat.gif file) and its value of width increments by 10px every time when
the cc() is executed (and decrements by 10px when dd() is executed).

Game Programming -- Penn Wu

41

\protect\hypertarget{index_split_004.htmlux5cux23p42}{}{}The condition
of cc() is that the value of width must be less than 400px. If the
condition is false (meaning m1's width value is \textgreater=400), then
call the dd() function (or you can say to skip cc() and move to dd()
function).

Reversely the condition of dd() is to be greater than 10px. If the
condition is false (meaning m1's width value \textless=10), then call
the cc() function.

\textless html\textgreater{}

\textless script\textgreater{}

function cc() \{

m1.style.width = m1.style.pixelWidth + 10;

if (m1.style.pixelWidth \textless{} 400) \{

setTimeout("cc()", 50);

\}

else \{dd(); \}

\}

function dd() \{

m1.style.width = m1.style.pixelWidth - 10;

if (m1.style.pixelWidth \textgreater{} 10) \{

setTimeout("dd()", 50);

\}

else \{cc(); \}

\}

\textless/script\textgreater{}

\textless body onLoad=cc()\textgreater{}

\textless center\textgreater\textless img id=m1 src="bat.gif"
style="width:10"\textgreater\textless/center\textgreater{}

\textless/body\textgreater{}

\textless/html\textgreater{}

DHTML use the following syntax to represent attribute of a given object
(Note: style is a keyword).

objectID. \textbf{style}.AttributeName

The \textbf{pixelWidth} attribute of m1 is represented by
\textbf{m1.style.pixelWidth}, and its value is a string by default, such
as \textbf{20px}. The value pixelWidth holds is an integer, so it can be
evaluated by comparison operators (e.g. \textgreater, \textless,
\textgreater=, \textless=). For example, if (\textbf{m1.style.pixelWidth
\textless{} 400}) \{

setTimeout("cc()", 50);

\}

else \{dd(); \}

\textbf{}

If the condition is true, the computer executes setTimeout(``cc()'', 50)
after 50 milliseconds; otherwise, it calls the dd() function.

The \emph{if..then..else} statement is good for anything that falls to
the two-state phenomena-\/-either true or false. However, most games are
not that black or white; they are very colorful, and require more than
one simple decisive logics to make decisions. In this case, you can
consider using nested \emph{if..then..else} statement, or using the
switch..case statement.

The

The \textbf{switch..case} statement is an alternative to the
If...Then...Else statement. It also makes code switch..case

easier to read especially when testing lots of different conditions. It
also makes the code more statement

efficient. In JavaScript, the \textbf{switch..case} statement uses the
following syntax: Game Programming -- Penn Wu

42

\protect\hypertarget{index_split_004.htmlux5cux23p43}{}{} \emph{switch}
(testexpression) \{

\emph{Case} expressionlist1:

Statements;

\emph{Case} expressionlist2:

Statements;

..............

..............

default : statement;

\}

where \emph{testexpression} is a single expression, which is evaluated
at the top of the list. The \emph{expressionlist} then lists possible
conditions and if the \emph{testexpression} is equal to this case, the
code in that section is run.

When using a \emph{switch..case} statement, you need to declare a
variable that will be matching against with the options in the case
list.

Inside the \emph{switch..Case} statement, you define separate
\emph{Case} statements for each condition to be checked against. In this
example, you have 5 and each one is set to respond to a different blood
type. If a match can be found, computer executes the code immediately
following the relevant Case statement.

Another point of interest is the \textbf{break} statement after each
case. Each \textbf{break} statement terminates the enclosing switch
statement. Control flow continues with the first statement following the
switch block. The \textbf{break} statements are necessary because
without them, case statements fall through; that is, without an explicit
break, control will flow sequentially through subsequent case
statements.

The \textbf{default} section handles all values that
aren\textquotesingle t explicitly handled by one of the case sections.
It is sometime good idea to have the \textbf{default} section, but it
depends on the situation.

The random gopher game uses the \emph{switch..case} statement to
controls the flow.

...............

\textless script\textgreater{}

..............

\textbf{}

\textbf{switch (i) \{}

\textbf{case 1:}

\textbf{g1.src="gopher2.gif";g2.src="gopher1.gif";g3.src="gopher1.gif";}
\textbf{break;}

\textbf{}

\textbf{case 2:}

\textbf{g2.src="gopher2.gif";g1.src="gopher1.gif";g3.src="gopher1.gif";}
\textbf{break;}

\textbf{}

\textbf{case 3:}

\textbf{g3.src="gopher2.gif";g1.src="gopher1.gif";g2.src="gopher1.gif";}
\textbf{break;}

\textbf{\}}

................

\textless/script\textgreater{}

................

The \textbf{gjump()} function contains a variable \emph{\textbf{i}},
whose value is assigned by a random number generator -\/- the
\textbf{Math.random()} function -\/- of JavaScript. By default, this
random function returns a random number between zero and 1 with the
decimal point. The following is a list of possibly results of
Math.random():

Game Programming -- Penn Wu

43

\protect\hypertarget{index_split_004.htmlux5cux23p44}{}{}

0.47145134906738295

0.8141135558627748

0.696928080402613

0.9209730732076619

0.8254258690882129

0.03148481558914484

The results are unpredictable, ant they are all smaller than 1. To
generate a smaller range of random number in JavaScript, simply use the
following code:

var variableName=Math.floor(Math.random()*11)

where 11 dictates that the random number will fall between 0-10. To
increase the range to, say, 100, simply change 11 to 101 instead.

You may be curious as to why \textbf{Math.round()}, instead of
Math.floor(), is used in the above code.

While both successfully round off its containing parameter to an integer
within the designated range, Math.floor does so more ``evenly'', so the
resulting integer isn't lopsided towards either end of the number
spectrum. In other words, a more random number is returned using
Math.floor().

This code uses the following line to randomly generate four possible
values 0, 1, 2, and 3.

var i = Math.round(Math.random()*3);

Since 0 is not valid case (only 1, 2, and 3 are listed used in the above
code), the following exclude the possibility of 0 as result.

if (i \textless{} 1) \{ i = Math.round(Math.random()*3); \}

Consider the following code block. If the value of \emph{\textbf{i}}
returns out to be 1, the computer will only carry out the code of the
case 1. The keyword break forces the flow to stop, so the codes that
belong to case 2 and case 3 will not be read and executed by the
computer.

\textbf{switch (i) \{}

\textbf{case 1:}

\textbf{g1.src="gopher2.gif";g2.src="gopher1.gif";g3.src="gopher1.gif";}
\textbf{break;}

Similarly, if the value of \emph{\textbf{i}} returns out to be 2, the
computer will only carry out the code of the case 2. Case 1 and case 3
are ignored.

case 1:

g1.src="gopher2.gif";g2.src="gopher1.gif";g3.src="gopher1.gif"; break;

\textbf{case 2:}

\textbf{g2.src="gopher2.gif";g1.src="gopher1.gif";g3.src="gopher1.gif";}
\textbf{break;}

case 3:

g3.src="gopher2.gif";g1.src="gopher1.gif";g2.src="gopher1.gif"; break;

Technically, the final break is not required because flow would fall out
of the switch statement anyway. However, we recommend using a break so
that modifying the code is easier and less error-prone.

The \emph{for} loop

The \emph{for} statement provides a compact way to iterate over a range
of values. Programmers often Game Programming -- Penn Wu

44

\protect\hypertarget{index_split_004.htmlux5cux23p45}{}{}refer to it as
the "for loop" because of the way in which it repeatedly loops until a
particular condition is satisfied. The general form of the for statement
can be expressed as follows: for (initialization; termination;
increment) \{

statement(s);

\}

When using this version of the for statement, keep in mind that:

•

The \emph{initialization} expression initializes the loop;
it\textquotesingle s executed once, as the loop begins.

•

When the \emph{termination} expression evaluates to false, the loop
terminates.

•

The \emph{increment} expression is invoked after each iteration through
the loop; it is perfectly acceptable for this expression to increment or
decrement a value.

Given the following code:

\textless html\textgreater{}

\textless style\textgreater{}

span \{position:absolute; color:red\}

\textless/style\textgreater{}

\textless script\textgreater{}

code1 = "";

function draw\_star() \{

for (i=1; i\textless=300; i++) \{

x = Math.round(Math.random()*screen.width);

y = Math.round(Math.random()*screen.height);

code1 += "\textless span id=dot"+i;

code1 += " style=\textquotesingle left:" + x + "; top:" + y
+"\textquotesingle\textgreater.\textless/span\textgreater";

\}

bd.innerHTML = code1;

\}

\textless/script\textgreater{}

\textless body id="bd" onLoad="draw\_star();"\textgreater{}

\textless/body\textgreater{}

\textless/html\textgreater{}

The \emph{for} loop is set to repetitively add an objects to the
browser's body area based on the following logic:

\textless span id=dot \emph{\textbf{n}} \textbf{}
style=\textquotesingle left: \emph{\textbf{x}}; top:
\emph{\textbf{y}}\textquotesingle\textgreater.\textless/span\textgreater{}
where \emph{\textbf{n}} represents a list of consecutive numbers
starting with 1 and end at 300, because the for loop definition:

for (i=1; i\textless=300; i++)

\textbf{x} and \textbf{y} are random numbers generated by the
\textbf{Math.random()} function multiplied by the screen width and
screen height. The \textbf{Math.round()} function is used to round
up/down the random values to integers.

Since \emph{x} and \emph{y} are inside the \emph{\{} and \emph{\}} of
the \emph{for} loop, the computer will generate a set of random numbers
each time when \emph{i} increments (i++). There will be 300 sets of
randomly generated \emph{x}- and \emph{y}- coordinates, each set is used
to place one dot (.) one the browser's body area. In other words, the
final client side HTML output consist 300 lines of:

\textless span id=dot1 style=\textquotesingle left:
\emph{\textbf{x}}\textbf{1}; top:
\emph{\textbf{y}}\textbf{1}\textquotesingle\textgreater.\textless/span\textgreater{}

\textless span id=dot2 style=\textquotesingle left:
\emph{\textbf{x}}\textbf{2}; top:
\emph{\textbf{y}}\textbf{2}\textquotesingle\textgreater.\textless/span\textgreater{}
Game Programming -- Penn Wu

45

\protect\hypertarget{index_split_004.htmlux5cux23p46}{}{}.....................

.....................

\textless span id=dot299 style=\textquotesingle left:
\emph{\textbf{x}}\textbf{299}; top:
\emph{\textbf{y}}\textbf{299}\textquotesingle\textgreater.\textless/span\textgreater{}

\textless span id=dot300 style=\textquotesingle left:
\emph{\textbf{x}}\textbf{300}; top:
\emph{\textbf{y}}\textbf{300}\textquotesingle\textgreater.\textless/span\textgreater{}
Obviously the \emph{for} loop is used to shorten the game code. For
example, the \textbf{cc()} function in the following code uses the
\emph{for} loop to repeats itself 13 times. Each time it produce a new
HTML

statement, which will be inserted to a blank space on the web browser's
body area with an ID

\textbf{area1} (specified by the \textbf{\textless span
id=area1\textgreater\textless/span\textgreater{}} statement).

\textless html\textgreater{}

\textless SCRIPT\textgreater{}

function cc()\{

\textbf{for (i=1; i\textless=13; i++) \{}

\textbf{area1.innerHTML += "\textless img src=c" + i + ".gif id=c" + i +
"\textgreater";}

\textbf{\}}

\}

\textless/SCRIPT\textgreater{}

\textless body onLoad=cc()\textgreater{}

\textless span id=area1\textgreater\textless/span\textgreater{}

\textless/body\textgreater\textless html\textgreater{}

The above bold code block will produce the following HTML codes, which
are to be inserted inside the \textless span
id=area1\textgreater\textless/span\textgreater{} code block.

\textless img src=c1.gif id=c1\textgreater{}

\textless img src=c2.gif id=c2\textgreater{}

\textless img src=c3.gif id=c3\textgreater{}

\textless img src=c4.gif id=c4\textgreater{}

\textless img src=c5.gif id=c5\textgreater{}

\textless img src=c6.gif id=c6\textgreater{}

\textless img src=c7.gif id=c7\textgreater{}

\textless img src=c8.gif id=c8\textgreater{}

\textless img src=c9.gif id=c9\textgreater{}

\textless img src=c10.gif id=c10\textgreater{}

\textless img src=c11.gif id=c11\textgreater{}

\textless img src=c12.gif id=c12\textgreater{}

\textless img src=c13.gif id=c13\textgreater{}

The \emph{for} loop is very helpful in producing similar code block
repetitively, especially when you need to reproduce a large amount of
them. The property \textbf{innerHTML}, in DHTML, sets or retrieves the
HTML contents between the start and end tags of the object (as in the
above example

\textless span\textgreater{} and \textless/span\textgreater). It is
valid for block elements only. The syntax is: objectID.innerHTML="HTML
contents"

Notice that the above code use a \textbf{+=} operator, which mean
``appending'' or ``adding without replacing the current content''. This
operator is used because when the \textbf{innerHTML} property is set,
the given HTML string completely replaces the existing content of the
object. You can remove the + sign (only the enlarged one) from the
following statement, and test the code again to see for yourself.

area1.innerHTML \textbf{+}= "\textless img src=c" + i + ".gif id=c" + i
+ "\textgreater"; Instead of using the innerHTML property, you can also
consider using the \textbf{insertAdjacentHTML}

method, as shown in the following code.

Game Programming -- Penn Wu

46

\protect\hypertarget{index_split_004.htmlux5cux23p47}{}{}

\textless html\textgreater{}

\textless script\textgreater{}

function cc() \{

for (i=1; i\textless=3; i++) \{

p1.insertAdjacentHTML("BeforeBegin","\textless img
src=\textquotesingle gopher"+i+"gif\textquotesingle{}

id=g"+i+"\textgreater\textless br\textgreater")

\}

\}

\textless/script\textgreater{}

\textless body onLoad=cc()\textgreater{}

\textless span id=p1\textgreater\textless/span\textgreater{}

\textless/body\textgreater{}

\textless/html\textgreater{}

The insertAdjacentHTML method inserts the given HTML text into the
element at the location.

The syntax is:

objectID.insertAdjacentHTML(location, string)

Available position values are:

• beforeBegin - Inserts the text immediately before the element.

• afterBegin - Inserts the text after the start of the element but
before all other content in the element.

• beforeEnd - Inserts the text immediately before the end of the element
but after all other content in the element.

• afterEnd - Inserts the text immediately after the end of the element.

The \emph{for..in}

The for..in loop is a bit different from the other loops
we\textquotesingle ve seen so far. It allows you to loop statement

through the properties of a JavaScript object.

If you are unfamiliar with objects in JavaScript, think of them as black
boxes that can have a number of properties associated with them. For
example, a cat object might have a color property with a value of
"black", and an ears property with a value of 2! The basic for..in
construct looks like this:

for ( variable\_name in object\_name )

\{

\textless{} do stuff with the property here \textgreater{}

\textless{} the current property name is stored in variable\_name
\textgreater{}

\}

Note that, on each pass through the loop, the loop variable
variable\_name holds the name of the current property. To obtain the
value of the current property, you would use the following syntax:
object\_name{[}variable\_name{]}

For example, this code loops through all the properties of the navigator
object (a built-in JavaScript object that holds information about your
browser, adds each property\textquotesingle s name and value to a
string, then displays the resulting string in an alert box: function
display\_nav\_props ( )

\{

var i;

var output\_string = "";

Game Programming -- Penn Wu

47

\protect\hypertarget{index_split_004.htmlux5cux23p48}{}{}\includegraphics{index-48_1.png}

for ( i in navigator )

\{

output\_string += "The value of " + i +

" is: " + navigator{[}i{]} + "\textbackslash n";

\}

alert ( output\_string );

\}

The \emph{while} loop The \textbf{while} loop is used when you want the
loop to execute and continue executing while the specified condition is
true. The syntax in JavaScript is:

while (condition)

\{

code to be executed

\}

The while statement evaluates \emph{expression}, which must return a
boolean value. If the expression evaluates to true, the while statement
executes the \emph{statement}(s) in the while block. The while statement
continues testing the expression and executing its block until the
expression evaluates to false.

For example, you can apply a while loop to a shooting game with the
logic ``while the \emph{x}-

coordinate is greater than or equal to 400, set the \emph{x}-coordinate
value to 95''.

\textless html\textgreater{}

\textless script\textgreater{}

var i = 95; // declare a variable i

function shoot() \{

b01.style.left=i; //let the current x-coordinate equal i value

i+=10; // i increment by 10

\textbf{while (i \textgreater= 400) \{}

\textbf{b01.style.display=\textquotesingle none\textquotesingle;}

\textbf{i=95;}

\textbf{\}}

setTimeout("shoot()",20); // call shoot() after 20 milliseconds

\}

\textless/script\textgreater{}

\textless body
onKeyDown="shoot();b01.style.display=\textquotesingle inline\textquotesingle;"\textgreater{}

\textless img src="gun.gif"\textgreater{}

\textless img src="bullet.gif" id="b01" style="position:absolute;
display:none; top:20; left:95"\textgreater\textless/span\textgreater{}

\textless/body\textgreater{}

\textless/html\textgreater{}

The output looks:

Game Programming -- Penn Wu

48

\protect\hypertarget{index_split_004.htmlux5cux23p49}{}{}

Sometimes, the \textbf{while} statement and \textbf{if..then} statement
produce the same results. You can write the Cat Running game using the
while statement. For example,

\textless html\textgreater{}

\textless body onLoad=cc()\textgreater{}

\textless script\textgreater{}

var i=0;

function cc() \{

m1.style.left=i;

i+=5;

\textbf{while (i\textgreater=document.body.clientWidth-20) \{i=0; \}}

setTimeout("cc()",70);

\}

\textless/script\textgreater{}

\textless img id=m1 src="ani\_cat.gif"

style="position:absolute;top:10;left:0"\textgreater{}

\textless/body\textgreater{}

\textless/html\textgreater{}

You can always use a \emph{while} loop to run the same block of code
while a specified condition is true.

The \emph{do..while}

Many programming language also provides a do-while statement, which can
be expressed as loop

follows:

do \{

statement(s)

\} while (expression);

The \textbf{do..while} statement is very similar to the while statement.
For example, function cancel\_to\_finish ( )

\{

var confirm\_result;

do

\{

confirm\_result = confirm ( "Press Cancel to finish!" );

\} while ( confirm\_result != 0 );

\}

The difference between do-while and while is that do-while evaluates its
expression at the bottom of the loop instead of the top. Therefore, the
statements within the do block are always executed at least once,

Use a do...while loop to run the same block of code while a specified
condition is true. This loop will always be executed at least once, even
if the condition is false, because the statements are executed before
the condition is tested.

Review

1. Which is a valid expression?

Questions

A. i = Math.floor(Math.random()*10);

B. i = i + 2;

C. i++;

D. for (i=0; i\textless=10; i++)

2. Given the following code block, which statement is correct?

Game Programming -- Penn Wu

49

\protect\hypertarget{index_split_004.htmlux5cux23p50}{}{}

if (i\textgreater=document.body.clientWidth-20) \{ i=i+1; \}

A. If this test evaluates to false, control jumps to \{ i=i+1; \}

B. It tests to determine if the value of variable i is larger or equal
to player's web browser width.

C. document.body.clientWidth represents the server\textquotesingle s
resolution values D. All of the above

3. Which is an example of "simple decisive logic"?

A. if (i\textgreater= 3) \{ i++; \} else \{ i-\/-;\}

B. if (i\textgreater= 3) \{ i++; \} else if \{ i-\/-;\} else \{i=0;\}

C. if (i\textgreater= 3) \{ i++; \}

D. None of the above

4. Given the following code block, which statement is correct?

if
(eval(m1.style.width.replace(\textquotesingle px\textquotesingle,\textquotesingle\textquotesingle))
\textless{} 400) \{

setTimeout("cc()", 50);

\}

A. The condition states that the value of width must be less than 400px.

B. If m1's width value is \textgreater=400, call the cc() function.

C. The condition states that it only allows 50 times of executions.

D. If m1\textquotesingle s width equals to 400 mulitpled by 50, then
stops.

5. Given the following, which is a keyword that must always be present?

m1.style.width

A. m1

B. style

C. width

D. all of the above

6. Given the following code segment, what does the replace() method do?

eval(m1.style.width.replace(\textquotesingle px\textquotesingle,\textquotesingle\textquotesingle));

A. It removes the \textquotesingle px\textquotesingle{} substring.

B. It replaces the \textquotesingle px\textquotesingle{} substring with
the user\textquotesingle s entry.

C. It replaces the \textquotesingle px\textquotesingle{} substring with
the new width value.

D. It replaces the \textquotesingle{} substring with the
\textquotesingle px\textquotesingle.

7. Given the following code segment, which is the output if the value of
i is 0?

swith(i)

case 1: p1.innerHTML = "Apple"; break;

case 2: p1.innerHTML = "Orange"; break;

case 3: p1.innerHTML = "Banana"; break;

default: p1.innerHTML = "Grape"; break;

A. Apple

B. Orange

C. Banana

D. Grape

8. Given the following code segment, which is not a possible output?

Game Programming -- Penn Wu

50

\protect\hypertarget{index_split_004.htmlux5cux23p51}{}{}var i =
Math.floor(Math.random()*3);

A. 3

B. 2

C. 1

D. 0

9. Given the following code segment, what will be inserted to p1?

for (i=0; i\textless=4; i++) \{

p1.innerHTML += i;

\}

A. 0123

B. 01234

C. 1234

D. 123

10. Given the following code segment, Which statement is correct?

while (i \textgreater= 400) \{

b01.style.display=\textquotesingle none\textquotesingle;

i=95;

b01.style.pixelLeft = i;

\}

A. while the y-coordinate is greater than or equal to 400, set the
y-coordinate value to 95

B. while the y-coordinate is greater than or equal to 95, set the
y-coordinate value to 400

C. while the x-coordinate is greater than or equal to 400, set the
x-coordinate value to 95

D. while the x-coordinate is greater than or equal to 95, set the
x-coordinate value to 400

Game Programming -- Penn Wu

51

\protect\hypertarget{index_split_004.htmlux5cux23p52}{}{}\includegraphics{index-52_1.png}

Lab \#3

Control of flows

\textbf{Preparation \#1:}

1. Create a new directory named \textbf{C:\textbackslash games}.

2. Use Internt Explorer to go to
\textbf{http://business.cypresscollege.edu/\textasciitilde pwu/cis261/download.htm}
to download lab3.zip (a zipped) file. Extract the files to
C:\textbackslash games directory.

\textbf{Learning Activity \#1: Shooting gun}

1. Change to the C:\textbackslash games directory.

2. Use Notepad to create a new file named
\textbf{C:\textbackslash games\textbackslash lab3\_1.htm} with the
following contents:

\textless html\textgreater{}

\textless script\textgreater{}

function shoot() \{

var i = b01.style.pixelLeft;

if (i \textless= document.body.clientWidth - b01.style.pixelWidth - 10)
\{

b01.style.left=i + 10;

setTimeout("shoot()",20);

\}

else \{b01.style.display=\textquotesingle none\textquotesingle;

b01.style.left=95;

\};

\}

\textless/script\textgreater{}

\textless body
onKeyDown="shoot();b01.style.display=\textquotesingle inline\textquotesingle;"\textgreater{}

\textless img src="gun.gif"\textgreater{}

\textless img src="bullet.gif" id="b01" style="position:absolute;
display:none; top:20; left:95"\textgreater\textless/span\textgreater{}

\textless/body\textgreater{}

\textless/html\textgreater{}

Note: Compare this version with the one in lecture note. This version
uses \textbf{if..then} statement; the other use \textbf{while}
\textbf{loop}.

3. Test the program. Press any key to shoot. Watch how the bullet moves.
A sample output looks: \textbf{Learning Activity \#2: UFO}

1. Change to the C:\textbackslash games directory.

2. Use Notepad to create a new file named
\textbf{C:\textbackslash games\textbackslash lab3\_2.htm} with the
following contents:

\textless html\textgreater{}

\textless style\textgreater{}

.dots \{position:absolute; color:white\}

Game Programming -- Penn Wu

52

\protect\hypertarget{index_split_004.htmlux5cux23p53}{}{}\includegraphics{index-53_1.png}

\textless/style\textgreater{}

\textless script\textgreater{}

code1 ="";

function draw\_star() \{

for (i=1; i\textless=600; i++) \{

x = Math.round(Math.random()*screen.width);

y = Math.round(Math.random()*screen.height);

code1 += "\textless span class=dots id=dot"+i;

code1 += " style=\textquotesingle left:" + x + "; top:" + y
+"\textquotesingle\textgreater.\textless/span\textgreater";

\}

bar.innerHTML = code1;

\}

function ufo\_fly() \{

ufo.style.left = Math.round(Math.random()*screen.width);

ufo.style.top = Math.round(Math.random()*screen.height);

setTimeout("ufo\_fly()", 500);

\}

\textless/script\textgreater{}

\textless body bgcolor=black
onLoad="draw\_star();ufo\_fly()"\textgreater{}

\textless div id="bar"
style="z-index:0"\textgreater\textless/div\textgreater{}

\textless img src="ufo.gif" id="ufo"
style="position:absolute;z-index:3;"\textgreater{}

\textless/body\textgreater{}

\textless/html\textgreater{}

3. Test the program. A UFO will appear and move around the sky that has
300 stars. A sample output looks: \textbf{Learning Activity \#3:
Shuffle-Organize playing card}

1. Change to the C:\textbackslash games directory.

2. Use Notepad to create a new file named
\textbf{C:\textbackslash games\textbackslash lab3\_3.htm} with the
following contents:

\textless html\textgreater{}

\textless script\textgreater{}

var code="";

function lineup()\{

area1.innerText = "";

for (i=1; i\textless=52; i++) \{

if (i\%13 == 0) \{

code += "\textless img src=c" + i + ".gif id=c" + i +
"\textgreater\textless br\textgreater";

\}

else \{

code += "\textless img src=c" + i + ".gif id=c" + i + "\textgreater";

\}

\}

Game Programming -- Penn Wu

53

\protect\hypertarget{index_split_004.htmlux5cux23p54}{}{}\includegraphics{index-54_1.png}

\includegraphics{index-54_2.png}

area1.innerHTML = code;

code = "";

\}

function scramble() \{

area1.innerText = "";

for (i=1; i\textless=52; i++) \{

x = Math.round(Math.random()*200);

y = Math.round(Math.random()*200);

code += "\textless img src=c" + i + ".gif id=c";

code += i + " style=\textquotesingle position:absolute; ";

code += "left:" + x + "; top:" + y + "\textquotesingle\textgreater";

\}

area1.innerHTML = code;

code = "";

\}

\textless/script\textgreater{}

\textless body onLoad=scramble() onClick=scramble()
onKeyDown=lineup()\textgreater{}

\textless span id="area1"\textgreater\textless/span\textgreater{}

\textless/body\textgreater{}

\textless/html\textgreater{}

3. Test the program. Press any key to organize, click the mouse to
shuffle. A sample output looks: Before organizing

After organizing

\textbf{Learning Activity \#4: Moving the ball}

1. Change to the C:\textbackslash games directory.

2. Use Notepad to create a new file named
\textbf{C:\textbackslash games\textbackslash lab3\_4.htm} with the
following contents:

\textless html\textgreater{}

\textless script\textgreater{}

function move() \{

var i = event.keyCode;

switch(i) \{

case 37:

if ( ball.style.pixelLeft \textless= 10) \{ ball.style.pixelLeft=10; \}

else \{ ball.style.pixelLeft -= 1; \}

break;

case 39:

if ((ball.style.pixelLeft + ball.style.pixelWidth) \textgreater=
area1.style.pixelWidth-10)

\{ ball.style.pixelLeft = (area1.style.pixelWidth -
ball.style.pixelWidth -10);\}

else \{ ball.style.pixelLeft += 1; \}

Game Programming -- Penn Wu

54

\protect\hypertarget{index_split_004.htmlux5cux23p55}{}{}\includegraphics{index-55_1.png}

break;

case 38:

if ( ball.style.pixelTop \textless= 10) \{ ball.style.pixelTop=10; \}

else \{ ball.style.pixelTop -= 1;\}

break;

case 40:

if ((ball.style.pixelTop + ball.style.pixelHeight) \textgreater=
area1.style.pixelHeight-10)

\{ ball.style.pixelTop = (area1.style.pixelHeight -
ball.style.pixelHeight -10);\}

else \{ ball.style.pixelTop += 1; \}

break;

\}

\}

\textless/script\textgreater{}

\textless body onkeydown=move()\textgreater{}

\textless div id=area1 style="border:solid 1 black;

background-color:\#abcdef;

width:300px; height:300px; top:10;left:10;

position:absolute"\textgreater{} \textless/div\textgreater{}

\textless img id=ball src="ball.gif" style="top:10;left:10;
position:absolute"\textgreater{}

\textless/body\textgreater{}

\textless/html\textgreater{}

3. Test the program. Use the ←, ↑ , →, ↓ arrow keys to move the ball. A
sample output looks: \textbf{Learning Activity \#5: Squeeze the clown}

1. Change to the C:\textbackslash games directory.

2. Use Notepad to create a new file named
\textbf{C:\textbackslash games\textbackslash lab3\_5.htm} with the
following contents:

\textless html\textgreater{}

\textless script\textgreater{}

var k=0;

function goUp() \{

bar.style.display=\textquotesingle inline\textquotesingle;

bar.style.pixelTop=bar.style.pixelTop - 5;

if (clown.style.pixelTop \textless= 20) \{

clown.style.pixelHeight = bar.style.pixelTop - 20;

clown.style.pixelWidth = clown.style.pixelWidth+1;

clown.style.pixelTop = 20;

\}

Game Programming -- Penn Wu

55

\protect\hypertarget{index_split_004.htmlux5cux23p56}{}{}\includegraphics{index-56_1.png}

\includegraphics{index-56_2.png}

else \{

bar.style.pixelHeight=k;

k+=5;

clown.style.pixelTop=bar.style.pixelTop-120;

\}

\}

\textless/script\textgreater{}

\textless body onKeyDown=goUp()\textgreater{}

\textless hr style="position:absolute; left:10; top:0;

background-color:red; width:100;z-index:2" size="20px"
align="left"\textgreater{}

\textless img id="clown" src="clown.jpg" style="position:absolute;
left:20; top:280; width:54; z-index:3"\textgreater{}

\textless span id="bar" style="position:absolute; left:10; top:400;
background-color:red; width:100; display:none;
z-index:1"\textgreater\textless/span\textgreater{}

\textless/body\textgreater{}

\textless/html\textgreater{}

3. Test the program. Press the {[}Enter{]} key to pump up the clown. If
the clown hit the ceiling, the clown gets squeezed. Two sample output
are:

Clown not squeezed

Clown squeezed

\textbf{Submittal}

Upon completing all the learning activities,

1. Upload all files you created in this lab to your remote web server.

2. Log in to Blackboard, launch Assignment 03, and then scroll down to
question 11.

3. Copy and paste the URLs to the textbox. For example,

•

http://www.geocities.com/cis261/lab3\_1.htm

•

http://www.geocities.com/cis261/lab3\_2.htm

•

http://www.geocities.com/cis261/lab3\_3.htm

•

http://www.geocities.com/cis261/lab3\_4.htm

•

http://www.geocities.com/cis261/lab3\_5.htm

No credit is given to broken link(s).

Game Programming -- Penn Wu

56

\protect\hypertarget{index_split_004.htmlux5cux23p57}{}{}

Lecture \#4

Event and Event Handler

\protect\hypertarget{index_split_005.html}{}{}

\hypertarget{index_split_005.htmlux5cux23calibre_pb_4}{%
\subsection{Introduction}\label{index_split_005.htmlux5cux23calibre_pb_4}}

The most dramatic switch for most people new to game programming is the
fact that Windows-based game programs are event-driven, which means such
game programs can respond to their operating system environment, as
opposed to the environment taking cues from the program. In this
lecture, you will learn to use event handlers to detect and respond to
user's activity.

What is an

When a user plays a game, any activity the user has is an
\textbf{event}. Objects on the game can contain event?

the so-called \textbf{event handler} that can trigger your functions to
respond to such events. For example, you add an \textbf{onClick} event
handler to trigger a function when the user clicks that button (an
object).

\textless button onClick="myFunction()"\textgreater Click
Me\textless/button\textgreater{} In this case, the button is an object,
and it has an event handler \textbf{onClick} that triggers a
user-defined function named \textbf{myFunction()}.

During normal program execution, a large number of events occur, so
skilled use of the event-oriented programming is very rewarding. It
opens up possibilities for creating very intricate and complex game
programs.

The DOM (Document Object Model) contains many objects that store a
variety of information about the browser, the computer running the
browser, visited URLs, and so on. You can use them to make your game
program more joyful.

The event

According to DHTML, the \textbf{event} object represents the state of an
event (meaning whenever the object

event occurs), the state of the keyboard keys, the location of the
mouse, and the state of the mouse buttons. In other words, the
\textbf{event} object stores data about events.

Whenever an event fires, the computer places appropriate data about the
event into the event object - for example, where the mouse pointer was
on the screen at the time of the event, which mouse buttons were being
pressed at the time of the event, and other useful information.

Since the event object is a fairly active object, and it constantly
changing its properties, you need to have a good handle of the even
object. There are some general-purpose properties of the event object
covered in this article are listed in the following table.

Event object's

Description

property

SrcElement

The element that fired the event

type

Type of event

returnValue

Determines whether the event is cancelled

cancelBubble

Can cancel an event bubble

\textbf{}

\textbf{The event.srcElement property}

The srcElement and type properties contain the data that effectively
encapsulates our object/event pair.

The srcElement property returns the element that fired the event. This
is an object, and has the same properties as the element. For example,
given an image object

\textless html\textgreater\textless body\textgreater{}

\textbf{\textless img id=\textquotesingle Image1\textquotesingle{}
src=\textquotesingle picture1.jpg\textquotesingle{}}

Game Programming -- Penn Wu

57

\protect\hypertarget{index_split_005.htmlux5cux23p58}{}{}
\textbf{onClick="p1.innerText=event.srcElement.id"}

\textbf{onDblClick="p1.innerText=event.srcElement.src"}

\textbf{onMouseOver="p1.innerText=event.srcElement.tagName"\textgreater{}}

\textless p id=p1\textgreater\textless/p\textgreater{}

\textless/body\textgreater\textless/html\textgreater{}

When clicking on this image with an \textbf{id} attribute of
\textquotesingle Image1\textquotesingle, and a \textbf{src} attribute of
\textquotesingle picture1.jpg\textquotesingle, then
\textbf{event.srcElement.id} will return
\textquotesingle Image1\textquotesingle, and when double clicking on it
the \textbf{event.srcElement.src} will return
\textquotesingle xxxxx/picture1.jpg\textquotesingle. Notice that the
\textbf{xxxxx/} part will be extended because the computer internally
converts relative URLs into absolute URLs.

Similarly, the srcElement has a tagName property:

event.srcElement.tagName

will return \textquotesingle IMG\textquotesingle. And we can also read
styles, so if the image has a style height of 100px, then
\textbf{event.srcElement.style.height} will return
\textquotesingle100px\textquotesingle.

\textbf{The event.type property}

The \textbf{type} property displays the event name. If the event handler
is the \textbf{onClick}, the type will be

\textquotesingle Click\textquotesingle, and if the event is onKeyPress,
then type will be \textquotesingle KeyPress\textquotesingle.

\textless body

\textbf{onClick}="status=\textquotesingle You used the
\textquotesingle+\textbf{event.type}+\textquotesingle{}

event handler to click on me!\textquotesingle"

\textbf{onKeyPress}="status=\textquotesingle You used the
\textquotesingle+\textbf{event.type}+\textquotesingle{}

event handler to click on me!\textquotesingle"\textgreater{}

\textless/body\textgreater{}

There are some event properties designed for the mouse buttons. A
detailed discussed is available in a later lecture.

clientX

Mouse pointer X coordinate relative to window

clientY

Mouse pointer Y coordinate relative to window

offsetX

Mouse pointer X coordinate relative to element that fired the event
offsetY

Mouse pointer Y coordinate relative to element that fired the event
button

Any mouse buttons that are pressed

For example, the following displays the current mouse position in the
browser\textquotesingle s status window.

\textless html\textgreater\textless body onmousemove="status =
\textquotesingle(\textquotesingle{} + event. \textbf{x} +
\textquotesingle, \textquotesingle{} +

event. \textbf{y} + \textquotesingle)\textquotesingle"\textgreater{}

\textless/body\textless/html\textgreater{}

There are also some event properties designed for the computer keyboard.
A detailed discussed is available in a later lecture.

altKey

True if the alt key was also pressed

ctrlKey

True if the ctrl key was also pressed

shiftKey

True if the shift key was also pressed

keyCode

Returns UniCode value of key pressed

Consider the following. It displays a message only when the user press
the Shift key.

\textless html\textgreater{}

\textless script\textgreater{}

function cc() \{

if (window.event. \textbf{shiftKey})

Game Programming -- Penn Wu

58

\protect\hypertarget{index_split_005.htmlux5cux23p59}{}{}
p1.innerText="Why did you press the Shift key?"

\}

\textless/script\textgreater{}

\textless body onKeyDown="cc()"\textgreater{}

\textless p id=p1\textgreater\textless/p\textgreater{}

\textless/body\textgreater\textless/html\textgreater{}

As we proceed through this class, you will see how you can successfully
employ the event objects to enhance your game programming.

Events and

\textbf{Events} are the user's activities, such as clicking a button,
pressing a key, moving the mouse event handlers

cursor around an object, and so on.

You can use the event handlers to detect a particular user activity, for
example, clicking a mouse button, and call a function to render some
results (such as an alert message) as response to the user's activity.

To allow you to run your bits of code when these events occur,
JavaScript provides us with event handlers. All the event handlers in
JavaScript start with the word on, and each event handler deals with a
certain type of event. Here\textquotesingle s a list of all the event
handlers in JavaScript, along with the objects they apply to and the
events that trigger them:

Table: Common Event Handlers

\textbf{Event handler}

\textbf{Applies to:}

\textbf{Triggered when:}

\textbf{onAbort}

Image

The loading of the image is

cancelled.

\textbf{onBlur}

Button, Checkbox, FileUpload, Layer,

The object in question loses

Password, Radio, Reset, Select, Submit,

focus (e.g. by clicking

Text, TextArea, Window

outside it or pressing the

TAB key).

\textbf{onCut}

Document

Fires on the source element

when the object or selection

is removed from the

document and added to the

system clipboard.

\textbf{onClick}

Button, Document, Checkbox, Link,

The object is clicked on.

Radio, Reset, Submit

\textbf{onDblClick}

Document, Link

The object is double-clicked

on.

\textbf{onDragDrop}

Window

An icon is dragged and

dropped into the browser.

\textbf{onError}

Image, Window

A JavaScript error occurs.

\textbf{onPaste}

Document

Fires on the target object

when the user pastes data,

transferring the data from

the system clipboard to the

document.

\textbf{onKeyDown}

Document, Image, Link, TextArea

The user presses a key.

\textbf{onKeyPress}

Document, Image, Link, TextArea

The user presses or holds

down a key.

\textbf{onKeyUp}

Document, Image, Link, TextArea

The user releases a key.

\textbf{onLoad}

Image, Window

The whole page has finished

loading.

\textbf{onMouseDown} Button, Document, Link

The user presses a mouse

button.

\textbf{onMouseMove} None

The user moves the mouse.

Game Programming -- Penn Wu

59

\protect\hypertarget{index_split_005.htmlux5cux23p60}{}{}\includegraphics{index-60_1.png}

\textbf{onMouseOut}

Image, Link

The user moves the mouse

away from the object.

\textbf{onMouseOver}

Image, Link

The user moves the mouse

over the object.

\textbf{onMouseUp}

Button, Document, Link

The user releases a mouse

button.

\textbf{onMove}

Window

The user moves the browser

window or frame.

\textbf{onResize}

Window

The user resizes the browser

window or frame.

\textbf{onUnload}

Window

The user leaves the page.

A complete list of event handler is available at

http://msdn.microsoft.com/workshop/author/dhtml/reference/events.asp?frame=true.

To use an event handler, you usually place the event handler name within
the HTML tag of the object you want to work with, followed by and
assignment sign "=" and the associated JavaScript codes. For example:

\textless button \textbf{onClick="alert(\textquotesingle Thank
You!\textquotesingle)"} \textgreater Click
Me\textless/button\textgreater{} Using events

In a previous lecture, you tried the following code, which uses the
event handler. You can now and event

replace the event handler with anyone in the above table to test how
they apply to your game handlers in

scenario.

game

programming

\textless html\textgreater{}

\textbf{}

\textless script\textgreater{}

var i=2;

function cc() \{

if (i\textgreater3) \{i=1;\}

m1.src="logo"+i+".gif";

i++;

\}

\textless/script\textgreater{}

\textless body \textbf{onKeyDown}="cc()"\textgreater{}

\textless img id=m1 src="logo1.gif"/\textgreater{}

\textless/body\textgreater{}

\textless/html\textgreater{}

An ideal sample output should look:

\textbf{}

\textbf{}

The truth is, as you probably already found out, not every event handler
can fit in this game. You must choose the event handler wisely.

Consider the following. It uses the \textbf{onMouseDown} event handler
to trigger the cc() function.

Inside the cc() function, the button property of the event object (as in
\textbf{event.button}) passes the mouse button value to the variable
\emph{i}. This value is then used to decide which .gif file to be
displayed: 1 for car\_left.gif, and 2 for car\_right.gif.

Game Programming -- Penn Wu

60

\protect\hypertarget{index_split_005.htmlux5cux23p61}{}{}\includegraphics{index-61_1.png}

\includegraphics{index-61_2.png}

\textless html\textgreater{}

\textless script\textgreater{}

function cc() \{

var i = event.button;

switch (i) \{

case 1:

p1.innerHTML = "\textless img
src=\textquotesingle car\_left.gif\textquotesingle\textgreater"; break;
case 2:

p1.innerHTML = "\textless img
src=\textquotesingle car\_right.gif\textquotesingle\textgreater"; break;

\}

\}

\textless/script\textgreater{}

\textless body onMouseDown=cc()\textgreater{}

\textless p id=p1\textgreater\textless/p\textgreater{}

\textless/body\textgreater\textless/html\textgreater{}

When clicking either the left or right mouse button, one of the .gif
file appears: or

The logic behind the scene is that the button property returns:
event.button value

Description

1

Left Mouse Button

2

Right Mouse Button

4

Middle Mouse Button

A detailed discussion about using mice as user input device is available
at later lecture.

Where can you

Many games require the computer to trigger user functions. The question
is when and where can use event

you use the event handlers? Basically, there are two ways:

handlers?

•

\textbf{Automatic}: When the page (document) is opened, or a panel (an
area) is opened, let the newly opened page/panel automatically triggers
functions.

•

\textbf{On-demand}: Create an object inside the page or panel, and embed
the demanded event handler onto that object.

In the following code, there are two functions: \textbf{init()} and
\textbf{spin()}. The statement,

\textless body onLoad=\textbf{init()}\textgreater{}

uses an \textbf{onLoad} event handler to trigger the \textbf{init()}
function whenever the page is loaded or refreshed, because the
\textless body\textgreater\ldots\textless/body\textgreater{} tags
construct the contents of the body area of browser.

The statement,

\textless button OnMouseDown=" \textbf{spin()}"

onMouseUp="clearTimeout(spinning)"\textgreater Spin\textless/button\textgreater{}
uses the \textbf{onMouseDown} event handler to trigger the spin()
function ONLY when the player holds a mouse button. In other words, it
is triggered on demand.

\textless html\textgreater{}

\textless script\textgreater{}

function \textbf{init()} \{

card1.src=Math.floor(Math.random()*55)+".gif";

Game Programming -- Penn Wu

61

\protect\hypertarget{index_split_005.htmlux5cux23p62}{}{}\includegraphics{index-62_1.png}

card2.src=Math.floor(Math.random()*55)+".gif";

\}

function \textbf{spin()} \{

card1.src=Math.floor(Math.random()*55)+".gif";

card2.src=Math.floor(Math.random()*55)+".gif";

spinning=setTimeout("spin()", 100);

\}

\textless/script\textgreater{}

\textless body onLoad=\textbf{init()}\textgreater{}

\textless table\textgreater\textless tr\textgreater{}

\textless td\textgreater\textless img
id=card1\textgreater\textless/td\textgreater{}

\textless td\textgreater\textless img
id=card2\textgreater\textless/td\textgreater{}

\textless/tr\textgreater\textless/table\textgreater{}

\textless button OnMouseDown=" \textbf{spin()}"

onMouseUp="clearTimeout(spinning)"\textgreater Spin\textless/button\textgreater{}

\textless/body\textgreater\textless/html\textgreater{}

This code has access to 55 .gif files with file names of 0.gif, 1.gif,
2.gif, \ldots., and 54.gif. To randomly pick a .gif and assign it to
\textbf{card1} (as in \textless img id=card1), this code use:
\textbf{card1}.src=Math.floor(Math.random()*55)+".gif"; Each time when
init() and spin() functions are trigger, a random number is generated
and used to make up a file name for card1, such as \emph{n}.gif, where
\emph{n} is the random number.

The last statement of the spin() function creates an object (named
\textbf{spinning}), which is used to represent the setTimeout method.
The spinning object triggers the spin() function every 100

milliseconds.

spinning=setTimeout("spin()", 100);

To end this loop, the following statement is used. The
\textbf{onMouseUp} event handler triggers the \textbf{clearTimeout}
method when the player release the mouse button.

\textless button OnMouseDown="spin()"

\textbf{onMouseUp="clearTimeout(spinning)}"\textgreater Spin\textless/button\textgreater{}
When being executed, click the Spin button and hold the mouse, the cards
spins till you release the mouse button.

Given the following HTML code which will load a GIF image file named
\textbf{pencil.gif} with an ID

\textbf{pen}.

\textless html\textgreater{}

\textless body\textgreater{}

Game Programming -- Penn Wu

62

\protect\hypertarget{index_split_005.htmlux5cux23p63}{}{}\includegraphics{index-63_1.png}

\includegraphics{index-63_2.png}

\textless img id="pen" src="pencil.gif"\textgreater{}

\textless/body\textgreater{}

\textless/html\textgreater{}

By adding the following CSS definition, this pencil.gif file will use
absolute positioning system on the browser's body area, so it can be
moved around with some JavaScript codes.

\textless style\textgreater{}

img \{position:absolute\}

\textless/style\textgreater{}

In order to move the pencil.gif file, we create a JavaScript function
named \textbf{MovePenn()} containing two events: \textbf{event.clientX}
and \textbf{event.clientY}.

•

The \textbf{clientX} property of event object sets or retrieves the
x-coordinate of the mouse pointer\textquotesingle s position relative to
the client area of the window, excluding window decorations and scroll
bars.

•

The \textbf{clientY} property sets or retrieves the y-coordinate of the
mouse pointer\textquotesingle s position relative to the client area of
the window, excluding window decorations and scroll bars.

\textless script\textgreater{}

function MovePen() \{

pen.style.left=event.clientX-30;

pen.style.top=event.clientY-25;

\}

\textless/script\textgreater{}

The absolute positioning system also uses the \emph{x}- and
\emph{y}-coordinates to place the pencil.gif one the browser's body
area. In the above code, we use \textbf{pen.style.left} to represent the
\emph{x}-coordinate, and \textbf{pen.style.top} to represent
\emph{y}-coordinate of the starting point of the pencil.gif file. The
starting point of images is always defaulted to start with the
upper-leftmost point.

(pen.style.left, pen.style.top)

54px

48px

The pencil.gif file has a dimension of 48px × 54px. You can detect the
dimension using any image software (such as Paint) or even the Internet
Explorer (by right clicking the image file and select Properties).

Dimension: 48 × 54

In order to move the pencil inside the pencil.gif file around the body
area of browser with the mouse cursor, we need to continuously assign
the current \emph{x}- and \emph{y}-coordinates of mouse cursor to
\textbf{pen.style.left} and \textbf{pen.style.top}. The MovePen()
function thus must contain the following code:
pen.style.left=event.clientX;

pen.style.top=event.clientY;

Game Programming -- Penn Wu

63

\protect\hypertarget{index_split_005.htmlux5cux23p64}{}{}\includegraphics{index-64_1.png}

\includegraphics{index-64_2.png}

However, we want the mouse cursor to point at the pencil inside the
pencil.gif file, it is necessary to adjust the values of event.clientX
and event.clienY.

(clientX, clientY)

(clientX-\textbf{30}, clientY-\textbf{25})

After a few trials, \textbf{(clientX-30, clientY-25)} seems to be a pair
of ideal values (Note: You can use other pairs, too), so the code is
modified to:

pen.style.left=event.clientX-30;

pen.style.top=event.clientY-25;

In JavaScript, as well as many other languages, a user-defined function
must be called or it will not execute. You can use the
\textbf{OnMouseMove} event handler to call the MovePen() function
whenever the user attempts to move the pencil.

\textless img id="pen" src="pencil.gif"
\textbf{OnMouseMove=MovePen()}\textgreater{} The complete code now
looks:

\textless html\textgreater{}

\textless style\textgreater{}

img \{position:absolute\}

\textless/style\textgreater{}

\textless script\textgreater{}

function MovePen() \{

pen.style.left=event.clientX-30;

pen.style.top=event.clientY-25;

\}

\textless/script\textgreater{}

\textless body\textgreater{}

\textless img id="pen" src="pencil.gif"
OnMouseMove=MovePen()\textgreater{}

\textless/body\textgreater{}

\textless/html\textgreater{}

When you test this code, you can move the pencil around the body area of
the browser by first putting the mouse cursor on the pencil.gif file and
then moving the cursor.

The event object and event handlers are very useful to the game
programming, and technically can be applied to any aspects of game
programming. For example, they can work with DTHML

text appending methods, innerHTML, to draw scribble lines.

Game Programming -- Penn Wu

64

\protect\hypertarget{index_split_005.htmlux5cux23p65}{}{}\includegraphics{index-65_1.png}

\textless html\textgreater{}

\textless style\textgreater{}

.dots \{position:absolute;font-size:78\}

\textless/style\textgreater{}

\textless script\textgreater{}

code = "";

function cc() \{

x = event.clientX;

y = event.clientY;

code += "\textless span class=dots style=\textquotesingle left: " + x
+"; top: " + y

+"\textquotesingle\textgreater.\textless/span\textgreater";

area1.innerText = "";

area1.innerHTML = code;

\}

\textless/script\textgreater{}

\textless body onMouseMove=cc() style="cursor:arrow"\textgreater{}

\textless div id=area1\textgreater\textless/div\textgreater{}

\textless/body\textgreater{}

\textless/html\textgreater{}

When being executed, you can move the mouse to draw free shape
(scribble) lines.

Drag and drop

Many games require the Drag-and-Drop function. There are several reasons
you might want to incorporate this Drag-and-Drop ability into your
games. One of the simplest reasons is to reorganize Data. As an example,
you might want to have a queue of items that your users can reorganize.
Instead of putting an input or select box next to each item to represent
its order, you could make the entire group of items draggable. Or
perhaps you want to have an object that can be moved around by your
users.

"Moving object" is the topic of a later lecture. For now, you just have
to know how the events and event handlers help to create the
Drag-and-Drop ability; particularly how you can use event handlers to
support this capacity so that the play can click on an item and drag it,
and finally move the item.

To make your DHTML games support Drag-and-Drop, the following code must
be used. It defines a ``dragme'' class.

var ie=document.all;

var nn6=document. \textbf{getElementById}\&\&!document.all; var
isdrag=false;

var x,y;

var dobj;

function movemouse(e)

Game Programming -- Penn Wu

65

\protect\hypertarget{index_split_005.htmlux5cux23p66}{}{}\{

if (isdrag)

\{

dobj.style.left = nn6 ? tx + e. \textbf{clientX} - x : tx +

\textbf{event.clientX} - x;

dobj.style.top = nn6 ? ty + e. \textbf{clientY} - y : ty +

\textbf{event.clientY} - y;

return false;

\}

\}

function selectmouse(e)

\{

var fobj = nn6 ? e.target : \textbf{event.srcElement};

var topelement = nn6 ? "HTML" : "BODY";

while (fobj. \textbf{tagName} != topelement \&\& fobj.className !=

"dragme")

\{

fobj = nn6 ? fobj.parentNode : fobj.parentElement;

\}

if (fobj.className=="dragme")

\{

isdrag = true;

dobj = fobj;

tx = parseInt(dobj.style.left+0);

ty = parseInt(dobj.style.top+0);

x = nn6 ? e.clientX : \textbf{event.clientX;}

y = nn6 ? e.clientY : \textbf{event.clientY};

document.onmousemove=movemouse;

return false;

\}

\}

document.onmousedown=selectmouse;

document.onmouseup=new Function("isdrag=false");

To understand this code, you need to have a strong background in
JavaScript and DHTML, but do not worry about it for now. It is displayed
here to give you an ideal how the event and event handler model
contribute to the gram programming.

As to you, simply treat this code as a library code. You do so by saving
it as an individual file with extension \textbf{.js}, such this file can
be used as a library file. To call this library file from another DHTML
file (e.g. lab4\_5.htm), simply add the following line between
\textless head\textgreater{} and \textless/head\textgreater{} of the
lab4\_4.htm file.

\textless script
src="dragndrop.js"\textgreater\textless/script\textgreater{}

Add the class="dragme" to whichever image you wish to make draggable.
For example,

\textless img id="cy1" \textbf{class=}" \textbf{dragme}"
src="cy1.jpg"\textgreater{} Review

1. Given the following code, which is the name of the user-defined
function?

Questions

\textless button id="click" onClick="Click()"\textgreater Click
Me\textless/button\textgreater{} A. click

B. onClick

C. Click

D. Click Me

Game Programming -- Penn Wu

66

\protect\hypertarget{index_split_005.htmlux5cux23p67}{}{}

2. Which is an example of the state of an event?

A. the state of the keyboard keys

B. the location of the mouse

C. the state of the mouse buttons

D. All of the above

3. Which event object\textquotesingle s property determines whether the
event is cancelled?

A. SrcElement

B. type

C. returnValue

D. cancelBubble

4. Given the following code block, what will be displayed on the status
bar?

\textless img id=\textquotesingle clown\textquotesingle{}
src=\textquotesingle clown.jpg\textquotesingle{}

onClick="status=event.srcElement.id"\textgreater{}

A. clown

B. clown.jpg

C. IMG

D. all of the above

5. Given the following code block, what will be displayed on the status
bar?

\textless body onClick="status=\textquotesingle You used the
\textquotesingle+event.type+\textquotesingle{}

event handler to click on me!\textquotesingle"\textgreater{}

A. Click

B. onClick

C. event.click

D. event.onClick

6. Which is an example of event?

A. clicking a button

B. pressing a key

C. moving the mouse cursor around an object

D. All of the above

7. Which event handler is triggered when the object or selection is
removed from the document and added to the system clipboard?

A. onClick

B. onCut

C. onPaste

D. onKeyPress

8. Which event handler is triggered when the user moves the browser
window or frame?

A. onWindowMove

B. onMouseMove

C. onDocumentMove

D. onMove

9. Which triggers the init() function whenever the page is loaded or
refreshed?

A. \textless body onStart=init()\textgreater{}

B. \textless body onLoad=init()\textgreater{}

C. \textless body onInit=init()\textgreater{}

D. \textless body onHold=init()\textgreater{}

Game Programming -- Penn Wu

67

\protect\hypertarget{index_split_005.htmlux5cux23p68}{}{}

10. How do you determinate the following code block?

objSpin=setTimeout("spin()", 100);

A. clearTimeout(objSpin);

B. clearTimeout("objSpin", 100);

C. clear("objSpin", 100);

D. clear(objSpin);

Game Programming -- Penn Wu

68

\protect\hypertarget{index_split_005.htmlux5cux23p69}{}{}\textbf{Appendix
A: More members exposed by the event object} MoreInfo

Retrieves the MoreInfo content of an entry banner in an ASX file through
the event object.

nextPage

Retrieves the position of the next page within a print template.

propertyName Sets or retrieves the name of the property that changes on
the object.

qualifier

Sets or retrieves the name of the data member provided by a data source
object.

reason

Sets or retrieves the result of the data transfer for a data source
object.

recordset

Sets or retrieves from a data source object a reference to the default
record set.

repeat

Retrieves whether the onkeydown event is being repeated.

returnValue

Sets or retrieves the return value from the event.

saveType

Retrieves the clipboard type when oncontentsave fires.

screenX

Retrieves the x-coordinate of the mouse pointer\textquotesingle s
position relative to the user\textquotesingle s screen.

screenY

Sets or retrieves the y-coordinate of the mouse
pointer\textquotesingle s position relative to the
user\textquotesingle s screen.

srcFilter

Sets or retrieves the filter object that caused the onfilterchange event
to fire.

srcUrn

Retrieves the Uniform Resource Name (URN) of the behavior that fired the
event.

toElement

Sets or retrieves a reference to the object toward which the user is
moving the mouse pointer.

type

Sets or retrieves the event name from the event object.

userName

Retrieves the sFriendlyName parameter that is passed to the useService
method.

wheelDelta

Retrieves the distance and direction the wheel button has rolled.

x

Sets or retrieves the x-coordinate, in pixels, of the mouse
pointer\textquotesingle s position relative to a relatively positioned
parent element.

y

Sets or retrieves the y-coordinate, in pixels, of the mouse
pointer\textquotesingle s position relative to a relatively positioned
parent element.

Game Programming -- Penn Wu

69

\protect\hypertarget{index_split_005.htmlux5cux23p70}{}{}\includegraphics{index-70_1.png}

Lab \#4

Event and Event Handler

\textbf{Preparation \#1:}

1. Create a new directory named \textbf{C:\textbackslash games}.

2. Use Internt Explorer to go to
\textbf{http://business.cypresscollege.edu/\textasciitilde pwu/cis261/download.htm}
to download lab4.zip (a zipped) file. Extract the files to
C:\textbackslash games directory.

\textbf{Learning Activity \#1: Drawing}

Note: This game is meant to be simple and easy for the sake of
demonstrating programming concepts. Please do not hesitate to enhance
the appearance or functions of this game.

\textbf{}

1. Change to the C:\textbackslash games directory.

2. Use Notepad to create a new file named
\textbf{C:\textbackslash games\textbackslash lab4\_1.htm} with the
following contents:

\textless html\textgreater{}

\textless style\textgreater{}

img \{position:absolute\}

\textless/style\textgreater{}

\textless script\textgreater{}

function MovePen() \{

pen.style.left=event.clientX-30;

pen.style.top=event.clientY-25;

\}

\textless/script\textgreater{}

\textless body\textgreater{}

\textless img id="pen" src="pencil.gif"
OnMouseMove=MovePen()\textgreater{}

\textless/body\textgreater{}

\textless/html\textgreater{}

3. Test the program. A sample output looks:

\textbf{Learning Activity \#2: Grimace Geckos}

Note: This game is meant to be simple and easy for the sake of
demonstrating programming concepts. Please do not hesitate to enhance
the appearance or functions of this game.

1. Change to the C:\textbackslash games directory.

2. Use Notepad to create a new file named
\textbf{C:\textbackslash games\textbackslash lab4\_2.htm} with the
following contents:

\textless html\textgreater{}

\textless style\textgreater{}

.aOut \{position:absolute\}

Game Programming -- Penn Wu

70

\protect\hypertarget{index_split_005.htmlux5cux23p71}{}{}\includegraphics{index-71_1.png}

\textless/style\textgreater{}

\textless script\textgreater{}

function popup() \{

var k = Math.floor(Math.random()*4);

area1.innerText="";

for (i=1; i\textless=k; i++) \{

var x = Math.floor(Math.random()*200);

var y = Math.floor(Math.random()*200);

codes = "\textless img class=aout id=geico" + i +"
src=\textquotesingle geico.gif\textquotesingle"; codes +=
"style=\textquotesingle left:" + x +"; top:" + y +"\textquotesingle"

codes += " onClick=bar()\textgreater";

area1.innerHTML += codes;

\}

setTimeout("popup()", 1000)

\}

function bar() \{

area2.insertAdjacentHTML(\textquotesingle BeforeBegin\textquotesingle,\textquotesingle\textless b
style=color:red\textgreater\textbar\textless/b\textgreater\textquotesingle);

\}

\textless/script\textgreater{}

\textless body onLoad=popup()\textgreater{}

\textless div id=area1
style="width:300;height:300"\textgreater\textless/div\textgreater{}

\textless div id=area2\textgreater\textless/div\textgreater{}

\textless/body\textgreater{}

\textless/html\textgreater{}

3. Test the program. When some geckos randomly appear, use the mouse
click to hit them. A red bar grows each time when you hit a geico. A
sample output looks:

\textbf{Learning Activity \#3: An incomplete slot machine}

Note: This game is meant to be simple and easy for the sake of
demonstrating programming concepts. Please do not hesitate to enhance
the appearance or functions of this game.

1. Change to the C:\textbackslash games directory.

2. Use Notepad to create a new file named
\textbf{C:\textbackslash games\textbackslash lab4\_3.htm} with the
following contents:

\textless html\textgreater{}

\textless script\textgreater{}

function init() \{

for (i=1; i\textless=4; i++) \{

var k = Math.floor(Math.random()*55);

area1.innerHTML += "\textless img id=g"+i+" src="+k+".gif\textgreater"

Game Programming -- Penn Wu

71

\protect\hypertarget{index_split_005.htmlux5cux23p72}{}{}\includegraphics{index-72_1.png}

\}

\}

function spin() \{

area1.innerText="";

for (i=1; i\textless=4; i++) \{

var k = Math.floor(Math.random()*55);

area1.innerHTML += "\textless img id=g"+i+" src="+k+".gif\textgreater"

\}

spinning=setTimeout("spin()",100);

\}

\textless/script\textgreater{}

\textless body onLoad=init()\textgreater{}

\textless div id=area1\textgreater\textless/div\textgreater{}

\textless button onMouseDown=spin()

onMouseUp=clearTimeout(spinning)\textgreater Spin\textless/button\textgreater{}

\textless/body\textgreater{}

\textless/html\textgreater{}

3. Test the program. Click the Spin button and hold the mouse, the cards
spins till you release the mouse button. A sample output looks:

\textbf{Learning Activity \#4: A simple Black Jack game}

Note: This game is meant to be simple and easy for the sake of
demonstrating programming concepts. Please do not hesitate to enhance
the appearance or functions of this game.

1. Change to the C:\textbackslash games directory.

2. Use Notepad to create a new file named
\textbf{C:\textbackslash games\textbackslash lab4\_4.htm} with the
following contents:

\textless html\textgreater{}

\textless script\textgreater{}

function serve() \{

dealer1.src=Math.floor(Math.random()*52)+1+".gif";

dealer2.src=Math.floor(Math.random()*52)+1+".gif";

player2.src=Math.floor(Math.random()*52)+1+".gif";

\}

function add\_card() \{

p1.innerHTML+="\textless img id=player3 src=" +
(Math.floor(Math.random()*52)+1)+ ".gif\textgreater";

\}

function again() \{

p1.innerText="";

player1.src="back.gif";

serve();

\}

\textless/script\textgreater{}

Game Programming -- Penn Wu

72

\protect\hypertarget{index_split_005.htmlux5cux23p73}{}{}\includegraphics{index-73_1.png}

\includegraphics{index-73_2.png}

\includegraphics{index-73_3.png}

\textless body onLoad=serve()\textgreater{}

\textless center\textgreater\textless table width="300px"\textgreater{}

\textless tr height="80px"\textgreater{}

\textless tb\textgreater{}

\textless img id=dealer1\textgreater{}

\textless img id=dealer2\textgreater{}

\textless/tb\textgreater{}

\textless tr
height="40px"\textgreater\textless td\textgreater\textless/td\textgreater\textless/tr\textgreater{}

\textless/tr\textgreater{}

\textless tr height="80px"\textgreater{}

\textless td\textgreater{}

\textless img id=player1 src="back.gif"\textgreater{}

\textless img id=player2\textgreater{}

\textless span id=p1\textgreater\textless/span\textgreater{}

\textless/tb\textgreater{}

\textless/tr\textgreater{}

\textless tr height="40px"\textgreater{}

\textless td\textgreater{}

\textless button onClick=add\_card()\textgreater Add
Card\textless/button\textgreater{}

\textless button
onClick="player1.src=Math.floor(Math.random()*52)+1+\textquotesingle.gif\textquotesingle"\textgreater Reveal\textless/button\textgreater{}

\textless button onClick=again()\textgreater Play
Again\textless/button\textgreater{}

\textless/td\textgreater{}

\textless/tr\textgreater{}

\textless/table\textgreater\textless/center\textgreater{}

\textless/body\textgreater{}

\textless/html\textgreater{}

3. Test the program. When starting the game, you are given two cards.
One card's value is not disclosed by default. Please Add Card if you
want to add more card(s). Please Reveal to disclose the last card. A
sample output looks:

Start

Add a card

Flip the card

\textbf{Learning Activity \#5: Puzzle}

Note: This game is meant to be simple and easy for the sake of
demonstrating programming concepts. Please do not hesitate to enhance
the appearance or functions of this game.

\textbf{}

1. Change to the C:\textbackslash games directory.

2. Use Notepad to create a new file named
\textbf{C:\textbackslash games\textbackslash lab4\_5.htm} with the
following contents:

\textless html\textgreater{}

\textless head\textgreater{}

\textless style\textgreater{}

.dragme\{position:relative;\}

Game Programming -- Penn Wu

73

\protect\hypertarget{index_split_005.htmlux5cux23p74}{}{}td
\{border:solid 1 red; text-align:center; font-size:12px\}

\textless/style\textgreater{}

\textless script
src="dragndrop.js"\textgreater\textless/script\textgreater{}

\textless script\textgreater{}

function cc() \{

for (i=1;i\textless=12; i++) \{

var x = Math.floor(Math.random()*50);

var y = Math.floor(Math.random()*100);

pic.innerHTML += "\textless img
class=\textquotesingle dragme\textquotesingle{} src=monalisa"+ i + ".gif
style=\textquotesingle left:" + x +

"; top:" + y + "\textquotesingle{}

ondragstart=\textquotesingle this.style.zIndex=3;this.style.left=event.clientX;this.style.top=event

.clientY\textquotesingle\textgreater";

\}

\}

function dd() \{

tb.style.display=\textquotesingle none\textquotesingle;

pic.innerText=\textquotesingle\textquotesingle;

for (k=1;k\textless=12; k++) \{

if (k\%3 == 0) \{

pic.innerHTML+="\textless img src=monalisa" + k +".gif
style=\textquotesingle border: solid 1
white\textquotesingle\textgreater\textless br\textgreater";

\}

else \{

pic.innerHTML+="\textless img src=monalisa" + k +".gif
style=\textquotesingle border: solid 1
white\textquotesingle\textgreater";

\}

\}

\}

\textless/script\textgreater{}

\textless/head\textgreater{}

\textless body onLoad=cc()\textgreater{}

\textless button
onClick=dd()\textgreater Solve\textless/button\textgreater{}

\textless table border=0 cellspacing=0 id=tb
style="position:relative;"\textgreater{}

\textless tr height="52px"\textgreater{}

\textless td width="60px"\textgreater1\textless/td\textgreater{}

\textless td width="64px"\textgreater2\textless/td\textgreater{}

\textless td width="60px"\textgreater3\textless/td\textgreater{}

\textless/td\textgreater{}

\textless/tr\textgreater{}

\textless tr height="94px"\textgreater{}

\textless td\textgreater4\textless/td\textgreater{}

\textless td\textgreater5\textless/td\textgreater{}

\textless td\textgreater6\textless/td\textgreater{}

\textless/tr\textgreater{}

\textless tr height="82px"\textgreater{}

\textless td\textgreater7\textless/td\textgreater{}

\textless td\textgreater8\textless/td\textgreater{}

\textless td\textgreater9\textless/td\textgreater{}

\textless/tr\textgreater{}

Game Programming -- Penn Wu

74

\protect\hypertarget{index_split_005.htmlux5cux23p75}{}{}\includegraphics{index-75_1.png}

\textless tr height="62px"\textgreater{}

\textless td\textgreater10\textless/td\textgreater{}

\textless td\textgreater11\textless/td\textgreater{}

\textless td\textgreater12\textless/td\textgreater{}

\textless/tr\textgreater{}

\textless/table\textgreater{}

\textless div id="pic"
style="position:relative;"\textgreater\textless/div\textgreater{}

\textless/body\textgreater{}

\textless/html\textgreater{}

3. Test the program. Move each piece to its expected cell. A sample
output looks: \textbf{Submittal}

Upon completing all the learning activities,

1. Upload all files you created in this lab to your remote web server.

2. Log in to to Blackboard, launch Assignment 04, and then scroll down
to question 11.

3. Copy and paste the URLs to the textbox. For example,

•

http://www.geocities.com/cis261/lab4\_1.htm

•

http://www.geocities.com/cis261/lab4\_2.htm

•

http://www.geocities.com/cis261/lab4\_3.htm

•

http://www.geocities.com/cis261/lab4\_4.htm

•

http://www.geocities.com/cis261/lab4\_5.htm

No credit is given to broken link(s).

Game Programming -- Penn Wu

75

\protect\hypertarget{index_split_005.htmlux5cux23p76}{}{}\includegraphics{index-76_1.png}

\includegraphics{index-76_2.png}

\includegraphics{index-76_3.png}

Lecture \#5

Using mouse buttons for input control

\protect\hypertarget{index_split_006.html}{}{}

\hypertarget{index_split_006.htmlux5cux23calibre_pb_5}{%
\subsection{Introduction}\label{index_split_006.htmlux5cux23calibre_pb_5}}

The players interact with a game by giving user inputs. To most games,
user inputs encompass the entire communications between a player and a
game. There are many user input devices that are used in game
programming, such as keyboard, mice, joysticks, flight sticks, touchpad,
pointing devices, and many other user input devices have brought
extended input capabilities to the game player; however, none is as
popular as the mouse.

This lecture discusses the basic concepts of user input handling, but it
will focus on the mouse. A later lecture will discuss the keyboard in
details.

Common User

Input devices are the physical hardware that allows a user to interact
with a game. Input devices Input Devices

all perform the same function: converting information provided by the
user into a form understandable by the computer. Input devices form the
link between the user and your game.

Even though you can't directly control the input device hardware, you
can certainly control how it is interpreted in your game. Currently,
there are three primary types of user input devices:

•

The keyboard.

•

The mouse.

•

Joysticks.

Why the

Operating system such as Windows, Linux, and Macintosh have all adopt
the mouse as a standard mouse?

input device, so it is a nature for the mouse to be the most popular
user input device for game programming.

The mouse, however, does not share the wide range of input applications
to games that the keyboard has. A mouse usually has 2 or 3 buttons, or 2
buttons with one wheel, while a computer keyboard has at least 101 keys.
Additionally, the mouse was primarily designed as a point-and-click
device. The truth is many games do not follow the point-and-click
paradigm.

Surprisingly the mouse has the mobility to quickly move around a given
area, so it does have a recognizable usefulness, which is dependent
totally on the type of game and the type of user interaction dictated by
the game. It is very important to learn how to handle user inputs with
the mouse.

Mouse

When you move the mouse, a series of events is set off, and most
computer languages provide a Anatomy

series of mouse messages that are used to convey mouse events to
recognizable actions. The previous lecture discussed how the events and
event handlers work closely together to detect and respond to
user\textquotesingle s activities. You should now try to understand how
a mouse works so you can have a better manipulation of them in your
games.

2-button mouse

2 buttons with one wheel

3-button mouse

A mouse is usually equipped with 1-3 buttons. Each button is an
individual interface that sends signals to the computer. A mouse is
commonly used to make selections and position the cursor, but, in a
graphical user interfaces environment, the mouse is also used to trace
and record x- and y-coordinates of any given objects.

Mouse Event

JavaScript includes a number of event handlers for detecting mouse
actions. Your game script can Game Programming -- Penn Wu

76

\protect\hypertarget{index_split_006.htmlux5cux23p77}{}{}\includegraphics{index-77_1.png}

Handlers

use them detect the movement of the mouse pointer and when a button is
clicked, released, or both. They are:

•

\textbf{onMouseOver}: The mouse is moved over an element.

•

\textbf{onMouseOut}: The mouse is moved off an element.

•

\textbf{onMouseMove}: The mouse is moved.

•

\textbf{onMouseDown}: A mouse button is pressed.

•

\textbf{onMouseUp}: A mouse button is released.

•

\textbf{onClick}: When a mouse button is clicked.

•

\textbf{onDblClick:} When a mouse button is double clicked.

\textbf{The onMouseOver event hander}

The \textbf{onMouseOver} handler is called when the mouse pointer moves
over a link or other object. In the following example, the onMouseOver
handler tells the computer to change the image file's source from
\textbf{gopher.gif} to \textbf{gopher2.gif} when the play mouse over the
image file. The keyword

``this'' is used in the
\textbf{this.src=\textquotesingle gopher2.gif\textquotesingle{}}
statement simply because the onMouseOver handler is used inline with the
\textless img\textgreater{} tag.

\textless img src="gopher.gif"
\textbf{onMouseOver="this.src=\textquotesingle gopher2.gif\textquotesingle"}
\textgreater{} The \textbf{onMouseOut} handler is the opposite---it is
called when the mouse pointer moves out of the object\textquotesingle s
border. Unless something strange happens, this always happens some time
after the onMouseOver event is called.

\textless img src="gopher.gif"
onMouseOver="this.src=\textquotesingle gopher2.gif\textquotesingle"

\textbf{onMouseOut="this.src=\textquotesingle gopher.gif\textquotesingle"}
\textgreater{}

In the above example, the onMouseOut handler changes the file source
from gopher2.gif back to gopher.gif. Guess what? This is a very commonly
used trick used to create a visual effect of an object.

\textbf{TheonMouseMove event hander}

The \textbf{onMouseMove} event occurs any time the mouse pointer moves.
As you might imagine, this happens quite often---the event can trigger
hundreds of times as the mouse pointer moves across a page. For example,

\textless body onMouseMove="status

=\textquotesingle(\textquotesingle+\textbf{event.clientX}+\textquotesingle{}
,\textquotesingle+\textbf{event.clientY}+\textquotesingle)\textquotesingle";\textgreater\textless/body\textgreater{}
When moving the mouse cursor around, a pair of \emph{x}- and
\emph{y}-coordinate appears on the status bar in the form of ( \emph{x},
\emph{y}). Noticeably, the value of \emph{x} and \emph{y} will
continuously change as response to the mouse's movement.

\textbf{The onMouseDown and onMouseUp event hander}

To give you even more control of what happens when the mouse button is
pressed, two more events are included:

•

\textbf{onMouseDown} is used when the user presses the mouse button.

•

\textbf{onMouseUp} is used when the user releases the mouse button.

Refer to the keyboard code of a previous lecture. You can apply
\textbf{onMouseDown} and \textbf{onMouseUp} methods to it. For example,

\textless span class="wKey" id="midC" style="left:50"
\textbf{onMouseDown}="CDown()"

\textbf{onMouseUp}="CUp()"\textgreater\textless/span\textgreater{}

\textless span class="wKey" id="midD" style="left:91"
\textbf{onMouseDown}="DDown()"

Game Programming -- Penn Wu

77

\protect\hypertarget{index_split_006.htmlux5cux23p78}{}{}\textbf{onMouseUp}="DUp()"\textgreater\textless/span\textgreater{}

\textless span class="wKey" id="midE" style="left:132"

\textbf{onMouseDown}="EDown()"
\textbf{onMouseUp}="EUp()"\textgreater\textless/span\textgreater{}
\textbf{The onClick and onDblClick event hander}

The onMouseDown and onMouseUp event handlers are the two halves of a
mouse click. You can also use events to detect when the mouse button is
clicked. The basic event handler for this is \textbf{onClick}. This
event handler is called when the mouse button is clicked while
positioned over the appropriate object.

The \textbf{onDblClick} event handler is similar to onClick, but is only
used if the user double-clicks on an object. In the following example,
onClick triggers the cc() function, while onDblClick triggers dd()
function.

\textless button onClick="cc()"
onDblClick="dd()"\textgreater Rotate\textless/button\textgreater{} If
you want to detect an entire click, use onClick. Use onMouseUp and
onMouseDown to detect just one or the other.

The event

Internet Explorer supports JavaScript's event object, and it provides
the following common object's

properties for record information of mouse movement. The data they
record can be retrieved as properties for

reference to your game if you know how to use them:

mouse

•

\textbf{event.button}: The mouse button that was pressed. This value is
1 for the left button and usually 2 for the right button.

•

\textbf{event.clientX}: The x-coordinate (column, in pixels) where the
event occurred.

•

\textbf{event.clientY}: The y-coordinate (row, in pixels) where the
event occurred.

The \textbf{event.clientX} property returns the horizontal coordinate
within the application\textquotesingle s client area at which the event
occurred (as opposed to the coordinates within the page). For example,
clicking in the top-left corner of the client area will always result in
a mouse event with a clientX value of 0, regardless of whether the page
is scrolled horizontally.

Similarly, \textbf{event.clientY} returns the horizontal coordinate
within the application\textquotesingle s client area at which the event
occurred (as opposed to the coordinates within the page). For example,
clicking in the top-left corner of the client area will always result in
a mouse event with a clientX value of 0, regardless of whether the page
is scrolled horizontally.

Syntax is:

event.clientX;

and

event.clientY

For example,

\textless html\textgreater{}

\textless body onMouseMove="p1.innerText

=\textquotesingle(\textquotesingle+\textbf{event.clientX}+\textquotesingle{}
,\textquotesingle+\textbf{event.clientY}+\textquotesingle)\textquotesingle";\textgreater{}

\textless p id=p1\textgreater\textless/p\textgreater{}

\textless/body\textgreater{}

\textless/html\textgreater{}

When executing the code, you will see a ( \emph{x}, \emph{y}) coordinate
set in the body area.

Game Programming -- Penn Wu

78

\protect\hypertarget{index_split_006.htmlux5cux23p79}{}{}\includegraphics{index-79_1.png}

\includegraphics{index-79_2.png}

\includegraphics{index-79_3.png}

Consider the following code. It creates a red area starting at the point
(100px, 100px) with a width of 100px and height of 50px.

\textless html\textgreater{}

\textless style\textgreater{}

.ar \{position:absolute; left:100px; top:100px;

width:100px; height:50px;

background-Color:red; border:solid 1 red\}

\textless/style\textgreater{}

\textless body onMouseMove=cc()\textgreater{}

\textless div class=ar id=area1\textgreater\textless/div\textgreater{}

\textless/body\textgreater{}

\textless/html\textgreater{}

By adding the following bold lines, which means ``when the mouse
cursor's \emph{y}-coordinate between 100px and 150px change the
background color to white; otherwise, set the background color to red''.

\textless html\textgreater{}

\textless style\textgreater{}

.ar \{position:absolute; left:100px; top:100px;

width:100px; height:50px;

background-Color:red; border:solid 1 red\}

\textless/style\textgreater{}

\textbf{\textless script\textgreater{}}

\textbf{function cc() \{}

\textbf{if ((event.clientY \textgreater= 100) \&\& (event.clientY
\textless= 150)) \{}

\textbf{area1.style.backgroundColor="white";}

\textbf{\}}

\textbf{else \{}

\textbf{area1.style.backgroundColor="red";}

\textbf{\}}

\textbf{\}}

\textbf{\textless/script\textgreater{}}

\textbf{}

\textless body onMouseMove=cc()\textgreater{}

\textless div class=ar id=area1\textgreater\textless/div\textgreater{}

\textless/body\textgreater{}

\textless/html\textgreater{}

Notice that the \textbf{onMouseMove} handler is placed inside the
\textless body\textgreater{} tag, which means ``when the mouse is moved
around the body area''.

to

Game Programming -- Penn Wu

79

\protect\hypertarget{index_split_006.htmlux5cux23p80}{}{}\includegraphics{index-80_1.png}

\includegraphics{index-80_2.png}

The \textbf{event.button} property is used to determine which mouse
button has been clicked. To safely detect a mouse button you have to use
the \textbf{onMouseDown} or \textbf{onMouseUp} events.

In the previous lecture, there was a explanation about how you could use
the \textbf{onMouseDown} event handler to trigger the cc() function in
the following example. This code is also a good example for the
\textbf{event.button} property.

Again, the event.button property is designed to set or retrieve the
mouse button pressed by the user. Possible values of the
\textbf{event.button} property for Internet Explorer are: 0

Default. No button is pressed.

1

Left button is pressed.

2

Right button is pressed.

3

Left and right buttons are both pressed.

4

Middle button is pressed.

5

Left and middle buttons both are pressed.

6

Right and middle buttons are both pressed.

7

All three buttons are pressed.

Inside the cc() function, the button property of the event object (as in
\textbf{event.button}) passes the mouse button value to the variable
\emph{i}. This value is then used to decide which .gif file to be
displayed: 1 for car\_left.gif, and 2 for car\_right.gif.

\textless html\textgreater{}

\textless script\textgreater{}

function cc() \{

var i = \textbf{event.button};

switch (i) \{

case 1:

p1.innerHTML = "\textless img
src=\textquotesingle car\_left.gif\textquotesingle\textgreater"; break;
case 2:

p1.innerHTML = "\textless img
src=\textquotesingle car\_right.gif\textquotesingle\textgreater"; break;

\}

\}

\textless/script\textgreater{}

\textless body onMouseDown=cc() \textbf{onContextMenu="return false"}
\textgreater{}

\textless p id=p1\textgreater\textless/p\textgreater{}

\textless/body\textgreater\textless/html\textgreater{}

When clicking either the left or right mouse button, one of the .gif
file appears: or

In this case, the event.button returns a numerical value to the
computer. The computer in turn gives it to the variable \emph{i} for
evaluation. Notice that this property is read/write. The property has a
default value of 0. This property is used with the \textbf{onmousedown},
\textbf{onmouseup}, and \textbf{onmousemove} events. For other events,
it defaults to 0 regardless of the state of the mouse buttons.

In addition to \textbf{clientX}, \textbf{clientY}, and \textbf{button}
properties, Internet Explorer also works with the following properties
of the event object.

event.offsetX sets or retrieves the x-coordinate of the mouse
pointer\textquotesingle s position relative to the object firing the
event.

Game Programming -- Penn Wu

80

\protect\hypertarget{index_split_006.htmlux5cux23p81}{}{}\includegraphics{index-81_1.png}

event.offsetY Sets or retrieves the y-coordinate of the mouse
pointer\textquotesingle s position relative to the object firing the
event.

event.screenX sets and retrieves the x-coordinate of the mouse
pointer\textquotesingle s position relative to the
user\textquotesingle s screen.

event.screenY sets or retrieves the y-coordinate of the mouse
pointer\textquotesingle s position relative to the
user\textquotesingle s screen.

event.x

sets or retrieves the x-coordinate, in pixels, of the mouse
pointer\textquotesingle s position relative to a relatively positioned
parent element. Technically, this property is similar to clientX.

event.y

sets or retrieves the y-coordinate, in pixels, of the mouse
pointer\textquotesingle s position relative to a relatively positioned
parent element. Technically, this property is similar to clientY.

The following figure illustrates the starting point of these properties.

screenX, screenY

clientX, clientY

offsetX, offsetY

The following example demonstrates how these properties are different
from each other. Be sure to run this code to see for yourself.

\textless html\textgreater{}

\textless script\textgreater{}

function cc() \{

p1.innerText = "(offsetX, offsetY)= (" + event.offsetX + " ," +

event.offsetY +")";

p2.innerText = "(clientX, clientY)= (" + event.clientX + " ," +

event.clientY +")";

p3.innerText = "(screenX, screenY)= (" + event.screenX + " ," +

event.screenY +")";

p4.innerText="(x, y)= (" + event.x + " ," + event.y +")";

\}

\textless/script\textgreater{}

\textless span
id=p1\textgreater\textless/span\textgreater\textless br\textgreater{}

\textless span
id=p2\textgreater\textless/span\textgreater\textless br\textgreater{}

\textless span
id=p3\textgreater\textless/span\textgreater\textless br\textgreater{}

\textless span
id=p4\textgreater\textless/span\textgreater\textless br\textgreater{}

\textless body onMouseMove=cc()\textgreater{}

\textless div style="position:absolute; width:300;height:300;

\textbf{left:100;top:100};background-Color:red"\textgreater\textless/div\textgreater{}

\textless/body\textgreater{}

Game Programming -- Penn Wu

81

\protect\hypertarget{index_split_006.htmlux5cux23p82}{}{}\includegraphics{index-82_1.png}

\includegraphics{index-82_2.png}

\textless/html\textgreater{}

When executing this code, move the cursor around the red area and
compare the x- and y-coordinates of them.

(offsetX, offsetY)=(100 ,102)

(clientX, clientY)=(202 ,204)

(screenX, screenY)=(549 ,433)

(x, y)=(202 ,204)

The difference between (offsetX, offsetY) and (clientX, clientY) was
caused by the following setting of the above code, which sets the
starting point of the red area as (100, 100), while the starting point
of the browser's body area remains being (0, 0):
\textbf{left:100;top:100};

Disable right

One annoying feature of the GUI (graphical user interfaces) is its
support of mouse right click, click

especially when the right click is used to trigger some menu. For
example, when you right click on a blank space of the browser's body
area, the following menu pops up.

This menu may disgrace your game programs, so many programmers choose to
disable it. To do so, simply add the bold section to the
\textless body\textgreater{} tag.

\textless body \textbf{onContextMenu="return false"} \textgreater{}

Test the code in Internet Explorer 5.0 or later. When you attempt to
right-click anywhere on the page, the shortcut menu is not displayed.

Another solution is to redefine what the right click should do. For
example, in the following code, the right click is now used to trigger
the \textbf{new\_right\_click()} function.

\textless html\textgreater{}

\textless script language="JavaScript"\textgreater{}

function new\_right\_click() \{

g1.style.display="inline";

\}

function click(e) \{

if (document.all) \{

if (event.button == 2) \{

Game Programming -- Penn Wu

82

\protect\hypertarget{index_split_006.htmlux5cux23p83}{}{}\includegraphics{index-83_1.png}

\includegraphics{index-83_2.png}

\includegraphics{index-83_3.png}

\includegraphics{index-83_4.png}

\includegraphics{index-83_5.png}

\includegraphics{index-83_6.png}

\includegraphics{index-83_7.png}

\includegraphics{index-83_8.png}

\includegraphics{index-83_9.png}

\includegraphics{index-83_10.png}

\includegraphics{index-83_11.png}

new\_right\_click();

return false;

\}

\}

if (document.layers) \{

if (e.which == 3) \{

new\_right\_click();

return false;

\}

\}

\}

if (document.layers) \{

document.captureEvents(Event.MOUSEDOWN);

\}

document.onmousedown=click;

\textless/script\textgreater{}

\textless body\textgreater{}

\textless img id=g1 src=cat.gif style="display:none"\textgreater{}

\textless/body\textgreater\textless/html\textgreater{}

Execute the code, and right click the mouse. The cat appears, but the
above menu won't.

The cursor

With CSS, you can change the value of the cursor declaration to use
different cursor styles to shape

enhance the appearance of your game. The syntax is:

element \{ cursor: Value \}

where \emph{element} is any HTML tag or user-declared style name,
\emph{cursor} is a keyword, and \emph{value} is the anyone from the
following list.

\textbf{Look}

\textbf{Values}

\textbf{Example}

default

cursor:default

crosshair

cursor:crosshair

hand

cursor:hand

pointer

cursor:pointer

Cross browser

cursor:pointer;cursor:hand

move

cursor:move

text

cursor:text

wait

cursor:wait

help

cursor:help

n-resize

cursor:n-resize

Game Programming -- Penn Wu

83

\protect\hypertarget{index_split_006.htmlux5cux23p84}{}{}\includegraphics{index-84_1.png}

\includegraphics{index-84_2.png}

\includegraphics{index-84_3.png}

\includegraphics{index-84_4.png}

\includegraphics{index-84_5.png}

\includegraphics{index-84_6.png}

\includegraphics{index-84_7.png}

\includegraphics{index-84_8.png}

\includegraphics{index-84_9.png}

\includegraphics{index-84_10.png}

\includegraphics{index-84_11.png}

\includegraphics{index-84_12.png}

\includegraphics{index-84_13.png}

\includegraphics{index-84_14.png}

ne-resize

cursor:ne-resize

e-resize

cursor:e-resize

se-resize

cursor:se-resize

s-resize

cursor:s-resize

sw-resize

cursor:sw-resize

w-resize

cursor:w-resize

nw-resize

cursor:nw-resize

progress

cursor:progress

not-allowed

cursor:not-allowed

no-drop

cursor:no-drop

vertical-text

cursor:vertical-text

all-scroll

cursor:all-scroll

col-resize

cursor:col-resize

row-resize

cursor:row-resize

For example, when you mouse the cursor on top of the image file, the
cursor turns into a cross.

\textless img src="gopher.gif" style=" \textbf{cursor:
crosshair}"\textgreater{} You can even use your own custom images as
cursors. The syntax is: element \{ cursor : url("FileName"), value \}

Notice that the \textbf{.cur} extension refers to the file extension for
the Windows Cursor file. This is one format that cursors (the mouse
pointer design) can be stored in, under Microsoft Windows 3.x and later.
In Windows 95, NT and later, this format has been largely superseded by
the ANI format which allows animated and color cursor designs.

The CUR format also allows transparency so that cursors do not have to
appear rectangular when displayed on screen. For example,

body \{ cursor : url("custom.cur"), pointer \}

Note: The custom images as cursors is only supported in Internet
Explorer 6.0 or later, which is why we also included pointer, so if the
browser dose not support custom cursors at least the default pointer
will be displayed.

In the following game, the mouse cursor is changed to a graphic of pen,
so it creates a visual effect of writing on the
browser\textquotesingle s body area using a pen.

Game Programming -- Penn Wu

84

\protect\hypertarget{index_split_006.htmlux5cux23p85}{}{}\includegraphics{index-85_1.png}

\textless html\textgreater{}

\textless style\textgreater{}

.dots \{position:absolute;font-size:30\}

\textbf{body \{}

\textbf{cursor : url("pencil.cur"), pointer}

\textbf{\}}

\textless/style\textgreater{}

\textless script\textgreater{}

function cc() \{

x = event.clientX;

y = event.clientY;

codes = "\textless span class=dots style=\textquotesingle left: " + x;

codes += "; top: " + y
+"\textquotesingle\textgreater.\textless/span\textgreater";
area1.innerHTML += codes;

\}

\textless/script\textgreater{}

\textless body onMouseMove=cc()\textgreater{}

\textless div id=area1\textgreater\textless/div\textgreater{}

\textless/body\textgreater{}

\textless/html\textgreater{}

A sample output looks:

The cursor changes to

a pencil

Review

1. Which is the event handler that functions only when the mouse is
moved off an element?

Questions

A. onMouseOver

B. onMouseOut

C. onMouseMove

D. onMouseDown

2. The onMouseDown and onMouseUp together makes the function of \_\_.

A. onMouseClick

B. onClick

C. onKeyClick

D. onButtonClick

3. Given the following code block, which statement is correct?

\textless img src="gopher.gif"
onMouseOver="this.src=\textquotesingle gopher2.gif\textquotesingle"\textgreater{}
A. The onMouseOver handler change the image file's source from
gopher2.gif to gopher.gif.

B. The onMouseOver handler change the image file's source from
gopher.gif to gopher2.gif.

Game Programming -- Penn Wu

85

\protect\hypertarget{index_split_006.htmlux5cux23p86}{}{}C. The word
"this" in this.src=\textquotesingle gopher2.gif\textquotesingle{} refers
to the gopher2.gif image file D. onMouseOver must be replaced with
onMouseMove to make this code work.

4. Which code block is equivalent to the following one?

\textless body onMouseMove="status

=\textquotesingle(\textquotesingle+event.clientX+\textquotesingle{}
,\textquotesingle+event.clientY+\textquotesingle)\textquotesingle";\textgreater\textless/body\textgreater{}
A. \textless body
onMouseMove="status=\textquotesingle(\textquotesingle+event.X+\textquotesingle{}
,\textquotesingle+event.Y+\textquotesingle)\textquotesingle";\textgreater\textless/body\textgreater{}
B. \textless body
onMouseMove="status=\textquotesingle(\textquotesingle+client.X+\textquotesingle{}
,\textquotesingle+client.Y+\textquotesingle)\textquotesingle";\textgreater\textless/body\textgreater{}
C. \textless body
onMouseMove="status=\textquotesingle(\textquotesingle+X+\textquotesingle{}
,\textquotesingle+Y+\textquotesingle)\textquotesingle";\textgreater\textless/body\textgreater{}
D. \textless body
onMouseMove="status=\textquotesingle(\textquotesingle+clientX+\textquotesingle{}
,\textquotesingle+clientY+\textquotesingle)\textquotesingle";\textgreater\textless/body\textgreater{}
5. Which statement about event.button is correct?

A. This value is 0 for the left button and usually 1 for the right
button.

B. This value is 1 for the left button and usually 2 for the right
button.

C. This value is 0 for the left button and usually 2 for the right
button.

D. This value is 1 for the left button and usually 4 for the right
button.

6. Which sets or retrieves the y-coordinate of the mouse
pointer\textquotesingle s position relative to the object firing the
event?

A. event.offsetY

B. event.clientY

C. event.screenY

D. event.Y

7. Which can disable the right click function?

A. \textless body onMenuClick="return false"\textgreater{}

B. \textless body onRightClick="return false"\textgreater{}

C. \textless body onContextMenu="return false"\textgreater{}

D. \textless body onContextClick="return false"\textgreater{}

8. Which is the correct way to change the mouse cursor to a question
mark?

A. cursor:wait

B. cursor:question

C. cursor:mark

D. cursor:help

9. The .cur file extension refers to \_\_.

A. Windows Cursor file

B. Windows icon file

C. Windows graphic file

D. Windows screen saver file

10. In Internet Explorer, which value of event.button indicates all
three buttons are pressed?

A. 7

B. 6

C. 5

D. 3

Game Programming -- Penn Wu

86

\protect\hypertarget{index_split_006.htmlux5cux23p87}{}{}\includegraphics{index-87_1.png}

Lab \#5

Using mouse buttons for input control

\textbf{Preparation \#1:}

1. Create a new directory named \textbf{C:\textbackslash games}.

2. Use Internt Explorer to go to
\textbf{http://business.cypresscollege.edu/\textasciitilde pwu/cis261/download.htm}
to download lab5.zip (a zipped) file. Extract the files to
C:\textbackslash games directory.

\textbf{Learning Activity \#1: Using custom mouse cursor}

Note: This game is meant to be simple and easy for the sake of
demonstrating programming concepts. Please do not hesitate to enhance
the appearance or functions of this game.

1. Change to the C:\textbackslash games directory.

2. Use Notepad to create a new file named
\textbf{C:\textbackslash games\textbackslash lab5\_1.htm} with the
following contents:

\textless html\textgreater{}

\textless style\textgreater{}

.dots \{position:absolute;font-size:30\}

\textbf{body \{}

\textbf{cursor : url("pencil.cur"), pointer}

\textbf{\}}

\textless/style\textgreater{}

\textless script\textgreater{}

code = "";

function cc() \{

x = event.clientX;

y = event.clientY;

code += "\textless span class=dots style=\textquotesingle left: " + x
+"; top: " + y
+"\textquotesingle\textgreater.\textless/span\textgreater";
area1.innerText = "";

area1.innerHTML = code;

\}

\textless/script\textgreater{}

\textless body onMouseMove=cc()\textgreater{}

\textless div id=area1\textgreater\textless/div\textgreater{}

\textless/body\textgreater{}

\textless/html\textgreater{}

3. Test the program. A sample output looks:

The cursor changes to

a pencil

\textbf{Learning Activity \#2: Angry cat}

Game Programming -- Penn Wu

87

\protect\hypertarget{index_split_006.htmlux5cux23p88}{}{}\includegraphics{index-88_1.png}

\includegraphics{index-88_2.png}

Note: This game is meant to be simple and easy for the sake of
demonstrating programming concepts. Please do not hesitate to enhance
the appearance or functions of this game.

4. Change to the C:\textbackslash games directory.

5. Use Notepad to create a new file named
\textbf{C:\textbackslash games\textbackslash lab5\_2.htm} with the
following contents:

\textless html\textgreater{}

\textless script\textgreater{}

function anger() \{

cat.src=\textquotesingle cat.gif\textquotesingle{}

setTimeout("cat.src=\textquotesingle cat1.gif\textquotesingle", 1000);

cat.style.pixelLeft =
Math.floor(Math.random()*(document.body.clientWidth -

cat.style.pixelWidth));

cat.style.pixelTop =
Math.floor(Math.random()*(document.body.clientHeight -

cat.style.pixelHeight));

\}

\textless/script\textgreater{}

\textless div id=area1\textgreater{}

\textless img id="cat" src=cat1.gif onMouseOver="anger()"
style="position:absolute"\textgreater{}

\textless/div\textgreater{}

\textless/html\textgreater{}

6. Test the program. Move the cursor to approach the cat, the cat jumps
and get angry. A sample output looks: Original

Angry cat

\textbf{Learning Activity \#3: A simple Tic-Tac-Toe}

Note: This game is meant to be simple and easy for the sake of
demonstrating programming concepts. Please do not hesitate to enhance
the appearance or functions of this game.

1. Change to the C:\textbackslash games directory.

2. Use Notepad to create a new file named
\textbf{C:\textbackslash games\textbackslash lab5\_3.htm} with the
following contents:

\textless html\textgreater{}

\textless head\textgreater{}

\textless style\textgreater{}

.dragme \{position:relative;\}

area2, area3 \{ position:absolute \}

\textless/style\textgreater{}

\textless script
src="dragndrop.js"\textgreater\textless/script\textgreater{}

\textless script\textgreater{}

function cc() \{

codes = "\textless table border=1 cellspacing=0\textgreater";

for (i=1; i\textless=3; i++) \{

Game Programming -- Penn Wu

88

\protect\hypertarget{index_split_006.htmlux5cux23p89}{}{}\includegraphics{index-89_1.png}

codes += "\textless tr height=54\textgreater\textless td
width=54\textgreater\&nbsp;\textless/td\textgreater"; codes +=
"\textless td width=54\textgreater\&nbsp;\textless/td\textgreater";

codes += "\textless td
width=54\textgreater\&nbsp;\textless/td\textgreater\textless/tr\textgreater";

\}

codes += "\textless table\textgreater";

area1.innerHTML= codes;

\}

function insertx() \{

area2.innerHTML += "\textless img src=x.gif
class=\textquotesingle dragme\textquotesingle\textgreater";

\}

function inserto() \{

area3.innerHTML += "\textless img src=o.gif
class=\textquotesingle dragme\textquotesingle\textgreater";

\}

\textless/script\textgreater{}

\textless/head\textgreater{}

\textless body onLoad="cc()"\textgreater{}

\textless span id=area1\textgreater\textless/span\textgreater{}

\textless div id=area2 onMouseUp=insertx()\textgreater\textless img
id=x0 src=x.gif class=dragme\textgreater\textless/div\textgreater{}

\textless div id=area3 onMouseUp=inserto()\textgreater\textless img
id=o0 src=o.gif class=dragme\textgreater\textless/div\textgreater{}

\textless/span\textgreater{}

\textless/body\textgreater{}

\textless/html\textgreater{}

3. Test the program. A sample output looks:

\textbf{Learning Activity \#4: A simple Roulette wheel}

Note: This game is meant to be simple and easy for the sake of
demonstrating programming concepts. Please do not hesitate to enhance
the appearance or functions of this game. \textbf{A later lecture will
explain how the ball can move in} \textbf{a circle}.

1. Change to the C:\textbackslash games directory.

2. Use Notepad to create a new file named
\textbf{C:\textbackslash games\textbackslash lab5\_4.htm} with the
following contents:

\textless html\textgreater{}

\textless script\textgreater{}

var r=120; a=0;

function cc() \{

b1.style.left=197+r*Math.sin(a);

b1.style.top=197+r*Math.cos(a);

if (a \textgreater= Math.PI *2) \{ a=0; \}

else \{ a += Math.PI/180; \}

Game Programming -- Penn Wu

89

\protect\hypertarget{index_split_006.htmlux5cux23p90}{}{}\includegraphics{index-90_1.png}

rotating=setTimeout("cc()", 2);

\}

\textless/script\textgreater{}

\textless body\textgreater{}

\textless img id=b1 src="white\_ball.gif" style="position:absolute;
z-index:1; left:20; top:20"\textgreater{}

\textless img src=wheel.gif style="position:absolute;

z-index:0; left:0; top:0"\textgreater{}

\textless div style="position:absolute; left:50; top:400;"\textgreater{}

\textless button
onClick="cc()"\textgreater Rotate\textless/button\textgreater{}

\textless button
onClick=clearTimeout(rotating)\textgreater Stop\textless/button\textgreater{}

\textless/div\textgreater{}

\textless/body\textgreater{}

\textless html\textgreater{}

3. Test the program. Click the Rotate button to start, and Stop to stop.
A sample output looks: \textbf{Learning Activity \#5: Mini Monopoly
Game}

Note: This game is meant to be simple and easy for the sake of
demonstrating programming concepts. Please do not hesitate to enhance
the appearance or functions of this game.

1. Change to the C:\textbackslash games directory.

2. Copy the lab5\_5.htm (the one you downloaded previously) to the
C:\textbackslash games directory.

3. Use Notepad to open the
\textbf{C:\textbackslash games\textbackslash lab5\_5.htm} file, and add
the following bold lines between \textless html\textgreater{} and

\textless body\textgreater{} tags. (this part of code import the
dragndrop.js library file and a function dice()):

\textless html\textgreater{}

\textbf{}

\textbf{\textless head\textgreater{}}

\textbf{}

\textbf{\textless style\textgreater{}}

\textbf{.dragme\{position:relative;\}}

\textbf{td \{ font-size:11px;text-align:center;font-family:arial\}}

\textbf{.cards \{position:absolute; width:120; height:60;}

\textbf{border:solid 1 black; text-align:center; font-size:16px;}

\textbf{color:white; \}}

\textbf{\textless/style\textgreater{}}

\textbf{}

\textbf{\textless script\textgreater{}}

\textbf{}

\textbf{function init() \{}

\textbf{for (i=1;i\textless=2;i++) \{}

Game Programming -- Penn Wu

90

\protect\hypertarget{index_split_006.htmlux5cux23p91}{}{}\textbf{var
k=Math.floor(Math.random()*6)+1;}

\textbf{area1.innerHTML += "\textless img src=d"+k+".gif
border=2\textgreater{} ";}

\textbf{\}}

\textbf{\}}

\textbf{}

\textbf{function dice() \{}

\textbf{area1.innerText="";}

\textbf{for (i=1;i\textless=2;i++) \{}

\textbf{var k=Math.floor(Math.random()*6)+1;}

\textbf{area1.innerHTML += "\textless img src=d"+k+".gif
border=2\textgreater{} ";}

\textbf{\}}

\textbf{tossing=setTimeout("dice()", 50);}

\textbf{\}}

\textbf{}

\textbf{function chancewShow() \{}

\textbf{cchance.style.backgroundColor="white";}

\textbf{cchance.style.fontSize="10px";}

\textbf{cchance.style.color="black";}

\textbf{var j = Math.floor(Math.random()*3);}

\textbf{switch (j) \{}

\textbf{case 0: cchance.innerText="You won \$50!"; break;}

\textbf{case 1: cchance.innerText="You lost \$100!"; break;}
\textbf{case 2: cchance.innerText="Speeding fine, \$20!"; break;}

\textbf{\}}

\textbf{\}}

\textbf{}

\textbf{function chestShow() \{}

\textbf{cchest.style.backgroundColor="white";}

\textbf{cchest.style.fontSize="10px";}

\textbf{cchest.style.color="black";}

\textbf{var j = Math.floor(Math.random()*3);}

\textbf{switch (j) \{}

\textbf{case 0: cchest.innerText="Christmas donation, \$50!"; break;}
\textbf{case 1: cchest.innerText="State Tax \$100!"; break;}
\textbf{case 2: cchest.innerText="You won the lottery of \$20!"; break;}

\textbf{\}}

\textbf{\}}

\textbf{}

\textbf{function cover() \{}

\textbf{cchance.style.backgroundColor="red";}

\textbf{cchance.style.fontSize="16px";}

\textbf{cchance.style.color="white";}

\textbf{cchance.innerText="Chance";}

\textbf{}

\textbf{cchest.style.backgroundColor="green";}

\textbf{cchest.style.fontSize="16px";}

\textbf{cchest.style.color="white";}

\textbf{cchest.innerText="Community Chest";}

\textbf{\}}

\textbf{\textless/script\textgreater{}}

\textbf{}

\textbf{\textless script
src="dragndrop.js"\textgreater\textless/script\textgreater{}}

\textbf{\textless/head\textgreater{}}

\textbf{}

\textless/body\textgreater{}

4. Replace the \textless body\textgreater{} tag with the following line:

\textless body onLoad=init()\textgreater{}

5. At the end of the code, add the following bold lines between
\textless/table\textgreater{} and \textless/body\textgreater{} tags:

\textless/table\textgreater{}

Game Programming -- Penn Wu

91

\protect\hypertarget{index_split_006.htmlux5cux23p92}{}{}\includegraphics{index-92_1.png}

\includegraphics{index-92_2.png}

\includegraphics{index-92_3.png}

\textbf{\textless/td\textgreater\textless td\textgreater{}}

\textbf{\textless span id=area1\textgreater\textless/span\textgreater{}}

\textbf{}

\textbf{\textless div class="cards" id="cchance" style="left:200;
top:150;} \textbf{background-Color:red;"
onClick="chancewShow()"\textgreater Chance\textless/div\textgreater{}}

\textbf{}

\textbf{\textless div class="cards" id="cchest" style="left:310;
top:270;} \textbf{background-Color:green;"
onClick="chestShow()"\textgreater Community
Chest\textless/div\textgreater{}}

\textbf{}

\textbf{\textless button onMouseDown="dice();cover()"
onMouseUp=clearTimeout(tossing)\textgreater Toss}
\textbf{Dice\textless/button\textgreater{}}

\textbf{\textless/td\textgreater\textless/tr\textgreater\textless/table\textgreater{}}

\textbf{}

\textbf{\textless img src="08\_castle.gif" class="dragme"\textgreater{}}

\textbf{\textless img src="15\_Knight.gif" class="dragme"\textgreater{}}

\textless/body\textgreater{}

\textless/html\textgreater{}

6. Test the program. To play, first move the two chess pieces to the Go
cell. Toss the dice, calculate the points, and move forwards. Click
either Chance or Community block when you arrive at one of them. A
sample output looks:

initial

Toss dice and move chess piece

Reading community chest

\textbf{Submittal}

Upon completing all the learning activities,

1. Upload all files you created in this lab to your remote web server.

2. Log in to to Blackboard, launch Assignment 05, and then scroll down
to question 11.

3. Copy and paste the URLs to the textbox. For example,

•

http://www.geocities.com/cis261/lab5\_1.htm

•

http://www.geocities.com/cis261/lab5\_2.htm

•

http://www.geocities.com/cis261/lab5\_3.htm

•

http://www.geocities.com/cis261/lab5\_4.htm

•

http://www.geocities.com/cis261/lab5\_5.htm

No credit is given to broken link(s).

Game Programming -- Penn Wu

92

\protect\hypertarget{index_split_006.htmlux5cux23p93}{}{}\includegraphics{index-93_1.png}

Lecture \#6

Using keyboard for input control

\protect\hypertarget{index_split_007.html}{}{}

\hypertarget{index_split_007.htmlux5cux23calibre_pb_6}{%
\subsection{Introduction}\label{index_split_007.htmlux5cux23calibre_pb_6}}

The keyboard has been the computer input device for the longest time,
even since the old days when there was no such thing called mouse and
graphical user interface. As a matter of fact, even today, the keyboard
is still a useful input device for a wide range of games. For example,
any game involving the player moving an object around will benefit from
using the arrow keys. In this lecture, you will learn many games that
uses the arrow keys to move objects.

Basic

JavaScript support some keyboard event handlers, which are very useful
in responding to the user keyboard-input when keys on a keyboard are
pressed. These events are what make it possible for JavaScript related
event

to ``react'' when a key is pressed:

handlers

Table: Keyboard related event handlers

\textbf{Event handler}

\textbf{functions}

Onkeypress

invokes JavaScript code when a key is pressed

Onkeydown

invokes JavaScript code when a key is held down (but not yet

released)

onkeyup

invokes JavaScript code when a key is has been released after being
pressed.

These events can be bind to most elements on a game page. But, you will
probably stick to either the ``document'' element in most cases. For
example, the code below uses the \textbf{onkeypress} property of the
\textbf{document} element to call the \textbf{show} function, which in
turns changes the display attribute's value from ``\textbf{none}'' to
``\textbf{inline}''. In other words, when the user presses any key, the
\textbf{cat.gif} appears.

\textless script type="text/javascript"\textgreater{}

function show()\{

m1.style.display="inline";

\}

document. \textbf{onkeypress}=show;

\textless/script\textgreater{}

\textless img id=m1 src="cat.gif" style="display:none"\textgreater{} In
the following example, the \textbf{onKeyDown} is used to display the
time the page was loaded, while \textbf{onKeyUp} displays the current
time.

\textless script\textgreater{}

function init\_time() \{

init\_time = Date();

\}

\textless/script\textgreater{}

\textless body onLoad=init\_time()

onKeyDown="t1.innerText=init\_time"

onKeyUp="t2.innerText=Date()"\textgreater{}

Starting time: \textless b
id=t1\textgreater\textless/b\textgreater\textless br\textgreater{}

Ending time: \textless b id=t2\textgreater\textless/b\textgreater{}

A sample output looks:

Game Programming -- Penn Wu

93

\protect\hypertarget{index_split_007.htmlux5cux23p94}{}{}\includegraphics{index-94_1.png}

\includegraphics{index-94_2.png}

To further demonstrate how the keyboard-related even handlers work in
game programming, create a new project called kb.htm with the following
contents:

\textless html\textgreater{}

\textless body \textbf{onkeydown}=cc()\textgreater{}

\textless div id=area1 style="border:solid 1 black;

background-color:\#abcdef;

width:300; height:300;"\textgreater{}

\textless/div\textgreater{}

\textless img id=ball src="ball.gif" style="top:315;

left:10; position:absolute"\textgreater{}

\textless/body\textgreater{}

\textless/html\textgreater{}

Use Internet Explorer to run the code. It creates a pale-blue area with
a rolling ball below it.

The \textless body\textgreater{} tag contains an \textbf{onKeyDown}

event, which will detect if the user ever

presses any key.

A user-defined function \textbf{cc()} is assigned to the

onKeyDown event handler, although it is not

yet created. You will soon create and define

what this cc() function can do.

The \textbf{\textless div\textgreater{}} and
\textbf{\textless/div\textgreater{}} defines the pale-blue area. All its
properties, such as border,

background color, width, and height, are

defined using CSS. The id attribute simply

defines the ID of this area.

The \textless img\textgreater{} tag loads the animated gif file. It also
uses CSS to define properties. Noticeably, the position is set to be
absolute, which means the position value set \textbf{(top, left)} is
based on the Web browser's origin. The ID of the gif file is ball.

The KeyCode

The \textbf{keyCode} property of the Event object returns UniCode value
of key pressed. The returned value is the key\textquotesingle s
numerical value, not the American National Standards Institute (ANSI)
value.

You can use keyCode to detect when the user has pressed an arrow or
function key, which cannot be specified by the key property.

The syntax is:

event.keyCode

For example, you can display each key's key code on the status bar by
using the following code:

\textless body onKeyDown="status=event.keyCode"\textgreater{}

Press the Alt key, you see a value 18 on the status bar:

Table: Common Key Codes

Game Programming -- Penn Wu

94

\protect\hypertarget{index_split_007.htmlux5cux23p95}{}{}\textbf{Key
Code}

\textbf{}

\textbf{Key Code}

\textbf{}

\textbf{Key Code}

\textbf{}

\textbf{Key Code}

A 65

0 48

+ 107

{[}Alt{]} 18

B 66

1 49

- 109

{[}Enter{]} 13

C 67

2 50

* 106

{[}Shift{]} 16

D 68

3 51

/ 111

{[}Ctrl{]} 17

E 69

4 52

` 192

{[}CapsLock{]} 20

F 70

5 53

, 188

{[}Esc{]} 27

G 71

6 54

. 190

{[}Backspace{]} 8

H 72

7 55

/ 191

{[}Insert{]} 45

I 73

8 56

; 186

{[}Home{]} 36

J 74

9 57

` 222

{[}PageUp{]} 33

K 75

F1 112

{[} 219

{[}Delete{]} 46

L 76

F2 113

{]} 221

{[}End{]} 35

M 77

F3 114

\textbackslash{} 220

{[}PageDown{]} 34

N 78

F4 115

= 187

{[}PrintScreen{]}

O 79

F5 116

{[}ScrollLock{]} 145

P 80

F6 117

{[}Pause{]} 19

Q 81

F7 118

R 82

F8 119

S 83

F9 120

T 84

F10 121

U 85

F11 122

V 86

F12 123

W 87

← 37

X 88

↑ 38

Y 89

→ 39

Z 90

↓ 40

You can create a function \textbf{cc()} and use the keyCode property of
the event object to determine weather or not the user presses the
\textbf{Left} arrow key (←). This Left arrow key has a numerical value
37. A simple \textbf{if..then} decisive logic can make this decision.

\textless html\textgreater{}

\textless script\textgreater{}

function cc() \{

if(event.keyCode==37) \{

ball.style.pixelLeft -= 10;

\}

\}

\textless/script\textgreater{}

\textless body onkeydown=cc()\textgreater{}

\textless div id=area1 style="border:solid 1 black;

background-color:\#abcdef;

width:300; height:300;"\textgreater{}

\textless img id=ball src="ball.gif" style="top:10;left:10;
position:relative"\textgreater{}

\textless/div\textgreater{}

\textless/body\textgreater{}

\textless/html\textgreater{}

In the following line, ``ball'' is the ID of the gif file, so
``ball.style.left'' represents the horizontal value of the position
value set \textbf{(top, left)}. ``\textbf{style}'' is a keyword that
indicates Cascading Style Sheet.

ball.style.pixelLeft -= 10;

The original value of ``ball.style.left'' is set to the 10, as specified
by ``left:10''. The above line is techinically the same as:

Game Programming -- Penn Wu

95

\protect\hypertarget{index_split_007.htmlux5cux23p96}{}{}

ball.style.pixelLeft = ball.style.pixelLeft - 10;

In the old days, the above line would look:

ball.style.left =
eval(ball.style.left.replace(\textquotesingle px\textquotesingle,\textquotesingle\textquotesingle))
- 10;

The difference between \textbf{left} and \textbf{pixelLeft} is that the
value of left is a string, such as \textbf{120px}. The value of
pixelLeft is an integer, such as \textbf{120}. Since a string value
cannot be calculated arithematically, you need to use the
\textbf{replace()} method to remove the `\textbf{px}' substring.

ball.style.left.
\textbf{replace(\textquotesingle px\textquotesingle,\textquotesingle\textquotesingle)}

Consequently, a value of 120px becomes 120. But, 120 is still a string
made of 1, 2, and 0, not an integer.

The \textbf{eval()} method then convert the number-like string to the
\textbf{integer} data type. In other words, 120 now means
one-hundred-and-twenty, no longer a string that reads one-two-zero. With
the new XHTML, CSS and JavaScript standard, you can ignore this old
technique. However, the instructor purposely brings up this issue for
your reference (see learning activity \#1 for details).

The following line tells the computer to subtract 10 from the current
``ball.style.pixlLeft'' each time when the user press the \textbf{Left}
arrow key (←).

if(event.keyCode==37) \{

ball.style.pixelLeft -= 10;

\}

According to the following table, the Up, Right, and Down arrow keys
have values of 38, 39, and 40. Similarly, you can use the following line
to add 10 to the current ``ball.style.left'' each time when the user
press the \textbf{Right} arrow key (→).

if(event.keyCode==39) \{

ball.style.pixelLeft += 10;

\}

The change the vertical position, you need to modify the current value
of ``ball.style.top'', which represents the vertical value of the
position value set \textbf{(top, left)}.

if(event.keyCode==38) \{

ball.style.pixelTop -= 10;

\}

if(event.keyCode==40) \{

ball.style.pixelTop += 10;

\}

The code now looks:

\textless html\textgreater{}

\textless script\textgreater{}

function cc() \{

if(event.keyCode==37) \{

ball.style.pixelLeft -= 10;

\}

if(event.keyCode==39) \{

ball.style.pixelLeft += 10;

\}

Game Programming -- Penn Wu

96

\protect\hypertarget{index_split_007.htmlux5cux23p97}{}{}\includegraphics{index-97_1.png}

if(event.keyCode==38) \{

ball.style.pixelTop -= 10;

\}

if(event.keyCode==40) \{

ball.style.pixelTop += 10;

\}

\}

\textless/script\textgreater{}

\textless body onkeydown=cc()\textgreater{}

\textless div id=area1 style="border:solid 1 black;

background-color:\#abcdef;

width:300; height:300;"\textgreater{}

\textless img id=ball src="ball.gif" style="top:10;left:10;
position:relative"\textgreater{}

\textless/div\textgreater{}

\textless/body\textgreater{}

\textless/html\textgreater{}

Use Internet to run the code, and use the ←, ↑, →, and ↓ keys to move
the ball.

By the way, some keys, like {[}Shift{]}, {[}Control{]} and {[}Alt{]},
aren't normally thought of as sending characters, but modify the
characters sent by other keys. For many keys, no key codes are returned
on keydown and keyup events. Instead the keyCode value is just zero,
Characters that give a zero keycode when typed include those listed
below, as well as any key when the Alt/Option key is held down.

- \_ \textasciitilde{} ! @ \# \$ \% \^{} \& * ( ) + \textbar{} :
\textless{} \textgreater{} ?

Internet Explorer provides the following properties.

Table: Keyboard related properties of the Event object

Properties

Description

altKey,

Boolean properties that indicate whether the Alt, Ctrl, Meta, and
ctrlKey,

shiftKey

Shift keys were pressed at time of the event.

keycode

Property indicating the Unicode for the key pressed.

type

A string indicating the type of event, such as "mouseover",

"click", etc.

Game Programming -- Penn Wu

97

\protect\hypertarget{index_split_007.htmlux5cux23p98}{}{}For example, to
create a code that requires the user to hold the Shift key and then
press the C key to play the C\# sound, use:

\textless script\textgreater{}

function cs() \{

if ((event.shiftKey) \&\& (event.keyCode==67)) \{

document.Cs.play();

\}

\}

\textless/script\textgreater{}

\textless body onKeyDown=cs()\textgreater{}

\textless embed src="Cs0.wav" autostart=false hidden=true name="Cs"

mastersound\textgreater{}

\textless/body\textgreater{}

There are few properties that work only with Netscape or Firefox, but
are not discussed in this lecture. They are:

Properties

Description

metaKey

Boolean property that indicate whether the Meta key

was pressed at the time of the event.

charCode

Property indicating the Unicode for the key pressed.

which

Legacy property indicating the Unicode for the key

pressed.

Both \emph{\textbf{if..then}} and \emph{\textbf{switch..case}}
statements are frequently used to make decision as what to respond to
the user's input from keyboard. For example,

\textless script\textgreater{}

function move\_witch() \{

if (event.keyCode==37) \{

w1.style.pixelLeft -= 10;\}

if (event.keyCode==38) \{

w1.style. pixelTop -= 10;\}

if (event.keyCode==39) \{

w1.style. pixelLeft += 10;\}

if (event.keyCode==40) \{

w1.style. pixelTop += 10;\}

\}

\textless/script\textgreater{}

You can rewrite the above code using \textbf{switch..case} statement.

\textless script\textgreater{}

function move\_witch() \{

var i = event.keyCode;

switch (i) \{

case 37:

w1.style.pixelLeft -= 10; break;

case 38:

w1.style. pixelTop -= 10; break;

case 39:

w1.style. pixelLeft += 10; break;

case 40:

w1.style. pixelTop += 10; break;

\}

\}

\textless/script\textgreater{}

Game Programming -- Penn Wu

98

\protect\hypertarget{index_split_007.htmlux5cux23p99}{}{}\includegraphics{index-99_1.png}

\includegraphics{index-99_2.png}

Technically speaking, the \textbf{switch..case} statement is a multi-way
decision statement. Unlike the multiple-decision statement that can be
created using \textbf{if..then}. The switch statement evaluates the
conditional expression and tests it against numerous constant values.
The branch corresponding to the value that the expression matches is
taken during execution.

Sample Games

Use Paint to create a gif file (named board.gif) that looks:

Modify the kb.htm file to:

\textless html\textgreater{}

\textless script\textgreater{}

function cc() \{

if(event.keyCode==37) \{

ball.style.left -= 10;

\}

if(event.keyCode==39) \{

ball.style.left += + 10;

\}

if(event.keyCode==38) \{

ball.style.top -= 10;

\}

if(event.keyCode==40) \{

ball.style.top += 10;

\}

\}

\textless/script\textgreater{}

\textless body onkeydown=cc()\textgreater{}

\textbf{\textless img id=board src="board.gif"\textgreater{}}

\textless img id=ball src="ball.gif" style="top:315;left:10"

style="position:absolute"\textgreater{}

\textless/body\textgreater{}

\textless/html\textgreater{}

Use Internet to run the code, you just create a maze game.

Game Programming -- Penn Wu

99

\protect\hypertarget{index_split_007.htmlux5cux23p100}{}{}

This code can be written using the switch..case statement. For example,

\textless html\textgreater{}

\textless script\textgreater{}

function cc() \{

var i = event.keyCode;

switch (i) \{

case 37:

ball.style.pixelLeft -= 10;

break;

case 39:

ball.style.pixelLeft += 10;

break;

case 38:

ball.style.pixelTop -= 10;

break;

case 40:

ball.style.pixelTop += 10;

break;

\}

\}

\textless/script\textgreater{}

\textless body onkeydown=cc()\textgreater{}

\textless img id=board src="board.gif"\textgreater{}

\textless img id=ball src="ball.gif" style="top:315;left:10"

style="position:absolute"\textgreater{}

\textless/body\textgreater{}

\textless/html\textgreater{}

Use keyboard

Many objects have the properties of width and height. You can increase
or decrease the values of to control the

them to create visual effects. In the following example, the
\textless img\textgreater{} tag is given and ID m1.

width and

Consequently, ``m1.style.width'' represents the \textbf{width} property
of this image object.

height of an

object

\textless html\textgreater{}

\textless script\textgreater{}

function cc() \{

if (event.keyCode==38) \{

\textbf{m1.style.pixelWidth} += 10;

\}

if (event.keyCode==40) \{

m1.style.pixelWidth -= 10;

\}

\}

\textless/script\textgreater{}

\textless body onKeyDown=cc()\textgreater{}

\textless center\textgreater\textless img id=m1 src="bat.gif"
style="width:10"\textgreater\textless/center\textgreater{}

\textless/body\textgreater{}

\textless/html\textgreater{}

When you run the code, the Up and Down arrow keys will increase and
decrease the value of width. On the screen, the change of value creates
effect of zoom in and zooms out.

Game Programming -- Penn Wu

100

\protect\hypertarget{index_split_007.htmlux5cux23p101}{}{}\includegraphics{index-101_1.png}

\includegraphics{index-101_2.png}

to

Review

1. Which event handler invokes JavaScript code when a key is held down,
but not yet released?

Questions

A. onKeyOn

B. Onkeypress

C. Onkeydown

D. onkeyup

2. Which is not a keyboard related event handlers?

A. onKeyOn

B. Onkeypress

C. Onkeydown

D. onkeyup

4. Given the following code block, which statement is correct?

\textless body onKeyDown="status = new Date();"

onLoad="document.title=new Date()"

onKeyUp="p1.innerText=new Date()"\textgreater{}

A. When the user loads the page, the current data and time is display on
the status bar.

B. When the user press and hold the {[}Enter{]} key, the current data
and time is displayed on the status bar.

C. When the user press and then release the {[}Enter{]} key, the current
data and time is displayed on the status bar.

D. All of the above

4. Which property of the Event object returns UniCode value of key
pressed?

A. event.keyType

B. event.keyCode

C. event.key

D. event.keypress

5. Given the following code block, which statement is incorrect?

\textless body onKeyDown="status=event.keyCode"\textgreater{}

A. When you press any key, the status bar displays the string
"event.keyCode".

B. When you press the Alt key, the status bar displays the value 18.

C. When you press the left arrow key, the status bar displays the value
37.

D. When you press the P key, the status bar displays the value 80.

6. When Unicode value is what you get when you press the Scroll Lock
key?

A. 145

B. 125

C. 135

D. 105

7. Given the following code, which statement is correct?

ball.style.left.replace(\textquotesingle px\textquotesingle,\textquotesingle\textquotesingle))

Game Programming -- Penn Wu

101

\protect\hypertarget{index_split_007.htmlux5cux23p102}{}{}A. It uses the
replace() method to remove "px" from "10px".

B. It uses the replace() method to remove "10px" from "10px".

C. It sets the value of ball.style.left to null.

D. All of the above

8. Which property of the event object returns a boolean outcome?

A. altKey

B. ctrlKey

C. shiftKey

D. All of the above

9. Given the following code block, which statement is correct?

if ((event.shiftKey) \&\& (event.keyCode==37)) \{

m1.src="1.gif";

\}

A. To change the value of src property of m1 object to "1.gif", you can
press either Shift key or C

key.

B. To change the value of src property of m1 object to "1.gif", you can
press both Shift key and C

key.

C. To change the value of src property of m1 object to "1.gif", you
cannot press Shift key nor C

key.

D. None of the above.

10. Which is the correct way to detect what key has been pressed?

A. \textless body onKeyDown="status=event.keyCode"\textgreater{}

B. \textless body onKeyDown="status=event.keyPressCode"\textgreater{} C.
\textless body onKeyDown="status=event.keyTypeCode"\textgreater{}

D. \textless body onKeyDown="status=event.keyOnCode"\textgreater{}

Game Programming -- Penn Wu

102

\protect\hypertarget{index_split_007.htmlux5cux23p103}{}{}

Lab \#6

\textbf{Preparation \#1:}

1. Create a new directory named \textbf{C:\textbackslash games}.

2. Use Internt Explorer to go to
\textbf{http://business.cypresscollege.edu/\textasciitilde pwu/cis261/download.htm}
to download lab6.zip (a zipped) file. Extract the files to
C:\textbackslash games directory.

\textbf{Learning Activity \#1: Labyrinth}

Note: This game is meant to be simple and easy for the sake of
demonstrating programming concepts. Please do not hesitate to enhance
the appearance or functions of this game.

1. Change to the C:\textbackslash games directory.

2. Use Notepad to create a new file named
\textbf{C:\textbackslash games\textbackslash lab6\_1.htm} with the
following contents:

\textless html\textgreater{}

\textless script\textgreater{}

function cc() \{

var i = event.keyCode;

switch (i) \{

case 37:

ball.style.left=eval(ball.style.left.replace(\textquotesingle px\textquotesingle,\textquotesingle\textquotesingle))
- 10;

break;

case 39:

ball.style.left=eval(ball.style.left.replace(\textquotesingle px\textquotesingle,\textquotesingle\textquotesingle))
+ 10;

break;

case 38:

ball.style.top=eval(ball.style.top.replace(\textquotesingle px\textquotesingle,\textquotesingle\textquotesingle))
- 10;

break;

case 40:

ball.style.top=eval(ball.style.top.replace(\textquotesingle px\textquotesingle,\textquotesingle\textquotesingle))
+ 10;

break;

\}

\}

\textless/script\textgreater{}

\textless body onkeydown=cc()\textgreater{}

\textless img id=board src="board.gif"\textgreater{}

\textless img id=ball src="ball.gif" style="top:315;left:10"
style="position:absolute"\textgreater{}

\textless/body\textgreater{}

\textless/html\textgreater{}

Note: This version use \emph{switch..case}, so be sure to compare this
version with the one in the lecture which use \emph{if..then} statement.

3. Test the program. Use ← , ↑ , → , and ↓ arrow keys to move the ball.
A sample output looks: Game Programming -- Penn Wu

103

\protect\hypertarget{index_split_007.htmlux5cux23p104}{}{}\includegraphics{index-104_1.png}

\textbf{Learning Activity \#2: Music Keyboard}

Note: This game is meant to be simple and easy for the sake of
demonstrating programming concepts. Please do not hesitate to enhance
the appearance or functions of this game.

1. Change to the C:\textbackslash games directory.

2. Use Notepad to open the
\textbf{C:\textbackslash games\textbackslash lab6\_2.htm} (you just
downloaded it) which has the contents inside the dash lines:

\textless html\textgreater{}

\textless style\textgreater{}

.wKey \{

position:absolute;

top:20;

background-color:white;

border-top:solid 1 \#cdcdcd;

border-left:solid 1 \#cdcdcd;

border-bottom:solid 4 \#cdcdcd;

border-right:solid 1 black;

height:150px;

width:40px

\}

.bKey \{

position:absolute;

top:20;

background-color:black;

border:solid 1 white;

height:70px;

width:36px

\}

\textless/style\textgreater{}

\textless script\textgreater{}

\textless!-\/- for middle C -\/-\textgreater{}

function CDown() \{

midC.style.borderTop="solid 1 black";

midC.style.borderLeft="solid 1 black";

midC.style.borderBottom="solid 1 black";

midC.style.borderRight="solid 1 \#cdcdcd";

\}

function CUp() \{

midC.style.borderTop="solid 1 \#cdcdcd";

midC.style.borderLeft="solid 1 \#cdcdcd";

midC.style.borderBottom="solid 4 \#cdcdcd";

midC.style.borderRight="solid 1 black";

\}

\textless!-\/- for middle D -\/-\textgreater{}

Game Programming -- Penn Wu

104

\protect\hypertarget{index_split_007.htmlux5cux23p105}{}{}function
DDown() \{

midD.style.borderTop="solid 1 black";

midD.style.borderLeft="solid 1 black";

midD.style.borderBottom="solid 1 black";

midD.style.borderRight="solid 1 \#cdcdcd";

\}

function DUp() \{

midD.style.borderTop="solid 1 \#cdcdcd";

midD.style.borderLeft="solid 1 \#cdcdcd";

midD.style.borderBottom="solid 4 \#cdcdcd";

midD.style.borderRight="solid 1 black";

\}

\textless!-\/- for middle E -\/-\textgreater{}

function EDown() \{

midE.style.borderTop="solid 1 black";

midE.style.borderLeft="solid 1 black";

midE.style.borderBottom="solid 1 black";

midE.style.borderRight="solid 1 \#cdcdcd";

\}

function EUp() \{

midE.style.borderTop="solid 1 \#cdcdcd";

midE.style.borderLeft="solid 1 \#cdcdcd";

midE.style.borderBottom="solid 4 \#cdcdcd";

midE.style.borderRight="solid 1 black";

\}

3. Add the following lines to the existing contents:

\textbf{function Downkeys() \{}

\textbf{var i = event.keyCode;}

\textbf{}

\textbf{if (event.shiftKey) \{}

\textbf{switch (i) \{}

\textbf{case 67: document.Cs.play(); break;}

\textbf{case 68: document.Ds.play(); break;}

\textbf{\}}

\textbf{\}}

\textbf{}

\textbf{else \{}

\textbf{switch (i) \{}

\textbf{case 67: CDown(); document.Csound.play(); break;}

\textbf{case 68: DDown(); document.Dsound.play(); break;}

\textbf{case 69: EDown(); document.Esound.play(); break;}

\textbf{\}}

\textbf{\}}

\textbf{\}}

\textbf{}

\textbf{function Upkeys() \{}

\textbf{var i = event.keyCode;}

\textbf{switch (i) \{}

\textbf{case 67: CUp(); break;}

\textbf{case 68: DUp(); break;}

\textbf{case 69: EUp(); break;}

\textbf{\}}

\textbf{\}}

\textbf{document.onkeydown=Downkeys;}

\textbf{document.onkeyup=Upkeys;}

\textbf{\textless/script\textgreater{}}

\textbf{}

Game Programming -- Penn Wu

105

\protect\hypertarget{index_split_007.htmlux5cux23p106}{}{}\includegraphics{index-106_1.png}

\textbf{}

\textbf{\textless div\textgreater{}}

\textbf{}

\textbf{\textless span class="wKey" id="midC"
style="left:50"\textgreater\textless/span\textgreater{}}

\textbf{\textless span class="wKey" id="midD"
style="left:91"\textgreater\textless/span\textgreater{}}

\textbf{\textless span class="wKey" id="midE"
style="left:132"\textgreater\textless/span\textgreater{}}

\textbf{}

\textbf{\textless span class="bKey" id="cSharp"
style="left:72"\textgreater\textless/span\textgreater{}}

\textbf{\textless span class="bKey" id="dSharp"
style="left:114"\textgreater\textless/span\textgreater{}}

\textbf{\textless/div\textgreater{}}

\textbf{}

\textbf{\textless!-\/- white key sound files -\/-\textgreater{}}

\textbf{\textless embed src="C0.wav" autostart=false hidden=true
name="Csound" mastersound\textgreater{}}

\textbf{\textless embed src="D0.wav" autostart=false hidden=true
name="Dsound" mastersound\textgreater{}}

\textbf{\textless embed src="E0.wav" autostart=false hidden=true
name="Esound" mastersound\textgreater{}}

\textbf{}

\textbf{\textless!-\/- black key sound files -\/-\textgreater{}}

\textbf{\textless embed src="Cs0.wav" autostart=false hidden=true
name="Cs" mastersound\textgreater{}}

\textbf{\textless embed src="Ds0.wav" autostart=false hidden=true
name="Ds" mastersound\textgreater{}}

\textbf{}

\textbf{\textless/html\textgreater{}}

4. Test the program. Press C, D, and Key first, and then Shift+C,
Shift+D to play corresponding sounds. A sample output looks:

\textbf{Learning Activity \#3: snake game}

Note: This game is meant to be simple and easy for the sake of
demonstrating programming concepts. Please do not hesitate to enhance
the appearance or functions of this game.

1. Change to the C:\textbackslash games directory.

2. Use Notepad to create a new file named
\textbf{C:\textbackslash games\textbackslash lab6\_3.htm} with the
following contents:

\textless html\textgreater{}

\textless script\textgreater{}

var x=190; y=384;

var k=1; n=0;

function draw()\{

var i = event.keyCode;

switch (i) \{

case 37: n=1; k=0; break;

case 38: n=0; k=1; break;

case 39: n=-1; k=0; break;

case 40: n=0; k=-1; break;

\}

\}

function cc() \{

if ((x\textless=10) \textbar\textbar{} (x\textgreater=394)
\textbar\textbar{} (y\textless=-10) \textbar\textbar{}
(y\textgreater=400)) \{

p1.innerText="Game over!";

\}

else \{

y-=k; x-=n;

codes = "\textless span
style=\textquotesingle position:absolute;left:"+x;

Game Programming -- Penn Wu

106

\protect\hypertarget{index_split_007.htmlux5cux23p107}{}{}\includegraphics{index-107_1.png}

codes += "; top:"+ y
+"\textquotesingle\textgreater.\textless/span\textgreater";
area1.innerHTML += codes;

setTimeout("cc()", 0.1);

\}

\}

\textless/script\textgreater{}

\textless body OnLoad=cc() onKeyDown=draw()\textgreater{}

\textless div id=area1 style="position:absolute; width:400;

height:400; border:solid 3 black; color:red;

background-Color: black; left:0;
top:10;"\textgreater\textless/div\textgreater{}

\textless p id=p1 style="position:absolute; left:160;top:200; color:
white"\textgreater\textless/p\textgreater{}

\textless/body\textgreater{}

\textless/html\textgreater{}

3. Test the program. Use ← , ↑ , → , and ↓ arrow keys to move. A sample
output looks: \textbf{Learning Activity \#4: Flying witch}

Note: This game is meant to be simple and easy for the sake of
demonstrating programming concepts. Please do not hesitate to enhance
the appearance or functions of this game.

1. Change to the C:\textbackslash games directory.

2. Use Notepad to create a new file named
\textbf{C:\textbackslash games\textbackslash lab6\_4.htm} with the
following contents:

\textless html\textgreater{}

\textless style\textgreater{}

img \{ position:absolute\}

\textless/style\textgreater{}

\textless script\textgreater{}

function init() \{

a1\_x=document.body.clientWidth;

a1\_y=Math.floor(Math.random()*document.body.clientHeight);

c1\_x=0;

c1\_y=Math.floor(Math.random()*document.body.clientHeight);

Game Programming -- Penn Wu

107

\protect\hypertarget{index_split_007.htmlux5cux23p108}{}{}w1\_x=200;

w1\_y=Math.floor(Math.random()*document.body.clientHeight);

b1\_x=Math.floor(Math.random()*document.body.clientWidth);

b1\_y=document.body.clientHeight;

codes="\textless img id=a1 src=airplane.gif
style=\textquotesingle left:"+a1\_x+";top:"+a1\_y+"\textquotesingle\textgreater";
codes+="\textless img id=c1 src=cloud.gif
style=\textquotesingle left:"+c1\_x+";top:"+c1\_y+"\textquotesingle\textgreater";
codes+="\textless img id=w1 src=witch.gif
style=\textquotesingle left:"+w1\_x+";top:"+w1\_y+"\textquotesingle\textgreater";
codes+="\textless img id=b1 src=bird.gif
style=\textquotesingle left:"+b1\_x+";top:"+b1\_y+"\textquotesingle\textgreater";
p1.innerHTML=codes;

airplane();

cloud();

witch();

bird();

\}

function airplane() \{

if (a1.style.pixelLeft \textgreater{} 0) \{

a1.style.pixelLeft-=10;

\}

else \{

a1.style.left=document.body.clientWidth;

a1.style.top=Math.floor(Math.random()*document.body.clientHeight);

\}

setTimeout("airplane()", 20);

\}

function cloud() \{

if (c1.style.pixelLeft \textless{} document.body.clientWidth) \{

c1.style.pixelLeft += 10;

\}

else \{

c1.style.pixelLeft=0;

c1.style.pixelTop=Math.floor(Math.random()*document.body.clientHeight);

\}

setTimeout("cloud()", 100);

\}

function witch() \{

if (w1.style.pixelLeft \textless{} document.body.clientWidth) \{

w1.style.pixelLeft += 10;

\}

else \{

w1.style.pixelLeft=0;

w1.style.pixelTop=Math.floor(Math.random()*document.body.clientHeight);

\}

setTimeout("witch()", 100);

\}

function bird() \{

if (b1.style.pixelTop \textgreater{} 0) \{

b1.style.pixelTop -= 10;

b1.style.pixelLeft -= 10;

\}

else \{

b1.style.pixelLeft=Math.floor(Math.random()*document.body.clientWidth);
b1.style.pixelTop=document.body.clientHeight;

Game Programming -- Penn Wu

108

\protect\hypertarget{index_split_007.htmlux5cux23p109}{}{}\includegraphics{index-109_1.png}

\}

setTimeout("bird()", 50);

\}

function move\_witch() \{

var i = event.keyCode;

switch (i) \{

case 37:

w1.style.pixelLeft -= 10; break;

case 38:

w1.style.pixelTop -= 10; break;

case 39:

w1.style.pixelLeft += 10; break;

case 40:

w1.style.pixelTop += 10; break;

\}

\}

\textless/script\textgreater{}

\textless body onLoad=init() onKeyDown="move\_witch()"\textgreater{}

\textless span id=p1\textgreater\textless/span\textgreater{}

\textless/body\textgreater{}

\textless/html\textgreater{}

3. Test the program. Use ← , ↑ , → , and ↓ arrow keys to move the witch
to avoid crashing into any object. A sample output looks:

\textbf{Learning Activity \#5: Jet Fighters}

Note: This game is meant to be simple and easy for the sake of
demonstrating programming concepts. Please do not hesitate to enhance
the appearance or functions of this game.

1. Change to the C:\textbackslash games directory.

2. Use Notepad to create a new file named
\textbf{C:\textbackslash games\textbackslash lab6\_5.htm} with the
following contents:

\textless html\textgreater{}

\textless script\textgreater{}

function init() \{

ap01.style.pixelLeft=Math.floor(Math.random()*300);

cc();

\}

var k = 2;

function cc() \{

if (ap01.style.pixelLeft \textgreater{} document.body.clientWidth - 20)
\{ k = -2; \}

else if (ap01.style.pixelLeft \textless{} 10) \{ k = 2; \}

ap01.style.pixelLeft += k;

setTimeout("cc()",5);

Game Programming -- Penn Wu

109

\protect\hypertarget{index_split_007.htmlux5cux23p110}{}{}\includegraphics{index-110_1.png}

\}

function fly() \{

if(event.keyCode==37) \{

st01.style.pixelLeft -= 10;

\}

if(event.keyCode==39) \{

st01.style.pixelLeft += 10;

\}

if(event.keyCode==38) \{

st01.style.pixelTop -= 10;

\}

if(event.keyCode==40) \{

st01.style.pixelTop += 10;

\}

if(event.keyCode==83) \{

b01.style.pixelLeft = st01.style.pixelLeft + 20;

status = st01.style.pixelWidth;

b01.style.pixelTop = st01.style.pixelTop;

b01.style.display=\textquotesingle inline\textquotesingle;

fire();

\}

\}

function fire() \{

b01.style.pixelTop -= 10;

setTimeout("fire()",50);

\}

\textless/script\textgreater{}

\textless body id="bd" onkeydown=fly() onLoad=init()\textgreater{}

\textless img id=ap01 src="e\_plan.gif"
style="position:absolute"\textgreater{}

\textless img id=st01 src="shooter.gif" style="position:absolute;
top:200;left:100"\textgreater{}

\textless span id=b01
style="position:absolute;display:none"\textgreater!\textless/span\textgreater{}

\textless/body\textgreater{}

\textless/html\textgreater{}

Note: Because collision detection and response is the topic of a later
lecture, how you can modify this code so that the bullet can destroy the
plane will be discussed in a later lecture.

3. Test the program. Use ← , ↑ , → , and ↓ arrow keys to move airplane.
Press the S key to shoot a bullet. A sample output looks:

Game Programming -- Penn Wu

110

\protect\hypertarget{index_split_007.htmlux5cux23p111}{}{}

\textbf{Submittal}

Upon completing all the learning activities,

1. Upload all files you created in this lab to your remote web server.

2. Log in to to Blackboard, launch Assignment 06, and then scroll down
to question 11.

3. Copy and paste the URLs to the textbox. For example,

•

http://www.geocities.com/cis261/lab6\_1.htm

•

http://www.geocities.com/cis261/lab6\_2.htm

•

http://www.geocities.com/cis261/lab6\_3.htm

•

http://www.geocities.com/cis261/lab6\_4.htm

•

http://www.geocities.com/cis261/lab6\_5.htm

No credit is given to broken link(s).

Game Programming -- Penn Wu

111

\protect\hypertarget{index_split_007.htmlux5cux23p112}{}{}\includegraphics{index-112_1.png}

Lecture \#7

Handling moving Objects

\protect\hypertarget{index_split_008.html}{}{}

\hypertarget{index_split_008.htmlux5cux23calibre_pb_7}{%
\subsection{Introduction}\label{index_split_008.htmlux5cux23calibre_pb_7}}

Moving objects is an extremely important topic in all games because they
provide an effective means of conveying movement while also allowing
objects to interact with one another. By learning how to move objects in
a game, you can create some interesting games.

For example, in the ball bouncing game (you created in a previous
lecture), there are two objects: the ball and the paddle (a customized
cursor). These two objects must be coded to be movable because they all
move and interact with each other. The ball floats inside a given area
while the paddle can be moved by the player to bounce back the ball.

In this example, the movement of ball is in essence the movement of an
image file, while the paddle is the mouse cursor (whose movement is
handled by the operating systems). But it points out the truth that
moving objects are typically bitmap images, and the game programmers'
job is to keep track of the position, velocity, width, height, Z-index
(or layers), and visibility. A previous lecture had explained how the
coordinate system is used by the computer to position an object using (
\emph{x}, \emph{y}) format.

In terms of DHTML games, CSS is responsible for the appearance of
objects, while a scripting language (e.g. Javascript) is responsible for
how objects move, behave, and interact with other objects. For example,
in the ball bouncing game,

•

CSS is used to define the dimension (width and height), background
color, and border of the area, in which the ball is defined by CSS its
initial position and visibility.

•

JavaScript codes creates two functions \textbf{dd()} and \textbf{ff()}.
The dd() function moves the ball continuously from top-left to
right-bottom direction. The ff() function is a reversed function of
dd().

\textless html\textgreater\textless body
onClick="t1.style.display=\textquotesingle block\textquotesingle;dd()"\textgreater{}

\textless script\textgreater{}

var j=0;

var i=Math.round(Math.random()*1000);

function dd() \{

t1.style.left=i;

t1.style.top=j;

status="("+i+","+j+")";

i++;j+=Math.round(Math.random()*10);

if ((j\textgreater=400) \textbar\textbar{} (i\textgreater=600))
\{clearTimeout(s1);ff()\}

if (i\textless=8) \{i=8;i+=20;j+=10\}

s1=setTimeout("dd()",50);

\}

Game Programming -- Penn Wu

112

\protect\hypertarget{index_split_008.htmlux5cux23p113}{}{}function ff()
\{

t1.style.left=i;

t1.style.top=j;

status="("+i+","+j+")";

j-=10;i-=5;

if (j\textless=0) \{clearTimeout(s2);dd();\} else

\{s2=setTimeout("ff()",50);\}

\}

\textless/script\textgreater{}

\textless div style="position:absolute; top:0 ;left:0; width:608;
height:408; background-color:\#abcdef;

cursor:vertical-text"\textgreater{}

\textless img id=t1 src="ball.gif" style="position:absolute;
top:0;display:none"

onMouseOver="clearTimeout(s1);ff()"\textgreater{}

\textless/div\textgreater{}

\textless/body\textgreater\textless/html\textgreater{}

The movement of the ball is determined by its ( \emph{x}, \emph{y})
coordinates in the browser's body area.

DHTML uses left and top attributes to represent \emph{x} and \emph{y}
respectively. The syntax to use them are: objectID.style.left

and

objectID.style.top

where \textbf{objectID} is the ID you assigned to the object you want to
move. For example, t1 is the ID

of the ball.gif according to the following statement. The
object\textquotesingle s ID varies according to the object, while the
keywords \textbf{style}, \textbf{left}, and \textbf{top} must stay.

\textless img id=t1 src="ball.gif" .............\textgreater{}

When the values of an object's ( \emph{x}, \emph{y}) change, the object
moves from its original location to the new location. In the following
example, two variables \emph{\textbf{i}} and \emph{\textbf{j}} will
assign values of ( \emph{x}, \emph{y}) to the \textbf{t1}

object.

function dd() \{

t1.style.left=i;

t1.style.top=j;

...................

\textbf{i++;j+=Math.round(Math.random()*10);}

....................

s1=setTimeout("dd()",50);

\}

The \textbf{setTimeout()} method will call the dd() function every 50
milliseconds, and each time when dd() executes, the following line
forces \emph{\textbf{i}} and \emph{\textbf{j}} to change their values.
In other words, the ( \emph{x}, \emph{y}) value of \textbf{t1} is
renewed each time when dd() executes. Since \textbf{t1} receives a new
position value set ( \emph{x}, \emph{y}), it keeps on moving from
previous ( \emph{x}, \emph{y}) to current ( \emph{x}, \emph{y}) to
create a visual effect of moving ball.

Creating a

A motion object must have identifiable properties with valid values for
the computer to know motion effects

how to manage it motion. Game programming using object-oriented
technique to handle motion is basically the implementation of an
object's properties. Most programming languages provide the following
properties to define the motion of a given object in a game that is
capable of Game Programming -- Penn Wu

113

\protect\hypertarget{index_split_008.htmlux5cux23p114}{}{}moving over
time. The basic techniques for finding an object's property are the same
in very language. In terms of DHTML, for example, the general syntax for
finding an element's HTML

property values is:

getElementByID(objectID).style.PropertyName;

where, \textbf{objectID} is the ID you assign to a given object, while
the word ``\textbf{style}'' is a keyword that must stay as it is. As to
the \textbf{PropertyName}, it must be a valid property. In DHTML game
programming, you will frequently use the following properties:

•

position (such as left, top, pixelLeft, pixelTop, etc.)

•

visibility

•

z-index

Consider the following example; \textbf{m1} is the ID of the image file
g1.gif (which is an object).

\textless html\textgreater{}

\textless script\textgreater{}

function init()

\{

var
x=document.getElementById(\textquotesingle m1\textquotesingle).style.pixelLeft;

var
y=document.getElementById(\textquotesingle m1\textquotesingle).style.pixelTop;

status="("+x+", "+y+")";

\}

\textless/script\textgreater{}

\textless body onLoad=init()\textgreater{}

\textless img id=m1 src=g1.gif\textgreater{}

\textless/body\textgreater{}

\textless/html\textgreater{}

The following line tells the computer to get the value of pixelLeft of
the \textbf{m1} object.

var
x=document.getElementById(\textquotesingle m1\textquotesingle).style.pixelLeft;

In previous lectures, you had learned that the above line can be
simplified to: var x=\textbf{m1}.style.pixelLeft;

So, what is the point using \textbf{getElementById()}? When using it
with \textbf{srcElement}, you will make your long codes very concise.
The \textbf{srcElement} property sets or retrieves the object that fired
the event. The syntax is:

{[} oObject = {]} event.srcElement

The optional \textbf{{[} oObject = {]}} part refers to the object that
specifies or receives the event that fired.

For example,

\textless html\textgreater{}

\textless script\textgreater{}

function change\_color() \{

var oldID = event.srcElement.id;

document.getElementById(oldID).style.color=oldID;

\}

\textless/script\textgreater{}

\textless button id="red"
onclick="change\_color()"\textgreater Red\textless/button\textgreater{}

\textless button id="blue"
onclick="change\_color()"\textgreater Color\textless/button\textgreater{}

\textless/html\textgreater{}

In the above code, the following line detects the ID of the button you
click (e.g. ``red'' or ``green''), Game Programming -- Penn Wu

114

\protect\hypertarget{index_split_008.htmlux5cux23p115}{}{}\includegraphics{index-115_1.png}

\includegraphics{index-115_2.png}

and assign the value to a variable named \textbf{oldID}.

var oldID = event.srcElement.id;

The following line tells the computer to change the text color of the
button to the value held by the oldID variable (which is either ``red''
or ``green'').

document.getElementById(oldID).style. \textbf{color}=oldID;

You will be using this programming technique often in the later
lectures. When executing the above code, you first see:

After clicking on the buttons, the colors change from black to red and
black to blue.

The SrcElement DOM object is generated any time that an event is called,
and it contains the element that called the event in the first place.
One of the principle advantages of using it is that you can radically
simplify your DHTML scripting code by assigning an event on a single
container object, then checking the SrcElement to see which object
within that container actually fired the event.

DHTML

DHTML uses the \textbf{position} property to sets or retrieves the type
of positioning used for the object.

position

Te syntax is:

systems

objectID.style.position=value

Possible values are:

•

\textbf{static}: Default. Object has no special positioning; it follows
the layout rules of HTML.

•

\textbf{absolute}: Object is positioned relative to parent element's
position or to the body object if its parent element is not positioned
using the \textbf{top} and \textbf{left} properties.

•

\textbf{relative}: Object is positioned according to the normal flow,
and then offset by the \textbf{top} and \textbf{left} properties.

A simple demonstration illustrates the point. Suppose that you had three
\textless DIV\textgreater{} elements within another
\textless DIV\textgreater{} element, one of each color red, green or
blue. You can place a general event handler on the containing DIV that
can then be used to intercept either the outside division or any of the
three smaller DIV elements.

\textless script\textgreater{}

function getColor()\{

var src=event.srcElement;

alert(src.style.backgroundColor);

\}

\textless/script\textgreater{}

\textless div

style="background-color:black; color:white; width:500px;

cursor:hand;"

onclick="getColor()"\textgreater Black

\textless div

style="background-color:red; color:black; width:150px;

Game Programming -- Penn Wu

115

\protect\hypertarget{index_split_008.htmlux5cux23p116}{}{}\textbf{position:relative}"\textgreater{}

Red Box\textless/div\textgreater{}

\textless div style="background-color:green; color:black; width:150px;
\textbf{position:relative}"\textgreater{}

Green Box\textless/div\textgreater{}

\textless div

style="background-color:blue; color:black; width:150px;

\textbf{position:relative}"\textgreater{}

Blue Box\textless/div\textgreater{}

\textless/div\textgreater{}

This technique is common for dealing with a large number of items that
have similar actions and are contained in a single element. Collapsible
trees, subordinate selections, input elements, and behaviors can all
make use of this technique.

Setting the property to absolute pulls the object out of the "flow" of
the document and positions it regardless of the layout of surrounding
objects. If other objects already occupy the given position, they do not
affect the positioned object, nor does the positioned object affect
them. Instead, all objects are drawn at the same place, causing the
objects to overlap. This overlap is controlled by using the z-index
attribute or property. Absolutely positioned objects do not have
margins, but they do have borders and padding.

To enable absolute positioning on an object you must specify at least
one of the top, bottom, left, or right properties, in addition to
setting the position property to absolute. Otherwise, these positioning
properties use their default value of absolute, which causes the object
to render immediately after the preceding elements, according to the
layout rules of HTML.

Object

The \textbf{visibility} property sets if an element should be visible or
invisible. The \textbf{display} property sets Visibility and

how/if an element is displayed. Invisible elements take up space on the
page. Use the "display"

Display

property to create invisible elements that do not take up space.

The syntax is:

objectID.style.visibility=value

and

objectID.style.display=value

Command values of Visibility project include:

\textbf{Value}

\textbf{Description}

visible

The element is visible

hidden

The element is invisible

collapse

When used in table elements, this value removes a row or column, but it
does not affect the table layout. The space taken up by the row or
column will be available for other content. If this value is used on
other elements, it renders as "hidden"

Command values of Display project include:

\textbf{Value}

\textbf{Description}

none

The element will not be displayed

block

The element will be displayed as a block-level element, with a line
break before and after the element

inline

The element will be displayed as an inline element, with no line break
before or after the element

Game Programming -- Penn Wu

116

\protect\hypertarget{index_split_008.htmlux5cux23p117}{}{}list-

The element will be displayed as a list

item

run-in

The element will be displayed as block-level or inline element depending
on context

compact

The element will be displayed as block-level or inline element depending
on context

For example,

\textless html\textgreater{}

\textless u onClick=this.style.display="none"\textgreater Make me
disappear!\textless/u\textgreater{} This is a
line.\textless br\textgreater{}

\textless b onClick=this.style.visibility="hidden"\textgreater Make me
invisible!\textless/b\textgreater{} This is another line.

\textless/html\textgreater{}

When click the sentence ``Make me disappear!'', it disappears (being
removed). When click the sentence ``Make me invisible!'', it becomes
invisible (their character space remain there) Layering

Most of the Web pages are meant to be 2D (2 dimensional), so objects on
the Web pages can be Objects -- z-positioned by X and Y coordinates, or
its horizontal and vertical positioning. CSS positioning index property

uses the Z axis to layer objects on top of each other. With the z-index
property, you can specify the layer on which an object lies.

By setting the z-index higher or lower, you can move the object up or
down in the stack.

Basically, the bigger the value is; the more front its layer will be.
Also,

•

A positive value positions the element above text that has no defined
z-index.

•

A negative value positions it below.

•

Set this parameter to null to remove the attribute.

The syntax is:

objectID.style.zIndex="value";

Z-index property is relative to x and y-axis, it will not work without x
and y-axis, that is, top and left properties. For example,

\textless html\textgreater{}

\textless style\textgreater{}

u \{z-index:3;position:absolute;left:55;top:25;background-

color:red\}

b \{z-index:2;position:absolute;left:65;top:35;background-

color:green\}

i \{z-index:1;position:absolute;left:75;top:45;background-

color:blue\}

\textless/style\textgreater{}

\textless u\textgreater This is layer1.\textless/u\textgreater{}

\textless b\textgreater this is layer2.\textless/b\textgreater{}

\textless i\textgreater This is layer3.\textless/i\textgreater{}

\textless/html\textgreater{}

One can also integrate the CSS style with HTML tags, for instance:

\textless html\textgreater{}

\textless style\textgreater{}

u \{z-index:3;position:absolute;top:55px;left:25px;background-

color:red\}

i \{z-index:2;position:absolute;top:65px;left:55px;background-

color:green\}

Game Programming -- Penn Wu

117

\protect\hypertarget{index_split_008.htmlux5cux23p118}{}{}\includegraphics{index-118_1.png}

b \{z-index:1;position:absolute;top:75px;background-color:blue\}

\textless/style\textgreater{}

\textless u\textgreater this goes to the top.\textless/u\textgreater{}

\textless i\textgreater this goes to the
middle.\textless/i\textgreater{}

\textless b\textgreater this goes the bottom.\textless/b\textgreater{}

\textless/html\textgreater{}

The output looks:

Continuous

A motion object also have built-in functions, such as start, stop, turn
left, turn right, and so on.

motion

The term "built-in" implies that the game programmer must create these
functions and given them to the object. Most programming languages
provide methods to allow game programmers to create "user-defined
function", which then serves as the "built-in" function of the object
for the players to use to control the object\textquotesingle s motion.

JavaScript is relatively weak on the support of such methods, but you
can use the following JavaScript method to handle motions of your DHTML
games.

•

\textbf{setTimeout()} - executes a code some time in the future

•

\textbf{clearTimeout()} - cancels the setTimeout()

Note: The setTimeout() and clearTimeout() are both methods of the HTML
DOM Window object.

To use the setTimeout() method, the formal syntax is:

var variableNmae=setTimeout("statement", milliseconds) Although, you can
simplify the syntax to:

setTimeout("statement", milliseconds);

it is highly recommended that you use the formal syntax, so you can use
the clearTimeout() method to cancel it. The syntax of clearTimeout()
method is:

clearTimeout(variableName)

Due to the object-oriented programming model, methods are controlled by
the programming language\textquotesingle s run-time environment (meaning
the background engine), so they cannot be destroyed, deleted, or
terminated by the user.

You can create an object, such as a variable in this setTimeout() case,
to transfer all the setTimeout() method\textquotesingle s function to
this object. Consequently, you can destroy, delete, or terminate the
object any time you want by using clearTimeout(variableName).

Additionally, the setTimeout() method returns a value - In the statement
above, the value is stored in a variable called t. If you want to cancel
this setTimeout(), you can refer to it using the variable name.

Drag and drop

``Drag and drop'' describes a particular action you can move an element
from one location to another by dragging it with the mouse. Click an
object, such as a image, then hold down the mouse button as you drag the
object to a new location. You drop the object by releasing the mouse
button.

This function is made possible by a good handling of ( \emph{x},
\emph{y}) coordinates using mouse buttons'

Game Programming -- Penn Wu

118

\protect\hypertarget{index_split_008.htmlux5cux23p119}{}{}positioning
systems: \textbf{event.clientX} and \textbf{event.clientY}. Given the
following code which is saved as an external code file (with a name such
as dragndrop.js). You can use this code file as library code to provide
drag-and-drop function to other DHTML codes.

var dragapproved=false

var z,x,y

function move()\{

if (event.button==1 \&\& dragapproved)\{

z.style.pixelLeft=temp1+event.clientX-x

z.style.pixelTop=temp2+event.clientY-y

return false

\}

\}

function drags()\{

if (!document.all)

return

if (event.srcElement.className=="drag")\{

dragapproved=true

z=event.srcElement

temp1=z.style.pixelLeft

temp2=z.style.pixelTop

x=event.clientX

y=event.clientY

document.onmousemove=move

\}

\}

document.onmousedown=drags

document.onmouseup=new Function("dragapproved=false") There are four
user-defined variables: dragapproved, \emph{z}, \emph{x}, and \emph{y}.
The variable \textbf{dragapproved} is merely used as a reference that
indicates whether or not any mouse button is currently pressed and held.
If no mouse button is pressed and held, the value of dragapproved is
\textbf{false}.

The variable \emph{\textbf{z}} is just a temporarily name of the object
that is being dragged. It is used only when an object is dragged.

The variables \emph{\textbf{x}} and \emph{\textbf{y}} represents the
mouse cursor's ( \emph{x}, \emph{y}) coordinates, which is exactly the
\textbf{event.clientX} and \textbf{event.clientY}. The values of
\emph{x} and \emph{y} variable change as the mouse cursor moves.

The following statement triggers the drags() function when an object is
dragged.

document.onmousedown=drags

In the drags() function, the following line verifies if an object being
dragged has a class name

``drag'', and this is the reason why you need to add
\textbf{class="drag"} to any object that may be dragged and dropped
(e.g. \textless img id=obj \textbf{class="drag"}
src=cat.gif\textgreater).

if (event.srcElement.className=="drag")

The following line assigns the real ID of the object being dragged to
\emph{\textbf{z}}.

z=event.srcElement

Two variables \textbf{temp1} and \textbf{temp2} are added to represent
the current (z.style.left, z.style.top) coordinates of the dragged
object, as shown below: Game Programming -- Penn Wu

119

\protect\hypertarget{index_split_008.htmlux5cux23p120}{}{}\includegraphics{index-120_1.png}

temp1=z.style.pixelLeft

temp2=z.style.pixelTop

Notice that \textbf{pixelLeft} and \textbf{pixelTop} are new CSS
positioning properties that specify or receive the \textbf{left} and
\textbf{top} position in pixels respectively. A detailed discussion is
available at a later lecture note.

Once a dragged object is checked and verified to have the
\textbf{class=''drag''} class name, the following statement triggers the
move() function.

document.onmousemove=move

The \textbf{move()} function uses the following statement to detect if
the left button of a mouse is still pressed and held.

if (event.button==1 \&\& dragapproved)\{

..........

\}

If the left button is pressed and held, the following statements update
the values of (left, top) of the dragged object.

z.style.pixelLeft=temp1+event.clientX-x

z.style.pixelTop=temp2+event.clientY-y

The logic is the new values of (\textbf{z.style.pixelLeft},
\textbf{z.style.pixelTop}) equals to the old one (\textbf{temp1},
\textbf{}

\textbf{temp2}) subtract the difference of new mouse cursor
(\textbf{event.clientX}, \textbf{event.clientY}) and old mouse cursor (
\emph{\textbf{x}}, \emph{\textbf{y}}).

Speed

Moving object sometimes involves in the control of speed. In the Flying
Witch game, for example, the bird moves faster than the witch does. In
other words, these two objects have different speeds, as shown below.

function witch() \{

..........

..........

\textbf{setTimeout("witch()", 100);}

\}

function bird() \{

..........

..........

\textbf{setTimeout("bird()", 50);}

\}

The functions \textbf{setTimeout} and \textbf{setInterval} are similar,
and they both can control the speed of a function. However, in fact,
these two functions perform very differently. The \textbf{setTimeout}
function delays for a specified time period and then triggers execution
of a specified function.

Game Programming -- Penn Wu

120

\protect\hypertarget{index_split_008.htmlux5cux23p121}{}{}Once the
function is triggered the setTimeout has finished. You can terminate the
execution of the setTimeout before it triggers the function by using the
\textbf{clearTimeout} function.

In this game programming class, you frequently see setTimeout used at
the end of a function to trigger another execution of the same function
perhaps passing it different parameters. Where this is done the time
from the first triggering of the function until the next will be the
specified delay time plus the time taken to execute the function. Here
is an example: moreSnow();

function moreSnow() \{

// content of moreSnow function

setTimeout("moreSnow()", speed);

\}

The \textbf{setInterval} function also delays for a specified time
before triggering the execution of a specific function. However, after
triggering that function the command doesn't complete. Instead it waits
for the specified time again and then triggers the function again and
continues to repeat this process of triggering the function at the
specified intervals until either the web page is unloaded or the
clearInterval function is called.

The above code using \textbf{setTimeout} could have been written using
\textbf{setInterval} instead and would have looped slightly faster since
the loop would not have waited for the content of the function to be
processed before triggering the repetition. Here is the above code
rewritten to us setInterval: moreSnow();

setInterval("moreSnow()", speed);

function moreSnow() \{

// content of moreSnow function

\}

On the other hand, if you replace the setTimeout function with
setInterval. When pressing any key. The code will fail. It is because
pressing key will continuously trigger the shoot() function, but the
setInterval doesn't complete a loop-\/-it simply delays the looping.
Starting a new loop before completing the old loop will fail the game.

function shoot() \{

var i = b01.style.pixelLeft;

if (i \textless= 400) \{

b01.style.pixelLeft=i + 10;

setTimeout("shoot()",20);

\}

else \{b01.style.display=\textquotesingle none\textquotesingle;

b01.style.pixelLeft=95;

\};

\}

Which of the two functions you should choose to use depends on what you
are trying to achieve.

If different parameters need to be passed to the function on each call
or the content of the function decides whether or not to call itself
again then setTimeout is the one to use. Where the function just needs
triggering at regular intervals then the setInterval function may
produce simpler code within the function particularly if it has multiple
exit points.

Directions

Objects may move in any direction, linear or non-linear. The term
``linear'' implies sequence, while ``non-linear'' implies skipping the
sequence.

In the snake game (you created previously), the snake line moves in a
\textbf{leaner} direction. Each time when you use arrow keys to change a
direction, the line move point by point with an increment of Game
Programming -- Penn Wu

121

\protect\hypertarget{index_split_008.htmlux5cux23p122}{}{}\textbf{1}, as
illustrated below.

..............

case 37: n=1; k=0; break;

case 38: n=0; k=1; break;

case 39: n=-1; k=0; break;

case 40: n=0; k=-1; break;

..............

y-=k; x-=n;

codes = "\textless span
style=\textquotesingle position:absolute;left:"+x;

codes += "; top:"+ y
+"\textquotesingle\textgreater.\textless/span\textgreater";
area1.innerHTML += codes;

setTimeout("cc()", 0.1);

..............

If ( \emph{x1}, \emph{y1}) = (10\textbf{0}, 100) and you press the left
arrow key (the key code is 37), then the very next dot is placed at (
\emph{x2}, \emph{y2}) = (10\textbf{1}, 100), and the next one is
(10\textbf{2}, 100), (10\textbf{3}, 100),\ldots\ldots{} In other words,
to move from (100, 100) to (119, 100). You need to move in a sequence of
100, 101, 102, 103, 104, 105, 106, \ldots, 117, 118, 119 along the
\emph{x} axis.

(100, 100)

(119, 100)

To handle a linear direction movement, simply use \textbf{increment} or
\textbf{decrement}. In the ball bouncing example, two variables
\emph{\textbf{i}} and \emph{\textbf{j}} will assign values of (
\emph{x}, \emph{y}) to the \textbf{t1} object.

function dd() \{

t1.style.left=i;

t1.style.top=j;

...................

\textbf{i++;j+=Math.round(Math.random()*10);}

....................

s1=setTimeout("dd()",50);

\}

where, i++ means i is incremented by 1, which is equivalent to i = i+ 1.
In JavaScript, the increment and decrement operators act on only one
operand. They are hence called Unary operators. They can be used either
before or after a variable as in: a++;

b-\/-;

or

++a;

-\/-b;

When these operators are used as prefixes, JavaScript first adds one to
the variable and then returns the value. When used as suffixes, the
value is first returned and then the variable is incremented.

Game Programming -- Penn Wu

122

\protect\hypertarget{index_split_008.htmlux5cux23p123}{}{}\includegraphics{index-123_1.png}

When the increment or decrement is larger than 1, use the following
syntax: x += \emph{n}; or x -= \emph{n};

where \emph{n} is an number, and \emph{n} is not necessary an integer.
For example, in the Roulette Wheel game, there is a code block that uses
increment that is π/180 which is a decimal value since π is 3.1412.

if (a \textgreater= Math.PI *2) \{ a=0; \}

else \{ \textbf{a += Math.PI/180;} \}

You may need to move objects in directions such as northeast, southeast,
northwest, and southwest. Assuming the object ID is \textbf{obj}, then
the direction control in DHTMLs are (where \emph{n} is an positive
number):

•

\textbf{NE}: obj.style.pixelLeft + \emph{n}; obj.style.pixelTop -
\emph{n};

•

\textbf{SE}: obj.style.pixelLeft + \emph{n}; obj.style.pixelTop +
\emph{n};

•

\textbf{NW}: obj.style.pixelLeft - \emph{n}; obj.style.pixelTop -
\emph{n};

•

\textbf{SW}: obj.style.pixelLeft - \emph{n}; obj.style.pixelTop +
\emph{n}; When you want an object to jump from on location to another,
and ignore all the points between the new and old locations, you are
handling a movement of non-linear direction. For example, in the Flying
UFO game, the \textbf{ufo} object's ( \emph{x}, \emph{y}) values are
generated randomly by the \textbf{Math.random()} method each time when
the \textbf{ufo\_fly()} function executes.

function ufo\_fly() \{

ufo.style.pixelLeft = Math.round(Math.random()*screen.width);

ufo.style.pixelTop = Math.round(Math.random()*screen.height);

setTimeout("ufo\_fly()", 500);

\}

Consequently, the \textbf{ufo} object moves in a random unpredictable
mode.

Moving along a To move an object along a circle, as the ball in the
Roulette Wheel game, you need to understand circle

how computers create a circle.

According to geometry, a \textbf{circle} is the set of all points in a
plane at a fixed distance (known as \textbf{radius)} from a fixed point
(known as \textbf{center}).

All the points on the circumference of the circle has a set of (
\emph{x}, \emph{y}) coordinates. The \emph{x} value is determined by:

centerX + radius × Sinθ

and \emph{y} is determined by:

centerY + radius × Cosθ

Game Programming -- Penn Wu

123

\protect\hypertarget{index_split_008.htmlux5cux23p124}{}{}where θ is
incremented by π/180, and 0 ≤ θ \emph{≤} 2π, because the circumference
of a circle is 2π.

radius

center

Consider the following code, it uses (197, 197) as the ( \emph{x},
\emph{y}) of the center and 120 as radius.

var r=120; a=0;

function cc() \{

b1.style.pixelLeft=197+r*Math.sin(a);

b1.style.pixelTop=197+r*Math.cos(a);

if (a \textgreater= Math.PI *2) \{ a=0; \}

else \{ a += Math.PI/180; \}

rotating=setTimeout("cc()", 2);

\}

The line \textbf{b1.style.pixelLeft=197+r*Math.sin(a);} is actually the
expression of centerX + radius × Sinθ

and \textbf{if (a \textgreater= Math.PI *2) \{ a=0; \}} is actually the
DHTML expression of 0 ≤ θ \emph{≤} 2π

while \textbf{a += Math.PI/180;} means to increment by π/180.
Consequently, each time when the \textbf{cc()} function executes, the
object moves along the circumference of a circle that has the radius 120
and is centered at (197, 197).

Review

1. In terms of DHTML games, \_\_ is responsible for the appearance of
objects, while \_\_ is Questions

responsible for how objects move, behave, and interact with other
objects.

A. CSS, HTML

B. HTML, CSS

C. CSS, scripting language

D. scripting, CSS

2. Given the following code block, which statement is correct?

function dd() \{

t1.style.left=i;

t1.style.top=j;

.........

\}

A. t1.style.top represents the x-coordinate of the t1 object.

B. t1.style.left represents the y-coordinate of the t1.object.

C. t1.style.left represents the x-coordinate of the t1.object.

D. t1.style.top is equivalent to t1.getElementById().style.top

3. Which is equivalent to the following line?

var x=m1.style.pixelLeft;

Game Programming -- Penn Wu

124

\protect\hypertarget{index_split_008.htmlux5cux23p125}{}{}

A. var
x=document.getElementById(\textquotesingle m1\textquotesingle).style.pixelLeft;

B. var x=m1.getElementById().style.pixelLeft;

C. var
x=m1.getElementById(\textquotesingle m1\textquotesingle).style.pixelLeft;

D. var x=document.getElementById().m1.style.pixelLeft;

4. Given the following code, which statement is correct?

function change\_color() \{

var oldID = event.srcElement.id;

document.getElementById(oldID).style.color=oldID;

\}

A. oldID will represent the ID of the object that triggers the event.

B. oldID represents the previous ID of the object that triggers the
event, because srcElement.id will assign a new ID to it.

C. event.srcElement.id retrieve the ID of the mouse button.

D. event.srcElement.id retrieve the key code of the mouse button.

5. Which statement is correct?

A. The default position is "absolute".

B. The srcElement property sets or retrieves the object that fired the
event.

C. The display property sets if an element should be visible or
invisible.

D. The visibility property sets how/if an element is displayed

6. Which sets an object\textquotesingle s position according to the
normal flow, and then offset by the op and left properties?

A. position:static

B. position:absolute

C. position:relative

D. position:physical

7. Which item will be on the front?

A. \textless img id="m1" style=z-index: 1\textgreater{}

B. \textless img id="m2" style=z-index: 2\textgreater{}

C. \textless img id="m3" style=z-index: 3\textgreater{}

D. \textless img id="m4" style=z-index: 4\textgreater{}

8. Which can control the speed of a function in JavaScript?

A. speed();

B. setTimeout();

C. controlTime();

D. setSpeed();

9. Given an object with ID "t1", which can move this object toward
northwest?

A. t1.style.pixelLeft + 2; t1.style.pixelTop + 2;

B. t1.style.pixelLeft + 2; t1.style.pixelTop - 2;

C. t1.style.pixelLeft - 2; t1.style.pixelTop - 2;

D. t1.style.pixelLeft - 2; t1.style.pixelTop + 2;

10. Given a center coordinate (100, 150) and a radius 25. Which
represents the y value of any point on the circle?

A. 100 + 25 * Sin

B. 150 + 25 * Sin

C. 100 + 25 * Cos

Game Programming -- Penn Wu

125

\protect\hypertarget{index_split_008.htmlux5cux23p126}{}{}D. 150 + 25 *
Cos

Game Programming -- Penn Wu

126

\protect\hypertarget{index_split_008.htmlux5cux23p127}{}{}\includegraphics{index-127_1.png}

Lab \#7

\textbf{Preparation \#1:}

1. Create a new directory named \textbf{C:\textbackslash games}.

2. Use Internt Explorer to go to
\textbf{http://business.cypresscollege.edu/\textasciitilde pwu/cis261/download.htm}
to download lab7.zip (a zipped) file. Extract the files to
C:\textbackslash games directory.

\textbf{Learning Activity \#1: Drag and Drop}

Note: This game is meant to be simple and easy for the sake of
demonstrating programming concepts. Please do not hesitate to enhance
the appearance or functions of this game.

1. Change to the C:\textbackslash games directory.

2. Use Notepad to create a new file named
\textbf{C:\textbackslash games\textbackslash lab7\_1.htm} with the
following contents:

\textless html\textgreater{}

\textless style\textgreater{}

.drag \{position:absolute\}

\textless/style\textgreater{}

\textless script
src="dragndrop.js"\textgreater\textless/script\textgreater{}

\textless script\textgreater{}

var codes = "";

var x, y;

function init() \{

for (i=1; i\textless=6; i++) \{

x = Math.floor(Math.random()*document.body.clientWidth/2);

y = Math.floor(Math.random()*document.body.clientHeight/2);

codes += "\textless img id=\textquotesingle cy" + i +
"\textquotesingle{} class=\textquotesingle drag\textquotesingle"; codes
+= " src=cy" +i + ".jpg style=\textquotesingle left:
"+x+";top:"+y+"\textquotesingle\textgreater\textless br\textgreater";

\}

area1.innerHTML = codes;

\}

\textless/script\textgreater{}

\textless body onLoad=init()\textgreater{}

\textless span id=area1\textgreater\textless/span\textgreater{}

\textless/body\textgreater{}

\textless/html\textgreater{}

3. Test the program. A sample output looks:

Game Programming -- Penn Wu

127

\protect\hypertarget{index_split_008.htmlux5cux23p128}{}{}\includegraphics{index-128_1.png}

\textbf{Learning Activity \#2: USA Map}

Note: This game is meant to be simple and easy for the sake of
demonstrating programming concepts. Please do not hesitate to enhance
the appearance or functions of this game.

1. Change to the C:\textbackslash games directory.

2. Use Notepad to create a new file named
\textbf{C:\textbackslash games\textbackslash lab7\_2.htm} with the
following contents:

\textless html\textgreater{}

\textless style\textgreater{}

.drag \{position:absolute\}

\textless/style\textgreater{}

\textless script
src="dragndrop.js"\textgreater\textless/script\textgreater{}

\textless body\textgreater{}

\textless img src="usa.gif"\textgreater{}

\textless table class="drag" border=1 cellspacing=0\textgreater{}

\textless tr\textgreater{}

\textless td\textgreater California\textless/td\textgreater{}

\textless td\textgreater Oregan\textless/td\textgreater{}

\textless td\textgreater Washington\textless/td\textgreater{}

\textless td\textgreater Nevada\textless/td\textgreater{}

\textless td\textgreater Arizona\textless/td\textgreater{}

\textless td\textgreater Utah\textless/td\textgreater{}

\textless td\textgreater Idaho\textless/td\textgreater{}

\textless/tr\textgreater{}

\textless tr\textgreater{}

\textless td\textgreater\textless img src="state1.gif"
class="drag"\textgreater\textless/td\textgreater{}

\textless td\textgreater\textless img src="state2.gif"
class="drag"\textgreater\textless/td\textgreater{}

\textless td\textgreater\textless img src="state3.gif"
class="drag"\textgreater\textless/td\textgreater{}

\textless td\textgreater\textless img src="state4.gif"
class="drag"\textgreater\textless/td\textgreater{}

\textless td\textgreater\textless img src="state5.gif"
class="drag"\textgreater\textless/td\textgreater{}

\textless td\textgreater\textless img src="state6.gif"
class="drag"\textgreater\textless/td\textgreater{}

\textless td\textgreater\textless img src="state7.gif"
class="drag"\textgreater\textless/td\textgreater{}

\textless/tr\textgreater{}

\textless/table\textgreater{}

\textless/body\textgreater{}

\textless/html\textgreater{}

3. Test the program. A sample output looks:

\textbf{Learning Activity \#3: Roulette Wheel 2}

Note: This game is meant to be simple and easy for the sake of
demonstrating programming concepts. Please do not hesitate to enhance
the appearance or functions of this game.

Game Programming -- Penn Wu

128

\protect\hypertarget{index_split_008.htmlux5cux23p129}{}{}\includegraphics{index-129_1.png}

1. Change to the C:\textbackslash games directory.

2. Use Notepad to create a new file named
\textbf{C:\textbackslash games\textbackslash lab7\_3.htm} with the
following contents:

\textless html\textgreater{}

\textless script\textgreater{}

var i=1;

function spin() \{

if (i\textgreater=16) \{i=1; \}

else \{

i++;

\}

m1.src="wheel\_"+i+".gif";

spinning = setTimeout("spin()", 100);

\}

var r=120; a=0;

function cc() \{

b1.style.left=197+r*Math.sin(a);

b1.style.top=197+r*Math.cos(a);

if (a \textgreater= Math.PI *2) \{ a=0; \}

else \{ a += Math.PI/180; \}

rotating=setTimeout("cc()", 2);

\}

\textless/script\textgreater{}

\textless body\textgreater{}

\textless img id=m1 src="wheel\_1.gif"\textgreater{}

\textless img id=b1 src="white\_ball.gif" style="position:absolute;
z-index:1; left:20; top:20"\textgreater{}

\textless div style="position:absolute; left:50; top:400;"\textgreater{}

\textless button
onClick="cc();spin()"\textgreater Rotate\textless/button\textgreater{}

\textless button onClick="clearTimeout(rotating);
clearTimeout(spinning)"\textgreater Stop\textless/button\textgreater{}

\textless/div\textgreater{}

\textless/body\textgreater{}

\textless/html\textgreater{}

3. Test the program. A sample output looks:

\textbf{Learning Activity \#4: Running Bugs}

Note: This game is meant to be simple and easy for the sake of
demonstrating programming concepts. Please do not hesitate to enhance
the appearance or functions of this game.

Game Programming -- Penn Wu

129

\protect\hypertarget{index_split_008.htmlux5cux23p130}{}{}

1. Change to the C:\textbackslash games directory.

2. Use Notepad to create a new file named
\textbf{C:\textbackslash games\textbackslash lab7\_4.htm} with the
following contents:

\textless html\textgreater{}

\textless style\textgreater{}

.ants \{position:absolute\}

\textless/style\textgreater{}

\textless script\textgreater{}

function init() \{

for (i=1; i\textless=8; i++) \{

x=Math.floor(Math.random()*document.body.clientWidth);

y=Math.floor(Math.random()*document.body.clientHeight);

codes = "\textless img src=ant"+i+".gif id=a"+i+" class=ants"

codes += "
style=\textquotesingle left:"+x+";top:"+y+"\textquotesingle\textgreater";
area1.innerHTML += codes;

\}

\}

function move() \{

a1.style.pixelLeft -= 5;

a1.style.pixelTop += 5;

a2.style.pixelTop += 5;

a3.style.pixelLeft += 5;

a3.style.pixelTop += 5;

a4.style.pixelLeft -= 5;

a5.style.pixelLeft += 5;

a6.style.pixelLeft -= 5;

a6.style.pixelTop -= 5;

a7.style.pixelTop -= 5;

a8.style.pixelLeft += 5;

a8.style.pixelTop -= 5;

if (a1.style.pixelLeft \textless= 0) \{a1.style.pixelLeft =
document.body.clientWidth; \}

if (a1.style.pixelTop \textgreater= document.body.clientHeight)
\{a1.style.pixelTop = 0; \}

if (a2.style.pixelTop \textgreater= document.body.clientHeight)
\{a2.style.pixelTop = 0; \}

if (a3.style.pixelLeft \textgreater= document.body.clientWidth)
\{a3.style.pixelLeft = 0; \}

if (a3.style.pixelTop \textgreater= document.body.clientHeight)
\{a3.style.pixelTop = 0; \}

if (a4.style.pixelLeft \textless= 0) \{a4.style.pixelLeft =
document.body.clientWidth; \}

if (a5.style.pixelLeft \textgreater= document.body.clientWidth)
\{a5.style.pixelLeft = 0; \}

if (a6.style.pixelLeft \textless= 0) \{a6.style.pixelLeft =
document.body.clientWidth; \}

if (a6.style.pixelTop \textless= 0) \{a6.style.pixelTop =
document.body.clientHeight; \}

if (a7.style.pixelLeft \textless= 0) \{a7.style.pixelLeft =
document.body.clientWidth; \}

if (a8.style.pixelLeft \textgreater= document.body.clientWidth)
\{a8.style.pixelLeft = 0; \}

if (a8.style.pixelTop \textless= 0) \{a8.style.pixelTop =
document.body.clientHeight; \}

Game Programming -- Penn Wu

130

\protect\hypertarget{index_split_008.htmlux5cux23p131}{}{}\includegraphics{index-131_1.png}

setTimeout("move()", 20);

\}

\textless/script\textgreater{}

\textless body onLoad="init();move()"\textgreater{}

\textless script id=area2\textgreater\textless/script\textgreater{}

\textless span id=area1\textgreater\textless/span\textgreater{}

\textless/body\textgreater{}

\textless/html\textgreater{}

3. Test the program. A sample output looks:

\textbf{Learning Activity \#5: Rush Hour Game}

Note: This game is meant to be simple and easy for the sake of
demonstrating programming concepts. Please do not hesitate to enhance
the appearance or functions of this game.

1. Change to the C:\textbackslash games directory.

2. Use Notepad to create a new file named
\textbf{C:\textbackslash games\textbackslash lab7\_5.htm} with the
following contents:

\textless html\textgreater{}

\textless style\textgreater{}

.dragr \{position:absolute\}

.dragc \{position:absolute\}

\textless/style\textgreater{}

\textless script\textgreater{}

var dragapproved=false

var z,x,y

function move()\{

if (event.button==1 \&\& dragapproved) \{

var clsName = event.srcElement.className;

switch (clsName) \{

case "dragr":

z.style.pixelLeft=temp1+event.clientX-x;

break;

case "dragc":

z.style.pixelTop=temp2+event.clientY-y

break;

\}

return false

\}

\}

function drags()\{

var clsName = event.srcElement.className;

Game Programming -- Penn Wu

131

\protect\hypertarget{index_split_008.htmlux5cux23p132}{}{}\includegraphics{index-132_1.png}

if (!document.all) \{ return \}

if ((clsName=="dragr") \textbar\textbar{} (clsName=="dragc")) \{

dragapproved=true

z=event.srcElement

temp1=z.style.pixelLeft

temp2=z.style.pixelTop

x=event.clientX

y=event.clientY

document.onmousemove=move

\}

\}

\textless!-\/- generating table -\/-\textgreater{}

function cc() \{

codes = "\textless table cellspacing=2 border=1";

codes += " style=\textquotesingle left:0; top:0;
position:absolute\textquotesingle\textgreater"; for (h=1; h\textless=8;
h++) \{

codes += "\textless tr height=36\textgreater";

for (i=1; i\textless=8; i++) \{

codes += "\textless td
width=36\textgreater\&nbsp;\textless/td\textgreater";

\}

codes += "\textless/tr\textgreater";

\}

codes += "\textless/table\textgreater";

area1.innerHTML = codes;

\}

document.onmousemove=move

document.onmousedown=drags

document.onmouseup=new Function("dragapproved=false")

\textless/script\textgreater{}

\textless body onLoad=cc()\textgreater{}

\textless div id=area1\textgreater\textless/div\textgreater{}

\textless img id="rc1" class=dragr src="rcar01.gif" style="left:4;
top:4"\textgreater{}

\textless img id="rc2" class=dragr src="rcar02.gif" style="left:172;
top:156"\textgreater{}

\textless img id="rc3" class=dragr src="rcar03.gif" style="left:46;
top:232"\textgreater{}

\textless img id="cc1" class=dragc src="ccar01.gif" style="left:46;
top:118"\textgreater{}

\textless img id="cc2" class=dragc src="ccar02.gif" style="left:256;
top:42"\textgreater{}

\textless img id="cc3" class=dragc src="ccar03.gif" style="left:130;
top:118"\textgreater{}

\textless/body\textgreater{}

\textless/html\textgreater{}

3. Test the program. A sample output looks:

\textbf{Submittal}

Upon completing all the learning activities,

Game Programming -- Penn Wu

132

\protect\hypertarget{index_split_008.htmlux5cux23p133}{}{}1. Upload all
files you created in this lab to your remote web server.

2. Log in to to Blackboard, launch Assignment 07, and then scroll down
to question 11.

3. Copy and paste the URLs to the textbox. For example,

•

http://www.geocities.com/cis261/lab7\_1.htm

•

http://www.geocities.com/cis261/lab7\_2.htm

•

http://www.geocities.com/cis261/lab7\_3.htm

•

http://www.geocities.com/cis261/lab7\_4.htm

•

http://www.geocities.com/cis261/lab7\_5.htm

No credit is given to broken link(s).

Game Programming -- Penn Wu

133

\protect\hypertarget{index_split_008.htmlux5cux23p134}{}{}

Lecture \#8

Collision detection and response

\protect\hypertarget{index_split_009.html}{}{}

\hypertarget{index_split_009.htmlux5cux23calibre_pb_8}{%
\subsection{Introduction}\label{index_split_009.htmlux5cux23calibre_pb_8}}

Collision detection and collision response is the act of detecting
whether or not a collision has happened, and if so, responding to the
collision. In physical simulations, video games and computational
geometry, collision detection includes algorithms from checking for
intersection between two given solids, to calculating trajectories,
impact times and impact points in a physical simulation. There are
several methods of collision detection, and the right one for the job
depends on the shape of the object.

DHTML allows developers to create client-side applications of a type
that was once possible only with server-side programming. By combining
JavaScript with Cascading Style Sheets, it\textquotesingle s possible to
perform effective collision detection.

The following JavaScript example uses two screen elements (which could
be images, layers, divisions, or anything else in the document object
hierarchy) called \textbf{myObject1} and \textbf{myObject2}.

The variables x and y hold either a true or a false value, depending on
whether the two objects overlap on either the horizontal plane or the
vertical plane. If both x and y evaluate to true, then the two objects
have collided.

x = ((myObject1.style.posLeft \textgreater= myObject2.style.posLeft)
\&\& (myObject.style.posLeft \textless= myObject2.style.posLeft));

y = ((myObject.style.posTop \textgreater= myObject2.style.posTop) \&\&
(myObject1.style.posTop \textless= myObject2.style.posTop));

if (x \&\& y)

\{

// Collision has occurred.

\}

The \textbf{move\_crab()} function, for example, use the \emph{if}
statement with programming logic similar to the above one.

function move\_crabs() \{

var redSpeed = Math.floor(Math.random()*20);

.............

red.style.pixelLeft += redSpeed;

.............

\textbf{if (red.style.pixelLeft + 32 \textgreater=
area2.style.pixelLeft)} \{

clearTimeout(s1);alert("Winner is the red crab!");\}

.............

\}

The following line is the spine of the collision detection, it determine
whether or not the \textbf{red} object and the \textbf{area2} object
collide.

One-

Handling collision detection in 2D graphics is based on a simple
idea-\/-determine whether or not dimensional

two or more graphics collide with one another. In a Web game, for
example, graphics are in a collision

shape of rectangle, so you can simply try to detect whether two or more
rectangular areas are in detection

any way touching or overlapping each other.

A rectangular image has four vertexes. In the following graphic, 1, 2,
3, and 4 represent each of Game Programming -- Penn Wu

134

\protect\hypertarget{index_split_009.htmlux5cux23p135}{}{}\includegraphics{index-135_1.png}

the vertexes:

Assuming this graphic is given an object ID \textbf{obj1}, the (x, y)
coordinates of these 4 vertexes can be represented by:

•

1: (obj1.style.pixelLeft, obj1.style.pixelTop)

•

2: (obj1.style.pixelLeft + obj1.style.width, obj1.style.pixelTop)

•

3: (obj1.style.pixelLeft + obj1.style.width, obj1.style.pixelTop +
obj1.style.height)

•

4: (obj1.style.pixelLeft, obj1.style.pixelTop + obj1.style.height) A
simple implementation of this method is as follows:

\textless html\textgreater\textless style\textgreater{}

.drag \{position: absolute\}

\textless/style\textgreater{}

\textless script\textgreater{}

function obj1Move() \{

obj1.style.left = obj1.style.pixelLeft + 5;

s1 = setTimeout("obj1Move()", 50);

\}

function obj2Move() \{

obj2.style.left = obj2.style.pixelLeft + 5;

\textbf{if (obj2.style.pixelLeft \textgreater= obj1.style.pixelLeft) \{}

\textbf{clearTimeout(s1);}

\textbf{clearTimeout(s2);}

\textbf{p1.innerText = "Game Over!";}

\textbf{\}}

else \{

s2 = setTimeout("obj2Move()", 20);

\}

\}

\textless/script\textgreater{}

\textless body onLoad="obj1Move();obj2Move();"\textgreater{}

\textless img id="obj1" class="drag" src="pac1.gif"

style="top:10;left:200"\textgreater{}

\textless img id="obj2" class="drag"
src=\textquotesingle gst1.gif\textquotesingle{}
style="top:10;left:10"\textgreater{}

\textless p id=p1 class="drag" style="top:150;
left:140;"\textgreater\textless/p\textgreater{}

\textless/body\textgreater\textless/html\textgreater{}

There are two image files: \textbf{pac1.gif} and \textbf{gst1.gif}.
Their IDs are \textbf{obj1} and \textbf{obj2} respectively. In the
obj1Move() function, the \textbf{obj1} is set to increment its value of
\emph{x}-coordinate by 5px every 50

milliseconds, as shown below:

obj1.style.left = obj1.style.pixelLeft + 5;

Similarly, the \textbf{obj2} is set to increment its value of
\emph{x}-coordinate by 5px every 20 milliseconds, as shown below:

obj2.style.left = obj2.style.pixelLeft + 5;

Game Programming -- Penn Wu

135

\protect\hypertarget{index_split_009.htmlux5cux23p136}{}{}\includegraphics{index-136_1.png}

\includegraphics{index-136_2.png}

\includegraphics{index-136_3.png}

\includegraphics{index-136_4.png}

\includegraphics{index-136_5.png}

\includegraphics{index-136_6.png}

\includegraphics{index-136_7.png}

\includegraphics{index-136_8.png}

Since obj2 moves faster then obj1, obj2 will soon catch up and overlaps
with obj1. You can use the following code block to detect such
collision, and the logic is ``if the \emph{x}-coordinate of obj2 is
greater than or equal to the \emph{x}-coordinate of obj1'':

if (obj2.style.pixelLeft \textgreater= obj1.style.pixelLeft)

Two-

In a given plant, rectangles are two-dimensional: vertical and
horizontal. In many games, dimensional

collisions can happen in any direction, vertically, horizontally, or
both. The following figure collision

illustrates four of the possible directions these two image files
overlap with each other.

detection

A rectangular area has four vertexes, but you can use the diagonal to
define the perimeter of the rectangle. This diagonal is drawn by
connecting the top-left-most point and bottom-right-most point. The
following figure uses (x1, y1) and (x2, y2) to indicate the four
coordinate sets of each images (obj1 and obj2).

obj1

obj2

When the value of x1 increases, the object moves from left to right, and
vica versa. When the value of y1 increases, the object moves from top to
bottom, and vica versa. Therefore, the set of (x1, y1) take care of the
left and top border.

Similarly, the change in values of (x2, y2) handles the right and bottom
border.

Once you understand the logic, you can write codes to detect the
collision of these two images from any direction. First, you can try to
define four variables to represent (x1, y1) and (x2, y2) of both image
file:

Game Programming -- Penn Wu

136

\protect\hypertarget{index_split_009.htmlux5cux23p137}{}{}\includegraphics{index-137_1.png}

\includegraphics{index-137_2.png}

The values of (x1, y1) of obj1 are assigned to (left1, top1) by the
following statements: left1 = obj1.style.pixelLeft;

top1 = obj1.style.pixelTop;

The value of (x2, y2) can be easily determined, they are:

•

x2 = x1 + width

•

y2 = y1 + height

So you can translate them into:

right1 = obj1.style.pixelLeft + obj1.style.width;

bottom1 = obj1.style.pixelTop + obj1.style.height;

The following is an example of how you handle the coding in DHTML.

\textless script\textgreater{}

var left1, left2;

var right1, right2;

var top1, top2;

var bottom1, bottom2;

left1 = obj1.style.pixelLeft;

left2 = obj2.style.pixelLeft;

right1 = obj1.style.pixelLeft + obj1.style.width;

right2 = obj2.style.pixelLeft + obj2.style.width;

top1 = obj1.style.pixelTop;

top2 = obj2.style.pixelTop;

bottom1 = obj1.style.pixelTop + obj1.style.height;

bottom2 = obj2.style.pixelTop + obj2.style.height;

.................

.................

function obj2Move() \{

obj2.style.left = obj2.style.pixelLeft + 5;

if \textbf{(((bottom1 \textless{} top2) \textbar\textbar{} (top1
\textgreater{} bottom2) \textbar\textbar{} (right1 \textless{} left2)}

\textbf{\textbar\textbar{} (left1 \textgreater{} right2))} \{

clearTimeout(s1);

clearTimeout(s2);

p1.innerText = "Game Over!";

\}

else \{

s2 = setTimeout("obj2Move()", 20);

\}

\}

Game Programming -- Penn Wu

137

\protect\hypertarget{index_split_009.htmlux5cux23p138}{}{}\includegraphics{index-138_1.png}

\includegraphics{index-138_2.png}

\includegraphics{index-138_3.png}

\textless/script\textgreater{}

In the Jet Fighter game, the \textbf{check()} function handles the
collision detection. The \textbf{ap01} object is the enemy's airplane,
while \textbf{b01} is the bullet.

The check() function uses a simple logic. First, the b01's pixelLeft
must be between ap01.style.pixelLeft and ap01.style.pixelLeft + 70.
Second, the b01's pixelTop must be less than or equal to
ap01.style.pixelTop. Notice that the dimension of \textbf{ap01} is 70 ×
70, meaning both the width and height are 70px.

function check() \{

if ((b01.style.pixelLeft \textgreater= ap01.style.pixelLeft) \&\&
(b01.style.pixelLeft \textless= ap01.style.pixelLeft + 70) \&\&
(b01.style.pixelTop \textless= ap01.style.pixelTop +70))

\{

ap01.src="explode.gif";

setTimeout("ap01.style.display=\textquotesingle none\textquotesingle",1000);

\}

\}

The following lines check if the b01's pixelLeft is between
ap01.style.pixelLeft and ap01.style.pixelLeft + 70:

(b01.style.pixelLeft \textgreater= ap01.style.pixelLeft) \&\&

(b01.style.pixelLeft \textless= ap01.style.pixelLeft + 70)

The following line check if the less than or equal to
ap01.style.pixelTop: (b01.style.pixelTop \textless= ap01.style.pixelTop
+70)

When all the three conditions are true, the following line changes the
image of ap01 to another one that is an animated flame.

ap01.src="explode.gif"; \textbf{}

Over-detecting

The method usually works fairly well to rectangles. But it has a problem
of ``over-detecting''

collisions. For example, the Pacman graphic is not rectangular, it is
roundish. When it is put in a rectangular image file, as shown below,
there will be a distance between each vertex and the graphics. The blue
area in the following figure represents the blank space.

Game Programming -- Penn Wu

138

\protect\hypertarget{index_split_009.htmlux5cux23p139}{}{}\includegraphics{index-139_1.png}

Because the roundish shape of the graphics, the vertexes of each image
file contains empty space.

But the bounding rectangles of each image are clearly intersecting, so a
traditional collision detecting algorithm as the above code uses will
have deviations.

One acceptable solution to this over-detecting problem is to define a
somewhat smaller rectangle than the full extents of the image, and use
this smaller rectangle for the collision detection.

A good idea, in this case, is to use 80\% of the full bounding box of
the image. To do so, you need to remove 20\% of the space from the
rectangular image. You can define six new variables: \textbf{obj1Width,
obj1Height, obj1ColWidth, obj1ColHeight, obj\_X\_offset,} and
\textbf{obj\_Y\_offset.}

obj1Width = obj1.style.pixelWidth;

obj1Height = obj1.style.pixelHeight;

obj1ColWidth = objWidth * 0.8;

obj1ColHeight = objHeight * 0.8;

obj1\_X\_offset = (objWidth - objColWidth) / 2;

obj1\_Y\_offset = (objHeight - objColHeight) / 2;

During initialization, wherever the above code retrieves the width and
height of \textbf{obj1}, and uses them to calculate the
\textbf{obj1ColWidths} and \textbf{obj1ColHeights} to be 20\% smaller.
It also defines the offset fields to describe where to set the bounding
box relative to the object's coordinates.

Advanced

In a previous lecture, you learned the techniques of handling drag and
drop. Similar technique can collision

be used to detect collisions. For example, when the monsters are bumping
into PacMan, you can detection

check whether the floating DIV element is touching or overlapping a
valid Drag Target.

assuming,

\textless DIV class="drag"\textgreater{}

\textless IMG SRC="mind.gif" WIDTH=115 HEIGHT=45 BORDER=0\textgreater{}

\textless/DIV\textgreater{}

As in the previous technique, there is nothing unusual about the DIV
element, except that it has been given a class name. Of course, you
would set the WIDTH and HEIGHT attributes of the IMG to fit your images.
It is important that all IMGs have the same WIDTH and HEIGHT.

Repeat this declaration for each DIV you want to make draggable. Finally
add an absolutely positioned DIV to your page like this:

\textless DIV ID="floater"
style="visibility:hidden;position:absolute"\textgreater{}

\textless IMG SRC="trans.gif" WIDTH=115 HEIGHT=45 BORDER=0\textgreater{}

\textless/DIV\textgreater{}

This floating DIV is the only moving element in the page, the rest of
the DIVs are not absolutely positioned, and therefore cannot be moved.
Also notice that the IMG inside the DIV is given a SRC, even though it
is initially invisible. This is because of a bug in IE4 that prevents
dynamically setting the SRC of an IMG if the original SRC is not
specified.

The reason we need this algorithm is that since the mouse is over the
floating DIV, the onmouseup event\textquotesingle s srcElement will not
be accurate. Also, since the surface area of the floating DIV is quite
large, it is quite possible that the DIV may overlap two or more
potential targets.

Game Programming -- Penn Wu

139

\protect\hypertarget{index_split_009.htmlux5cux23p140}{}{}Only by
comparing the area of overlap of each target, can we accurately
determine which the target the user intended was. Happily, the Collision
Detection Algorithm can do this.

The algorithm goes something like this:

Given 2 rectangular objects of the same size,

the Horizontal and Vertical overlap between

them is given by :

hOverlap = W - \textbar x2-x1\textbar{}

vOverlap = H - \textbar y2-y1\textbar{}

where

W=width of the objects

H=Height of the objects

x1,y1=Top Left co-ordinates of Obj 1

x2,y2=Top Left co-ordinates of Obj 2

if both hOverlap and vOverlap are positive

their product gives the area of overlap of

the 2 objects.

if either or both are negative, the objects

do not overlap (have not collided).

Note: This algorithm was devised by Stanley Rajasingh.

Now that we have the algorithm, coding the \textbf{onMouseup} handler is
a trivial exercise. Looking at it we see -

function DragEnd()\{

if(!DragEl) return;

X1=floater.style.pixelLeft;

Y1=floater.style.pixelTop;

MaxArea=0;

TargetElem=null;

Targets=document.all;

for(i=0;i\textless Targets.length;i++)

if(Targets{[}i{]}.className=="dnd")\{

X2=getRealPos(Targets{[}i{]}.children{[}0{]},"Left");

Y2=getRealPos(Targets{[}i{]}.children{[}0{]},"Top");

hOverlap=floater.offsetWidth-Math.abs(X2-X1);

vOverlap=floater.offsetHeight-Math.abs(Y2-Y1);

areaOverlap=((hOverlap\textgreater0)\&\&(vOverlap\textgreater0))?hOverlap*vOverlap:0;
if(areaOverlap\textgreater MaxArea)\{

MaxArea=areaOverlap;

TargetElem=Targets{[}i{]};

\}

\}

floater.style.visibility=\textquotesingle hidden\textquotesingle;

if(TargetElem)\{

//Swap Images

DragEl.src=TargetElem.children{[}0{]}.src;

TargetElem.children{[}0{]}.src=floater.children{[}0{]}.src;

\} else \{

//Restore Original Image

DragEl.src=floater.children{[}0{]}.src;

return;

\}

DragEl=null;

Game Programming -- Penn Wu

140

\protect\hypertarget{index_split_009.htmlux5cux23p141}{}{}\}

The magic \emph{for} loop cycles through each element in the above code
is that it checks to see if it is a valid Drag Target
(className=="dnd"). If it is, the code applies the Collision Detection
Algorithm to obtain the overlap area. The element which has the largest
overlap with the floating DIV becomes the \textbf{TargetElem}.

Once the target has been obtained, the function swaps the source and
target images, and then hides the floating DIV, creating the illusion
that the \textbf{Drag Source} and \textbf{Drag Target} have exchanged
places. Finally it resets the Drag Source pointer \textbf{(DragEl)} to
null, indicating that the dragging has ended.

Be aware that since this algorithm takes over the \textbf{onMouseOver},
\textbf{onMouseUp}, and \textbf{onMouseMove} handlers of the document,
any other element on the page that uses these events, such as a DHTML
menu, will not function. In order to prevent this you might want to
overload these handlers rather than taking over them entirely.

Collision Math

Given two objects \emph{\textbf{X}} and \emph{\textbf{Y}} with position
vectors \emph{rx} and \emph{ry}, you want to be able to test whether or
not the objects are collided as a function of \emph{rx} - \emph{ry}.

The solution is that objects \emph{\textbf{X}} and \emph{\textbf{Y}}
overlap if the set

\{( \emph{x}, \emph{y}) \textbar{} \emph{x} ∈ \emph{X} ∧ \emph{y} ∈
\emph{Y} ∧ \emph{x} + \emph{rx} = \emph{y} + \emph{ry} \}

is non-empty. This is equivalent to the condition of

\emph{}

\emph{rx} - \emph{ry} ∈ \emph{Z}

where \textbf{Z} is the set

\{ \emph{y} -- \emph{x} \textbar{} \emph{x} ∈ \emph{X} ∧ \emph{y} ∈
\emph{Y}\}

If \textbf{X} and \textbf{Y} are arbitrary shapes, then the shape
\textbf{Z} should be pre-computed. However there is no easy way to
efficiently determine membership of a point in an arbitrary shape in
JavaScript (unless perhaps you separately encode the shape \textbf{Z} as
a suitable array of arrays).

One can avoid this problem by making all sprites a rectangular shape, in
which case if \textbf{X} is a rectangle with width \emph{w} x and height
\emph{h} x and \textbf{Y} is a rectangle with width \emph{w} y and
height \emph{h} y, then \textbf{Z}

will be rectangle with width \emph{w} x + \emph{w} y and height \emph{h}
x + \emph{h} y. Which explains the following code: function collided
(element1, element2) \{

var x1 = element1.offsetLeft;

var x2 = element2.offsetLeft - element1.offsetWidth;

var x3 = element2.offsetLeft + element2.offsetWidth;

if (x2 \textless= x1 \&\& x1 \textless= x3) \{

var y1 = element1.offsetTop;

var y2 = element2.offsetTop - element1.offsetHeight;

var y3 = element2.offsetTop + element2.offsetHeight;

return y2 \textless= y1 \&\& y1 \textless= y3;

\}

else \{

return false;

\}

\}

In the above code, \textbf{element1} is \textbf{X}, \textbf{element2} is
\textbf{Y}, and the variables \textbf{x2}, \textbf{x3}, \textbf{y2} and
\textbf{y3} represent the boundaries of \textbf{element2} as if it had a
size of the hypothetical \textbf{Z} object.

The other thing to watch with collision detection is that if you are
checking it once per game loop, Game Programming -- Penn Wu

141

\protect\hypertarget{index_split_009.htmlux5cux23p142}{}{}then you want
to make sure that the objects aren\textquotesingle t moving towards each
other so fast that they can skip past each other, i.e. their total
distance travelled relative to each other per game loop should not
exceed the size of the vector corresponding to their combined width and
height.

Review

1. Handling collision detection in 2D graphics is based on determining
\_\_.

Questions

A. if two objects\textquotesingle{} shapes are symmetrical.

B. whether or not two or more objects collide with one another.

C. if two objects\textquotesingle{} motions are synchronous.

D. whether or not two or more objects resemble one another.

2. Which can express the following sentence:

"If the x-coordinate of obj2 is greater than or equal to the
x-coordinate of obj1"

A. if ((obj2.style.pixelLeft \textgreater{} z.style.pixelLeft) \&\&
(obj2.style.pixelLeft = z.style.pixelLeft)) B. if (z.style.pixelLeft
\textgreater= obj2.style.pixelLeft)

C. if (obj2.style.pixelLeft \textgreater= z.style.pixelLeft)

D. if ((z.style.pixelLeft \textgreater{} obj2.style.pixelLeft) \&\&
(z.style.pixelLeft = obj2.style.pixelLeft)) 3. Given the following
rectangular object with an ID "z", which can represent the (x, y)
coordinate of the vertex 3 using DHTML?

1-\/-\/-2

\textbar{} \textbar{}

\textbar{} \textbar{}

4-\/-\/-3

A. (z.style.pixelLeft + object.style.width, z.style.pixelTop)

B. (z.style.pixelLeft, z.style.pixelTop)

C. (z.style.pixelLeft, z.style.pixelTop + object.style.height)

D. (z.style.pixelLeft + object.style.width, z.style.pixelTop +
object.style.height) 4. Given the following rectangular object with an
ID "z", which can represent the (x, y) coordinate of the vertex 2 using
DHTML?

1-\/-\/-2

\textbar{} \textbar{}

\textbar{} \textbar{}

4-\/-\/-3

A. (z.style.pixelLeft + object.style.width, z.style.pixelTop)

B. (z.style.pixelLeft, z.style.pixelTop)

C. (z.style.pixelLeft, z.style.pixelTop + object.style.height)

D. (z.style.pixelLeft + object.style.width, z.style.pixelTop +
object.style.height) 5. Given an object that has dimension of 60 X 60
and the ID "b", which can represent the (x, y) coordinate of the vertex
4 using DHTML?

1-\/-\/-2

\textbar{} \textbar{}

\textbar{} \textbar{}

4-\/-\/-3

A. (b.style.pixelLeft + 60, b.style.pixelTop + 60)

B. (b.style.pixelLeft, b.style.pixelTop + 60)

C. (b.style.pixelLeft + 60, b.style.pixelTop)

D. (b.style.pixelLeft, b.style.pixelTop)

Game Programming -- Penn Wu

142

\protect\hypertarget{index_split_009.htmlux5cux23p143}{}{}

6. Which can determine if the less than or equal to ap01.style.pixelTop?

A. if (b01.style.pixelTop \textless= ap01.style.pixelTop)

B. if (ap01.style.pixelTop \textless= b01.style.pixelTop)

C. if ((ap01.style.pixelTop \textless{} b01.style.pixelTop) \&\&
(ap01.style.pixelTop \textless{} b01.style.pixelTop) D. if
((b01.style.pixelTop \textless{} ap01.style.pixelTop) \&\&
(b01.style.pixelTop \textless{} ap01.style.pixelTop)) 7. Which can
determine if the b01\textquotesingle s pixelLeft is between
ap01.style.pixelLeft and ap01.style.pixelLeft + 70?

A. if (b01.style.pixelTop \textless= ap01.style.pixelTop + 70)

B. if ((b01.style.pixelLeft \textgreater= ap01.style.pixelLeft) \&\&
(b01.style.pixelLeft \textless=

ap01.style.pixelLeft + 70))

C. if ((b01.style.pixelLeft \textless= ap01.style.pixelLeft) \&\&
(b01.style.pixelLeft \textgreater=

ap01.style.pixelLeft + 70))

D. if ((b01.style.pixelLeft \textgreater{} ap01.style.pixelLeft) \&\&
(b01.style.pixelLeft \textless{} ap01.style.pixelLeft

+ 70) \&\& (b01.style.pixelLeft = ap01.style.pixelLeft))

8. Which can force the "z" object to change its value of the "src"
property to "apple.gif"?

A. z.src = "apple.gif"

B. z.style.src = "apple.gif"

C. z.style.Src = "apple.gif"

D. z = "apple.gif"

9. The term "over-detecting" means \_\_.

A. a rectangular image file with over-sized graphics that cause
unexpected collision.

B. a rectangular image file with over-programmed code to make the
collision detection function over sensitive.

C. a rectangular image file that contains polygons that makes the
collision detection function over sensitive.

D. a rectangular image file that contains graphics that does not fully
fill the rectangular area, such that the unfilled area become blank
space when collision happens.

10. What statement is correct about "collision detection and response"?

A. It is the act of detecting whether or not a collision has happened.

B. It uses algorithms from checking for intersection between two given
solids.

C. Collision detection is responsible for determining whether or not a
collision happens, while collision response handles the counter-action
when the collision is detected.

D. All of the above.

Game Programming -- Penn Wu

143

\protect\hypertarget{index_split_009.htmlux5cux23p144}{}{}

Lab \#8

Collision detection and response

\textbf{Preparation \#1:}

1. Create a new directory named \textbf{C:\textbackslash games}.

2. Use Internt Explorer to go to
\textbf{http://business.cypresscollege.edu/\textasciitilde pwu/cis261/download.htm}
to download lab8.zip (a zipped) file. Extract the files to
C:\textbackslash games directory.

\textbf{Learning Activity \#1: Simple Pacman I}

1. Change to the C:\textbackslash games directory.

2. Use Notepad to create a new file named
\textbf{C:\textbackslash games\textbackslash lab8\_1.htm} with the
following contents:

\textless html\textgreater{}

\textless style\textgreater{}

.drag \{position: absolute\}

\textless/style\textgreater{}

\textless script\textgreater{}

function obj1Move() \{

obj1.style.left =
eval(obj1.style.left.replace(\textquotesingle px\textquotesingle,\textquotesingle\textquotesingle))
+ 5;

s1 = setTimeout("obj1Move()", 50);

\}

function obj2Move() \{

obj2.style.left =
eval(obj2.style.left.replace(\textquotesingle px\textquotesingle,\textquotesingle\textquotesingle))
+ 5;

if
(eval(obj2.style.left.replace(\textquotesingle px\textquotesingle,\textquotesingle\textquotesingle))
\textgreater=

eval(obj1.style.left.replace(\textquotesingle px\textquotesingle,\textquotesingle\textquotesingle)))
\{

clearTimeout(s1);

clearTimeout(s2);

p1.innerText = "Game Over!";

\}

else \{

s2 = setTimeout("obj2Move()", 20);

\}

\}

\textless/script\textgreater{}

\textless body onLoad="obj1Move();obj2Move();"\textgreater{}

\textless img id="obj1" class="drag" src="pac1.gif"
style="top:10;left:200"\textgreater{}

\textless img id="obj2" class="drag"
src=\textquotesingle gst1.gif\textquotesingle{}
style="top:10;left:10"\textgreater{}

\textless p id=p1 class="drag" style="top:150;
left:140;"\textgreater\textless/p\textgreater{}

\textless/body\textgreater{}

\textless/html\textgreater{}

3. Test the program. Obj2 chases Obj1, and when they overlap each other,
the game is over. Click Refresh to restart the game. A sample output
looks:

Game Programming -- Penn Wu

144

\protect\hypertarget{index_split_009.htmlux5cux23p145}{}{}\includegraphics{index-145_1.png}

\includegraphics{index-145_2.png}

to

\textbf{Learning Activity \#2: Simple Pacman II}

Note: This game is meant to be simple and easy for the sake of
demonstrating programming concepts. Please do not hesitate to enhance
the appearance or functions of this game.

1. Change to the C:\textbackslash games directory.

2. Use Notepad to create a new file named
\textbf{C:\textbackslash games\textbackslash lab8\_2.htm} with the
following contents:

\textless html\textgreater{}

\textless style\textgreater{}

.drag \{position: absolute\}

\textless/style\textgreater{}

\textless script
src="dragndrop.js"\textgreater\textless/script\textgreater{}

\textless script\textgreater{}

var left1, left2;

var top1, top2;

function obj1Move2Right() \{

left1 =
eval(obj1.style.left.replace(\textquotesingle px\textquotesingle,\textquotesingle\textquotesingle));

top1 =
eval(obj1.style.top.replace(\textquotesingle px\textquotesingle,\textquotesingle\textquotesingle));

if (left1 \textgreater= (document.body.clientWidth - 100)) \{

clearTimeout(ToRight);

obj1Move2Left();

\}

else \{

obj1.src="pac1.gif"

obj1.style.left = left1 + 2;

eat();

ToRight = setTimeout("obj1Move2Right()", 20);

\}

\}

function obj1Move2Left() \{

left1 =
eval(obj1.style.left.replace(\textquotesingle px\textquotesingle,\textquotesingle\textquotesingle));

top1 =
eval(obj1.style.top.replace(\textquotesingle px\textquotesingle,\textquotesingle\textquotesingle));

if (left1 \textless= 0) \{

clearTimeout(ToLeft);

obj1Move2Right();

\}

else \{

obj1.src="pac2.gif"

obj1.style.left = left1 - 2;

ToLeft = setTimeout("obj1Move2Left()", 20);

\}

\}

Game Programming -- Penn Wu

145

\protect\hypertarget{index_split_009.htmlux5cux23p146}{}{}\includegraphics{index-146_1.png}

function dots() \{

scripts ="function eat() \{";

k = (document.body.clientWidth - 100) / 20;

for (i=1; i\textless=k; i++) \{

codes = "\textless img id=d"+i+ "
src=\textquotesingle dot.gif\textquotesingle{}
class=\textquotesingle drag\textquotesingle"; codes += "
style=\textquotesingle z-index:1;top:30; left:"+
i*20+"\textquotesingle\textgreater"; area1.innerHTML += codes;

\}

\}

\textless/script\textgreater{}

\textless script src="eat.js"\textgreater\textless/script\textgreater{}

\textless body onLoad="dots();obj1Move2Right();"\textgreater{}

\textless img id="obj1" class="drag" src="pac1.gif"
style="top:10;left:0; z-index:2"\textgreater{}

\textless span id=area1\textgreater\textless/span\textgreater{}

\textless/body\textgreater{}

\textless/html\textgreater{}

3. Test the program. A sample output looks:

\textbf{Learning Activity \#3: Racing Crabs}

1. Change to the C:\textbackslash games directory.

2. Use Notepad to create a new file named
\textbf{C:\textbackslash games\textbackslash lab8\_3.htm} with the
following contents:

\textless html\textgreater{}

\textless style\textgreater{}

div, img \{ position: absolute;left:10px \}

\textless/style\textgreater{}

\textless script\textgreater{}

function init() \{

area1.style.width = document.body.clientWidth - 80;

area1.style.borderRight = "solid 2 white";

area2.style.pixelLeft = document.body.clientWidth - 70;

\}

function move\_crabs() \{

var redSpeed = Math.floor(Math.random()*20);

var greenSpeed = Math.floor(Math.random()*20);

var blueSpeed = Math.floor(Math.random()*20);

var whiteSpeed = Math.floor(Math.random()*20);

red.style.pixelLeft += redSpeed;

green.style.pixelLeft += greenSpeed;

blue.style.pixelLeft += blueSpeed;

Game Programming -- Penn Wu

146

\protect\hypertarget{index_split_009.htmlux5cux23p147}{}{}\includegraphics{index-147_1.png}

white.style.pixelLeft += whiteSpeed;

if (red.style.pixelLeft + 32 \textgreater= area2.style.pixelLeft) \{

clearTimeout(s1);alert("Winner is the red crab!");\}

else if (green.style.pixelLeft + 32 \textgreater= area2.style.pixelLeft)
\{

clearTimeout(s1);alert("Winner is the green crab!");\}

else if (blue.style.pixelLeft + 32 \textgreater= area2.style.pixelLeft)
\{

clearTimeout(s1);alert("Winner is the blue crab!");\}

else if (white.style.pixelLeft + 32 \textgreater= area2.style.pixelLeft)
\{

clearTimeout(s1);alert("Winner is the white crab!");\}

else \{ s1 = setTimeout("move\_crabs()", 200); \}

\}

\textless/script\textgreater{}

\textless body onLoad=init()\textgreater{}

\textless div style="background-color:black; width:100\%; height:130;
z-index:2" id="area1"\textgreater{}

\textless img style="background-color:red; top:20; z-index:3"
src="crab.gif" id="red"\textgreater\textless br\textgreater{}

\textless img style="background-color:green; top:45; z-index:3"
src="crab.gif"

id="green"\textgreater\textless br\textgreater{}

\textless img style="background-color:blue; top:70; z-index:3"
src="crab.gif" id="blue"\textgreater\textless br\textgreater{}

\textless img style="background-color:white; top:95; z-index:3"
src="crab.gif"

id="white"\textgreater\textless br\textgreater{}

\textless/div\textgreater{}

\textless div style="background-color:black; width:60; height:130;
z-index:1"

id="area2"\textgreater\textless/div\textgreater{}

\textless button style="position:absolute; top:200;"
onClick="move\_crabs()"\textgreater Play\textless/button\textgreater{}

\textless/body\textgreater{}

\textless/html\textgreater{}

3. Test the program. Click the Play button to start. Click the browser's
Refresh button to reset. A sample output looks:

\textbf{Learning Activity \#4: Jet Fighter}

1. Change to the C:\textbackslash games directory.

2. Use Notepad to create a new file named
\textbf{C:\textbackslash games\textbackslash lab8\_4.htm} with the
following contents:

\textless html\textgreater{}

\textless SCRIPT\textgreater{}

function init() \{

ap01.style.left=Math.floor(Math.random() *
document.body.clientWidth)-120; st01.style.left=Math.floor(Math.random()
* document.body.clientWidth); st01.style.top=Math.floor(Math.random() *
document.body.clientHeight); Game Programming -- Penn Wu

147

\protect\hypertarget{index_split_009.htmlux5cux23p148}{}{}if
((st01.style.pixelTop \textless= 100) \textbar\textbar{}
(st01.style.pixelTop \textgreater=

document.body.clientHeight)) \{

st01.style.top=Math.floor(Math.random() * document.body.clientHeight);
\}

b01.style.pixelLeft=st01.style.pixelLeft + 35;

b01.style.pixelTop=st01.style.pixelTop - 10;

cc();

\}

function cc() \{

check()

r\_width = eval(document.body.clientWidth) - 120;

ap01.style.pixelLeft=ap01.style.pixelLeft + 10;

if (ap01.style.pixelLeft \textgreater= r\_width)

\{

clearTimeout(s1);dd();

\}

else \{

s1 = setTimeout("cc()",100);

\}

\}

function dd() \{

check()

ap01.style.pixelLeft=ap01.style.pixelLeft - 10;

if (ap01.style.pixelLeft \textless= 10) \{

clearTimeout(s2);cc();

\}

else \{

s2 = setTimeout("dd()",100);

\}

\}

function fly() \{

var e=event.keyCode;

switch (e) \{

case 37:

st01.style.pixelLeft=st01.style.pixelLeft - 10;

break;

case 39:

st01.style.pixelLeft=st01.style.pixelLeft + 10;

break;

case 38:

st01.style.pixelTop=st01.style.pixelTop - 10;

break;

case 40:

st01.style.pixelTop=st01.style.pixelTop + 10;

break;

case 83:

b01.style.display=\textquotesingle inline\textquotesingle;

b01.style.pixelLeft=st01.style.pixelLeft + 35;

b01.style.pixelTop=st01.style.pixelTop - 10;

shoot();

break;

\}

\}

function shoot() \{

if (b01.style.pixelTop == 0) \{

Game Programming -- Penn Wu

148

\protect\hypertarget{index_split_009.htmlux5cux23p149}{}{}\includegraphics{index-149_1.png}

b01.style.display=\textquotesingle none\textquotesingle;

clearTimeout(sh);

\}

else \{

b01.style.top=b01.style.pixelTop - 5;

\}

sh = setTimeout("shoot()",5);

\}

function check() \{

if ((b01.style.pixelLeft \textgreater= ap01.style.pixelLeft) \&\&
(b01.style.pixelLeft \textless= ap01.style.pixelLeft + 70) \&\&
(b01.style.pixelTop \textless= ap01.style.pixelTop))

\{

ap01.src="explode.gif";

setTimeout("ap01.style.display=\textquotesingle none\textquotesingle",1000);

\}

\}

\textless/script\textgreater{}

\textless body bgcolor=000000; onkeydown="fly();"
onLoad=init()\textgreater{}

\textless img id=ap01 src="e\_plan.gif" style="position:absolute;
z-index:3"\textgreater{}

\textless img id=st01 src="shooter.gif"
style="position:absolute;"\textgreater{}

\textless b id=b01 style="position:absolute; color:white; display:none;
z-index=2"\textgreater!\textless/b\textgreater{}

\textless/body\textgreater{}

\textless/html\textgreater{}

3. Test the program. A sample output looks:

\textbf{Learning Activity \#5: Shooting Crabs}

1. Change to the C:\textbackslash games directory.

2. Use Notepad to create a new file named
\textbf{C:\textbackslash games\textbackslash lab8\_5.htm} with the
following contents:

\textless html\textgreater{}

\textless style\textgreater{}

.bee \{position:relative; background-color:yellow; z-index: 1\}

\textless/style\textgreater{}

\textless script\textgreater{}

function init() \{

codes ="";

for (i=1; i\textless=5; i++) \{

codes += "\textless img class=\textquotesingle bee\textquotesingle{}
id=c" + i + " src=\textquotesingle crab.gif\textquotesingle{}
/\textgreater";

\}

Game Programming -- Penn Wu

149

\protect\hypertarget{index_split_009.htmlux5cux23p150}{}{}
area1.innerHTML += codes;

moveLeft();

\}

function moveLeft() \{

leftSide = area1.style.pixelLeft + 5*32;

if (leftSide \textgreater= 300) \{

if (area1.style.pixelTop \textgreater= 200) \{

clearTimeout(sf);

\}

else \{

area1.style.pixelTop += 20;

clearTimeout(sf); moveRight();

\}

\}

else \{

area1.style.pixelLeft += 5;

sf = setTimeout("moveLeft()", 100);

\}

\}

function moveRight() \{

if (area1.style.pixelLeft == 0) \{

if (area1.style.pixelTop \textgreater= 200) \{

clearTimeout(sr);

\}

else \{

area1.style.pixelTop += 20;

clearTimeout(sr); moveLeft();

\}

\}

else \{

area1.style.pixelLeft -= 5;

sr = setTimeout("moveRight()", 100);

\}

\}

function shoot() \{

var e=event.keyCode;

switch(e) \{

case 37:

shooter.style.pixelLeft-=2; break;

case 39:

shooter.style.pixelLeft+=2; break;

case 83:

b1.style.pixelLeft = shooter.style.pixelLeft + 14;

b1.style.display = \textquotesingle inline\textquotesingle;

fire();

\}

\}

function fire() \{

if (b1.style.pixelTop == 0) \{

b1.style.display=\textquotesingle none\textquotesingle;

b1.style.top = 270;

clearTimeout(ss);

\}

else \{

b1.style.pixelTop -= 5;

if ((b1.style.pixelTop \textless= area1.style.pixelTop + 20) \&\&
(b1.style.pixelTop \textgreater=area1.style.pixelTop)) \{

Game Programming -- Penn Wu

150

\protect\hypertarget{index_split_009.htmlux5cux23p151}{}{}\includegraphics{index-151_1.png}

if ((b1.style.pixelLeft \textgreater= area1.style.pixelLeft) \&\&
(b1.style.pixelLeft \textless= area1.style.pixelLeft + 31)) \{

c1.style.visibility=\textquotesingle hidden\textquotesingle;
b1.style.display=\textquotesingle none\textquotesingle;\}

if ((b1.style.pixelLeft \textgreater= area1.style.pixelLeft + 32) \&\&
(b1.style.pixelLeft \textless= area1.style.pixelLeft + 61)) \{

c2.style.visibility=\textquotesingle hidden\textquotesingle;
b1.style.display=\textquotesingle none\textquotesingle;\}

if ((b1.style.pixelLeft \textgreater= area1.style.pixelLeft + 64) \&\&
(b1.style.pixelLeft \textless= area1.style.pixelLeft + 93)) \{

c3.style.visibility=\textquotesingle hidden\textquotesingle;
b1.style.display=\textquotesingle none\textquotesingle;\}

if ((b1.style.pixelLeft \textgreater= area1.style.pixelLeft + 96) \&\&
(b1.style.pixelLeft \textless= area1.style.pixelLeft + 127)) \{

c4.style.visibility=\textquotesingle hidden\textquotesingle;
b1.style.display=\textquotesingle none\textquotesingle;\}

if ((b1.style.pixelLeft \textgreater= area1.style.pixelLeft + 128) \&\&
(b1.style.pixelLeft \textless= area1.style.pixelLeft + 159)) \{

c5.style.visibility=\textquotesingle hidden\textquotesingle;
b1.style.display=\textquotesingle none\textquotesingle;\}

\}

ss = setTimeout("fire()", 20);

\}

\}

\textless/script\textgreater{}

\textless body onLoad=init() onKeyDown=shoot()\textgreater{}

\textless div style="position:absolute; width:300; height:300;
background-color: black"\textgreater{}

\textless span id="area1"
style="position:absolute;"\textgreater\textless/span\textgreater{}

\textless span id="b1" style="position:absolute; top:270; display:none;
color:white; z-index:2"\textgreater!\textless/span\textgreater{}

\textless span id="shooter" style="position:absolute; top:275;
color:white"\textgreater\textless b\textgreater\_\textbar\_\textless/b\textgreater\textless/span\textgreater{}

\textless/div\textgreater{}

\textless/body\textgreater{}

\textless/html\textgreater{}

3. Test the program. Press S to shoot, Press → or ← to move to left or
right. A sample output looks: \textbf{Submittal}

Upon completing all the learning activities,

1. Upload all files you created in this lab to your remote web server.

2. Log in to to Blackboard, launch Assignment 08, and then scroll down
to question 11.

3. Copy and paste the URLs to the textbox. For example,

Game Programming -- Penn Wu

151

\protect\hypertarget{index_split_009.htmlux5cux23p152}{}{}•

http://www.geocities.com/cis261/lab8\_1.htm

•

http://www.geocities.com/cis261/lab8\_2.htm

•

http://www.geocities.com/cis261/lab8\_3.htm

•

http://www.geocities.com/cis261/lab8\_4.htm

•

http://www.geocities.com/cis261/lab8\_5.htm

No credit is given to broken link(s).

Game Programming -- Penn Wu

152

\protect\hypertarget{index_split_009.htmlux5cux23p153}{}{}

Lecture \#9

Applying sound effects

\protect\hypertarget{index_split_010.html}{}{}

\hypertarget{index_split_010.htmlux5cux23calibre_pb_9}{%
\subsection{Introduction}\label{index_split_010.htmlux5cux23calibre_pb_9}}

In our environment, all the sound you can hear through your ears are
analog sounds. Even the sounds that come from the speaker of your
computer are analog sounds. However, computers are digital machines,
which mean computers do not process analog sounds.

The concept is when a microphone converts sound waves to voltage
signals, the resulting signal is an analog (or continuous) signal. The
sound card in your computer converts this analog signal to a digital
signal for a computer to process.

Analog to digital (A/D) converters handle the task of converting analog
signals to digital signals, which is also referred to as sampling. The
process of converting an analog signal to a digital signal
doesn\textquotesingle t always yield exact results. How closely a
digital wave matches its analog counterpart is determined by the
frequency at which it is sampled, as well as the amount of information
stored at each sample.

Common music Operating systems, such as Windows XP, take care of the
communication between the game file types

codes and A/D converter of the sound card. Digital sounds in Windows are
known as \textbf{waves}, which refer to the physical sound waves from
which digital sounds originate. Windows waves are stored in files with a
\textbf{.WAV} file extension, and can be stored in a wide range of
formats to accommodate different sound qualities. More specifically, you
can save waves in frequencies from 8kHz to 44kHz with either 8 or 16
bits per sample and as either mono or stereo.

Musical Instrument Digital Interface, or \textbf{MIDI}, has established
a standard interface between musical instruments. MIDI is commonly used
with a dedicated keyboard to control a synthesizer.

The keyboard inputs musical information (such as musical notes), such
that the synthesizer can generate electronic sound output (such as
songs). MIDI is popular in generating sound effects for games.

Creating and

To be able to create and modify your own sounds, you need some type of
software sound editing editing sounds

utility. Sound editing utilities usually provide a means of sampling
sounds from a microphone, CD-ROM, or line input. There are many free,
open source, sound editing tools you can easily find on the Internet.

The truth is that no matter what tool you choose to use, the process of
editing sounds is technically the same.

•

waveform: Determine whether the sound volume is too loud or too soft.
Waveform of a sound is the graphical appearance of the sound when
plotted over time.

•

Clipping: zooming in on the waveform in a sound editor and cutting out
the silence that appears before and after the sound.

•

Latency is the amount of time between when you queue a sound for playing
and when the user actually hears it. Latency should be kept to a minimum
so that sounds are heard when you want them to be heard without any
delay.

Background

DHTML uses HTML's \textless BGSOUND\textgreater{} and
\textless EMBED\textgreater{} tags to allow you to associate a sound
file sound/music

to a web game page as background sounds or soundtracks. Although it
plays a background sound when the page is opened, you then uses the
\textbf{autostart="false"} property and value to keep the sound file
from being played automatically.

The \textless BGSOUND\textgreater{} tag may be placed either within the
\textless HEAD\textgreater..\textless/HEAD\textgreater{} or the

\textless BODY\textgreater..\textless/BODY\textgreater{} portion of the
page. For example,

\textless HTML\textgreater\textless BODY\textgreater{}

Game Programming -- Penn Wu

153

\protect\hypertarget{index_split_010.htmlux5cux23p154}{}{}\includegraphics{index-154_1.png}

\textless bgsound SRC="game1.mid" autostart="true" /\textgreater{}

\textless/BODY\textgreater\textless/HTML\textgreater{}

and

\textless HTML\textgreater\textless head\textgreater{}

\textless bgsound SRC="game1.mid" \textbf{autostart="false"
/}\textgreater{}

\textless/head\textgreater\textless/HTML\textgreater{}

Microsoft documents specify more parameters for the
\textless BGSOUND\textgreater{} tag than are currently supported. The
following table lists the supported ones:

Parameter

Description

autostart="value" Set the value to TRUE to being playing the music
immediately upon page load.

ID=" \emph{idvalue}"

An identifier to be used for references in associated style sheet,
hypertext links, and JavaScript code. Must begin with a letter and may
include underscores. ID should be unique within the scope of the
document. If duplicated, the collection of identical-ID elements may be
referenced by their ordinal numbers.

LOOP=" \emph{n}"

Specifies how many times the sound will loop.

SRC=" \emph{url}"

Specifies the URL of the sound file. As shown above, sound files can be
in any recognizable format (for example: "midi, wav, au") VOLUME="
\emph{n}"

Determines the loudness of the background sound. Values may range from
-10,000 (weakest) to 0 (loudest). Not supported by the MAC.

You can use the HTML \textbf{TITLE} attribute to describe an HTML tag.
This capability is particularly important for elements that do not
normally have textual information associated with them, such as the
\textbf{bgSound} object, which specifies a sound file to play in the
background of a Web page.

The following example uses the TITLE attribute with the bgSound object
to describe the background sound.

\textless BGSOUND SRC="soundfile.mid" \textbf{TITLE="Sound of falling
water" /}\textgreater{} The \textbf{\textless EMBED\textgreater{}} tag
allows documents of any type to be embedded, including music files. For
example, the following use of the EMBED element mimics the behavior of
the BGSOUND tag.

\textless EMBED type="audio/x-midi" src="cis261.mid"
\textbf{autostart="false"}

\textbf{hidden="true"} \textgreater{}

When you embed a sound, it plays as soon as the page loads, you can use
the \textbf{autostart}

\textbf{=}" \textbf{false}" property and values to prevent the sound
file from being played as soon as the page is loaded.

The \textbf{hidden} property, when being set to true, controls the
visibility of the sound file icon. If you do not supply HIDDEN, then the
sound file icon will be visible, as shown below.

The EMBED element must appear inside the BODY element of the document.
Additionally, Internet Explorer relies on the resource specified in the
SRC attribute to ultimately determine which application to use to
display the embedded document.

You can call a sound playback function from the onMouseOver event
handler. For example, Game Programming -- Penn Wu

154

\protect\hypertarget{index_split_010.htmlux5cux23p155}{}{}\textless button

onMouseOver="playSound();return true"

onMouseOut ="stopSound();return true"

\textgreater Play\textless/button\textgreater{}

This technique applies to the design of the Piano game. When a user move
mouse over the key, the computer plays the Csound file.

............

............

\textless{} \textbf{span class="wKey" id="midC" style="left:50"}

\textbf{onMouseOver="Down();document.Csound.play()"}

onMouseOut="Up()"\textgreater\textless/span\textgreater{}

............

............

\textless span class="bKey" id="cSharp" style="left:72"

onMouseOver="document.Cs.play()"\textgreater\textless/span\textgreater{}

...........

...........

\textless embed src="C0.wav" autostart=false hidden=true name="Csound"

mastersound\textgreater{}

...........

...........

\textless/html\textgreater{}

You can add an sound effect to the Jet Fighter game (you created it in a
previous lecture) by adding the following bold faced lines:

function check() \{

if ((b01.style.pixelLeft \textgreater= ap01.style.pixelLeft) \&\&
(b01.style.pixelLeft \textless= ap01.style.pixelLeft + 70) \&\&
(b01.style.pixelTop \textless= ap01.style.pixelTop + 70))

\{

\textbf{document.explosion.play()}

ap01.src="explode.gif";

setTimeout("ap01.style.display=\textquotesingle none\textquotesingle",1000);

\}

\}

\textless/script\textgreater{}

\textless body bgcolor=000000; onkeydown="fly();"
onLoad=init()\textgreater{}

\textless img id=ap01 src="e\_plan.gif" style="position:absolute;
z-index:3"\textgreater{}

\textless img id=st01 src="shooter.gif"
style="position:absolute;"\textgreater{}

\textless b id=b01 style="position:absolute; color:white; display:none;
z-index=2"\textgreater!\textless/b\textgreater{}

\textbf{\textless embed src="explosion.wav" autostart=false hidden=true}
\textbf{name="explosion" mastersound\textgreater{}}

\textless/body\textgreater{}

Certainly, you must load the explosion.wav sound file to the same
directory as the Jet Fighter file resides.

Play sound

With Microsoft Internet Explorer, there are several ways to play a sound
file using DHTML and

/music file

JavaScript, without opening a separate window for sound control.
Assuming you add the using

following MIDI file:

JavaScript

\textless embed name="mysound" src="game1.mid"

Game Programming -- Penn Wu

155

\protect\hypertarget{index_split_010.htmlux5cux23p156}{}{}autostart=false
hidden=true mastersound\textgreater{}

To start playing the game1.mid file via the EMBED tag, use:

document.mySound.play()

To stop playing, use:

document.mySound.stop()

A complete code sample is:

\textless script\textgreater{}

function playSound()\{

\textbf{document.sound1.play();}

\}

function stopSound()\{

\textbf{document.sound1.stop();}

\}

\textless/script\textgreater{}

\textless body\textgreater{}

\textless embed src="game1.mid" autostart="false" hidden="True"

name="sound1"\textgreater{}

\textless button
onClick="playSound()"\textgreater Play\textless/button\textgreater{}

\textless button
onClick="stopSound()"\textgreater Stop\textless/button\textgreater{}

\textless/body\textgreater{}

If you use the \textless bgsound\textgreater{} tag, you have to do it
differently. Assuming you add the sound file using the following code:

\textless bgsound id="kick" loop=1\textgreater{}

To start playing a sound file game1.mid via the BGSOUND tag, use:
document.all{[}\textquotesingle kick\textquotesingle{]}.src=\textquotesingle game1.mid\textquotesingle{}

To stop playing, use:

document.all{[}\textquotesingle kick\textquotesingle{]}.src=\textquotesingle nosound.mid\textquotesingle{}

Notice that \textbf{game1.mid} stands for the name of the sound file
that you actually want to play; \textbf{nosound.mid} is a ``do-nothing''
sound file, which means it does not play any sound at all, but can be
used to stop the playback of other sound files.

A complete code sample is:

\textless script\textgreater{}

function playSound()\{

\textbf{document.all{[}\textquotesingle kick\textquotesingle{]}.src=\textquotesingle game1.mid\textquotesingle{}}

\}

function stopSound()\{

\textbf{document.all{[}\textquotesingle kick\textquotesingle{]}.src=\textquotesingle nosound.mid\textquotesingle{}}

\}

\textless/script\textgreater{}

\textless button
onClick="playSound()"\textgreater Play\textless/button\textgreater{}

\textless button
onClick="stopSound()"\textgreater Stop\textless/button\textgreater{}
Game Programming -- Penn Wu

156

\protect\hypertarget{index_split_010.htmlux5cux23p157}{}{}\textless bgsound
id="kick" loop=1 autostart="false"\textgreater{} One trick you need to
know is that you don't need to use the
\textbf{\textless bgsound\textgreater{}} tag to include all files you
need to play, simply include the first sound file is good enough.
Remember, the more sound files you include to your code using
\textless bgsound\textgreater{} and \textless embed\textgreater{} tags,
the longer it takes to load the game file. For example, you can use
JavaScript to switch between the game1.mid and game2.mid soundtracks
every 20 seconds.

\textless BODY\textgreater{}

\textless BGSOUND SRC="game1.mid" ID="cis261" autostart="true"
/\textgreater{}

\textless SCRIPT\textgreater{}

setInterval("changeSound()",20000);

function changeSound()\{

if (document.all.cis261.src == "game1.mid")

document.all.cis261.src = "game2.mid"

else document.all.cis261.src = "game1.mid";

\}

\textless/SCRIPT\textgreater{}

\textless/BODY\textgreater{}

JavaScript

Want to add a short sound effect to your page for certain actions, such
as when the user moves the Sound effect

mouse over a link? This is a simple yet versatile script that lets you
do just like! Relying on IE\textquotesingle s BGSOUND attribute (and
hence IE5+ only), the script can easily add a sound effect to a single
item (ie: 1 link), or thanks to a helper function, all items of the
specified element (ie: all \textless a\textgreater{} tags). This makes
it very easy to add a sound effect to an entire menu\textquotesingle s
links, for example.

Add the below script to the \textless HEAD\textgreater{} section of your
page, and change "var soundfile" to point to where your sound file is
located.

\textbf{\textless bgsound src="\#" id="soundeffect" loop=1
autostart="true" /\textgreater{}}

\textless script\textgreater{}

var soundfile="game1.mid"

function playsound(soundfile)\{

if (document.all \&\& document.getElementById)\{

document.getElementById("soundeffect").src="" //reset first in case of
problems

document.getElementById("soundeffect").src=soundfile

\}

\}

function bindsound(tag, soundfile, masterElement)\{

if (!window.event) return

var source=event.srcElement

while (source!=masterElement \&\& source.tagName!="HTML")\{

if (source.tagName==tag.toUpperCase())\{

playsound(soundfile)

break

\}

source=source.parentElement

\}

\}

\textless/script\textgreater{}

Set up an element to receive the JavaScript sound, whether onMouseover,
onClick etc. For example, the below plays a sound when the mouse moves
over a link: Game Programming -- Penn Wu

157

\protect\hypertarget{index_split_010.htmlux5cux23p158}{}{}\textless button
onClick="playsound(soundfile)"\textgreater Play
1\textless/button\textgreater{} or

\textless button onMouseover="playsound(soundfile)"\textgreater Play
2\textless/button\textgreater{} The second line shows that you can pass
in a different wav file to play for any link other than the default
specified within the script (var soundfile).

Digital sound

To sample a sound, you just store the amplitude of the sound wave at
regular intervals. Taking theory

samples at more frequent intervals causes the digital signal to more
closely approximate the analog signal and, therefore, sound more like
the analog wave when played. So, when sampling sounds the rate
(frequency) at which the sound is sampled is very important, as well as
how much data is stored for each sample. The unit of measurement for
frequency is Hertz (Hz), which specifies how many samples are taken per
second. As an example, CD-quality audio is sampled at 44,000Hz (44kHz),
which means that when you\textquotesingle re listening to a music CD
you\textquotesingle re actually hearing 44,000 digital sound samples
every second.

In addition to the frequency of a sampled sound, the number of bits used
to represent the amplitude of the sound impacts the sound quality, as
well as whether the sound is a stereo or mono sound. Knowing this,
it\textquotesingle s possible to categorize the quality of a digital
sound according to the following properties:

•

Frequency

•

Bits-per-sample

•

Mono/stereo

The frequency of a sampled sound typically falls somewhere in the range
of 8kHz to 44kHz, with 44kHz representing CD-quality sound. The
bits-per-sample of a sound is usually either 8bps (bits per sample) or
16bps, with 16bps representing CD-quality audio; this is also known as
16-bit audio. A sampled sound is then classified as being either mono or
stereo, with mono meaning that there is only one channel of sound,
whereas stereo has two channels. As you might expect, a stereo sound
contains twice as much information as a mono sound. Not surprisingly,
CD-quality audio is always stereo. Therefore, you now understand that a
CD-quality sound is a 44kHz 16-bit stereo sound.

Although it would be great to incorporate sounds that are CD-quality
into all of your games, the reality is that high-quality sounds take up
a lot of memory and can therefore be burdensome to play if your game
already relies on a lot of images and other memory-intensive resources.

Granted, most computers these days are capable of ripping through
memory-intensive multimedia objects such as MP3 songs like they are
nothing, but games must be designed for extreme efficiency. Therefore,
it\textquotesingle s important to consider ways to minimize the memory
and processing burden on games every chance you get. One way to minimize
this burden is to carefully choose a sound quality that sounds good
without hogging too much memory.

Review

1. Where can you place a sound file using the
\textless BGSOUND\textgreater{} tag?

Questions

A. between \textless head\textgreater{} and \textless/head\textgreater{}

B. between \textless body\textgreater{} and \textless/body\textgreater{}

C. A and B

D. None of the above

2. Where can you place a sound file using the
\textless EMBED\textgreater{} tag?

A. between \textless head\textgreater{} and \textless/head\textgreater{}

B. between \textless body\textgreater{} and \textless/body\textgreater{}

C. A and B

D. None of the above

3. Which causes the cis261.mid sound file to play automatically when the
page is loaded?

A. \textless bgsound src="cis261.mid" autostart="true"\textgreater{}
Game Programming -- Penn Wu

158

\protect\hypertarget{index_split_010.htmlux5cux23p159}{}{}B.
\textless bgsound src="cis261.mid" autostart="yes"\textgreater{} C.
\textless bgsound src="cis261.mid" start=1\textgreater{}

D. \textless bgsound src="cis261.mid" start="auto"\textgreater{} 4.
Which sets the embedded MIDI file as an visible object on the page?

A. \textless EMBED src="cis261.mid" display="true"\textgreater{} B.
\textless EMBED src="cis261.mid" hidden="true"\textgreater{} C.
\textless EMBED src="cis261.mid" hidden="visible"\textgreater{} D.
\textless EMBED src="cis261.mid" display="invisible"\textgreater{} 5.
Given the following code segment, which can play the sound?

\textless embed name="s1" src="cis261.mid" autostart="false"
mastersound\textgreater{} A. window.s1.play();

B. document.s1.play();

C. window.play(s1);

D. document.play(s1);

6. Given the following code segment, which can play the sound?

\textless bgsound id="c1" autostart="false"\textgreater{}

A.
document.all{[}\textquotesingle c1\textquotesingle{]}.src=\textquotesingle c261.mid\textquotesingle{}

B. document.play(\textquotesingle c1\textquotesingle);
c1.src=\textquotesingle c261.mid\textquotesingle{}

C. document.all.c1.src == "c261.mid"

D.
document.play(\textquotesingle c1\textquotesingle).src=\textquotesingle c261.mid\textquotesingle{}

7. Given the following code segment, which is a "do-nothing" sound file?

function playSound() \{

var b1="m2.mid";

document.all{[}\textquotesingle a1\textquotesingle{]}.src=\textquotesingle m1.mid\textquotesingle;

document.b1.play(); \}

function stopSound() \{

var d1="m4.mid";

document.all{[}\textquotesingle c1\textquotesingle{]}.src=\textquotesingle m3.mid\textquotesingle;

document.d1.stop(); \}

\textless/script\textgreater{}

A. m1.mid

B. m2.mid

C. m3.mid

D. m4.mid

8. When click the following button, which sound file will be played?

\textless button
onMouseOver="Down();document.E.play()"\textgreater Play\textless/button\textgreater{}
A. \textless embed src="C.wav" autostart=false hidden=true name="C"
mastersound\textgreater{} B. \textless embed src="D.wav" autostart=false
hidden=true name="D" mastersound\textgreater{} C. \textless embed
src="E.wav" autostart=false hidden=true name="E"
mastersound\textgreater{} D. \textless embed src="F.wav" autostart=false
hidden=true name="F" mastersound\textgreater{} 9. Given the following
code segment, which sound will be played automatically?

\textless embed id="s1" src="hit.wav" autostart=false
mastersound\textgreater{}

\textless bgsound id="s2" src="circusride.mid"\textgreater{} Game
Programming -- Penn Wu

159

\protect\hypertarget{index_split_010.htmlux5cux23p160}{}{}\textless bgsound
id="s3" /\textgreater{}

A. s1

B. s2

C. s3

D. All of the above

10. If you wish to switch between two MIDI files every 20 seconds, which
method will you choose to use?

A. setInterval()

B. setLoop()

C. setSwitch()

D. setReturn()

Game Programming -- Penn Wu

160

\protect\hypertarget{index_split_010.htmlux5cux23p161}{}{}

Lab \#9

\textbf{Preparation \#1:}

1. Create a new directory named \textbf{C:\textbackslash games}.

2. Use Internt Explorer to go to
\textbf{http://business.cypresscollege.edu/\textasciitilde pwu/cis261/download.htm}
to download lab9.zip (a zipped) file. Extract the files to
C:\textbackslash games directory.

\textbf{Learning Activity \#1:}

Note: This game is meant to be simple and easy for the sake of
demonstrating programming concepts. Please do not hesitate to enhance
the appearance or functions of this game.

1. Change to the C:\textbackslash games directory.

2. Rename the \textbf{jet\_fighter.htm} file to \textbf{lab9\_1.htm}.

3. Use Notepad to open the
\textbf{C:\textbackslash games\textbackslash lab9\_1.htm} file.

4. Add the following bold-faced lines:

...............

...............

function check() \{

if ((b01.style.pixelLeft \textgreater= ap01.style.pixelLeft) \&\&
(b01.style.pixelLeft \textless= ap01.style.pixelLeft + 70) \&\&
(b01.style.pixelTop \textless= ap01.style.pixelTop + 70))

\{

\textbf{document.explosion.play()}

ap01.src="explode.gif";

setTimeout("ap01.style.display=\textquotesingle none\textquotesingle",1000);

\}

\}

\textless/script\textgreater{}

\textless body bgcolor=000000; onkeydown="fly();"
onLoad=init()\textgreater{}

\textless img id=ap01 src="e\_plan.gif" style="position:absolute;
z-index:3"\textgreater{}

\textless img id=st01 src="shooter.gif"
style="position:absolute;"\textgreater{}

\textless b id=b01 style="position:absolute; color:white; display:none;
z-index=2"\textgreater!\textless/b\textgreater{}

\textbf{\textless embed src="explosion.wav" autostart=false hidden=true
name="explosion"}

\textbf{mastersound\textgreater{}}

\textless/body\textgreater{}

\textless/html\textgreater{}

5. Test the program. Whenever you hit the enemy's plan, the
explosion.wav sound file is played.

\textbf{Learning Activity \#2: Gun shot}

Note: This game is meant to be simple and easy for the sake of
demonstrating programming concepts. Please do not hesitate to enhance
the appearance or functions of this game.

1. Change to the C:\textbackslash games directory.

2. Use Notepad to create a new file named
\textbf{C:\textbackslash games\textbackslash lab9\_2.htm} with the
following contents:

\textless html\textgreater{}

\textless script\textgreater{}

Game Programming -- Penn Wu

161

\protect\hypertarget{index_split_010.htmlux5cux23p162}{}{}\includegraphics{index-162_1.png}

function shoot() \{

var i = b01.style.pixelLeft;

if (i \textless= 400) \{

b01.style.display=\textquotesingle inline\textquotesingle;

b01.style.left=i + 10;

setTimeout("shoot()",20);

\}

else \{

b01.style.display=\textquotesingle none\textquotesingle;

b01.style.left=95;

\};

\}

function playsound() \{

document.all{[}\textquotesingle gunshot\textquotesingle{]}.src=\textquotesingle gunshot.wav\textquotesingle{}

\}

\textless/script\textgreater{}

\textless body onKeyDown="playsound();shoot();"\textgreater{}

\textless img src="gun.gif"\textgreater{}

\textless img src="bullet.gif" id="b01" style="position:absolute;
display:none; top:20; left:95"\textgreater\textless/span\textgreater{}

\textbf{\textless bgsound id="gunshot" /\textgreater{}}

\textless/body\textgreater{}

\textless/html\textgreater{}

3. Test the program. Press any key to fire the gunshot; you should hear
the sound effect, too. A sample output looks:

\textbf{Learning Activity \#3: Squeeze the clown}

Note: This game is meant to be simple and easy for the sake of
demonstrating programming concepts. Please do not hesitate to enhance
the appearance or functions of this game.

1. Change to the C:\textbackslash games directory.

2. Use Notepad to create a new file named
\textbf{C:\textbackslash games\textbackslash lab9\_3.htm} with the
following contents:

\textless html\textgreater{}

\textless script\textgreater{}

var k=0;

function goUp() \{

bar.style.display=\textquotesingle inline\textquotesingle;

bar.style.top=bar.style.pixelTop - 5;

Game Programming -- Penn Wu

162

\protect\hypertarget{index_split_010.htmlux5cux23p163}{}{}\includegraphics{index-163_1.png}

if (clown.style.pixelTop \textless= 20) \{

clown.style.height=eval(bar.style.top.replace(\textquotesingle px\textquotesingle,\textquotesingle\textquotesingle))
- 20;

clown.style.width =
eval(clown.style.width.replace(\textquotesingle px\textquotesingle,\textquotesingle\textquotesingle))+1;
clown.style.Top = 20;

\textbf{document.hit.play()}

\}

else \{

bar.style.height=k;

k+=5;

clown.style.top=bar.style.pixelTop-120;

\}

if (
eval(clown.style.height.replace(\textquotesingle px\textquotesingle,\textquotesingle\textquotesingle))
\textless=10) \{

clearTimeout(s1);\}

else \{

s1=setTimeout("goUp()",20); \}

\}

\textless/script\textgreater{}

\textless body onKeyDown=goUp()\textgreater{}

\textless hr style="position:absolute; left:10; top:0;

background-color:red; width:100;z-index:2" size="20px"
align="left"\textgreater{}

\textless img id="clown" src="clown.jpg" style="position:absolute;
left:20; top:280; width:54; z-index:3"\textgreater{}

\textless span id="bar" style="position:absolute; left:10; top:400;
background-color:red; width:100; display:none;
z-index:1"\textgreater\textless/span\textgreater{}

\textbf{\textless embed src="hit.wav" autostart=false hidden=true
name="hit" mastersound\textgreater{}}

\textbf{\textless bgsound src="circusride.mid"\textgreater{}}

\textless/body\textgreater{}

\textless/html\textgreater{}

3. Test the program. When the page is loaded, the background music
plays. When squeezing the clown, you will hear a sound effect. A sample
output looks:

\textbf{Learning Activity \#4}

\textbf{}

Note: This game is meant to be simple and easy for the sake of
demonstrating programming concepts. Please do not hesitate to enhance
the appearance or functions of this game.

1. Change to the C:\textbackslash games directory.

2. Use Notepad to create a new file named
\textbf{C:\textbackslash games\textbackslash lab9\_4.htm} with the
following contents:

\textless html\textgreater{}

\textless script\textgreater{}

var x=190; y=384;

Game Programming -- Penn Wu

163

\protect\hypertarget{index_split_010.htmlux5cux23p164}{}{}\includegraphics{index-164_1.png}

var k=1; n=0;

function draw()\{

var i = event.keyCode;

switch (i) \{

case 37: n=1; k=0; document.left.play(); break;

case 38: n=0; k=1; document.up.play(); break;

case 39: n=-1; k=0; document.right.play(); break;

case 40: n=0; k=-1; document.down.play(); break;

\}

\}

function cc() \{

if ((x\textless=10) \textbar\textbar{} (x\textgreater=394)
\textbar\textbar{} (y\textless=-10) \textbar\textbar{}
(y\textgreater=400)) \{

p1.innerText="Game over!";

\}

else \{

y-=k; x-=n;

codes = "\textless span
style=\textquotesingle position:absolute;left:"+x;

codes += "; top:"+ y
+"\textquotesingle\textgreater.\textless/span\textgreater";
area1.innerHTML += codes;

setTimeout("cc()", 0.1);

\}

\}

\textless/script\textgreater{}

\textless body OnLoad=cc() onKeyDown=draw()\textgreater{}

\textless div id=area1 style="position:absolute; width:400;

height:400; border:solid 3 black; color:red;

background-Color: black; left:0;
top:10;"\textgreater\textless/div\textgreater{}

\textless p id=p1 style="position:absolute; left:160;top:200; color:
white"\textgreater\textless/p\textgreater{}

\textbf{\textless embed src="left.wav" autostart=false hidden=true
name="left" mastersound\textgreater{}}

\textbf{\textless embed src="right.wav" autostart=false hidden=true
name="right" mastersound\textgreater{}}

\textbf{\textless embed src="up.wav" autostart=false hidden=true
name="up" mastersound\textgreater{}}

\textbf{\textless embed src="down.wav" autostart=false hidden=true
name="down" mastersound\textgreater{}}

\textbf{\textless bgsound src="canzone.mid"\textgreater{}}

\textless/body\textgreater{}

\textless/html\textgreater{}

3. Test the program. A sample output looks:

\textbf{Learning Activity \#5: Piano}

Game Programming -- Penn Wu

164

\protect\hypertarget{index_split_010.htmlux5cux23p165}{}{}Note: This
game is meant to be simple and easy for the sake of demonstrating
programming concepts. Please do not hesitate to enhance the appearance
or functions of this game.

1. Change to the C:\textbackslash games directory.

2. Use Notepad to create a new file named
\textbf{C:\textbackslash games\textbackslash lab9\_5.htm} with the
following contents:

\textless html\textgreater{}

\textless style\textgreater{}

.wKey \{

position:absolute;

top:20;

background-color:white;

border-top:solid 1 \#cdcdcd;

border-left:solid 1 \#cdcdcd;

border-bottom:solid 4 \#cdcdcd;

border-right:solid 1 black;

height:150px;

width:40px

\}

.bKey \{

position:absolute;

top:20;

background-color:black;

border:solid 1 white;

height:70px;

width:36px

\}

\textless/style\textgreater{}

\textless script\textgreater{}

function Down() \{

var keyID = event.srcElement.id;

document.getElementById(keyID).style.borderTop="solid 1 black";
document.getElementById(keyID).style.borderLeft="solid 1 black";
document.getElementById(keyID).style.borderBottom="solid 1 black";
document.getElementById(keyID).style.borderRight="solid 1 \#cdcdcd";

\}

function Up() \{

var keyID = event.srcElement.id;

document.getElementById(keyID).style.borderTop="solid 1 \#cdcdcd";
document.getElementById(keyID).style.borderLeft="solid 1 \#cdcdcd";
document.getElementById(keyID).style.borderBottom="solid 4 \#cdcdcd";
document.getElementById(keyID).style.borderRight="solid 1 black";

\}

\textless/script\textgreater{}

\textless div\textgreater{}

\textless span class="wKey" id="midC" style="left:50"

onMouseOver="Down();document.Csound.play()"

onMouseOut="Up()"\textgreater\textless/span\textgreater{}

\textless span class="wKey" id="midD" style="left:91"

onMouseOver="Down();document.Dsound.play()"

onMouseOut="Up()"\textgreater\textless/span\textgreater{}

\textless span class="wKey" id="midE" style="left:132"

onMouseOver="Down();document.Esound.play()"

onMouseOut="Up()"\textgreater\textless/span\textgreater{}

Game Programming -- Penn Wu

165

\protect\hypertarget{index_split_010.htmlux5cux23p166}{}{}\includegraphics{index-166_1.png}

\textless span class="wKey" id="midF" style="left:173"

onMouseOver="Down();document.Fsound.play()"

onMouseOut="Up()"\textgreater\textless/span\textgreater{}

\textless span class="wKey" id="midG" style="left:214"

onMouseOver="Down();document.Gsound.play()"

onMouseOut="Up()"\textgreater\textless/span\textgreater{}

\textless span class="wKey" id="midA" style="left:255"

onMouseOver="Down();document.Asound.play()"

onMouseOut="Up()"\textgreater\textless/span\textgreater{}

\textless span class="wKey" id="midB" style="left:296"

onMouseOver="Down();document.Bsound.play()"

onMouseOut="Up()"\textgreater\textless/span\textgreater{}

\textless span class="wKey" id="highC" style="left:337"

onMouseOver="Down();document.CHsound.play()"

onMouseOut="Up()"\textgreater\textless/span\textgreater{}

\textless span class="bKey" id="cSharp" style="left:72"

onMouseOver="document.Cs.play()"\textgreater\textless/span\textgreater{}

\textless span class="bKey" id="dSharp" style="left:114"

onMouseOver="document.Ds.play()"\textgreater\textless/span\textgreater{}

\textless span class="bKey" id="fSharp" style="left:196"

onMouseOver="document.Fs.play()"\textgreater\textless/span\textgreater{}

\textless span class="bKey" id="gSharp" style="left:238"

onMouseOver="document.Fs.play()"\textgreater\textless/span\textgreater{}

\textless span class="bKey" id="aSharp" style="left:280"

onMouseOver="document.As.play()"\textgreater\textless/span\textgreater{}

\textless/div\textgreater{}

\textless!-\/- white key sound files -\/-\textgreater{}

\textless embed src="C0.wav" autostart=false hidden=true name="Csound"
mastersound\textgreater{}

\textless embed src="D0.wav" autostart=false hidden=true name="Dsound"
mastersound\textgreater{}

\textless embed src="E0.wav" autostart=false hidden=true name="Esound"
mastersound\textgreater{}

\textless embed src="F0.wav" autostart=false hidden=true name="Fsound"
mastersound\textgreater{}

\textless embed src="G0.wav" autostart=false hidden=true name="Gsound"
mastersound\textgreater{}

\textless embed src="A0.wav" autostart=false hidden=true name="Asound"
mastersound\textgreater{}

\textless embed src="B0.wav" autostart=false hidden=true name="Bsound"
mastersound\textgreater{}

\textless embed src="C1.wav" autostart=false hidden=true name="CHsound"
mastersound\textgreater{}

\textless!-\/- black key sound files -\/-\textgreater{}

\textless embed src="Cs0.wav" autostart=false hidden=true name="Cs"
mastersound\textgreater{}

\textless embed src="Ds0.wav" autostart=false hidden=true name="Ds"
mastersound\textgreater{}

\textless embed src="Fs0.wav" autostart=false hidden=true name="Fs"
mastersound\textgreater{}

\textless embed src="Gs0.wav" autostart=false hidden=true name="Gs"
mastersound\textgreater{}

\textless embed src="As0.wav" autostart=false hidden=true name="As"
mastersound\textgreater{}

\textless/html\textgreater{}

3. Test the program. Use the mouse to play a piece of music now.

Game Programming -- Penn Wu

166

\protect\hypertarget{index_split_010.htmlux5cux23p167}{}{}\textbf{Submittal}

Upon completing all the learning activities,

1. Upload all files you created in this lab to your remote web server.

2. Log in to to Blackboard, launch Assignment 09, and then scroll down
to question 11.

3. Copy and paste the URLs to the textbox. For example,

•

http://www.geocities.com/cis261/lab9\_1.htm

•

http://www.geocities.com/cis261/lab9\_2.htm

•

http://www.geocities.com/cis261/lab9\_3.htm

•

http://www.geocities.com/cis261/lab9\_4.htm

•

http://www.geocities.com/cis261/lab9\_5.htm

No credit is given to broken link(s).

Game Programming -- Penn Wu

167

\protect\hypertarget{index_split_010.htmlux5cux23p168}{}{}

Lecture \#10

Score boards

\protect\hypertarget{index_split_011.html}{}{}

\hypertarget{index_split_011.htmlux5cux23calibre_pb_10}{%
\subsection{Introduction}\label{index_split_011.htmlux5cux23calibre_pb_10}}

Many people consider scoring the extension of collision detection, and
should be part of the collision response. Once you create a game code
that allows two or more objects to collide, you frequently need a method
to keep and display the score the play earned. However, there are many
situations in which you have nothing to collide when keeping the game
score. The instructor, thus, believes score keeping should be an
individual topic, and is completely independent of the collision
response.

The scoring keeping function can be an individual function, or it can be
placed into other function.

This lecture will introduce you to the world of scoring in game
programming.

Use variables

In a computer game, the player earns or loses score points in according
to the game rules the to keep the

programmers defined. The score points may increase, decrease, or stay
the same depending on how score

the players stick to the game rules. In other words, the score point is
not a fixed value, because it varies as the players continue to play the
game.

In terms of programming, programmers declare variables to keep such
continuously changing values in a temporary holding area (namely
computer\textquotesingle s memory). If the values need to be updated,
the new value replaces the old one to be stored in that temporary
holding area. It is necessary to assign a name to that holding area,
such that the computer can always locate the correct temporary holding
area if there are more than one variable.

In any programming language, the process to tell the computer to find
some memory space as a temporary particular holding area is known as
``declaring a variable.'' The name of the variable is also the ID of the
temporary holding area. The values of the variable, such as a string, a
number, or a date which directly represents a constant value, are known
as \textbf{literals}.

In JavaScript, you declare a variable by using the following syntax: var
\emph{\textbf{VariableName}} = "InitialValue";

where \textbf{var} is a keyword, \emph{VariableNmae} is the name you
assign to that variable. If the variable has an initial value (meaning
its first value before being updated by the computer), add it next to
the = sign.

For example,

var playerName = "Helen";

var gameName = \textquotesingle Tic Tac Toe\textquotesingle;

var score = 0;

The use of single (\textquotesingle) or double quotes (") specifies that
these two values-\/-\textbf{Helen} and \textbf{Tic Tac Toe}-\/-

are strings. The term ``string'' refers to a series of alphanumeric
characters, usually forming a piece of text.

A game code segment consists of lines of statements with sequence
arranged by the programmer.

When computer execute the codes, statements are executed on a sequential
basis. Consider the following code segments:

var myScore = 0;

myScore = myScore + 5;

myScore += 50;

myScore = myScore - 10;

myScore -= 5;

Game Programming -- Penn Wu

168

\protect\hypertarget{index_split_011.htmlux5cux23p169}{}{}The first line
declares the variable \textbf{myScore} with an initial value \textbf{0}.
The second line means

``myScore now equals to whatever myScore currently has plus 5'', so the
new value is 5 after this statement is executed (0 + 5 =5).

The following line uses the increment operator (+=) to tell the computer
to ``add 50 to the current value of myScore'', so the value after
execution is 55 (5 + 50 = 55).

myScore += 50;

The increment (++) operator is a unary operator that can increment a
value by adding 1 to it. You can either place the increment operator to
the left of a variable (known as pre-increment), or place it to the
right of a variable (known as post-increment). In the following example,
both x and y variables will have a new value 1 after the execution.

var x=0; y=0;

++x;

y++;

When you need to increment by adding 2 or larger, the increment operator
changes to (+= \emph{n}, where \emph{n} is usually an integer). For
example,

myScore += 50;

The following line tells the computer to subtract 10 from the current
value of myScore, so the value after execution is 45 (55 -- 10 = 45).

myScore = myScore - 10;

The last line uses the decrement operator (-=) to subtract 5 from the
current value of myScore, so the value after execution is 40 (45 -- 5 =
40).

The concept of decrement (-\/-) operator is opposite to that of
increment operator. So, the values of x and y after execution in the
following example are 0.

var x=1; y=1;

x-\/-;

y-\/-;

When you need to decrement by 2 or larger, the operator changes to (-=
\emph{n}, where \emph{n} is usually an integer).

Adding /

Keeping score in essence is to add or subtract certain number from a
variable's current value. In the Decreasing

number guessing game, the following code creates two textboxes: one (t1)
is visible; the other (t2) is Score:

invisible.

The \textbf{start()} function use the \textbf{Math.random()} method to
generate a random number in the range of

{[}1-64{]} (or from 1 to 64). The random number is then temporarily
assigned to the \textbf{t2} textbox as its value. The \textbf{t1}
textbox, on the other hand, is for the player to enter a number to guess
what number the computer will randomly generates.

...............

...............

function start() \{

var k=Math.floor(Math.random()*64);

f1.t2.value=k;

...............

...............

\textless form name=f1\textgreater{}

Game Programming -- Penn Wu

169

\protect\hypertarget{index_split_011.htmlux5cux23p170}{}{}Enter a number
(1-64):

\textless input type=text name=t1 size=5\textgreater{}

\textless input type=hidden name=t2\textgreater{}

\textless/form\textgreater{}

...............

...............

\textless p\textgreater Your Score: \textless b
id=p1\textgreater\textless/b\textgreater\textless/p\textgreater{}

...............

...............

When the player clicks the Stop button to trigger the stop() function,
the \textbf{score} variable is used to update and keep the increment
(adding) or decrement (decreasing) of the player's score.

function stop() \{

clearTimeout(sf);

\textbf{if (f1.t1.value==f1.t2.value) \{ score+=1000; p1.innerText =
score;\}}

\textbf{else \{ score -=1; p1.innerText = score;\}}

\}

The logic is simple. Compare the value of the player's entry, t1, and
the random number generated by the computer (which is temporarily stored
to t2 textbox). If the are exactly the same, the \textbf{score} variable
increments it value of 1000; otherwise, it decrement by 1.

The following line displays the updated score in the \textbf{p1} object,
the \textbf{innerText} property forces the value to be a \textbf{string}
(text-based) literal.

\textbf{p1.innerText = score;}

In the Punching Duck game, the punch() function forces the glove to move
toward northeast till the \textbf{gv.style.pixelTop} value is less than
or equal to 0.

function punch() \{

gv.style.pixelLeft+=2;

gv.style.pixelTop-=2;

if (gv.style.pixelTop \textless=0)

\{clearTimeout(s2);gv.style.pixelTop=200;gv.style.pixelLeft=10;\}

else \{s2=setTimeout("punch()", 20);\}

for (i=1; i\textless=10; i++) \{

code3="if ((gv.style.pixelLeft \textgreater= d"+i+".style.pixelLeft)
\&\&"; code3+=" (gv.style.pixelLeft \textless= (d"+i+".style.pixelLeft +
61)) \&\& "; code3+=" (gv.style.pixelTop \textless=
d"+i+".style.pixelTop)) ";
code3+="\{d"+i+".style.display=\textquotesingle none\textquotesingle;d"+i+".src=\textquotesingle noduck.gif\textquotesingle;"

code3+="score+=10;p1.innerText=score\}";

eval(code3);

\}

\}

When the glove moves, the following code block generates 10 similar
codes segments which will detect the collision of the \textbf{gv} object
with any of the 10 ducks (d1, d2, d3, \ldots, d10).

for (i=1; i\textless=10; i++) \{

code3="if ((gv.style.pixelLeft \textgreater= d"+i+".style.pixelLeft)
\&\&"; code3+=" (gv.style.pixelLeft \textless= (d"+i+".style.pixelLeft +
61)) \&\& "; code3+=" (gv.style.pixelTop \textless=
d"+i+".style.pixelTop)) ";
code3+="\{d"+i+".style.display=\textquotesingle none\textquotesingle;d"+i+".src=\textquotesingle noduck.gif\textquotesingle;"

code3+="score+=10;p1.innerText=score\}";

eval(code3);

\}

Game Programming -- Penn Wu

170

\protect\hypertarget{index_split_011.htmlux5cux23p171}{}{}\includegraphics{index-171_1.png}

\includegraphics{index-171_2.png}

The following is a sample output of the above code block for d3: if
((gv.style.pixelLeft \textgreater= \textbf{d3}.style.pixelLeft) \&\&
(gv.style.pixelLeft \textless= (d3.style.pixelLeft + 61)) \&\&
(gv.style.pixelTop \textless= d3.style.pixelTop))

\{d3.style.display=\textquotesingle none\textquotesingle;d3.src=\textquotesingle noduck.gif\textquotesingle{}

\textbf{score+=10};p1.innerText=score \}

In plain English, it means ``if gv's pixelLeft value is greater than or
equal to that of d3 and gv's pixelLeft value is less than or equal to
d3's plus 61 (the width of d3) and gv's pixelTop value is less than or
equal to d3's, then \ldots. add 10 to the value of the variable
\textbf{score}\ldots\ldots''

In other words, the above \emph{for} loop builds a system of collision
detection for each of the 10 ducks.

The collision detection is based on three borders---left, right, and
bottom, of each of the d \emph{n} (where \emph{n} is 1, 2, 3, \ldots{}
10) object.

Notice that it takes 5 movements for the glove to move across the duck,
as shown below: Each movement forces the computer to add 10 to the value
of \textbf{score}, so the player earns 50 points (50 + 50 + 50 + 50 +
50) each time when he/she punches the duck with ID \textbf{d3}.

In the Shooting Crab game, the \textbf{fire()} function detects the
collision of \textbf{c1} and \textbf{area1} object. If the collision
happens, the score variable is incremented by 10.

function fire() \{

if (b1.style.pixelTop == 0) \{

b1.style.display=\textquotesingle none\textquotesingle;

b1.style.top = 270;

clearTimeout(ss);

\}

else \{

b1.style.pixelTop -= 5;

if ((b1.style.pixelTop \textless= area1.style.pixelTop + 20) \&\&
(b1.style.pixelTop \textgreater=area1.style.pixelTop)) \{

if ((b1.style.pixelLeft \textgreater= area1.style.pixelLeft) \&\&
(b1.style.pixelLeft \textless= area1.style.pixelLeft + 31))

\{

c1.style.display=\textquotesingle none\textquotesingle;b1.style.display="none";

clearTimeout(ss); \textbf{score += 10;}

\textbf{gd.innerText = score;} newCrab() \}

\}

ss = setTimeout("fire()", 20);

\}

\}

Store the score

JavaScript can create, read, and erase HTTP (HyperText Transfer
Protocol) cookies. A cookie is a in a JavaScript

small text file a Web server sends to a web client computer to store
information about the user. A Game Programming -- Penn Wu

171

\protect\hypertarget{index_split_011.htmlux5cux23p172}{}{}Cookie

JavaScript game may set a cookie file to keep the game score
temporarily.

Cookies allow you to store information on the client computer. However,
a cookie can only hold string based name/value pairs (i.e
\emph{name=value} settings). The cookie property of the document object
set the cookie. The format is:

document.cookie = " \emph{CookieName =Value}";

Consider the following example. The \textbf{scr} variable keeps the
updated value.

\textless html\textgreater{}

\textless script\textgreater{}

var scr = 0;

var d = new Date();

function set\_score() \{

\textbf{document.cookie = "score ="+scr;}

scr += 5;

\}

function read\_score() \{

p1.innerText = "Score: " + document.cookie.split("="){[}1{]};

\}

\textless/script\textgreater{}

\textless body onLoad="set\_score();read\_score()"\textgreater{}

\textless button
onClick="set\_score();read\_score()"\textgreater Click\textless/button\textgreater{}

\textless p id="p1"\textgreater\textless/p\textgreater{}

\textless/body\textgreater{}

\textless/html\textgreater{}

Since the data stored in a cookie has a format of
\emph{CookieName=Value}, you can use the split() function to break the
data into a string array in which \emph{CookieName} is the first element
and Value is the \emph{second}. The following get the seconds elements:

document.cookie.split("="){[}1{]}

Similarly, to retrieve the name of cookie, use

document.cookie.split("="){[}0{]}

Store the score

Values held by a variable can then be saved to local file permanently or
kept in the computer's in a text file

physical memory temporarily. JavaScript support reading from and writing
to local file with the support of ActiveX objects (Internet Explorer
only).

The \textbf{OpenTextFile} method opens a specified file and returns a
TextStream object that can be used to read from, write to, or append to
the file. The syntax is:

object.OpenTextFile(filename{[}, iomode{[}, create{[}, format{]}{]}{]})

where,

•

\emph{object}: is always the name of a FileSystemObject.

•

\emph{filename}: String expression that identifies the file to open.

•

\emph{iomode}: Can be one of three constants: ForReading (1), ForWriting
(2), or ForAppending (8).

•

\emph{create}: Boolean value that indicates whether a new file can be
created if the specified filename doesn\textquotesingle t exist. The
value is True if a new file is created, False if it
isn\textquotesingle t created. If omitted, a new file
isn\textquotesingle t created.

•

\emph{format}: One of three Tristate values used to indicate the format
of the opened file. If omitted, Game Programming -- Penn Wu

172

\protect\hypertarget{index_split_011.htmlux5cux23p173}{}{}the file is
opened as ASCII.

The \textbf{CreateTextFile} method Creates a specified file name and
returns a TextStream object that can be used to read from or write to
the file. The syntax is:

object.CreateTextFile(filename{[}, overwrite{[}, unicode{]}{]})

where,

•

\emph{object}: the name of a FileSystemObject or Folder object.

•

\emph{filename}: String expression that identifies the file to create.

•

\emph{overwrite}: Boolean value that indicates whether you can overwrite
an existing file. The value is true if the file can be overwritten,
false if it can\textquotesingle t be overwritten. If omitted, existing
files are not overwritten.

•

\emph{unicode}: Boolean value that indicates whether the file is created
as a Unicode or ASCII file.

The value is true if the file is created as a Unicode file, false if
it\textquotesingle s created as an ASCII file. If omitted, an ASCII file
is assumed.

Consider the following code, which illustrates how to use the
CreateTextFile method to create and open a text file.

var fso = new ActiveXObject("Scripting.FileSystemObject"); var a =
fso.CreateTextFile("c:\textbackslash\textbackslash testfile.txt", true);
a.WriteLine("This is a test.");

a.Close();

The following code illustrates the use of the OpenTextFile method to
open a file for appending text: var fs, a, ForAppending;

ForAppending = 8;

fs = new ActiveXObject("Scripting.FileSystemObject"); a =
fs.OpenTextFile("c:\textbackslash\textbackslash testfile.txt",
ForAppending, false);

...

a.Close();

You can simplify the above code to:

var fs, a;

fs = new ActiveXObject("Scripting.FileSystemObject"); a =
fs.OpenTextFile("c:\textbackslash\textbackslash testfile.txt",
\textbf{8}, false);

...

a.Close();

Consider the following code.

\textless html\textgreater\textless head\textgreater{}

\textless SCRIPT
LANGUAGE=\textquotesingle JavaScript\textquotesingle\textgreater{}

var scr=0;

function WriteToFile() \{

var filename = \textquotesingle c://tmp//score.txt\textquotesingle;

var fso = new
ActiveXObject(\textquotesingle Scripting.FileSystemObject\textquotesingle);

if (fso.FileExists(filename)) \{

var a, file;

file = fso.OpenTextFile(filename, 2, false);

file.WriteLine(scr);

\}

else \{

var file = fso.CreateTextFile(filename, true);

file.WriteLine(scr);

\}

Game Programming -- Penn Wu

173

\protect\hypertarget{index_split_011.htmlux5cux23p174}{}{} file.Close();

scr+=5;

\}

function ReadIt() \{

var filename = \textquotesingle c://tmp//score.txt\textquotesingle;

var fso, a, ForReading;

fso = new
ActiveXObject(\textquotesingle Scripting.FileSystemObject\textquotesingle);

file = fso.OpenTextFile(filename, 1, false);

p1.innerText = file.readline();

file.Close();

\}

\textless/SCRIPT\textgreater{}

\textless/head\textgreater{}

\textless body
onload=\textquotesingle WriteToFile();ReadIt()\textquotesingle\textgreater{}

\textless button
onClick=\textquotesingle WriteToFile();ReadIt()\textquotesingle\textgreater{}
Click \textless/button\textgreater{}

\textless p id="p1"\textgreater\textless/p\textgreater{}

\textless/body\textgreater{}

\textless/html\textgreater{}

Keeping score

As the game programmer, you have the power of defining what the game
rules are. However, you based on game

need to keep the score based on the game rule, and you need to
absolutely stick to the rules. In the rules

Slot Machine game, for example, the rules are:

•

If 4 numbers are all 7's, win 30000 points.

•

If there are any 3 numbers having 7 as values, win 3000 points.

•

If there are any 3 numbers having exactly the same values, win 300
points.

•

If there are any 2 numbers having exactly the same value, assign 0
point.

•

If all the 4 numbers are all different, lose 30 points.

The \textbf{check()} code is responsible for validating the above game
rules and keep score based on the rules.

function check() \{

clearTimeout(s1);

var v1 = f1.n1.value;

var v2 = f1.n2.value;

var v3 = f1.n3.value;

var v4 = f1.n4.value;

\textbf{if ((v1 == 7) \&\& (v2 == 7) \&\& (v3 == 7) \&\& (v4==7))}

\textbf{\{gd += 30000; score.innerText = 30000;\}}

\textbf{}

\textbf{else if (((v1 == 7) \&\& (v2 == 7) \&\& (v3 == 7))
\textbar\textbar{}}

\textbf{((v1 == 7) \&\& (v2 == 7) \&\& (v4 == 7)) \textbar\textbar{}}

\textbf{((v2 == 7) \&\& (v3 == 7) \&\& (v4 == 7)))}

\textbf{\{gd += 3000; score.innerText = 3000;\}}

\textbf{}

\textbf{else if (((v1 == v2) \&\& (v2 == v3)) \textbar\textbar{}}

\textbf{((v1 == v2) \&\& (v2 == v4)) \textbar\textbar{}}

\textbf{((v2 == v3) \&\& (v3 == v4))) \{ gd += 300;}

\textbf{score.innerText = 300; \}}

\textbf{}

\textbf{else if ((v1 == v2) \textbar\textbar{} (v1 == v3)
\textbar\textbar{}}

\textbf{(v1 == v4) \textbar\textbar{} (v2 == v3) \textbar\textbar{}}

\textbf{(v2 == v4) \textbar\textbar{} (v3 == v4))}

\textbf{\{gd += 0; score.innerText = 0;\}}

Game Programming -- Penn Wu

174

\protect\hypertarget{index_split_011.htmlux5cux23p175}{}{}\textbf{else
\{gd -=30; score.innerText = "-30";\}}

credit.innerText = gd;

\}

\textless/script\textgreater{}

The following line, for example, defines the rule ``If 4 numbers are all
7's,\ldots''

\textbf{if ((v1 == 7) \&\& (v2 == 7) \&\& (v3 == 7) \&\& (v4==7))} The
following detects if there are any 3 variables having the value 7:
\textbf{if (((v1 == 7) \&\& (v2 == 7) \&\& (v3 == 7))
\textbar\textbar{}}

\textbf{((v1 == 7) \&\& (v2 == 7) \&\& (v4 == 7)) \textbar\textbar{}}

\textbf{((v2 == 7) \&\& (v3 == 7) \&\& (v4 == 7)))}

The following detects if there are any 3 variables having exactly the
same values (any possible value will work):

\textbf{if (((v1 == v2) \&\& (v2 == v3)) \textbar\textbar{}}

\textbf{((v1 == v2) \&\& (v2 == v4)) \textbar\textbar{}}

\textbf{((v2 == v3) \&\& (v3 == v4)))}

\textbf{}

The following detects if there are any two variables having the same
values (any possible value will work):

\textbf{if ((v1 == v2) \textbar\textbar{} (v1 == v3) \textbar\textbar{}}

\textbf{(v1 == v4) \textbar\textbar{} (v2 == v3) \textbar\textbar{}}

\textbf{(v2 == v4) \textbar\textbar{} (v3 == v4))}

In the Pitching and Catching Baseball game, the game rules are:

•

If the catcher uses his glove to catch the ball, the play wins 10
points.

•

If the catcher does not catch the ball, the player loses 10 points.

•

If the base ball moves out of the body area of the browser, the player
get 5 points for bad pitching.

function play() \{

if ( bb.style.pixelLeft \textgreater= document.body.clientWidth-50) \{

bb.style.display="none"; score-=10; p1.innerText=score;\}

else \{

if ((bb.style.pixelLeft \textgreater= ct.style.pixelLeft + 50) \&\&
(bb.style.pixelTop \textless= ct.style.pixelTop + 20) \&\&

(bb.style.pixelTop \textgreater= ct.style.pixelTop)) \{

bb.style.display="none"; score+=10; p1.innerText=score;

\}

else if ((bb.style.pixelTop \textgreater= document.body.clientHeight-20)
\textbar\textbar{}

(bb.style.pixelTop\textless=0)) \{

bb.style.display="none"; score+=5; p1.innerText=score;
msg.innerText="Bad pitch, you got 5 points!";

\}

else \{

bb.style.pixelLeft+=10;

bb.style.pixelTop+=Math.tan(i);

s1 = setTimeout("play()",50);

\}

\}

\}

The following line simply means the baseball move over the catcher's
defending zone. In other words, the catcher does not catch the ball. The
player loses 10 points.

Game Programming -- Penn Wu

175

\protect\hypertarget{index_split_011.htmlux5cux23p176}{}{}\includegraphics{index-176_1.png}

if ( bb.style.pixelLeft \textgreater= document.body.clientWidth-50) \{

bb.style.display="none"; score-=10; p1.innerText=score;\}

The following detects if the baseball moves through the glove area, as
shown below.

if ((bb.style.pixelLeft \textgreater= ct.style.pixelLeft + 50) \&\&
(bb.style.pixelLeft \textless= ct.style.pixelLeft + 77) \&\&

(bb.style.pixelTop \textless= ct.style.pixelTop + 20) \&\&

(bb.style.pixelTop \textgreater= ct.style.pixelTop)) \{

bb.style.display="none"; score+=10; p1.innerText=score;

\}

If the above collision detection returns ``True'', the player wins 10
points.

The following code determines if the base ball moves out of the body
area of the browser. If so, the player gets 5 points and the screen
displays ``Bad pitch, you got 5 points!''

else if ((bb.style.pixelTop \textgreater= document.body.clientHeight-20)
\textbar\textbar{}

(bb.style.pixelTop\textless=0)) \{

bb.style.display="none"; score+=5; p1.innerText=score;
msg.innerText="Bad pitch, you got 5 points!";

\}

Review

1. Keeping score for a game player is a task \_\_.

Questions

A. of incrementing or decrementing the value of a variable

B. that requires the player to manually enter the score points to a
textbox.

C. that requires the player to buy a USB scoreboard device and plugs it
in to the computer.

D. All of the above

2. You need to add 10 points to your scoreboard function of your game,
which would you use?

A. score = score ++ 10;

B. score = score += 10;

C. score += 10;

D. score ++ 10;

3. You need to deduct 1 point from your scoreboard function of your
game, which would you use?

A. score -=;

B. -\/-score;

C. score -= score;

D. score -\/- score;

4. Given the following code segment, which variable keeps a string
literal?

var width = "20";

var height = 17;

var length = 129 + 11;

var area = height * 19;

Game Programming -- Penn Wu

176

\protect\hypertarget{index_split_011.htmlux5cux23p177}{}{}A. width

B. height

C. length

D. area

5. Given the following code segment, which statement is correct?

if (f1.t1.value==f1.t2.value) \{ .. \}

..........

\textless form name="f1"\textgreater{}

..........

\textless input type="text" name="t1"\textgreater{}

\textless input type="hidden" name="t2"\textgreater{}

\textless/form\textgreater{}

..........

A. f1.t1.value represents the value that is entered to the t1 textbox.

B. t2 is an invisible textbox inside the f1 HTML form.

C. it checks if the value of t1 equals to the value of t2.

D. All of the above

6. Given the following code segment, which is the value that will be
displayed in the p1 object?

var score = 0;

score += 50;

score-\/-;

score -= 5;

p1.innerText = score;

A. 50

B. 49

C. 44

D. 0

7. Given the following code segment, how many image file(s) will pop up?

\textless script\textgreater{}

function init() \{

for (i=1; i\textless=4; i++) \{

code1 = "\textless img src=b.gif\textgreater";

area1.innerHTML = code1;

\}

\}

\textless/script\textgreater{}

\textless body onLoad=init()\textgreater{}

\textless div id="area1"\textgreater\textless/div\textgreater{}

\textless/body\textgreater{}

A. 1

B. 2

C. 3

D. 4

8. Given the following code segment, how many image file(s) will pop up?

\textless script\textgreater{}

function init() \{

for (i=1; i\textless=4; i++) \{

code1 = "\textless img src=b.gif\textgreater";

\}

Game Programming -- Penn Wu

177

\protect\hypertarget{index_split_011.htmlux5cux23p178}{}{}
area1.innerHTML += code1;

\}

\textless/script\textgreater{}

\textless body onLoad=init()\textgreater{}

\textless div id="area1"\textgreater\textless/div\textgreater{}

A. 1

B. 2

C. 3

D. 4

9. Given the following code segment, how many image file(s) will pop up?

\textless script\textgreater{}

function init() \{

var code1 = "";

for (i=1; i\textless=4; i++) \{

code1 += "\textless img src=b.gif\textgreater";

\}

area1.innerHTML = code1;

\}

\textless/script\textgreater{}

\textless body onLoad=init()\textgreater{}

\textless div id="area1"\textgreater\textless/div\textgreater{}

A. 1

B. 2

C. 3

D. 4

10. The following code literally means "\_\_".

if ((v1 == 7) \&\& (v2 == 7) \&\& (v3 == 7) \&\& (v4==7)) A. If 4
numbers are all 7's

B. If there are any 3 numbers having 7 as values

C. If there are any 2 numbers having 7 as values

D. If there is any 1 number having 7 as value

Game Programming -- Penn Wu

178

\protect\hypertarget{index_split_011.htmlux5cux23p179}{}{}

Lab \#10

Score boards

\textbf{Preparation \#1:}

1. Create a new directory named \textbf{C:\textbackslash games}.

2. Use Internt Explorer to go to
\textbf{http://business.cypresscollege.edu/\textasciitilde pwu/cis261/download.htm}
to download lab10.zip (a zipped) file. Extract the files to
C:\textbackslash games directory.

\textbf{Learning Activity \#1: Guessing Number Game}

Note: This game is meant to be simple and easy for the sake of
demonstrating programming concepts. Please do not hesitate to enhance
the appearance or functions of this game.

1. Change to the C:\textbackslash games directory.

2. Use Notepad to create a new file named
\textbf{C:\textbackslash games\textbackslash lab10\_1.htm} with the
following contents:

\textless html\textgreater{}

\textless style\textgreater{}

span \{border:solid 2 black; width:30; height:30;

font-size:20px; text-align:center; color:white;

background-color:green;\}

\textless/style\textgreater{}

\textless script\textgreater{}

var score=0;

function init() \{

code1="";

for (i=1; i\textless=64; i++) \{

if (i\%8==0) \{

code1+="\textless span
id=n"+i+"\textgreater"+i+"\textless/span\textgreater\textless br\textgreater";\}

else \{ code1+="\textless span
id=n"+i+"\textgreater"+i+"\textless/span\textgreater";\}

\}

area1.innerHTML=code1;

p1.innerText = 0;

\}

function start() \{

var k=Math.floor(Math.random()*64);

f1.t2.value=k;

if (k==0 \textbar\textbar{} k\textgreater64)
\{k=Math.floor(Math.random()*64);\}

for (i=1; i\textless=64; i++) \{

code1="n"+i+".style.backgroundColor=\textquotesingle green\textquotesingle;"

eval(code1)

\}

eval("n"+k+".style.backgroundColor=\textquotesingle red\textquotesingle;");
sf = setTimeout("start()",50);

\}

function stop() \{

clearTimeout(sf);

\textbf{if (f1.t1.value==f1.t2.value) \{ score+=1000; p1.innerText =
score;\}}

\textbf{else \{ score -=1; p1.innerText = score;\}}

\}

\textless/script\textgreater{}

Game Programming -- Penn Wu

179

\protect\hypertarget{index_split_011.htmlux5cux23p180}{}{}\includegraphics{index-180_1.png}

\textless body onLoad=init()\textgreater{}

\textless div id=area1\textgreater\textless/div\textgreater{}

\textless form name=f1\textgreater{}

Enter a number (1-64):

\textless input type=text name=t1 size=5\textgreater{}

\textless input type=hidden name=t2\textgreater{}

\textless/form\textgreater{}

\textless p\textgreater Your Score: \textless b
id=p1\textgreater\textless/b\textgreater\textless/p\textgreater{}

\textless button
onClick="start()"\textgreater Start\textless/button\textgreater{}

\textless button
onClick="stop()"\textgreater Stop\textless/button\textgreater{}

\textless/body\textgreater\textless/html\textgreater{}

3. Test the program. A sample output looks:

\textbf{Learning Activity \#2: Punching duckling}

Note: This game is meant to be simple and easy for the sake of
demonstrating programming concepts. Please do not hesitate to enhance
the appearance or functions of this game.

1. Change to the C:\textbackslash games directory.

2. Use Notepad to create a new file named
\textbf{C:\textbackslash games\textbackslash lab10\_2.htm} with the
following contents:

\textless html\textgreater{}

\textless style\textgreater{}

img \{position: absolute; display:none\}

\textless/style\textgreater{}

\textless script\textgreater{}

var score=0;

function init() \{

for (i=1; i\textless=10; i++) \{

codes="\textless img id=\textquotesingle d"+i+"\textquotesingle{}
src=\textquotesingle duck.gif\textquotesingle{}
style=\textquotesingle left:10; top:10\textquotesingle\textgreater";
area1.innerHTML += codes;

\}

DuckMove()

\}

function DuckMove() \{

var w=Math.floor(document.body.clientWidth / 10);

for (i=1; i\textless=9; i++) \{

code1="if (d"+i+".style.pixelLeft \textgreater= w) "; code1+=" \{
d"+(i+1)+".style.display=\textquotesingle inline\textquotesingle; ";
code1+="d"+(i+1)+".style.pixelLeft += 5; \}"; eval(code1);

\}

Game Programming -- Penn Wu

180

\protect\hypertarget{index_split_011.htmlux5cux23p181}{}{}

if (d1.style.pixelLeft \textgreater= document.body.clientWidth-w)

\{ d1.style.display = "none";\}

else \{
d1.style.display=\textquotesingle inline\textquotesingle;d1.style.pixelLeft
+= 5;\}

for (j=2; j\textless=9; j++) \{

code2="if (d"+j+".style.pixelLeft \textgreater=
document.body.clientWidth-w)"; code2+=" \{ d"+j+".style.display =
\textquotesingle none\textquotesingle;\}"; eval(code2);

\}

if (d10.style.pixelLeft \textgreater= document.body.clientWidth-w)

\{ d10.style.display = \textquotesingle none\textquotesingle;
gameOver();\}

else \{ s1 = setTimeout("DuckMove()", 100); \}

\}

function gameOver() \{

clearTimeout(s1);

alert("Game Over!");

\}

function GloveMove() \{

var e=event.keyCode;

switch (e) \{

case 37:

gv.style.pixelLeft-=2;

break;

case 39:

gv.style.pixelLeft+=2;

break;

case 83:

punch()

break;

\}

\}

function punch() \{

gv.style.pixelLeft+=2;

gv.style.pixelTop-=2;

if (gv.style.pixelTop \textless=0)

\{clearTimeout(s2);gv.style.pixelTop=200;gv.style.pixelLeft=10;\}

else \{s2=setTimeout("punch()", 20);\}

for (i=1; i\textless=10; i++) \{

code3="if ((gv.style.pixelLeft \textgreater= d"+i+".style.pixelLeft)
\&\&"; code3+=" (gv.style.pixelLeft \textless= (d"+i+".style.pixelLeft +
61)) \&\& "; code3+=" (gv.style.pixelTop \textless=
d"+i+".style.pixelTop)) ";
code3+="\{d"+i+".style.display=\textquotesingle none\textquotesingle;d"+i+".src=\textquotesingle noduck.gif\textquotesingle;"

code3+="score+=10;p1.innerText=score\}";

eval(code3);

\}

\}

\textless/script\textgreater{}

\textless body onLoad=init() onKeyDown=GloveMove()
bgcolor="white"\textgreater{}

\textless div id="area1"
style="width:100\%"\textgreater\textless/div\textgreater{}

\textless img id=\textquotesingle gv\textquotesingle{} src="glove.gif"
style="left:10;top:200;display:\textquotesingle inline\textquotesingle"
width=50\textgreater{}

\textless p style="position:absolute; top:300"\textgreater Your score:
\textless b
id=p1\textgreater\textless/b\textgreater\textless/p\textgreater{}

\textless/body\textgreater\textless/html\textgreater{}

Game Programming -- Penn Wu

181

\protect\hypertarget{index_split_011.htmlux5cux23p182}{}{}\includegraphics{index-182_1.png}

3. Test the program. Press S to punch. Press → or ← to move to left or
right. A sample output looks: \textbf{Learning Activity \#3: Shooting
Crab Game}

Note: This game is meant to be simple and easy for the sake of
demonstrating programming concepts. Please do not hesitate to enhance
the appearance or functions of this game.

1. Change to the C:\textbackslash games directory.

2. Use Notepad to create a new file named
\textbf{C:\textbackslash games\textbackslash lab10\_3.htm} with the
following contents:

\textless html\textgreater{}

\textless style\textgreater{}

.bee \{position:relative; z-index: 1\}

\textless/style\textgreater{}

\textless script\textgreater{}

var score=0;

function moveLeft() \{

leftSide = area1.style.pixelLeft + 32;

if (leftSide \textgreater= 300) \{

if (area1.style.pixelTop \textgreater= 200) \{

clearTimeout(sf);

\}

else \{

area1.style.pixelTop += 20;

clearTimeout(sf); moveRight();

\}

\}

else \{

area1.style.pixelLeft += 5;

sf = setTimeout("moveLeft()", 50);

\}

\}

function moveRight() \{

if (area1.style.pixelLeft == 0) \{

if (area1.style.pixelTop \textgreater= 200) \{

clearTimeout(sr);

\}

else \{

area1.style.pixelTop += 20;

clearTimeout(sr); moveLeft();

\}

\}

else \{

area1.style.pixelLeft -= 5;

sr = setTimeout("moveRight()", 50);

\}

\}

Game Programming -- Penn Wu

182

\protect\hypertarget{index_split_011.htmlux5cux23p183}{}{}function
shoot() \{

var e=event.keyCode;

switch(e) \{

case 37:

shooter.style.pixelLeft-=2; break;

case 39:

shooter.style.pixelLeft+=2; break;

case 83:

b1.style.pixelLeft = shooter.style.pixelLeft + 14;

b1.style.display = \textquotesingle inline\textquotesingle;

fire();

\}

\}

function fire() \{

if (b1.style.pixelTop == 0) \{

b1.style.display=\textquotesingle none\textquotesingle;

b1.style.top = 270;

clearTimeout(ss);

\}

else \{

b1.style.pixelTop -= 5;

if ((b1.style.pixelTop \textless= area1.style.pixelTop + 20) \&\&
(b1.style.pixelTop \textgreater=area1.style.pixelTop)) \{

if ((b1.style.pixelLeft \textgreater= area1.style.pixelLeft) \&\&
(b1.style.pixelLeft \textless= area1.style.pixelLeft + 31))

\{

c1.style.display=\textquotesingle none\textquotesingle;b1.style.display="none";

clearTimeout(ss); \textbf{score += 10;}

\textbf{gd.innerText = score;} newCrab() \}

\}

ss = setTimeout("fire()", 20);

\}

\}

function newCrab() \{

area1.style.pixelLeft = 0;

area1.style.pixelTop = 0;

c1.style.display=\textquotesingle inline\textquotesingle;

if (c1.style.backgroundColor==\textquotesingle yellow\textquotesingle)
\{c1.style.backgroundColor=\textquotesingle red\textquotesingle;\}

else if (c1.style.backgroundColor==\textquotesingle red\textquotesingle)
\{c1.style.backgroundColor=\textquotesingle blue\textquotesingle;\}

else if
(c1.style.backgroundColor==\textquotesingle blue\textquotesingle)
\{c1.style.backgroundColor=\textquotesingle orange\textquotesingle;\}

else if
(c1.style.backgroundColor==\textquotesingle orange\textquotesingle)
\{c1.style.backgroundColor=\textquotesingle white\textquotesingle;\}

else if
(c1.style.backgroundColor==\textquotesingle white\textquotesingle)
\{c1.style.backgroundColor=\textquotesingle yellow\textquotesingle;\}

\}

\textless/script\textgreater{}

\textless body onLoad=moveLeft() onKeyDown=shoot()\textgreater{}

\textless div style="position:absolute; width:300; height:300;
background-color: black"\textgreater{}

\textless span id="area1" style="position:absolute;"\textgreater{}

\textless img class=\textquotesingle bee\textquotesingle{} id=c1
src=\textquotesingle crab.gif\textquotesingle{}
style="background-color:yellow" /\textgreater{}

\textless/span\textgreater{}

\textless span id="b1" style="position:absolute; top:270; display:none;
color:white; z-index:2"\textgreater!\textless/span\textgreater{}

\textless span id="shooter" style="position:absolute; top:275;
color:white"\textgreater\textless b\textgreater\_\textbar\_\textless/b\textgreater\textless/span\textgreater{}

\textless/div\textgreater{}

\textless p style="position:absolute;top: 350"\textgreater Your score:
\textless b
id=gd\textgreater\textless/b\textgreater\textless/p\textgreater{}

\textless/body\textgreater{}

\textless/html\textgreater{}

Game Programming -- Penn Wu

183

\protect\hypertarget{index_split_011.htmlux5cux23p184}{}{}\includegraphics{index-184_1.png}

3. Test the program. A sample output looks:

\textbf{Learning Activity \#4: Slot Machine}

Note: This game is meant to be simple and easy for the sake of
demonstrating programming concepts. Please do not hesitate to enhance
the appearance or functions of this game.

1. Change to the C:\textbackslash games directory.

2. Use Notepad to create a new file named
\textbf{C:\textbackslash games\textbackslash lab10\_4.htm} with the
following contents:

\textless html\textgreater{}

\textless style\textgreater{}

.box \{position:relative; border:solid 5 black;

font-size:72; font-family:arial; text-align:center;

width:100; height:100\}

\textless/style\textgreater{}

\textless script\textgreater{}

var gd=0;

function init() \{

codes="";

for (i=1; i\textless=4; i++) \{

codes+="\textless input type=text name=n"+i+"
class=\textquotesingle box\textquotesingle{} /\textgreater";

\}

area2.innerHTML=codes;

\}

function spin() \{

for (i=1; i\textless=4; i++) \{

code2="f1.n"+i+".value = Math.floor(Math.random()*10);"; eval(code2);

\}

s1 = setTimeout("spin()", 50);

\}

function check() \{

clearTimeout(s1);

var v1 = f1.n1.value;

var v2 = f1.n2.value;

var v3 = f1.n3.value;

var v4 = f1.n4.value;

if ((v1 == 7) \&\& (v2 == 7) \&\& (v3 == 7) \&\& (v4==7)) Game
Programming -- Penn Wu

184

\protect\hypertarget{index_split_011.htmlux5cux23p185}{}{}\includegraphics{index-185_1.png}

\{gd += 30000; score.innerText = 30000;\}

else if (((v1 == 7) \&\& (v2 == 7) \&\& (v3 == 7)) \textbar\textbar{}

((v1 == 7) \&\& (v2 == 7) \&\& (v4 == 7)) \textbar\textbar{}

((v2 == 7) \&\& (v3 == 7) \&\& (v4 == 7)))

\{gd += 3000; score.innerText = 3000;\}

else if (((v1 == v2) \&\& (v2 == v3)) \textbar\textbar{}

((v1 == v2) \&\& (v2 == v4)) \textbar\textbar{}

((v2 == v3) \&\& (v3 == v4))) \{ gd += 300; score.innerText = 300; \}

else if ((v1 == v2) \textbar\textbar{} (v1 == v3) \textbar\textbar{}

(v1 == v4) \textbar\textbar{} (v2 == v3) \textbar\textbar{}

(v2 == v4) \textbar\textbar{} (v3 == v4))

\{gd += 0; score.innerText = 0;\}

else \{gd -=30; score.innerText = "-30";\}

credit.innerText = gd;

\}

\textless/script\textgreater{}

\textless body onLoad=init()\textgreater{}

\textless div id="area1" style="position:absolute;
background-color:white; width:400px; height:100px"\textgreater{}

\textless form name="f1"\textgreater{}

\textless span id="area2"
style="position:relative"\textgreater\textless/span\textgreater{}

\textless/form\textgreater{}

\textless p\textgreater{}

Your score: \textless b
id=score\textgreater\textless/b\textgreater\textless br\textgreater{}

Your credit: \textless b
id=credit\textgreater\textless/b\textgreater\textless br\textgreater{}

\textless/p\textgreater{}

\textless p\textgreater\textless button onMouseDown="spin()"
onMouseUp="check()"\textgreater Spin\textless/button\textgreater{}

\textless/div\textgreater{}

\textless/body\textgreater{}

\textless/html\textgreater{}

3. Test the program. Press any key to spin. A sample output looks:
\textbf{Learning Activity \#5: Pitching \& Catching Baseball Game}

Note: This game is meant to be simple and easy for the sake of
demonstrating programming concepts. Please do not hesitate to enhance
the appearance or functions of this game.

1. Change to the C:\textbackslash games directory.

2. Use Notepad to create a new file named
\textbf{C:\textbackslash games\textbackslash lab10\_5.htm} with the
following contents:

\textless html\textgreater{}

Game Programming -- Penn Wu

185

\protect\hypertarget{index_split_011.htmlux5cux23p186}{}{}\textless script\textgreater{}

var i=Math.floor(Math.random()*360);

var score=0;

function init() \{

ct.style.left = document.body.clientWidth-150;

\}

function check() \{

var e=event.keyCode;

switch (e) \{

case 83:

play(); break;

case 38:

ct.style.pixelTop -=5; break;

case 40:

ct.style.pixelTop +=5; break;

\}

\}

function play() \{

if ( bb.style.pixelLeft \textgreater= document.body.clientWidth-50) \{

bb.style.display="none"; score-=10; p1.innerText=score;\}

else \{

if ((bb.style.pixelLeft \textgreater= ct.style.pixelLeft + 50) \&\&
(bb.style.pixelTop \textless= ct.style.pixelTop + 20) \&\&

(bb.style.pixelTop \textgreater= ct.style.pixelTop)) \{

bb.style.display="none"; score+=10; p1.innerText=score;

\}

else if ((bb.style.pixelTop \textgreater= document.body.clientHeight-20)
\textbar\textbar{}

(bb.style.pixelTop\textless=0)) \{

bb.style.display="none"; score+=5; p1.innerText=score;
msg.innerText="Bad pitch, you got 5 points!";

\}

else \{

bb.style.pixelLeft+=10;

bb.style.pixelTop+=Math.tan(i);

s1 = setTimeout("play()",50);

\}

\}

\}

\textless/script\textgreater{}

\textless body onLoad=init(); onKeyDown=check()\textgreater{}

\textless img id="bb" width=10 src="ball.gif" style="position:absolute;
top:400;left:10; z-index:3"\textgreater{}

\textless img id="ct" src="catch.gif" style="position:absolute; top:200;
z-index:1"\textgreater{}

\textless p\textgreater Your score: \textless b
id=p1\textgreater\textless/b\textgreater\textless br\textgreater{}

\textless b
id=msg\textgreater\textless b\textgreater\textless/p\textgreater{}

\textless/body\textgreater{}

\textless/html\textgreater{}

3. Test the program. Press S to pitch. Use ↑ and ↓ keys to move the
catcher. Use the browser's Refresh button to play the game again. Only
when the ball hit the catcher's glove can you get 10 \emph{n} points (n
is an integer); otherwise you lose 10 points. If the ball moves out of
the body area, you get 5 point for bad pitching. A sample output looks:

Game Programming -- Penn Wu

186

\protect\hypertarget{index_split_011.htmlux5cux23p187}{}{}\includegraphics{index-187_1.png}

\includegraphics{index-187_2.png}

\textbf{Submittal}

Upon completing all the learning activities,

1. Upload all files you created in this lab to your remote web server.

2. Log in to to Blackboard, launch Assignment 10, and then scroll down
to question 11.

3. Copy and paste the URLs to the textbox. For example,

•

http://www.geocities.com/cis261/lab10\_1.htm

•

http://www.geocities.com/cis261/lab10\_2.htm

•

http://www.geocities.com/cis261/lab10\_3.htm

•

http://www.geocities.com/cis261/lab10\_4.htm

•

http://www.geocities.com/cis261/lab10\_5.htm

No credit is given to broken link(s).

Game Programming -- Penn Wu

187

\protect\hypertarget{index_split_011.htmlux5cux23p188}{}{}\includegraphics{index-188_1.png}

Lecture \#11

Storyline and Texture

Objective

This lecture is aimed at walking students through some of the basic
knowledge for developing complex games within a restricted coding
environment. Along the way, this lecture provides real codes that will
be useful to everyone who wants to develop basic skills for creating
complex games.

The Art of

A storylines is the setting for the game, including such things as an
introduction to the characters, Storyboarding

their location, and the reason they do what they do. A storyline forms
the spirit of the game.

In the arcade, you frequently see the dancing pad game. It is a game
that uses is a flat electronic game controller for sending inputs to a
dance platform (such as the one used in the arcade version of Dance
Revolution).

Picture of Dance Revolution

To create a computer game as simulation of such game, you have the
freedom to decide the storyline. In the following code example, the
storylines are:

•

Four arrows-\/-left, right, up, and down-\/-are selected randomly, one
at a time.

•

Change the color of the select arrow to red (from black).

•

Detect the player's response using arrow keys on the keyboard.

•

If the player presses the correct key, he/she wins 10 points.

•

Additionally, add an animated dancing girl as 3D texture of this game.

The following codes randomly select a number from 1, 2, 3, and 4, which
represent up, left, right, and down respectively. The
\textbf{switch..case} statement, then determine which arrow's color must
be changed to red. The \textbf{setTimeout()} method executes the
\textbf{init()} function every 500 milliseconds.

function init() \{

clear();

var i=Math.floor(Math.random()*4)+1;

switch (i) \{

case 1:

up.src="up\_red.gif"; k=38;

break;

case 2:

left.src="left\_red.gif"; k=37;

break;

case 3:

right.src="right\_red.gif"; k=39;

break;

case 4:

down.src="down\_red.gif"; k=40;

break;

Game Programming -- Penn Wu

188

\protect\hypertarget{index_split_011.htmlux5cux23p189}{}{}\includegraphics{index-189_1.png}

\}

s1=setTimeout("init()", 500);

\}

The \textbf{keyed()} function checks whether or not the player presses
the correct arrow key on the keyboard by comparing the \textbf{keyCode}
value with the value of a variable \textbf{k}. The init() function
assigns a value to k every time when it is executed.

function keyed() \{

if (event.keyCode==k) \{ score+=10; p1.innerText=score; \}

\}

The \textbf{score} variable keeps score and let the \textbf{innerText}
property insert the updates score to p1. The following line loads an
animated gif file that delivers an effect of dancing girl.

\textless img src="dancinggirl.gif" width=250"\textgreater{}

A complete code of this game will screen a theme like:

As a programmer, it is easy for you to convert your thoughts into flows
of game; however, the players may not be able to handle your game
concepts if you fail to stick to the following unwritten rules:

•

Start with a simple game theme. It is always best not to take risks with
games that are too difficult and instead stick to a design that is
simple and intuitive.

•

Consider your target devices. Computers differ in CPU and memory sizes
and color resolutions. A complicated game may not provide acceptable
performance on all target computers.

•

Develop a good storyline. Create a storyboard for every important screen
of your game, so that you have a rough visual idea of your final
product. And, develop a gripping back-story for supporting the gameplay.
The storyline must be closely tied to the game view (top view, side
view, isometric view), gameplay, game characters, levels, obstacles,
game animations and so on.

•

Focus on the right thing. Make sure that the back-story is not the focal
point of the game. Too many background graphics or too complicate in
texturing will take away the player's attention on the game. Textures
should be created only if it is applicable to the game.

This lecture, from this point on, will focus on how to develop
storylines and create game codes in according to the storylines.

Developing a

Developing a storyline for a game usually consists of a few steps.
First, decide what theme you storyline

want for your story. For example, in the Tetris game, you need to
continuously drop shapes of bricks from the top to bottom. Each shape
can be rotated horizontally or vertically, so you need to define all the
possible shapes.

Your storyline for now will be:

•

Display a square area with 16 cells in it.

•

When the player press any key, make sure all the 16 cells' backgrounds
are in white, and then change the background color of any predefined set
of 4 cells to red.

•

Be ready for the next theme.

Game Programming -- Penn Wu

189

\protect\hypertarget{index_split_011.htmlux5cux23p190}{}{}\includegraphics{index-190_1.png}

\includegraphics{index-190_2.png}

\includegraphics{index-190_3.png}

\includegraphics{index-190_4.png}

\includegraphics{index-190_5.png}

Once the storyline is developed, you can begin writing code for this
part of the entire storylines.

Given the following code. It generates an output of 16 cells with
numbers in each of them.

\textless html\textgreater{}

\textless style\textgreater{}

.cell \{border:solid 1 black;

width:20; height:20;

background-Color: white\}

\textless/style\textgreater{}

\textless script\textgreater{}

function init() \{

var code1="";

for (i=1; i\textless=16; i++) \{

if (i\textbf{\%}4==0) \{

code1 += "\textless span id=c"+i+"
class=\textquotesingle cell\textquotesingle\textgreater"+i+"\textless/span\textgreater{}
\textbf{\textless br\textgreater{}} ";\}

else \{code1 += "\textless span id=c"+i+"
class=\textquotesingle cell\textquotesingle\textgreater"+i+"\textless/span\textgreater";\}

\}

area1.innerHTML = code1;

\}

\textless/script\textgreater{}

\textless body onLoad=init()\textgreater{}

\textless div id="area1"\textgreater\textless/div\textgreater{}

\textless/body\textgreater{}

\textless/html\textgreater{}

The following lines use the modulus operator (\%), which returns the
remainder of two numbers.

if (i\textbf{\%}4==0) \{

code1 += "\textless span id=c"+i+"
class=\textquotesingle cell\textquotesingle\textgreater"+i+"\textless/span\textgreater{}
\textbf{\textless br\textgreater{}} ";\}

else \{code1 += "\textless span id=c"+i+"
class=\textquotesingle cell\textquotesingle\textgreater"+i+"\textless/span\textgreater";\}

The above code uses the \emph{for} loop to create 16 cells, but it only
allows 4 cells each row. The modulus operator in this case helps to
detect if the current value of \emph{i} is a multiple of 4. If so, add

\textless br\textgreater{} to
\textless span\textgreater..\textless/span\textgreater{} to break the
line.

The modulus operator is also commonly used to take a randomly generated
number and reduce that number to a random number on a smaller range, and
it can quickly tell you if one number is a factor of another.

The above code only lays out the background area, which consists of 16
cells with numbers in them. Notice that these numbers are added only to
help you figure out how you can design shapes (as those in the Tetris
game). For example,

Fig 1

Fig 2

Fig 3

Fig 4

Logically speaking:

•

To create the share of Fig 1, you need to change the background color of
cell 1, 2, 3 and 6.

•

To create the share of Fig 2, you need to change the background color of
cell 2, 5, 6 and 7.

•

To create the share of Fig 3, you need to change the background color of
cell 1, 5, 6 and 9.

•

To create the share of Fig 4, you need to change the background color of
cell 2, 5, 6 and 10.

Game Programming -- Penn Wu

190

\protect\hypertarget{index_split_011.htmlux5cux23p191}{}{}

This part of code is included in \textbf{shapes()} function. Variables
\textbf{n1}, \textbf{n2}, \textbf{n3}, and \textbf{n4} are used to
represent the numbers of the 4 picked cells.

\textbf{}

function shapes() \{

clear(); // call the clear() function

var i=Math.floor(Math.random()*4)+1;

switch (i) \{

case 1:

n1=1; n2=2; n3=3; n4=6; break;

case 2:

n1=2; n2=5; n3=6; n4=7; break;

case 3:

n1=1; n2=5; n3=6; n4=9; break;

case 4:

n1=2; n2=5; n3=6; n4=10; break;

\}

code2 =
"c"+n1+".style.backgroundColor=\textquotesingle red\textquotesingle;";
code2 +=
"c"+n2+".style.backgroundColor=\textquotesingle red\textquotesingle;";
code2 +=
"c"+n3+".style.backgroundColor=\textquotesingle red\textquotesingle;";
code2 +=
"c"+n4+".style.backgroundColor=\textquotesingle red\textquotesingle;";
eval(code2);

\}

If case 1 is selected (randomly by the computer), the following code
blocks will become codes as shown in the dashed line area. Consequently,
you see the shape of Fig 1 on screen.

code2 =
"c"+n1+".style.backgroundColor=\textquotesingle red\textquotesingle;";
code2 +=
"c"+n2+".style.backgroundColor=\textquotesingle red\textquotesingle;";
code2 +=
"c"+n3+".style.backgroundColor=\textquotesingle red\textquotesingle;";
code2 +=
"c"+n4+".style.backgroundColor=\textquotesingle red\textquotesingle;";
eval(code2);

c\textbf{1}.style.backgroundColor=\textquotesingle red\textquotesingle;

c\textbf{2}.style.backgroundColor=\textquotesingle red\textquotesingle;

c\textbf{3}.style.backgroundColor=\textquotesingle red\textquotesingle;

c6.style.backgroundColor=\textquotesingle red\textquotesingle;

The \textbf{clear()} function is created to handle the ``\ldots make
sure all the 16 cells' backgrounds are in white\ldots'' part of the
storyline.

function clear() \{

for (i=1; i\textless=16; i++) \{

code3 =
"c"+i+".style.backgroundColor=\textquotesingle white\textquotesingle;";
eval(code3);

\}

\}

By adding the following functions, \textbf{shapes()} and
\textbf{clear()}, you can randomly display a shape in four directions.

The next step is to decide how the theme will be played out. In a real
Tetris game, you don't see the numbers in cells, so modify the following
line to prevent the numbers from being display (Compare with the above
codes).

function init() \{

var code1="";

for (i=1; i\textless=16; i++) \{

Game Programming -- Penn Wu

191

\protect\hypertarget{index_split_011.htmlux5cux23p192}{}{}\includegraphics{index-192_1.png}

\includegraphics{index-192_2.png}

\includegraphics{index-192_3.png}

\includegraphics{index-192_4.png}

if (i\%4==0) \{

\textbf{code1 += "\textless span id=c"+i+"
class=\textquotesingle cell\textquotesingle\textgreater\textless/span\textgreater\textless br\textgreater";\}}

\textbf{else \{code1 += "\textless span id=c"+i+"
class=\textquotesingle cell\textquotesingle\textgreater\textless/span\textgreater";\}}

\}

area1.innerHTML = code1;

\}

The outputs now look:

Fig 1

Fig 2

Fig 3

Fig 4

Now that you have the theme set up, you can establish the protagonist
and antagonist. The player is the protagonist in the Tetris game, while
the computer surely is the antagonist. So, you need to decide the game
rules for scoring, game stages (phases), and levels of difficulty, etc..
For example, you can modify the code of dancing pad game to let players
choose the level of difficulty. Consider the following code,

function init() \{

clear();

var i=Math.floor(Math.random()*4)+1;

switch (i) \{

case 1:

up.src="up\_red.gif"; k=38;

break;

case 2:

left.src="left\_red.gif"; k=37;

break;

case 3:

right.src="right\_red.gif"; k=39;

break;

case 4:

down.src="down\_red.gif"; k=40;

break;

\}

s1=setTimeout("init()", \textbf{500});

\}

Simply change to the re-execution time value from 500 to 1000, the
player will have more time to decide what arrow key to press. When the
value is lowered to 300, the player has less time to respond.

Seeing that you want to the player to select the level of
difficulty-\/-certainly one at a time, you can add a web form to collect
the player's entry.

function init() \{

clear();

var j=f1.t1.value;

switch(j) \{

case 1:

sec = 1000; break;

case 2:

sec = 500; break;

case 3:

Game Programming -- Penn Wu

192

\protect\hypertarget{index_split_011.htmlux5cux23p193}{}{} sec = 300;
break;

\}

var i=Math.floor(Math.random()*4)+1;

switch (i) \{

case 1:

up.src="up\_red.gif"; k=38;

break;

case 2:

left.src="left\_red.gif"; k=37;

break;

case 3:

right.src="right\_red.gif"; k=39;

break;

case 4:

down.src="down\_red.gif"; k=40;

break;

\}

s1=setTimeout("init()", \textbf{j});

\}

..................

..................

\textless form name="f1"\textgreater{}

Level {[}1-3{]}: \textless input type="text" name="t1"
size=3\textgreater\textless br\textgreater{}

\textless button
onclick="init()"\textgreater Start\textless/button\textgreater{}

\textless/form\textgreater{}

..................

In the Tetris game, your storylines eventually must include moving the
shapes downwards. This part of code, for example, can be completed by
adding the following bold lines:

...............

function init() \{

var code1="";

for (i=1; i\textless=\textbf{64}; i++) \{ // change the maximum to 64

................

................

code2 =
"c"+n1+".style.backgroundColor=\textquotesingle red\textquotesingle;";
code2 +=
"c"+n2+".style.backgroundColor=\textquotesingle red\textquotesingle;";
code2 +=
"c"+n3+".style.backgroundColor=\textquotesingle red\textquotesingle;";
code2 +=
"c"+n4+".style.backgroundColor=\textquotesingle red\textquotesingle;";
eval(code2);

\textbf{st1=setInterval("moveDown()", 1000);}

\textbf{\}}

\textbf{}

function clear() \{

for (i=1; i\textless=\textbf{64}; i++) \{ // change the maximum to 64

code3 =
"c"+i+".style.backgroundColor=\textquotesingle white\textquotesingle;";
eval(code3);

\}

\}

\textbf{}

\textbf{function moveDown() \{}

\textbf{clearInterval(st1);}

\textbf{}

\textbf{if ((n1\textgreater=60) \textbar\textbar{} (n2\textgreater=60)
\textbar\textbar{} (n1\textgreater=60) \textbar\textbar{}
(n2\textgreater=60)) \{}

\textbf{clearTimeout(s1);}

\textbf{\}}

\textbf{else \{}

\textbf{clear();}

\textbf{n1+=4;}

\textbf{n2+=4;}

Game Programming -- Penn Wu

193

\protect\hypertarget{index_split_011.htmlux5cux23p194}{}{}\includegraphics{index-194_1.png}

\includegraphics{index-194_2.png}

\includegraphics{index-194_3.png}

\textbf{n3+=4;}

\textbf{n4+=4;}

\textbf{}

\textbf{code4 =
"c"+n1+".style.backgroundColor=\textquotesingle red\textquotesingle;";}
\textbf{code4 +=
"c"+n2+".style.backgroundColor=\textquotesingle red\textquotesingle;";}
\textbf{code4 +=
"c"+n3+".style.backgroundColor=\textquotesingle red\textquotesingle;";}
\textbf{code4 +=
"c"+n4+".style.backgroundColor=\textquotesingle red\textquotesingle;";}
\textbf{eval(code4);}

\textbf{s1 = setTimeout("moveDown()", 1000);}

\textbf{\}}

\textbf{\}}

So a sample outcome now looks:

Your game will also include storylines that require computer to make
human-like decisions. For example, in a Tic-Tac-Toe game, the player is
the protagonist and the computer will serves the antagonist to play
against the human player. The computer must be able to respond to the
human player's movement with an intention to win the game. A later
lecture will discuss about how you can program by using concepts of
artificial intelligence to make you game code think like a human.

Developing

Before developing a multi-theme game, be sure to develop a linear
sequence of themes with a storylines for

smooth transition between any two of them.

multi-theme

games

In the \textbf{Super Mario} game, Mario needs to overcome two stages to
rescue the princess. You will then need to develop two separated
storylines-\/-one for each stage..

You can create one single program with all the themes in it, or you can
create separated programs on a one-for-each-theme basis. The decision
should be made in according to the complexity of storyline. The more
characters, antagonists, and levels of difficulty the game has, the more
complicated it is.

In the Super Mario Theme 1 game, there is a \textbf{theme2()} function
which tells the browser to change the web page to \textbf{lab11\_4.htm}
(from \textbf{lab11\_3.htm}) when Mario jumps into the chimney.

function theme2() \{

if (h1.style.pixelTop\textgreater=200) \{

Game Programming -- Penn Wu

194

\protect\hypertarget{index_split_011.htmlux5cux23p195}{}{}\includegraphics{index-195_1.png}

\includegraphics{index-195_2.png}

h1.style.display =\textquotesingle none\textquotesingle;

clearTimeout(t1);

\textbf{window.location=\textquotesingle lab11\_4.htm\textquotesingle\}
// change to the lab10\_4.htm file} else \{h1.style.pixelTop +=2;\}

t1= setTimeout("theme2()", 20);

\}

In the Bat Battle game, the storylines consist of two themes. In the
first theme:

•

A bat flies to a castle.

•

When the bat approaches the gate, the gate opens by itself.

In the second theme:

•

A bat flies toward a knight, who is watching the bat closely.

•

When the timing is ready, the knight attacks that flying bat.

Theme 1

Theme 2

These two themes are placed in the same code with
\textless body\textgreater{} and \textless/body\textgreater, as shown
below.

\textless body bgcolor="black" onLoad="init();flyBat()" /\textgreater{}

// theme 1

\textless div id="theme1"\textgreater{}

\textless img id="castle" src="gate\_close.gif"
style="position:absolute; width:150; left:40" \textgreater{}

\textless img id="bat" src="bat.gif" style="position:absolute; width:75;
top:250; left:100"\textgreater{}

\textless/div\textgreater{}

// theme 2

\textless div id="theme2" style="display:none"\textgreater{}

\textless img id="knight" src="knight1.gif"

style="position:absolute;left:40"\textgreater{}

\textless img id="bat2" src="bat.gif" style="position:absolute;
width:75; top:220; left:100"\textgreater{}

\textless/div\textgreater{}

\textless/body\textgreater{}

Each theme is a panel by itself. A panel is a container that can include
many objects. In the above code, \textless div\textgreater{} and
\textless/div\textgreater{} defines the two panels. Objects (such as
image files) belonging to the first theme (the solid line area) are
placed within:

\textless div id="theme1"\textgreater......\textless/div\textgreater{}

Objects that belong to the second theme (the dashed line area) are
placed within:

\textless div id="theme2" \textbf{style="display:none"}
\textgreater........\textless/div\textgreater{} Game Programming -- Penn
Wu

195

\protect\hypertarget{index_split_011.htmlux5cux23p196}{}{}\includegraphics{index-196_1.png}

The second panel (the ID is ``theme2'') is not displayed by default,
because the \textbf{display} property has the value ``\textbf{none}''.
Thus, when the game starts, the player only sees the first theme.

The \textbf{flybat()} function controls the transition between theme 1
and 2, as shown below.

function flyBat() \{

if (bat.style.pixelWidth == 0) \{

\textbf{theme1.style.display = "none";}

\textbf{theme2.style.display = "inline";}

clearTimeout(s2); flyBat2();

\}

else\{

bat.style.pixelWidth-\/-;

bat.style.pixelTop-=2;

s2 = setTimeout("flyBat()", 50);

\}

\}

The logic is simple-\/-set the \textbf{display} property of theme 1 to
``\textbf{none}'', and set the display property of theme to
``\textbf{inline}''. Consequently the first panel is no longer
displayed, the second themes takes the place and is displayed on the
screen.

Noticeably, at the end of the following code block, the
\textbf{flyBat2()} function is triggered, which launch the actions for
theme 2.

if (bat.style.pixelWidth == 0) \{

theme1.style.display = "none";

theme2.style.display = "inline";

clearTimeout(s2); \textbf{flyBat2();}

\}

The arrangement of this multi-theme is technically simple-\/-whatever
belongs to theme 1 is placed in the area for theme1, whatever belongs to
theme 2 is placed in the area of theme 2. Between

\textless script\textgreater{} and \textless/script\textgreater{} tags,
functions that will be used for theme 1 are tied together, while those
for theme 2 are in another batch.

Texturing and

In graphics, the digital representation of the surface of an object is
known as the \textbf{texture} of that texture

graphics. In game programming, the effect of having a texture can be
reached by applying a one-, mapping

two-, or three-dimensional image to a given object as surface. Such
programming skill frequently requires the programmer to define a set of
parameters that determine how visible surface are derived from the
image. For example, by using some pre-designed 3D graphics with colorful
texture as background, the Tetris game will look completely different
(possibly more appealing to players).

In addition to two-dimensional qualities, such as color and brightness,
a texture is also encoded with three-dimensional properties, such as how
transparent and reflective the object is. Once a texture has been
defined, it can be wrapped around any 3-dimensional object. This is
called \textbf{texture} \textbf{mapping}.

Texture mapping is a technique that applies an image onto an
object\textquotesingle s surface as if the image were a decal or
cellophane shrink-wrap. The image is created in texture space, with an (
\emph{x}, \emph{y}, \emph{z}) coordinate Game Programming -- Penn Wu

196

\protect\hypertarget{index_split_011.htmlux5cux23p197}{}{}\includegraphics{index-197_1.png}

system.

To add textures with DHTML, you can apply the so-called
``\textbf{multimedia-style effects}'' (Microsoft names them the
``\textbf{visual filters}'') to standard HTML controls, such as text
containers, images, and other non-window-specified objects.

By combining filters with JavaScript scripting, you can create visually
engaging and interactive games. For example, the \textbf{alpha} filer
can adjust the opacity of the content of the object. Its syntax is:

\textless HTMLTag STYLE=

"filter:progid:DXImageTransform.Microsoft.Alpha(
\emph{Str})"..\textgreater{} where \emph{str} is the string that
specifies one or more properties exposed by the filter.

In the following code, the second image uses the alpha filter to fade
out its right-bottom corner.

\textless img src="101.jpg" alt="without alpha"\textgreater{}

\textless img src="101.jpg" alt="with alpha"

style="filter:progid:DXImageTransform.Microsoft.Alpha(

Opacity=100, FinishOpacity=0, Style=1, StartX=0, FinishX=100,

StartY=0, FinishY=100)"\textgreater{}

Compare the output of these two image file, the second image is
\textbf{textured}.

When writing game codes, you can use the following syntax to dynamically
control the object using visual filters:

\emph{ObjectID}.style.filter =

"progid:DXImageTransform.Microsoft.Alpha(str)";

For example, the following use the \textbf{light} filter to create a
texture effect of a blue light shining on the content of the object.

\textless script\textgreater{}

window.onload=init;

function init() \{

tp101.filters{[}0{]}.addCone(0,0,1,250,499,0,0,255,20,180);

\}

\textless/script\textgreater{}

\textbf{\textless img src="101.jpg" id="tp101"}

\textbf{style="filter:progid:DXImageTransform.Microsoft.light();"\textgreater{}}
Surprisingly this is how this image looks now:

Game Programming -- Penn Wu

197

\protect\hypertarget{index_split_011.htmlux5cux23p198}{}{}\includegraphics{index-198_1.png}

The \textbf{addCone()} method adds a cone light to the Light filter
effect object to cast a directional light on the page. The syntax:

ObjectID.filters{[} \emph{n}{]}.addCone( \emph{iX1, iY1, iZ1, iX2, iY2,
iRed, iGreen,} \emph{iBlue, iStrength, iSpread});

where \emph{n} is an integer that represents a filter in the order of an
array. It is because an object can have more than one filter. The first
filter has a key 0, the second 1, and so on.

All the parameters are:

\emph{iX1}

Required. Integer that specifies the left coordinate of the light
source.

\emph{iY1}

Required. Integer that specifies the top coordinate of the light source.

\emph{iZ1}

Required. Integer that specifies the z-axis level of the light source.

\emph{iX2}

Required. Integer that specifies the left coordinate of the target light
focus.

\emph{iY2}

Required. Integer that specifies the top coordinate of the target light
focus.

\emph{iRed}

Required. Integer that specifies the red value. The value can range from
0

(lowest saturation) to 255 (highest saturation).

\emph{iGreen}

Required. Integer that specifies the green value. The value can range
from 0 (lowest saturation) to 255 (highest saturation).

\emph{iBlue}

Required. Integer that specifies the blue value. The value can range
from 0 (lowest saturation) to 255 (highest saturation).

\emph{iStrength} Required. Integer that specifies the intensity of the
light filter. The value can range from 0 (lowest intensity) to 100
(highest intensity).

\emph{iSpread}

Required. Integer that specifies the angle, or spread, between the
vertical position of the light source and the surface of the object. The
angle can range from 0 to 90 degrees. Smaller angle values produce a
smaller cone of light; larger values produce an oblique oval or circle
of light.

The cone light fades with distance from the target \emph{x} and y
position. The light displays a hard edge at the near edge of its focus
and fades gradually as it reaches its distance threshold.

You can also integrate the item method with \textbf{addCode()} if you
have more than one collections. The item method retrieves an object from
the all collection or various other collections. Its syntax is:
ObjectID.item(vIndex {[}, iSubindex{]})

where \emph{\textbf{vIndex}} is required. It should be an integer or
string that specifies the object or collection to retrieve. If this
parameter is an integer, it is the zero-based index of the object. If
this parameter is a string, all objects with matching name or id
properties are retrieved, and a collection is returned if more than one
match is made.

Also, \emph{\textbf{iSubindex}} is optional. Integer that specifies the
zero-based index of the object to retrieve Game Programming -- Penn Wu

198

\protect\hypertarget{index_split_011.htmlux5cux23p199}{}{}\includegraphics{index-199_1.png}

when a collection is returned.

The syntax to combine \textbf{item} and \textbf{addCode} methods is:
ObjectID.filters.item("DXImageTransform.Microsoft.Light").addCone(

iX1, iY1, iZ1, iX2, iY2, iRed, iGreen, iBlue, iStrength, iSpread); In
the Radar game, the \textbf{radar.gif} is a graphic with concentric
circles and \emph{x}, \emph{y} axis. Its background color is white.

The code uses the \textbf{init()} function to create a radar-like
texture and apply the texture to \textbf{radar.gif} file (its ID is
\textbf{rd}). The \textbf{addCone} method generates the coned light
beam.

function init() \{

\textbf{rd
.filters.item(\textquotesingle DXImageTransform.Microsoft.light\textquotesingle).addCone(}

\textbf{121, 121, 0, Light\_X, Light\_Y, 0, 255, 0, 150, 10);}

...............

\}

................

................

\textless img src="radar.gif" id=rd style="height:242;width:242;
filter:progid:DXImageTransform.Microsoft.light()"\textgreater{}

In the movies, a radar screen normally uses blue as background color, so
let the addAmbient method adds a blue background to the rd object.

function init() \{

...............

\textbf{rd
.filters.item(\textquotesingle DXImageTransform.Microsoft.light\textquotesingle).addAmbient(}

\textbf{0, 0, 255, 80 )}

...............

\}

The \textbf{addAmbient} method adds an ambient light to the Light
filter. Ambient light is non-directional and distributed uniformly
throughout space. Ambient light falling upon a surface approaches from
all directions. The light is reflected from the object independent of
surface location and orientation with equal intensity in all directions.
Its syntax is:

addAmbient(iRed, iGreen, iBlue, iStrength);

where,

•

iRed is an integer that specifies the red value. The value can range
from 0 (lowest saturation) to 255 (highest saturation).

•

iGreen is an integer that specifies the green value. The value can range
from 0 (lowest saturation) to 255 (highest saturation).

•

iBlue is an integer that specifies the blue value. The value can range
from 0 (lowest saturation) to 255 (highest saturation).

•

iStrength is an integer that specifies the intensity of the light
filter. The value can range from 0

(lowest intensity) to 100 (highest intensity). The intensity specified
pertains to the target coordinates.

Game Programming -- Penn Wu

199

\protect\hypertarget{index_split_011.htmlux5cux23p200}{}{}\includegraphics{index-200_1.png}

On the radar screen, red points represent flying objects. To dynamically
add an red point to the rd object, use the \textbf{addPoint} method.

function init() \{

...............

\textbf{rd
.filters.item(\textquotesingle DXImageTransform.Microsoft.light\textquotesingle).addPoint(}

\textbf{PlaneLight\_X, PlaneLight\_Y, 3, 255, 0, 0, 100);}

...............

\}

The \textbf{addPoint} method adds a light source to the Light filter.
The light source originates at a single point and radiates in all
directions.

addPoint(iX, iY, iZ, iRed, iGreen, iBlue, iStrength);

where,

•

iX is an integer that specifies the left coordinate of the light source.

•

iY is an integer that specifies the top coordinate of the light source.

•

iZ is an integer that specifies the z-axis level of the light source.

•

iRed is an integer that specifies the red value. The value can range
from 0 (lowest saturation) to 255 (highest saturation).

•

iGreen is an integer that specifies the green value. The value can range
from 0 (lowest saturation) to 255 (highest saturation).

•

iBlue is an integer that specifies the blue value. The value can range
from 0 (lowest saturation) to 255 (highest saturation).

•

iStrength is an integer that specifies the intensity of the light
filter. The value can range from 0

(lowest intensity) to 100 (highest intensity). The intensity specified
pertains to the target coordinates.

With the above codes, a dynamically created texture is applied to the rd
object (radar.gif file), so it now looks:

Microsoft defined many visual filters and methods to support these
filters. You can visit the Microsoft MSDN site for details about every
filters. As of April, 2007, the URL is
http://msdn.microsoft.com/workshop/author/filter/reference/reference.asp?frame=true
Review

1. Which provides a framework for computer game?

Questions

A. sprite

B. storyboard

C. storyline

D. genre

2. Given the following code segment, which assigns a file source
"flower.gif" to it so the image will appear?

Game Programming -- Penn Wu

200

\protect\hypertarget{index_split_011.htmlux5cux23p201}{}{}

\textless img id="m1"\textgreater{}

A. m1.style.src = "flower.gif"

B. m1.src = "flower.gif"

C. src(m1) = "flower.gif"

D. src.m1 = "flower.gif"

3. Which is the best way to let computer randomly pick a number from 1
to 4?

A. Math.floor(Math.random()*4);

B. Math.floor(Math.random()*4 + 1);

C. Math.floor(Math.random()*4) + 1;

D. Math.floor(Math.random()*4) - 1;

4. Given the following code segment, how will the "area1" object look?

var code1="";

for (i=1; i\textless=15; i++) \{

if (i\%3==0) \{

code1 += "\textless span id=c"+i+"
class=\textquotesingle cell\textquotesingle\textgreater\textless/span\textgreater\textless br\textgreater";\}

else \{code1 += "\textless span id=c"+i+"
class=\textquotesingle cell\textquotesingle\textgreater\textless/span\textgreater";\}

\}

area1.innerHTML = code1;

A. 3 rows 5 columns

B. 5 rows 3 columns

C. 3 rows 3 columns

D. 5 rows 5 columns

5. Given the following code segment, which can change the background
color of the "c1" object to red?

\textless span id="c1"
style="width:100"\textgreater\textless/span\textgreater{} A.
c1.style.backgroundColor=\textquotesingle red\textquotesingle;

B. c1.style.bgcolor=\textquotesingle red\textquotesingle;

C. c1.backgroundColor=\textquotesingle red\textquotesingle;

D. c1.style.bgcolor=\textquotesingle red\textquotesingle;

6. Given the following code segment, if you wish to run the init() code
at a slower but constant speed, you should \_\_.

var spd = 500;

s1=setTimeout("init()", spd);

A. increase the value of spd variable

B. decrease the value of spd variable

C. let the value of spd variable increment by 10 repetitively

D. let the value of spd variable decrement by 10 repetitively

7. Given the following code segment, which statement is correct?

st1=setInterval("init()", 1000);

A. It sets a one-time pause for 1 second and then executes init().

B. It sets a repeating pause every 1 second before it executes init().

C. It sets an infinite pause every 1 second before it executes init().

D. It sets an endless pause every 1 second before it executes init().

Game Programming -- Penn Wu

201

\protect\hypertarget{index_split_011.htmlux5cux23p202}{}{}

8. Which can hide a theme which is included in a pair of \textless div
id="theme1"\textgreater{} and \textless/div\textgreater{} tags?

A. theme1.style.show = "false"

B. theme1.style.hide = "true"

C. theme1.style.display = "none"

D. theme1.style.visible = "none"

9. Which Microsoft visual filter can fade out an image partially?

A. beta

B. sigma

C. fade

D. alpha

10. Given the following code, which value specifies the intensity of the
light filter?

addAmbient( 0, 120, 255, 80 )

A. 0

B. 120

C. 255

D. 80

Game Programming -- Penn Wu

202

\protect\hypertarget{index_split_011.htmlux5cux23p203}{}{}

Lab \#11

Storyline and Texture

\textbf{Preparation \#1:}

1. Create a new directory named \textbf{C:\textbackslash games}.

2. Use Internt Explorer to go to
\textbf{http://business.cypresscollege.edu/\textasciitilde pwu/cis261/download.htm}
to download lab11.zip (a zipped) file. Extract the files to
C:\textbackslash games directory.

\textbf{Learning Activity \#1: Dancing Pad Game}

Note: This game is meant to be simple and easy for the sake of
demonstrating programming concepts. Please do not hesitate to enhance
the appearance or functions of this game.

1. Change to the C:\textbackslash games directory.

2. Use Notepad to create a new file named
\textbf{C:\textbackslash games\textbackslash lab11\_1.htm} with the
following contents:

\textless script\textgreater{}

var score = 0;

var k;

function init() \{

clear();

var i=Math.floor(Math.random()*4)+1;

switch (i) \{

case 1:

up.src="up\_red.gif"; k=38;

break;

case 2:

left.src="left\_red.gif"; k=37;

break;

case 3:

right.src="right\_red.gif"; k=39;

break;

case 4:

down.src="down\_red.gif"; k=40;

break;

\}

s1=setTimeout("init()", 500);

\}

function clear() \{

up.src="up\_black.gif";

left.src="left\_black.gif";

right.src="right\_black.gif";

down.src="down\_black.gif";

\}

function keyed() \{

if (event.keyCode==k) \{ score+=10; p1.innerText=score; \}

\}

\textless/script\textgreater{}

\textless body onLoad=init() onKeyDown=keyed() /\textgreater{}

\textless table\textgreater{}

\textless tr\textgreater\textless td\textgreater{}

\textless img src="dancinggirl.gif" width=250"\textgreater{}

\textless/td\textgreater{}

Game Programming -- Penn Wu

203

\protect\hypertarget{index_split_011.htmlux5cux23p204}{}{}\includegraphics{index-204_1.png}

\textless td\textgreater{}

\textless p\textgreater Your Score: \textless b
id=p1\textgreater\textless/b\textgreater\textless/p\textgreater{}

\textless table border=1\textgreater{}

\textless tr\textgreater\textless td\textgreater\textless/td\textgreater{}

\textless td\textgreater\textless img id="up"
src="up\_black.gif"\textgreater\textless/td\textgreater{}

\textless td\textgreater\textless/td\textgreater\textless/tr\textgreater{}

\textless tr\textgreater\textless td\textgreater\textless img id="left"
src="left\_black.gif"\textgreater\textless/td\textgreater{}

\textless td\textgreater\textless/td\textgreater{}

\textless td\textgreater\textless img id="right"
src="right\_black.gif"\textgreater\textless/td\textgreater\textless/tr\textgreater{}

\textless tr\textgreater\textless td\textgreater\textless/td\textgreater{}

\textless td\textgreater\textless img id="down"
src="down\_black.gif"\textgreater\textless/td\textgreater{}

\textless td\textgreater\textless/td\textgreater\textless/tr\textgreater{}

\textless/td\textgreater{}

\textless/table\textgreater{}

3. Test the program. Press the correct arrow keys on the keyboard as
response to the red arrow on screen. A sample output looks:

\textbf{Learning Activity \#2: Multi-theme game}

Note: This game is meant to be simple and easy for the sake of
demonstrating programming concepts. Please do not hesitate to enhance
the appearance or functions of this game.

1. Change to the C:\textbackslash games directory.

2. Use Notepad to create a new file named
\textbf{C:\textbackslash games\textbackslash lab11\_2.htm} with the
following contents:

\textless html\textgreater{}

\textless script\textgreater{}

// theme 1 functions

function init() \{

if (bat.style.pixelTop \textless= 200) \{

castle.src="gate\_open.gif";

setInterval("castle.src=\textquotesingle gate\_opened.gif\textquotesingle",1600);

clearTimeout(s1);

\}

else \{

s1 = setTimeout("init()", 150);

\}

\}

function flyBat() \{

if (bat.style.pixelWidth == 0) \{

theme1.style.display = "none";

theme2.style.display = "inline";

clearTimeout(s2); flyBat2();

\}

Game Programming -- Penn Wu

204

\protect\hypertarget{index_split_011.htmlux5cux23p205}{}{}\includegraphics{index-205_1.png}

\includegraphics{index-205_2.png}

else\{

bat.style.pixelWidth-\/-;

bat.style.pixelTop-=2;

s2 = setTimeout("flyBat()", 50);

\}

\}

// theme 2 functions

function flyBat2() \{

if (bat2.style.pixelTop \textless= 170) \{

knight.src="knight2.gif";

setInterval("knight.src=\textquotesingle knight1.gif\textquotesingle",
4000);

\}

if (bat2.style.pixelWidth == 0) \{

clearTimeout(s3);

\}

else\{

bat2.style.pixelWidth-\/-;

bat2.style.pixelTop-=2;

s3 = setTimeout("flyBat2()", 50);

\}

\}

\textless/script\textgreater{}

\textless body bgcolor="black" onLoad="init();flyBat()" /\textgreater{}

// theme 1

\textless div id="theme1"\textgreater{}

\textless img id="castle" src="gate\_close.gif"
style="position:absolute; width:150; left:40" \textgreater{}

\textless img id="bat" src="bat.gif" style="position:absolute; width:75;
top:250; left:100"\textgreater{}

\textless/div\textgreater{}

// theme 2

\textless div id="theme2" style="display:none"\textgreater{}

\textless img id="knight" src="knight1.gif"
style="position:absolute;left:40"\textgreater{}

\textless img id="bat2" src="bat.gif" style="position:absolute;
width:75; top:220; left:100"\textgreater{}

\textless/div\textgreater{}

\textless/body\textgreater{}

\textless/html\textgreater{}

3. Test the program. A sample output looks:

Theme 1

Theme 2

Game Programming -- Penn Wu

205

\protect\hypertarget{index_split_011.htmlux5cux23p206}{}{}

\textbf{Learning Activity \#3: Super Mario Theme 1 (this file and
lab11\_4.htm are linked)} Note: This game is meant to be simple and easy
for the sake of demonstrating programming concepts. Please do not
hesitate to enhance the appearance or functions of this game.

1. Change to the C:\textbackslash games directory.

2. Use Notepad to create a new file named
\textbf{C:\textbackslash games\textbackslash lab11\_3.htm} with the
following contents:

\textless html\textgreater{}

\textless script\textgreater{}

var i=0;

var direction = "up";

function keyed() \{

var e=event.keyCode;

switch (e) \{

case 39:

if ((h1.style.pixelLeft + 36 \textless{} b1.style.pixelLeft)
\textbar\textbar{}

(h1.style.pixelLeft \textgreater{} b1.style.pixelLeft + 18)) \{

h1.src="4.gif"; h1.style.pixelLeft+=2;

\}

else \{

h1.style.pixelLeft = b1.style.pixelLeft - 36 ; h1.src="1.gif"; \}

break;

case 83:

jump(); break;

\}

\}

function jump() \{

h1.src="2.gif";

if (i==50) \{ direction = "down"; \}

if (direction=="up") \{ h1.style.pixelTop -=2;

if (i\textgreater=20) \{ h1.style.pixelLeft +=1;\}

i++; \}

else \{h1.style.pixelTop +=2; i-\/-;\}

if (i\textless0) \{clearTimeout(s1);h1.style.pixelTop -=2;

i=0; direction="up"; h1.src="1.gif"\}

else \{

s1 = setTimeout("jump()", 10);

\}

if ((h1.style.pixelLeft + 36 \textgreater= b1.style.pixelLeft) \&\&
(h1.style.pixelLeft + 36 \textgreater= b1.style.pixelLeft + 18) \&\&
(h1.style.pixelTop + 46 \textgreater= b1.style.pixelTop)) \{

clearTimeout(s1);

h1.src="1.gif";

theme2();

\}

\}

function theme2() \{

if (h1.style.pixelTop\textgreater=200) \{

h1.style.display =\textquotesingle none\textquotesingle;

clearTimeout(t1);

\textbf{window.location=\textquotesingle lab11\_4.htm\textquotesingle\}
// change to the lab10\_4.htm file} Game Programming -- Penn Wu

206

\protect\hypertarget{index_split_011.htmlux5cux23p207}{}{}\includegraphics{index-207_1.png}

\includegraphics{index-207_2.png}

else \{h1.style.pixelTop +=2;\}

t1= setTimeout("theme2()", 20);

\}

function keyUp() \{

if (h1.src=="4.gif") \{ h1.src="1.gif"; \}

\}

\textless/script\textgreater{}

\textless body onKeyDown="keyed()" onKeyUp="keyUp()"\textgreater{}

\textless img id="h1" src="1.gif" style="position:absolute; top:200;
left:10"\textgreater{}

\textless img id="b1" src="block.gif" style="position:absolute; top:194;
left:200; width:26;height:50"\textgreater{}

\textless hr size=1 width=400 style="position:absolute;
top:243;"\textgreater{}

\textless/body\textgreater{}

\textless/html\textgreater{}

3. Test the program. Use key to move toward the chimney, and then press
S to jump into the chimney. A sample output looks:

Theme 1

Theme 2 (lab11\_4.htm)

\textbf{Learning Activity \#4: Super Mario Theme 2 (this file is the
second theme of lab11\_3.htm)} Note: This game is meant to be simple and
easy for the sake of demonstrating programming concepts. Please do not
hesitate to enhance the appearance or functions of this game.

1. Change to the C:\textbackslash games directory.

2. Use Notepad to create a new file named
\textbf{C:\textbackslash games\textbackslash lab11\_4.htm} with the
following contents:

\textless html\textgreater{}

\textless style\textgreater{}

.area \{position:absolute\}

\textless/style\textgreater{}

\textless script\textgreater{}

var score=0;

var barNh1="false";

function init() \{

for (i=1; i\textless=4; i++) \{

code1 = "\textless img
src=\textquotesingle brick.gif\textquotesingle\textgreater";

area2.innerHTML += code1;

area3.innerHTML += code1;

\}

moveBar();

\}

function keyed() \{

var e=event.keyCode;

switch (e) \{

Game Programming -- Penn Wu

207

\protect\hypertarget{index_split_011.htmlux5cux23p208}{}{} case 39:

walk();

break;

case 37:

break;

case 83:

clearTimeout(s1);

break;

\}

\}

function walk() \{

w1 = area2.style.pixelLeft + 4*60;

w2 = area3.style.pixelLeft;

w3 = area3.style.pixelLeft + 4*60;

if (h1.style.pixelLeft \textgreater{} w2) \{ barNh1="false";
h1.style.pixelTop = 160;\}

if ((barNh1=="false") \&\& (h1.style.pixelLeft \textgreater= w1) \&\&
(h1.style.pixelLeft \textless= w2))

\{

End(); clearTimeout(s1); \}

else if (h1.style.pixelLeft \textgreater= w3-60)
\{clearTimeout(s1);h1.src="1.gif"; \}

else

\{

h1.src="4.gif";

h1.style.pixelLeft +=2;

s1 = setTimeout("walk()", 20);

\}

\}

function End() \{

if (h1.style.pixelTop + 46 \textless= bar.style.pixelTop) \{

h1.src="1.gif"; barNh1 = "true";

\}

else \{

if (h1.style.pixelTop == area2.style.pixelTop + 88) \{

h1.style.display = "none";

clearTimeout(s2); score -=10;

p1.innerText=score; \}

else \{

h1.src="1.gif";

h1.style.pixelTop += 2;

s2 = setTimeout("End()", 20); \}

\}

\}

function moveBar() \{

if (barNh1 == "true") \{

h1.style.pixelTop = bar.style.pixelTop - 46;

\}

if (bar.style.pixelTop \textless= 10) \{

bar.style.pixelTop = 290;\}

bar.style.pixelTop -= 5;

s3 = setTimeout("moveBar()", 300);

\}

\textless/script\textgreater{}

\textless body onLoad=init() onKeyDown=keyed()\textgreater{}

\textless div id="area1"\textgreater{}

\textless img id="h1" src="1.gif" class="area" style="left:10;
top:160"\textgreater{}

\textless span id="area2" class="area" style="left:10;
top:200"\textgreater\textless/span\textgreater{}

\textless span id="area3" class="area" style="left:300;
top:200"\textgreater\textless/span\textgreater{} Game Programming --
Penn Wu

208

\protect\hypertarget{index_split_011.htmlux5cux23p209}{}{}\includegraphics{index-209_1.png}

\textless span id="area4" class="area" style="left:10; top:289;
background-Color:\#abcdef; width:530;
height:50"\textgreater\textless/span\textgreater{}

\textless img src="p.gif" class="area" style="left:510; top:160"
/\textgreater{}

\textless hr id="bar" class="area" size=10 color="brown"
style="left:250; top:290; width:50;" /\textgreater{}

\textless p\textgreater Your Score: \textless b
id=p1\textgreater\textless/b\textgreater\textless/p\textgreater{}

\textless/div\textgreater{}

\textless/body\textgreater{}

\textless/html\textgreater{}

3. Test the program. Use key to move toward the gap. Wait till the
moving bar comes to an appropriate place.

Use key to step on the bar, and then jump to the second area to rescue
the princess. A sample output looks: \textbf{Learning Activity \#5:
Radar (dynamic texture and texture mapping)} Note: This game is meant to
be simple and easy for the sake of demonstrating programming concepts.
Please do not hesitate to enhance the appearance or functions of this
game.

1. Change to the C:\textbackslash games directory.

2. Use Notepad to create a new file named
\textbf{C:\textbackslash games\textbackslash lab11\_5.htm} with the
following contents:

\textless HTML\textgreater{}

\textless HEAD\textgreater{}

\textless SCRIPT language="JavaScript"\textgreater{}

var Light\_X = 20

var Light\_Y = 20

var Light\_Z = 40

var xInc = 10;

var yInc = 10;

var r = 100;

var deg = 0;

var deg1;

var rad;

var PointAngle;

var PlaneLight\_X = 20

var PlaneLight\_Y = 120

var PlaneLight\_Z = 3

var conversion = (2 * Math.PI)/360

function movefilt()

\{

// Do some basic geometry to convert from Polar coordinates to Cartesian
Game Programming -- Penn Wu

209

\protect\hypertarget{index_split_011.htmlux5cux23p210}{}{}

Light\_X = r + r * Math.cos(deg * conversion);

Light\_Y = r + r * Math.sin(deg * conversion);

deg += 10;

if (deg == 360)

deg = 0;

// Rotate the cone

rd .filters{[}0{]}.moveLight(0, Light\_X, Light\_Y, Light\_Z, 1);

// Figure out where the Plane is in relation to the cone

PointAngle = Math.atan((PlaneLight\_Y - r)/(PlaneLight\_X -
r))/conversion;

// More basic geometry

if ((PlaneLight\_X \textless{} r) \&\& (PlaneLight\_Y \textless{} r))

PointAngle += 180;

if ((PlaneLight\_X \textgreater{} r) \&\& (PlaneLight\_Y \textless{} r))

PointAngle += 360

if ((PlaneLight\_X \textless{} r) \&\& (PlaneLight\_Y \textgreater{} r))

PointAngle += 180

// If the plane is in the cone, update the planes position

if ((deg - 10 \textless= PointAngle) \&\& (PointAngle \textless= deg))
rd .filters{[}0{]}.moveLight(2, PlaneLight\_X, PlaneLight\_Y,
PlaneLight\_Z, 1);

//Do it all again in about 1/10th of a second

mytimeout=setTimeout(\textquotesingle movefilt()\textquotesingle, 100);

\}

function movePlanes()

\{

// Increment the planes position

PlaneLight\_X++;

PlaneLight\_Y++;

// Wrap the plane if it goes off the screen

if (PlaneLight\_Y \textgreater{} 200) PlaneLight\_Y = 0;

if (PlaneLight\_X \textgreater{} 200) PlaneLight\_X = 0;

timeout2 = setTimeout(\textquotesingle movePlanes()\textquotesingle,
500);

\}

function init() \{

rd
.filters.item(\textquotesingle DXImageTransform.Microsoft.light\textquotesingle).addCone(121,121,0,Light\_X,
Light\_Y, 0, 255, 0, 150, 10);

rd
.filters.item(\textquotesingle DXImageTransform.Microsoft.light\textquotesingle).addAmbient(0,0,255,80)
rd
.filters.item(\textquotesingle DXImageTransform.Microsoft.light\textquotesingle).addPoint(PlaneLight\_X,
PlaneLight\_Y, 3, 255, 0, 0, 100);

var x = 0;

movefilt();

movePlanes();

\}

\textless/SCRIPT\textgreater{}

\textless/HEAD\textgreater{}

\textless BODY onload="init()" TOPMARGIN=10 LEFTMARGIN=10
bgcolor="white"\textgreater{} Game Programming -- Penn Wu

210

\protect\hypertarget{index_split_011.htmlux5cux23p211}{}{}\includegraphics{index-211_1.png}

\textless img src="radar.gif" id=rd style="height:242;width:242;
filter:progid:DXImageTransform.Microsoft.light()"\textgreater{}

\textless/BODY\textgreater{}

\textless/HTML\textgreater{}

3. Test the program. A sample output looks:

\textbf{Submittal}

Upon completing all the learning activities,

1. Upload all files you created in this lab to your remote web server.

2. Log in to to Blackboard, launch Assignment 11, and then scroll down
to question 11.

3. Copy and paste the URLs to the textbox. For example,

•

http://www.geocities.com/cis261/lab11\_1.htm

•

http://www.geocities.com/cis261/lab11\_2.htm

•

http://www.geocities.com/cis261/lab11\_3.htm

•

http://www.geocities.com/cis261/lab11\_4.htm

•

http://www.geocities.com/cis261/lab11\_5.htm

No credit is given to broken link(s).

Game Programming -- Penn Wu

211

\protect\hypertarget{index_split_011.htmlux5cux23p212}{}{}

Lecture \#12

Applying artificial intelligence

Concepts

Artificial intelligence (AI) is a relatively new subject in computer
sciences, but it has been used in game programming for decades. AI
refers to the programming techniques that make the computers emulate the
human thought processes. The statement ``applying artificial
intelligence to a game`` is simply the attempt to build a human-like
decision making systems in a game.

However, human thought is very complicated in terms of process. Even
brain scientists do not know much about how human brains function, so
the level of AI technology is still in its early infancy.

Many games, or even a section of storylines, require characters to be
``smart'' (or smart enough to respond to the changing condition. In the
second theme of Castle game, when the bat approaches the knight, the bat
(with its animal instincts) should know the knight is ready to hit it
down. So the bat will keep a distance from the knight. On the other
hand, the knight should be \textbf{smart} enough to move forward to a
place where he can hit down the bat.

Your can add the following codes for this scenario:

function hit() \{

if ( knight.style.pixelTop \textgreater=70) \{

clearTimeout(s2);

setInterval("knight.src=\textquotesingle knight1.gif\textquotesingle",
2000);

\}

else \{

knight.style.pixelTop+=2;

setInterval("bat2.style.display=\textquotesingle none\textquotesingle",
2000);

s2 = setTimeout("hit()", 20);

\}

\}

By doing so, both characters-\/-bat and knight-\/-have more liveliness,
because their actions and behaviors are much closer to animal and human.

Information-

Most traditional AI techniques use a variety of information-based
algorithms to make decisions, based

just as people use a variety of previous experiences and mental rules to
make a decision. The algorithms

information-based AI algorithms were completely deterministic, which
means that every decision could be traced back to a predictable flow of
logic. This type of AI technique is the simplest way to apply artificial
intelligence.

Consider the following example, the two fishes swim toward each other,
and they will soon face to each other.

\textless html\textgreater{}

\textless style\textgreater{}

.fish \{position:absolute;top:100;\}

\textless/style\textgreater{}

\textless script\textgreater{}

function init() \{

f1.style.pixelLeft=10;

f2.style.pixelLeft=document.body.clientWidth-100;

move();

\}

function move() \{

if (f1.style.pixelLeft+100 \textgreater= f2.style.pixelLeft) \{

Game Programming -- Penn Wu

212

\protect\hypertarget{index_split_011.htmlux5cux23p213}{}{}\includegraphics{index-213_1.png}

clearTimeout(s1);\}

else \{

f1.style.pixelLeft+=2;

f2.style.pixelLeft-=2;

s1=setTimeout("move()", 20);

\}

\}

\textless/script\textgreater{}

\textless body onLoad=init()\textgreater{}

\textless img id="f1" src="fish1.gif" class="fish"\textgreater{}

\textless img id="f2" src="fish2.gif" class="fish"\textgreater{}

\textless/body\textgreater{}

\textless/html\textgreater{}

Assuming that fish1 is the predator of fish2, fish2's biological
mechanism will force fish2 to avoid crashing into fish1. You can
simulate such biological mechanism by adding the following code blocks:

function avoid() \{

if (f2.style.pixelTop \textgreater= document.body.clientHeight - 100) \{

clearTimeout(s2); \}

else \{

f2.style.pixelTop +=10;

f2.style.pixelLeft-=2;

f1.style.pixelLeft+=2;

s2=setTimeout("avoid()",10);

\}

\}

In this example, the \textbf{if..then} statement formulates a very
simple information-based AI, which helps the computer to make decisions.
However, most games require a much complicated decision making system,
especially when the computer must play the role of antagonist.

Obviously, the way you do your thinking is not never this simple and
predictable. Using the deterministic approach to program games will
never make the game think like a human. You can only try to apply as
many strategies as you can to let the computer ``understand'' the
condition, and then ``find'' the best solution to make decisions.

Developing a

In a chess game, you fight against computer. As a human, your decisions
can result from a large-scale

combination of past experience, personal bias, or the current state of
emotion in addition to the information-completely logical decision
making process. So, there is technically no way to precisely predict
based AI for

how you will play the chess.

games

Luckily, there are strict rules a player must follow in order to play
chess in an acceptable way.

People can develop game strategies (which are basically the result of
past experiences) and use them to make the best move against the
antagonist.

As an AI-oriented game programmer, you need to convert such
``human-developed game strategies'' into computer codes, so the computer
can also use these strategies against the human player. And, this is
what ``applying artificial intelligence to game programming'' is all
about.

Game Programming -- Penn Wu

213

\protect\hypertarget{index_split_011.htmlux5cux23p214}{}{}\includegraphics{index-214_1.png}

\includegraphics{index-214_2.png}

Noticeably, human don't always make scientifically predictable decisions
based on analyzing their surroundings and arriving at a logical
conclusion. So, you probably will never be able to include every
strategy in your games.

To develop a large-scale information-based AI for games starts with
simply figuring out how to convert strategies into computer codes. For
example, to make the Tic-Tac-Toe game ``smart'', first, create a
framework for the Tic-Tac-Toe game by writing the following HTML codes:

\textless html\textgreater{}

\textless img id="c1" src="n.gif" class="cell"

onClick="this.src=\textquotesingle x.gif\textquotesingle;this.alt=\textquotesingle x\textquotesingle;check()"
alt=\textquotesingle n\textquotesingle\textgreater{}

\textless img id="c2" src="n.gif" class="cell"

onClick="this.src=\textquotesingle x.gif\textquotesingle;this.alt=\textquotesingle x\textquotesingle;check()"
alt=\textquotesingle n\textquotesingle\textgreater{}

\textless img id="c3" src="n.gif" class="cell"

onClick="this.src=\textquotesingle x.gif\textquotesingle;this.alt=\textquotesingle x\textquotesingle;check()"
alt=\textquotesingle n\textquotesingle\textgreater\textless br\textgreater{}

\textless img id="c4" src="n.gif" class="cell"

onClick="this.src=\textquotesingle x.gif\textquotesingle;this.alt=\textquotesingle x\textquotesingle;check()"
alt=\textquotesingle n\textquotesingle\textgreater{}

\textless img id="c5" src="n.gif" class="cell"

onClick="this.src=\textquotesingle x.gif\textquotesingle;this.alt=\textquotesingle x\textquotesingle;check()"
alt=\textquotesingle n\textquotesingle\textgreater{}

\textless img id="c6" src="n.gif" class="cell"

onClick="this.src=\textquotesingle x.gif\textquotesingle;this.alt=\textquotesingle x\textquotesingle;check()"
alt=\textquotesingle n\textquotesingle\textgreater\textless br\textgreater{}

\textless img id="c7" src="n.gif" class="cell"

onClick="this.src=\textquotesingle x.gif\textquotesingle;this.alt=\textquotesingle x\textquotesingle;check()"
alt=\textquotesingle n\textquotesingle\textgreater{}

\textless img id="c8" src="n.gif" class="cell"

onClick="this.src=\textquotesingle x.gif\textquotesingle;this.alt=\textquotesingle x\textquotesingle;check()"
alt=\textquotesingle n\textquotesingle\textgreater{}

\textless img id="c9" src="n.gif" class="cell"

onClick="this.src=\textquotesingle x.gif\textquotesingle;this.alt=\textquotesingle x\textquotesingle;check()"
alt=\textquotesingle n\textquotesingle\textgreater\textless br\textgreater{}

\textless/html\textgreater{}

The output looks:

There are three image files used in this game:

•

n.gif -- a blank image

•

x.gif -- image of x

•

o.gif -- imageof o

Inside each \textbf{\textless img\textgreater{}} tag the following part
changes the file source (image file name) from \textbf{n.gif} to
\textbf{x.gif}, and the \textbf{alt} property (which is the alternative
description of the image file) from \textbf{n} to \textbf{x}.

onClick="this.src=\textquotesingle x.gif\textquotesingle;this.alt=\textquotesingle x\textquotesingle;check()"

When the player click on any cell, the image changes from \textbf{blank}
to \textbf{x}. For example, When the game starts, \textbf{x} is the
player, while \textbf{o} is the antagonist (the computer). You need to
add some codes to simulate human-like decision making mechanism, thus,
whenever, the player makes a move, the computer can responds with an
intention to prevent the user from winning the game.

Game Programming -- Penn Wu

214

\protect\hypertarget{index_split_011.htmlux5cux23p215}{}{}\includegraphics{index-215_1.png}

\includegraphics{index-215_2.png}

\includegraphics{index-215_3.png}

To help developing the code with strategy against the player, try
marking each cell with its IDs.

For example,

c1 c2 c3

c4 c5 c6

c7 c8 c9

One commonly used strategy (as recommended by a sophisticated player)
is:

•

If c1 is marked x, while c2 and c4 are blank, check c3.

•

If c3 is blank, mark c3 with o; otherwise mark c2 with o.

The code that performs the above human-like decision making is: function
check() \{

if (c1.alt==\textquotesingle x\textquotesingle{} \&\&
c2.alt==\textquotesingle n\textquotesingle{} \&\&
c4.alt==\textquotesingle n\textquotesingle) \{

if (c3.alt==\textquotesingle n\textquotesingle) \{
c3.src=\textquotesingle o.gif\textquotesingle; \}

else \{ c2.src=\textquotesingle o.gif\textquotesingle; \}

\}

\}

Consequently, when the player clicks c1 to mark \textbf{x} on the c1
cell, the above code immediately marks c3 with \textbf{o} to prevent the
player from marking c1, c2, and c3 with \textbf{x}.

If the player marked c3 with \textbf{x} first, and then marks c1 with
\textbf{x}, the code will be smart enough to mark c2 with \textbf{o}. In
other words, the computer is now playing the game by making human-like
decisions, so you have just applied \textbf{artificial intelligence} to
the programming of this game.

Of course, this game cannot be so simple. You need to develop as many
strategies to win as possible (or at least let the computer be smart
enough to have a tie with the player). For example, a new strategy is
``if c4 is marked \textbf{x}, and the player just marks c4 with
\emph{x}, then mark c7 with \textbf{o}.''

Simply add the following bold lines, the computer can possesses this
``artificial intelligence''

function check() \{

if (c1.alt==\textquotesingle x\textquotesingle{} \&\&
c2.alt==\textquotesingle n\textquotesingle{} \&\&
c4.alt==\textquotesingle n\textquotesingle) \{

if (c3.alt==\textquotesingle n\textquotesingle) \{
c3.src=\textquotesingle o.gif\textquotesingle; \}

else \{ c2.src=\textquotesingle o.gif\textquotesingle; \}

\}

\textbf{if (c1.alt==\textquotesingle x\textquotesingle{} \&\&
c2.alt==\textquotesingle n\textquotesingle{} \&\&
c4.alt==\textquotesingle x\textquotesingle) \{}

\textbf{c7.src=\textquotesingle o.gif\textquotesingle\}}

\}

Apply as many game strategies as you can to this Tic-Tac-Toe game, so
the computer will be very smart and eventually have the intelligence to
win over the player. For example, the following code is enough to reach
a tie between the player and the computer if the player starts with
clicking c1.

Game Programming -- Penn Wu

215

\protect\hypertarget{index_split_011.htmlux5cux23p216}{}{}\includegraphics{index-216_1.png}

function check() \{

if (c1.alt==\textquotesingle x\textquotesingle{} \&\&
c2.alt==\textquotesingle n\textquotesingle{} \&\&
c4.alt==\textquotesingle n\textquotesingle) \{

if (c3.alt==\textquotesingle n\textquotesingle) \{
c3.src=\textquotesingle o.gif\textquotesingle; \}

else \{ c2.src=\textquotesingle o.gif\textquotesingle; \}

\}

if (c1.alt==\textquotesingle x\textquotesingle{} \&\&
c2.alt==\textquotesingle n\textquotesingle{} \&\&
c4.alt==\textquotesingle x\textquotesingle) \{

c7.src=\textquotesingle o.gif\textquotesingle\}

\textbf{if (c1.alt==\textquotesingle x\textquotesingle{} \&\&
c2.alt==\textquotesingle x\textquotesingle{} \&\&
c5.alt==\textquotesingle n\textquotesingle) \{}

\textbf{c5.src=\textquotesingle o.gif\textquotesingle\}}

\textbf{if (c1.alt==\textquotesingle x\textquotesingle{} \&\&
c2.alt==\textquotesingle x\textquotesingle{} \&\&
c7.alt==\textquotesingle x\textquotesingle) \{}

\textbf{c9.src=\textquotesingle o.gif\textquotesingle\}}

\}

Although the above strategies work with the demonstrated conditions,
they may not be the best strategies. Be sure to develop you own game
strategies to make this Tic-Tac-Toe game smart and possibly smarter.

When applying artificial intelligence to your games, one good advice
\textbf{is to keep it as simple as} \textbf{possible}. If you know the
answer, put the answer into the program. If you know how to compute the
answer, put the algorithm for computing it into the program. For
example, you can use the simplest \textbf{if..then} statement to
re-write the above codes.

function check() \{

// player starts with c1

if (c1.alt==\textquotesingle x\textquotesingle{} \&\&
c2.alt==\textquotesingle n\textquotesingle{} \&\&
c3.alt==\textquotesingle n\textquotesingle{} \&\&
c4.alt==\textquotesingle n\textquotesingle) \{

c3.src=\textquotesingle o.gif\textquotesingle; \}

if (c1.alt==\textquotesingle x\textquotesingle{} \&\&
c2.alt==\textquotesingle x\textquotesingle{} \&\&
c3.alt==\textquotesingle n\textquotesingle) \{
c3.src=\textquotesingle o.gif\textquotesingle;

\}

if (c1.alt==\textquotesingle x\textquotesingle{} \&\&
c2.alt==\textquotesingle n\textquotesingle{} \&\&
c3.alt==\textquotesingle x\textquotesingle) \{
c2.src=\textquotesingle o.gif\textquotesingle;

\}

if (c1.alt==\textquotesingle x\textquotesingle{} \&\&
c4.alt==\textquotesingle x\textquotesingle{} \&\&
c7.alt==\textquotesingle n\textquotesingle) \{
c7.src=\textquotesingle o.gif\textquotesingle;

\}

if (c1.alt==\textquotesingle x\textquotesingle{} \&\&
c4.alt==\textquotesingle n\textquotesingle{} \&\&
c7.alt==\textquotesingle x\textquotesingle) \{
c4.src=\textquotesingle o.gif\textquotesingle;

\}

if (c1.alt==\textquotesingle x\textquotesingle{} \&\&
c5.alt==\textquotesingle x\textquotesingle{} \&\&
c9.alt==\textquotesingle n\textquotesingle) \{
c9.src=\textquotesingle o.gif\textquotesingle;

\}

if (c1.alt==\textquotesingle x\textquotesingle{} \&\&
c2.alt==\textquotesingle x\textquotesingle{} \&\&
c5.alt==\textquotesingle n\textquotesingle) \{
c5.src=\textquotesingle o.gif\textquotesingle;

\}

if (c1.alt==\textquotesingle x\textquotesingle{} \&\&
c2.alt==\textquotesingle x\textquotesingle{} \&\&
c5.alt==\textquotesingle n\textquotesingle) \{
c5.src=\textquotesingle o.gif\textquotesingle;

\}

if (c1.alt==\textquotesingle x\textquotesingle{} \&\&
c4.alt==\textquotesingle x\textquotesingle{} \&\&
c8.alt==\textquotesingle x\textquotesingle) \{
c5.src=\textquotesingle o.gif\textquotesingle;

\}

if (c1.alt==\textquotesingle x\textquotesingle{} \&\&
c2.alt==\textquotesingle x\textquotesingle{} \&\&
c6.alt==\textquotesingle x\textquotesingle) \{
c7.src=\textquotesingle o.gif\textquotesingle;

\}

if (c1.alt==\textquotesingle x\textquotesingle{} \&\&
c2.alt==\textquotesingle x\textquotesingle{} \&\&
c8.alt==\textquotesingle x\textquotesingle) \{
c7.src=\textquotesingle o.gif\textquotesingle;

\}

if (c1.alt==\textquotesingle x\textquotesingle{} \&\&
c2.alt==\textquotesingle x\textquotesingle{} \&\&
c9.alt==\textquotesingle x\textquotesingle) \{
c7.src=\textquotesingle o.gif\textquotesingle;

\}

............

............

\}

In 1997, the famous supercomputer-\/-IBM\textquotesingle s Deep Blue,
defeated the world great chess master Garry Kasparov. It marked the
first time that a computer had defeated a World Champion in a match of
several games. However, in reality, Deep Blue is just a computer that
had been Game Programming -- Penn Wu

216

\protect\hypertarget{index_split_011.htmlux5cux23p217}{}{}\includegraphics{index-217_1.png}

programmed with millions of chess game strategies (developed by human in
the past, of course).

IBM Deep Blue vs. Garry Kasparov

It is the computation power of the machine and the ``artificial
intelligence'' the computer was programmed to have won the game. But, be
sure to understand one important fact-\/-Deep Blue do not think, it only
quickly run through all the decision-making codes (just like the ones
above) to search for the best-fit rules and then make decision based on
the rules.

Create games

You can create games that detect and record the player's skill level,
playing style, or thinking that learns

model, etc. You can also create games that can modify the game
environment in according to the collected data about the player. There
are many ways to reach this goal.

You can, for example, use a server-side scripting language (such as PHP,
Perl, ASP.Net, etc.) to create a file (or set a cookie file) in the
client computer that keeps data for the player. In the Ping Pong game,
the Write\_level() function forces the client computer to create a new
file (if not existing) named MyLevel.txt and immediately write a line
``level=2'' to it.

function play() \{

if (score \textgreater{} 50) \{ speed = 10; level=2;
\textbf{Write\_level();\}}

else \{speed = 20; level=1\}

...........

...........

\}

\textbf{function Write\_level(sText)\{}

\textbf{var MYFILE = new ActiveXObject("Scripting.FileSystemObject");}
\textbf{var myLevel =
MYFILE.CreateTextFile("C:\textbackslash\textbackslash MyLevel.txt",
true);} \textbf{myLevel.WriteLine("level=2");}

\textbf{myLevel.Close();}

\textbf{\}}

This function uses ActiveXObject, which is an advanced topic for
Windows-based web programming. However, you can learn the concepts as
how to insert a file to a computer machine.

Additionally, if the MyLevel.txt file already exists, and you need to
update the content, you should use the following format:

var myLevel =
MYFILE.OpenTextFile("C:\textbackslash\textbackslash MyLevel.txt", 8,
true,0); where, 8=append, true=create if not exist, and 0 = ASCII.

To set a cookie, for example, you can use the following code:

function setCookie() \{

document.cookie = "level=2; expires=Fri, 13 Jul 2009 05:28:21 UTC;
path=/";

\}

The cookie code only works after you upload the file to a Web server. In
other words, only when the player access the codes from a remote web
server will the cookie file be inserted to the client machine. The path
to file the cookie is usually (Windows XP):

C:\textbackslash Documents and
Settings\textbackslash\%UserName\%\textbackslash Cookies

Game Programming -- Penn Wu

217

\protect\hypertarget{index_split_011.htmlux5cux23p218}{}{}\includegraphics{index-218_1.png}

Where \%UserName\% is the name the player uses to log in to the Windows
XP machine. For example,

A sample content of the cookie looks:

level

2

www.geocities.com/

1088

3155626112

30016378

843931136

29854209

*

If your game uses client-side scripting language (such as JavaScript),
you can use variables to store data temporarily, or create hidden fields
of web forms to store data temporarily. Some client-side scripting
languages (such as JavaScript) allows you to set a cookie in a client
machine, too. For example, in the Ping Pong game, several variables are
used to store the player's data temporarily. The speed of ping pong ball
is determined by the score of the player. If the score is greater than
50, the speed increases and the level is upgraded to 2.

..........

function play() \{

\textbf{if (score \textgreater{} 50) \{ speed = 10; level=2};
Write\_level();\}

\textbf{else \{speed = 20; level=1\}}

if (b1.style.pixelLeft \textgreater{} bd.style.pixelWidth-5) \{

clearTimeout(s1); p1.innerText="You lose!"; \}

else \{s1=setTimeout("play()", \textbf{speed});\}

..........

\}

Review

1. Which statement best describes what "applying artificial intelligence
to a game" is?

Questions

A. It describes a game that has intelligence.

B. It makes the computer functions like a real human.

C. It attempts to build a robot game.

D. It is simply the attempt to build a human-like decision making
systems in a game.

2. Which statement is correct about the term "information-based AI
algorithms"?

A. It is completely referencing.

B. It is completely deterministic.

C. It is completely self-controlled.

D. It is completely empirical.

3. Techniques of artificial intelligence can apply to a game that \_\_.

A. mimics an animal\textquotesingle s self protection mechanism

Game Programming -- Penn Wu

218

\protect\hypertarget{index_split_011.htmlux5cux23p219}{}{}B. plays chess
game against a real human player C. determine the level the player
should play

D. All of the above

4. Which statement is correct?

A. Human logics are complicated, so you should try to make your computer
code as complicated as possible.

B. Although Human logics are complicated; you should try to make your
computer code as simple as possible.

C. Human logics are complicated, you should always use "switch..case"
statement to apply AI to game codes.

D. All of the above

5. Given the following code block of a Tic-Tac-Toe game, which statement
is correct?

Let ci.alt=x represent X, let
ci.alt=\textquotesingle n\textquotesingle{} represent blank (where i is
an integer between 1 and 9).

if (c1.alt==\textquotesingle x\textquotesingle{} \&\&
c2.alt==\textquotesingle n\textquotesingle{} \&\&
c3.alt==\textquotesingle n\textquotesingle{} \&\&
c4.alt==\textquotesingle n\textquotesingle)

\{ c3.src=\textquotesingle o.gif\textquotesingle; \}

A. If the player places x on c1, while c2, c3, and c4 are blank, the
computer placed o on c3.

B. If the player places x on c1, while c2, c3, and c4 are NOT blank, the
computer placed o on c3.

C. If the player places X on c1, while c2, c3, and c4 are all marked
with the letter \textquotesingle n\textquotesingle, the computer placed
o on c3.

D. If the player places X on c1, while c2, c3, and c4 are all marked
with the letter \textquotesingle n\textquotesingle, the computer marks
the letter \textquotesingle o\textquotesingle{} on c3.

6. Given the following code block of a Tic-Tac-Toe game, which statement
is correct?

Let ci.alt=x represent X, let
ci.alt=\textquotesingle n\textquotesingle{} represent blank (where i is
an integer between 1 and 9).

if (c5.alt==\textquotesingle x\textquotesingle{} \&\&
c7.alt==\textquotesingle x\textquotesingle{} \&\&
c3.alt==\textquotesingle n\textquotesingle) \{
c3.src=\textquotesingle o.gif\textquotesingle; \}

A. If the player marked x on c5, and later marks x on c7, the computer
must place o on c3 even when c3 is not blank.

B. If the player marked x on c5, and later marks x on c7, the computer
must place o on c3 only when c3 is blank.

C. If the player marked o on c5, and later marks o on c7, the computer
must place x on c3 even when c3 is not blank.

D. If the player marked o on c5, and later marks o on c7, the computer
must place x on c3 only when c3 is blank.

7. Given the following code block of a Tic-Tac-Toe game, which statement
is correct?

Let ci.alt=x represent X, let
ci.alt=\textquotesingle n\textquotesingle{} represent blank (where i is
an integer between 1 and 9).

if (c3.alt==\textquotesingle x\textquotesingle{} \&\&
c5.alt==\textquotesingle x\textquotesingle{} \&\&
c7.alt==\textquotesingle n\textquotesingle) \{
c7.src=\textquotesingle o.gif\textquotesingle; \}

A. When c3 and c5 are marked x, the computer must check whether or not
c7 is marked o to decide what the next move should be.

B. When c3 and c5 are marked x, the computer must check whether or not
c7 is marked x to decide what the next move should be.

C. When c3 and c5 are marked x, the computer must check whether or not
c7 is b to blank to decide what the next move should be.

D. When c3 and c5 are marked n, the computer must check whether or not
c7 is marked n to decide the next move.

8. Given the following code segment, which statement is correct?

function cc(sText)\{

Game Programming -- Penn Wu

219

\protect\hypertarget{index_split_011.htmlux5cux23p220}{}{}var f = new
ActiveXObject("Scripting.FileSystemObject"); var s =
f.CreateTextFile("C:\textbackslash\textbackslash s.txt", true);

s.WriteLine("Write\_file");

s.Close();

\}

A. It creates a new file named s.txt in the client machine.

B. It writes the message "write\_file" to the newly created file.

C. It uses ActiveXObject.

D. All of the above.

9. Given the following code segment, which statement is correct?

var obj = f.OpenTextFile("C:\textbackslash\textbackslash f1.txt", 8,
true,0);

A. 8=append

B. true=create if not exist

C. 0 = ASCII

D. All of the above

10. Which would insert a cookie to the client machine?

A. document.cookie();

B. window.cookie()

C. setCookie();

D. insertCookie();

Game Programming -- Penn Wu

220

\protect\hypertarget{index_split_011.htmlux5cux23p221}{}{}

Lab \#12

Applying artificial intelligence

\textbf{Preparation \#1:}

1. Create a new directory named \textbf{C:\textbackslash games}.

2. Use Internt Explorer to go to
\textbf{http://business.cypresscollege.edu/\textasciitilde pwu/cis261/download.htm}
to download lab12.zip (a zipped) file. Extract the files to
C:\textbackslash games directory.

\textbf{Learning Activity \#1:}

Note: This game is meant to be simple and easy for the sake of
demonstrating programming concepts. Please do not hesitate to enhance
the appearance or functions of this game.

1. Change to the C:\textbackslash games directory.

2. Use Notepad to create a new file named
\textbf{C:\textbackslash games\textbackslash lab12\_1.htm} with the
following contents:

\textless html\textgreater{}

\textless script\textgreater{}

function init() \{

if (bat2.style.pixelTop \textless= 170) \{

knight.src="knight2.gif";

\}

if (bat2.style.pixelWidth == 20) \{

clearTimeout(s1);

setInterval("hit()",1000);

\}

else\{

bat2.style.pixelWidth-\/-;

bat2.style.pixelTop-=2;

s1 = setTimeout("init()", 50);

\}

\}

function hit() \{

if ( knight.style.pixelTop \textgreater=70) \{

clearTimeout(s2);

setInterval("knight.src=\textquotesingle knight1.gif\textquotesingle",
2000);

\}

else \{

knight.style.pixelTop+=2;

setInterval("bat2.style.display=\textquotesingle none\textquotesingle",
2000);

s2 = setTimeout("hit()", 20);

\}

\}

\textless/script\textgreater{}

\textless body bgcolor="black" onLoad="init()" /\textgreater{}

\textless div id="theme2"\textgreater{}

\textless img id="knight" src="knight1.gif"
style="position:absolute;left:40; top:20"\textgreater{}

\textless img id="bat2" src="bat.gif" style="position:absolute;
width:75; top:220; left:100"\textgreater{}

\textless/div\textgreater{}

\textless/body\textgreater{}

\textless/html\textgreater{}

3. Test the program. A sample output looks:

Game Programming -- Penn Wu

221

\protect\hypertarget{index_split_011.htmlux5cux23p222}{}{}\includegraphics{index-222_1.png}

\textbf{Learning Activity \#2: Fish}

Note: This game is meant to be simple and easy for the sake of
demonstrating programming concepts. Please do not hesitate to enhance
the appearance or functions of this game.

1. Change to the C:\textbackslash games directory.

2. Use Notepad to create a new file named
\textbf{C:\textbackslash games\textbackslash lab12\_2.htm} with the
following contents:

\textless html\textgreater{}

\textless style\textgreater{}

.fish \{position:absolute;\}

\textless/style\textgreater{}

\textless script\textgreater{}

function init() \{

f1.style.pixelLeft=10;

f2.style.pixelLeft=document.body.clientWidth-100;

move();

\}

function move() \{

if (f1.style.pixelLeft+100 \textgreater= f2.style.pixelLeft-100) \{

clearTimeout(s1); avoid();\}

else \{

f1.style.pixelLeft+=2;

f2.style.pixelLeft-=2;

s1=setTimeout("move()", 20);

\}

\}

function avoid() \{

if (f2.style.pixelTop \textgreater= document.body.clientHeight - 100) \{

clearTimeout(s2); \}

else \{

f2.style.pixelTop +=10;

f2.style.pixelLeft-=2;

f1.style.pixelLeft+=2;

s2=setTimeout("avoid()",10);

\}

\}

\textless/script\textgreater{}

\textless body onLoad=init()\textgreater{}

\textless img id="f1" src="fish1.gif" class="fish"
style="top:100"\textgreater{}

\textless img id="f2" src="fish2.gif" class="fish" width="40"
style="top:120"\textgreater{}

\textless/body\textgreater{}

\textless/html\textgreater{}

3. Test the program. A sample output looks:

Game Programming -- Penn Wu

222

\protect\hypertarget{index_split_011.htmlux5cux23p223}{}{}\includegraphics{index-223_1.png}

\textbf{Learning Activity \#3: Tic-Tac-Toe}

Note: This game is meant to be simple and easy for the sake of
demonstrating programming concepts. Please do not hesitate to enhance
the appearance or functions of this game. \textbf{Please feel free to
add as many strategies as} \textbf{possible.}

1. Change to the C:\textbackslash games directory.

2. Use Notepad to create a new file named
\textbf{C:\textbackslash games\textbackslash lab12\_3.htm} with the
following contents:

\textless html\textgreater{}

\textless style\textgreater{}

.cell \{border:solid 1 black; width:30px; height:30px\}

\textless/style\textgreater{}

\textless script\textgreater{}

function check() \{

// player starts with c1

if (c1.alt==\textquotesingle x\textquotesingle{} \&\&
c2.alt==\textquotesingle n\textquotesingle{} \&\&
c3.alt==\textquotesingle n\textquotesingle{} \&\&
c4.alt==\textquotesingle n\textquotesingle) \{
c3.src=\textquotesingle o.gif\textquotesingle; \}

if (c1.alt==\textquotesingle x\textquotesingle{} \&\&
c2.alt==\textquotesingle x\textquotesingle{} \&\&
c3.alt==\textquotesingle n\textquotesingle) \{
c3.src=\textquotesingle o.gif\textquotesingle; \}

if (c1.alt==\textquotesingle x\textquotesingle{} \&\&
c2.alt==\textquotesingle n\textquotesingle{} \&\&
c3.alt==\textquotesingle x\textquotesingle) \{
c2.src=\textquotesingle o.gif\textquotesingle; \}

if (c1.alt==\textquotesingle x\textquotesingle{} \&\&
c4.alt==\textquotesingle x\textquotesingle{} \&\&
c7.alt==\textquotesingle n\textquotesingle) \{
c7.src=\textquotesingle o.gif\textquotesingle; \}

if (c1.alt==\textquotesingle x\textquotesingle{} \&\&
c4.alt==\textquotesingle n\textquotesingle{} \&\&
c7.alt==\textquotesingle x\textquotesingle) \{
c4.src=\textquotesingle o.gif\textquotesingle; \}

if (c1.alt==\textquotesingle x\textquotesingle{} \&\&
c5.alt==\textquotesingle x\textquotesingle{} \&\&
c9.alt==\textquotesingle n\textquotesingle) \{
c9.src=\textquotesingle o.gif\textquotesingle; \}

if (c1.alt==\textquotesingle x\textquotesingle{} \&\&
c2.alt==\textquotesingle x\textquotesingle{} \&\&
c5.alt==\textquotesingle n\textquotesingle) \{
c5.src=\textquotesingle o.gif\textquotesingle; \}

if (c1.alt==\textquotesingle x\textquotesingle{} \&\&
c2.alt==\textquotesingle x\textquotesingle{} \&\&
c5.alt==\textquotesingle n\textquotesingle) \{
c5.src=\textquotesingle o.gif\textquotesingle; \}

if (c1.alt==\textquotesingle x\textquotesingle{} \&\&
c4.alt==\textquotesingle x\textquotesingle{} \&\&
c8.alt==\textquotesingle x\textquotesingle) \{
c5.src=\textquotesingle o.gif\textquotesingle; \}

if (c1.alt==\textquotesingle x\textquotesingle{} \&\&
c2.alt==\textquotesingle x\textquotesingle{} \&\&
c6.alt==\textquotesingle x\textquotesingle) \{
c7.src=\textquotesingle o.gif\textquotesingle; \}

if (c1.alt==\textquotesingle x\textquotesingle{} \&\&
c2.alt==\textquotesingle x\textquotesingle{} \&\&
c8.alt==\textquotesingle x\textquotesingle) \{
c7.src=\textquotesingle o.gif\textquotesingle; \}

if (c1.alt==\textquotesingle x\textquotesingle{} \&\&
c2.alt==\textquotesingle x\textquotesingle{} \&\&
c9.alt==\textquotesingle x\textquotesingle) \{
c7.src=\textquotesingle o.gif\textquotesingle; \}

if (c1.alt==\textquotesingle x\textquotesingle{} \&\&
c6.alt==\textquotesingle x\textquotesingle{} \&\&
c5.alt==\textquotesingle n\textquotesingle) \{
c5.src=\textquotesingle o.gif\textquotesingle; \}

if (c1.alt==\textquotesingle x\textquotesingle{} \&\&
c6.alt==\textquotesingle x\textquotesingle{} \&\&
c9.alt==\textquotesingle x\textquotesingle) \{
c7.src=\textquotesingle o.gif\textquotesingle; \}

if (c1.alt==\textquotesingle x\textquotesingle{} \&\&
c9.alt==\textquotesingle x\textquotesingle{} \&\&
c5.alt==\textquotesingle n\textquotesingle) \{
c5.src=\textquotesingle o.gif\textquotesingle; \}

if (c1.alt==\textquotesingle x\textquotesingle{} \&\&
c8.alt==\textquotesingle x\textquotesingle{} \&\&
c5.alt==\textquotesingle n\textquotesingle) \{
c5.src=\textquotesingle o.gif\textquotesingle; \}

if (c1.alt==\textquotesingle x\textquotesingle{} \&\&
c5.alt==\textquotesingle x\textquotesingle{} \&\&
c8.alt==\textquotesingle x\textquotesingle) \{
c6.src=\textquotesingle o.gif\textquotesingle; \}

if (c1.alt==\textquotesingle x\textquotesingle{} \&\&
c5.alt==\textquotesingle x\textquotesingle{} \&\&
c6.alt==\textquotesingle x\textquotesingle) \{
c4.src=\textquotesingle o.gif\textquotesingle; \}

// player starts with c2

if (c1.alt==\textquotesingle n\textquotesingle{} \&\&
c2.alt==\textquotesingle x\textquotesingle{} \&\&
c3.alt==\textquotesingle n\textquotesingle{} \&\&
c4.alt==\textquotesingle n\textquotesingle) \{
c1.src=\textquotesingle o.gif\textquotesingle; \}

if (c2.alt==\textquotesingle x\textquotesingle{} \&\&
c5.alt==\textquotesingle x\textquotesingle{} \&\&
c8.alt==\textquotesingle n\textquotesingle) \{
c8.src=\textquotesingle o.gif\textquotesingle; \}

if (c5.alt==\textquotesingle x\textquotesingle{} \&\&
c7.alt==\textquotesingle x\textquotesingle{} \&\&
c3.alt==\textquotesingle n\textquotesingle) \{
c3.src=\textquotesingle o.gif\textquotesingle; \}

if (c2.alt==\textquotesingle x\textquotesingle{} \&\&
c4.alt==\textquotesingle x\textquotesingle{} \&\&
c5.alt==\textquotesingle n\textquotesingle) \{
c5.src=\textquotesingle o.gif\textquotesingle; \}

if (c2.alt==\textquotesingle x\textquotesingle{} \&\&
c4.alt==\textquotesingle x\textquotesingle{} \&\&
c5.alt==\textquotesingle x\textquotesingle) \{
c6.src=\textquotesingle o.gif\textquotesingle; \}

if (c2.alt==\textquotesingle x\textquotesingle{} \&\&
c3.alt==\textquotesingle x\textquotesingle{} \&\&
c9.alt==\textquotesingle n\textquotesingle) \{
c9.src=\textquotesingle o.gif\textquotesingle; \}

if (c2.alt==\textquotesingle x\textquotesingle{} \&\&
c5.alt==\textquotesingle x\textquotesingle{} \&\&
c6.alt==\textquotesingle x\textquotesingle) \{
c4.src=\textquotesingle o.gif\textquotesingle; \}

if (c2.alt==\textquotesingle x\textquotesingle{} \&\&
c5.alt==\textquotesingle x\textquotesingle{} \&\&
c9.alt==\textquotesingle x\textquotesingle) \{
c7.src=\textquotesingle o.gif\textquotesingle; \}

// Player starts with c3

if (c1.alt==\textquotesingle n\textquotesingle{} \&\&
c2.alt==\textquotesingle n\textquotesingle{} \&\&
c3.alt==\textquotesingle x\textquotesingle{} \&\&
c7.alt==\textquotesingle n\textquotesingle) \{
c1.src=\textquotesingle o.gif\textquotesingle; \}

if (c3.alt==\textquotesingle x\textquotesingle{} \&\&
c6.alt==\textquotesingle x\textquotesingle{} \&\&
c9.alt==\textquotesingle n\textquotesingle) \{
c9.src=\textquotesingle o.gif\textquotesingle; \}

if (c3.alt==\textquotesingle x\textquotesingle{} \&\&
c5.alt==\textquotesingle x\textquotesingle{} \&\&
c7.alt==\textquotesingle n\textquotesingle) \{
c7.src=\textquotesingle o.gif\textquotesingle; \}

if (c3.alt==\textquotesingle x\textquotesingle{} \&\&
c4.alt==\textquotesingle x\textquotesingle{} \&\&
c5.alt==\textquotesingle x\textquotesingle) \{
c6.src=\textquotesingle o.gif\textquotesingle; \}

Game Programming -- Penn Wu

223

\protect\hypertarget{index_split_011.htmlux5cux23p224}{}{}\includegraphics{index-224_1.png}

// Player starts with c4

if (c1.alt==\textquotesingle n\textquotesingle{} \&\&
c4.alt==\textquotesingle x\textquotesingle{} \&\&
c7.alt==\textquotesingle n\textquotesingle{} \&\&
c5.alt==\textquotesingle n\textquotesingle) \{
c5.src=\textquotesingle o.gif\textquotesingle; \}

if (c4.alt==\textquotesingle x\textquotesingle{} \&\&
c5.alt==\textquotesingle x\textquotesingle{} \&\&
c8.alt==\textquotesingle x\textquotesingle{} \&\&
c2.alt==\textquotesingle n\textquotesingle) \{
c2.src=\textquotesingle o.gif\textquotesingle; \}

if (c4.alt==\textquotesingle x\textquotesingle{} \&\&
c7.alt==\textquotesingle x\textquotesingle{} \&\&
c9.alt==\textquotesingle x\textquotesingle{} \&\&
c8.alt==\textquotesingle n\textquotesingle) \{
c8.src=\textquotesingle o.gif\textquotesingle; \}

if (c1.alt==\textquotesingle n\textquotesingle{} \&\&
c2.alt==\textquotesingle x\textquotesingle{} \&\&
c4.alt==\textquotesingle x\textquotesingle) \{
c1.src=\textquotesingle o.gif\textquotesingle; \}

if (c1.alt==\textquotesingle n\textquotesingle{} \&\&
c4.alt==\textquotesingle x\textquotesingle{} \&\&
c7.alt==\textquotesingle x\textquotesingle) \{
c1.src=\textquotesingle o.gif\textquotesingle; \}

if (c4.alt==\textquotesingle x\textquotesingle{} \&\&
c8.alt==\textquotesingle x\textquotesingle{} \&\&
c9.alt==\textquotesingle n\textquotesingle) \{
c9.src=\textquotesingle o.gif\textquotesingle; \}

if (c4.alt==\textquotesingle x\textquotesingle{} \&\&
c3.alt==\textquotesingle x\textquotesingle{} \&\&
c9.alt==\textquotesingle x\textquotesingle) \{
c6.src=\textquotesingle o.gif\textquotesingle; \}

if (c4.alt==\textquotesingle x\textquotesingle{} \&\&
c9.alt==\textquotesingle x\textquotesingle{} \&\&
c7.alt==\textquotesingle n\textquotesingle) \{
c7.src=\textquotesingle o.gif\textquotesingle; \}

if (c4.alt==\textquotesingle x\textquotesingle{} \&\&
c6.alt==\textquotesingle x\textquotesingle{} \&\&
c3.alt==\textquotesingle n\textquotesingle) \{
c3.src=\textquotesingle o.gif\textquotesingle; \}

\}

\textless/script\textgreater{}

\textless img id="c1" src="n.gif" class="cell"

onClick="this.src=\textquotesingle x.gif\textquotesingle;this.alt=\textquotesingle x\textquotesingle;check()"
alt=\textquotesingle n\textquotesingle\textgreater{}

\textless img id="c2" src="n.gif" class="cell"

onClick="this.src=\textquotesingle x.gif\textquotesingle;this.alt=\textquotesingle x\textquotesingle;check()"
alt=\textquotesingle n\textquotesingle\textgreater{}

\textless img id="c3" src="n.gif" class="cell"

onClick="this.src=\textquotesingle x.gif\textquotesingle;this.alt=\textquotesingle x\textquotesingle;check()"
alt=\textquotesingle n\textquotesingle\textgreater\textless br\textgreater{}

\textless img id="c4" src="n.gif" class="cell"

onClick="this.src=\textquotesingle x.gif\textquotesingle;this.alt=\textquotesingle x\textquotesingle;check()"
alt=\textquotesingle n\textquotesingle\textgreater{}

\textless img id="c5" src="n.gif" class="cell"

onClick="this.src=\textquotesingle x.gif\textquotesingle;this.alt=\textquotesingle x\textquotesingle;check()"
alt=\textquotesingle n\textquotesingle\textgreater{}

\textless img id="c6" src="n.gif" class="cell"

onClick="this.src=\textquotesingle x.gif\textquotesingle;this.alt=\textquotesingle x\textquotesingle;check()"
alt=\textquotesingle n\textquotesingle\textgreater\textless br\textgreater{}

\textless img id="c7" src="n.gif" class="cell"

onClick="this.src=\textquotesingle x.gif\textquotesingle;this.alt=\textquotesingle x\textquotesingle;check()"
alt=\textquotesingle n\textquotesingle\textgreater{}

\textless img id="c8" src="n.gif" class="cell"

onClick="this.src=\textquotesingle x.gif\textquotesingle;this.alt=\textquotesingle x\textquotesingle;check()"
alt=\textquotesingle n\textquotesingle\textgreater{}

\textless img id="c9" src="n.gif" class="cell"

onClick="this.src=\textquotesingle x.gif\textquotesingle;this.alt=\textquotesingle x\textquotesingle;check()"
alt=\textquotesingle n\textquotesingle\textgreater\textless br\textgreater{}

\textless/html\textgreater{}

3. Test the program. A sample output looks:

\textbf{Learning Activity \#4: Ping Pong Game}

Note: This game is meant to be simple and easy for the sake of
demonstrating programming concepts. Please do not hesitate to enhance
the appearance or functions of this game.

1. Change to the C:\textbackslash games directory.

2. Use Notepad to create a new file named
\textbf{C:\textbackslash games\textbackslash lab12\_4.htm} with the
following contents:

\textless html\textgreater{}

\textless script\textgreater{}

var score=0;

var speed;

var level;

var left\_right = "right";

var up\_down = "down";

function init() \{

b1.style.pixelTop = Math.floor(Math.random()*235);

b1.style.display = "inline";

p1.innerText ="";

Game Programming -- Penn Wu

224

\protect\hypertarget{index_split_011.htmlux5cux23p225}{}{}play();

\}

function move() \{

ply.style.pixelTop = event.clientY;

\}

function play() \{

if (score \textgreater{} 50) \{ speed = 10; level=2; Write\_level();\}

else \{speed = 20; level=1\}

if (b1.style.pixelLeft \textgreater{} bd.style.pixelWidth-5) \{

clearTimeout(s1); p1.innerText="You lose!"; \}

else \{s1=setTimeout("play()", speed);\}

if ((b1.style.pixelLeft+15 \textgreater= ply.style.pixelLeft) \&\&
(b1.style.pixelTop \textgreater{} ply.style.pixelTop) \&\&

(b1.style.pixelTop+15 \textless{} ply.style.pixelTop+50))

\{ left\_right = "left"; score+=10; p1.innerText=score;
p2.innerText=level;\}

if (b1.style.pixelLeft \textless= 10) \{ left\_right = "right"; \}

switch (left\_right) \{

case "right":

b1.style.pixelLeft++; break;

case "left":

b1.style.pixelLeft-\/-; break;

\}

if (b1.style.pixelTop \textless=10) \{ up\_down="down"; \}

if (b1.style.pixelTop \textgreater=250) \{ up\_down="up"; \}

switch (up\_down) \{

case "up":

b1.style.pixelTop-\/-; break;

case "down":

b1.style.pixelTop++; break;

\}

\}

function Write\_level(sText)\{

var MYFILE = new ActiveXObject("Scripting.FileSystemObject"); var
myLevel =
MYFILE.CreateTextFile("C:\textbackslash\textbackslash MyLevel.txt",
true); myLevel.WriteLine("level=2");

myLevel.Close();

\}

\textless/script\textgreater{}

\textless body onMouseMove="move()"\textgreater{}

\textless!-\/- b1-ball, bd-board, ply-player -\/-\textgreater{}

\textless img id="b1" SRC="ball.gif" style="position:absolute; Left:10;
Width:15; Height:15;z-index=3; display: none"\textgreater{}

\textless div id="bd" style="position: absolute; Top:10; Left:10;
Width:350; Height:250; background-Color:green;
z-index=2;"\textgreater\textless/div\textgreater{}

\textless span id="ply" style="position: absolute; Top:10; Left:340;
Width:15; Height:50; background-Color: blue;
z-index:3"\textgreater\textless/span\textgreater{}

\textless button TYPE="button" onClick="init()" style="position:
absolute; Top:15;
Left:400;z-index:3"\textgreater Play\textless/button\textgreater{}

\textless p style="position: absolute; Top:95;
Left:400;z-index:3"\textgreater Your Score: \textless b
id=p1\textgreater\textless/b\textgreater\textless/p\textgreater{}

Game Programming -- Penn Wu

225

\protect\hypertarget{index_split_011.htmlux5cux23p226}{}{}\includegraphics{index-226_1.png}

\textless p style="position: absolute; Top:115;
Left:400;z-index:3"\textgreater Your level: \textless b
id=p2\textgreater\textless/b\textgreater\textless/p\textgreater{}

\textless/body\textgreater{}

\textless/html\textgreater{}

3. Test the program. Press the Play button to begin. Move mouse cursor
up or down to move the blue bar. When your score is bigger than 50, the
computer automatically upgrades your level to 2. A sample output looks:
\textbf{Learning Activity \#5: Tetris}

Note: This game is meant to be simple and easy for the sake of
demonstrating programming concepts. Please do not hesitate to enhance
the appearance or functions of this game.

1. Change to the C:\textbackslash games directory.

2. Use Notepad to create a new file named
\textbf{C:\textbackslash games\textbackslash lab12\_5.htm} with the
following contents:

\textless HTML\textgreater{}

\textless style\textgreater{}

.MB \{ border: solid 1 white; background-color: red; height: 22px;
width: 22px \}

.SB \{ border: solid 1 white; background-color: teal; height: 22px;
width: 22px

\}

.BK \{ border: solid 1 \#abcdef; height: 22px; width: 22px \}

.GT \{ border: solid 1 black; \}

\textless/style\textgreater{}

\textless script\textgreater{}

var BX=new Array(4);

var BY=new Array(4);

var PX=new Array(4);

var PY=new Array(4);

var mTimer

var firstView

function init() \{

var code1="\textless table cellspacing=0 cellpadding=0
class=gt\textgreater"; code1 += "\textless tbody
id=\textquotesingle GameBar\textquotesingle\textgreater";

for (i=1; i\textless=160; i++) \{

if (i\%10==1) \{ code1 += "\textless tr\textgreater\textless td
class=BK\textgreater\&nbsp;\textless/td\textgreater"; \}

else if (i\%10==0) \{ code1 += "\textless td
class=BK\textgreater\&nbsp;\textless/td\textgreater\textless/tr\textgreater";
\}

else \{ code1 += "\textless td
class=BK\textgreater\&nbsp;\textless/td\textgreater"; \}

\}

code1 += "\textless/tbody\textgreater\textless/table\textgreater";

area1.innerHTML += code1;

var code2="\textless table cellspacing=0 cellpadding=0
class=gt\textgreater"; code2 += "\textless tbody
id=\textquotesingle previewBar\textquotesingle\textgreater";

for (j=1; j\textless=16; j++) \{

if (j\%4==1) \{ code2 += "\textless tr\textgreater\textless td
class=BK\textgreater\&nbsp;\textless/td\textgreater"; \}

else if (j\%4==0) \{ code2 += "\textless td
class=BK\textgreater\&nbsp;\textless/td\textgreater\textless/tr\textgreater";
\}

else \{ code2 += "\textless td
class=BK\textgreater\&nbsp;\textless/td\textgreater"; \}

Game Programming -- Penn Wu

226

\protect\hypertarget{index_split_011.htmlux5cux23p227}{}{} \}

code2 += "\textless/tbody\textgreater\textless/table\textgreater";

area2.innerHTML += code2;

\}

function beginGame()

\{

gameState=0;

speed=1;

outTime=1100-speed*100;

if(gameState!=0) return;

firstView=true;

for(j=0;j\textless16;j++)

for(i=0;i\textless10;i++)

setClass(i,j,"BK");

randBar();

gameState=1;

Play.disabled=true;

window.clearInterval(mTimer);

mTimer=window.setInterval("moveBar()",outTime);

\}

function keyControl()

\{

if(gameState!=1)return;

switch(event.keyCode)\{

case 37: //left

for(i=0;i\textless4;i++) if(BX{[}i{]}==0)return;

for(i=0;i\textless4;i++) if(getClass(BX{[}i{]}-1,BY{[}i{]})=="SB")
return; for(i=0;i\textless4;i++)setClass(BX{[}i{]},BY{[}i{]},"BK");

for(i=0;i\textless4;i++)BX{[}i{]}=BX{[}i{]}-1;

for(i=0;i\textless4;i++)setClass(BX{[}i{]},BY{[}i{]},"MB");

break;

case 38: //up

var preMBarX=new Array(4);

var preMBarY=new Array(4);

var cx=Math.round((BX{[}0{]}+BX{[}1{]}+BX{[}2{]}+BX{[}3{]})/4);

var cy=Math.round((BY{[}0{]}+BY{[}1{]}+BY{[}2{]}+BY{[}3{]})/4);

for(i=0;i\textless4;i++)\{

preMBarX{[}i{]}=Math.round(cx-cy+BY{[}i{]});

preMBarY{[}i{]}=Math.round(cx+cy-BX{[}i{]});

if(preMBarX{[}i{]}\textless0 \textbar\textbar{}
preMBarX{[}i{]}\textgreater9 \textbar\textbar{}
preMBarY{[}i{]}\textless0 \textbar\textbar{}

preMBarY{[}i{]}\textgreater15)

return;

if(getClass(preMBarX{[}i{]},preMBarY{[}i{]})=="SB") return;

\}

for(i=0;i\textless4;i++)setClass(BX{[}i{]},BY{[}i{]},"BK");

for(i=0;i\textless4;i++)\{

BX{[}i{]}=preMBarX{[}i{]};

BY{[}i{]}=preMBarY{[}i{]};

\}

for(i=0;i\textless4;i++)setClass(BX{[}i{]},BY{[}i{]},"MB");

break;

case 39: //right

for(i=0;i\textless4;i++) if(BX{[}i{]}==9)return;

for(i=0;i\textless4;i++) if(getClass(BX{[}i{]}+1,BY{[}i{]})=="SB")
return; for(i=0;i\textless4;i++)setClass(BX{[}i{]},BY{[}i{]},"BK");

for(i=0;i\textless4;i++)BX{[}i{]}=BX{[}i{]}+1;

for(i=0;i\textless4;i++)setClass(BX{[}i{]},BY{[}i{]},"MB");

break;

case 40: //down

moveBar();

break;

Game Programming -- Penn Wu

227

\protect\hypertarget{index_split_011.htmlux5cux23p228}{}{} \}

\}

function delLine()

\{

for(i=0;i\textless4;i++) setClass(BX{[}i{]},BY{[}i{]},"SB");

for(j=0;j\textless16;j++)\{

dLine=true;

for(i=0;i\textless9;i++)\{

if(getClass(i,j)!="SB")\{

dLine=false;

break;

\}

\}

if(dLine)\{

for(k=j;k\textgreater0;k-\/-)

for(l=0;l\textless10;l++)

setClass(l,k,getClass(l,k-1));

for(l=0;l\textless10;l++)setClass(l,0,"BK");

\}

\}

randBar();

window.clearInterval(mTimer);

mTimer=window.setInterval("moveBar()",outTime);

\}

function getClass(x,y)\{return
GameBar.children{[}y{]}.children{[}x{]}.className;\}

function
setClass(x,y,cName)\{GameBar.children{[}y{]}.children{[}x{]}.className=cName;\}

function moveBar()

\{

if(gameState!=1)return;

dropLine=true;

for(i=0;i\textless4;i++)if(BY{[}i{]}==15)dropLine=false;

if(dropLine)for(i=0;i\textless4;i++)if(getClass(BX{[}i{]},BY{[}i{]}+1)=="SB")dropLine=false;
if(!dropLine)\{

window.clearInterval(mTimer);

delLine();

return;

\}

for(i=0;i\textless4;i++)setClass(BX{[}i{]},BY{[}i{]},"BK");

for(i=0;i\textless4;i++)BY{[}i{]}=BY{[}i{]}+1;

for(i=0;i\textless4;i++)setClass(BX{[}i{]},BY{[}i{]},"MB");

\}

function randBar()

\{

randNum=Math.floor(Math.random()*20)+1;

if(!firstView)

for(i=0;i\textless4;i++)\{

BX{[}i{]}=PX{[}i{]};

BY{[}i{]}=PY{[}i{]};

\}

switch(randNum)\{

case 1:

PX{[}0{]}=4;PY{[}0{]}=0;PX{[}1{]}=4;PY{[}1{]}=1;PX{[}2{]}=5;PY{[}2{]}=1;PX{[}3{]}=6;PY{[}3{]}=1;break;
case 2:

PX{[}0{]}=4;PY{[}0{]}=0;PX{[}1{]}=5;PY{[}1{]}=0;PX{[}2{]}=4;PY{[}2{]}=1;PX{[}3{]}=4;PY{[}3{]}=2;break;
case 3:

PX{[}0{]}=4;PY{[}0{]}=0;PX{[}1{]}=5;PY{[}1{]}=0;PX{[}2{]}=6;PY{[}2{]}=0;PX{[}3{]}=6;PY{[}3{]}=1;break;
case 4:

PX{[}0{]}=5;PY{[}0{]}=0;PX{[}1{]}=5;PY{[}1{]}=1;PX{[}2{]}=5;PY{[}2{]}=2;PX{[}3{]}=4;PY{[}3{]}=2;break;
Game Programming -- Penn Wu

228

\protect\hypertarget{index_split_011.htmlux5cux23p229}{}{} case 5:

PX{[}0{]}=6;PY{[}0{]}=0;PX{[}1{]}=6;PY{[}1{]}=1;PX{[}2{]}=4;PY{[}2{]}=1;PX{[}3{]}=5;PY{[}3{]}=1;break;
case 6:

PX{[}0{]}=4;PY{[}0{]}=0;PX{[}1{]}=4;PY{[}1{]}=1;PX{[}2{]}=4;PY{[}2{]}=2;PX{[}3{]}=5;PY{[}3{]}=2;break;
case 7:

PX{[}0{]}=4;PY{[}0{]}=0;PX{[}1{]}=4;PY{[}1{]}=1;PX{[}2{]}=5;PY{[}2{]}=0;PX{[}3{]}=6;PY{[}3{]}=0;break;
case 8:

PX{[}0{]}=4;PY{[}0{]}=0;PX{[}1{]}=5;PY{[}1{]}=0;PX{[}2{]}=5;PY{[}2{]}=1;PX{[}3{]}=5;PY{[}3{]}=2;break;
case 9:

PX{[}0{]}=4;PY{[}0{]}=0;PX{[}1{]}=5;PY{[}1{]}=0;PX{[}2{]}=5;PY{[}2{]}=1;PX{[}3{]}=6;PY{[}3{]}=1;break;
case 10:

PX{[}0{]}=5;PY{[}0{]}=0;PX{[}1{]}=5;PY{[}1{]}=1;PX{[}2{]}=4;PY{[}2{]}=1;PX{[}3{]}=4;PY{[}3{]}=2;break;
case 11:

PX{[}0{]}=4;PY{[}0{]}=1;PX{[}1{]}=5;PY{[}1{]}=1;PX{[}2{]}=5;PY{[}2{]}=0;PX{[}3{]}=6;PY{[}3{]}=0;break;
case 12:

PX{[}0{]}=4;PY{[}0{]}=0;PX{[}1{]}=4;PY{[}1{]}=1;PX{[}2{]}=5;PY{[}2{]}=1;PX{[}3{]}=5;PY{[}3{]}=2;break;
case 13:

PX{[}0{]}=4;PY{[}0{]}=0;PX{[}1{]}=5;PY{[}1{]}=0;PX{[}2{]}=6;PY{[}2{]}=0;PX{[}3{]}=5;PY{[}3{]}=1;break;
case 14:

PX{[}0{]}=4;PY{[}0{]}=0;PX{[}1{]}=4;PY{[}1{]}=1;PX{[}2{]}=4;PY{[}2{]}=2;PX{[}3{]}=5;PY{[}3{]}=1;break;
case 15:

PX{[}0{]}=5;PY{[}0{]}=0;PX{[}1{]}=5;PY{[}1{]}=1;PX{[}2{]}=4;PY{[}2{]}=1;PX{[}3{]}=6;PY{[}3{]}=1;break;
case 16:

PX{[}0{]}=5;PY{[}0{]}=0;PX{[}1{]}=5;PY{[}1{]}=1;PX{[}2{]}=5;PY{[}2{]}=2;PX{[}3{]}=4;PY{[}3{]}=1;break;
case 17:

PX{[}0{]}=4;PY{[}0{]}=0;PX{[}1{]}=5;PY{[}1{]}=0;PX{[}2{]}=4;PY{[}2{]}=1;PX{[}3{]}=5;PY{[}3{]}=1;break;
case 18:

PX{[}0{]}=4;PY{[}0{]}=0;PX{[}1{]}=5;PY{[}1{]}=0;PX{[}2{]}=4;PY{[}2{]}=1;PX{[}3{]}=5;PY{[}3{]}=1;break;
case 19:

PX{[}0{]}=3;PY{[}0{]}=0;PX{[}1{]}=4;PY{[}1{]}=0;PX{[}2{]}=5;PY{[}2{]}=0;PX{[}3{]}=6;PY{[}3{]}=0;break;
case 20:

PX{[}0{]}=5;PY{[}0{]}=0;PX{[}1{]}=5;PY{[}1{]}=1;PX{[}2{]}=5;PY{[}2{]}=2;PX{[}3{]}=5;PY{[}3{]}=3;break;

\}

if(firstView)\{

firstView=false;

randBar();

return;

\}

for(i=0;i\textless4;i++)\{

for(j=0;j\textless4;j++)\{

previewBar.children{[}j{]}.children{[}i{]}.className="BK";

\}

\}

for(i=0;i\textless4;i++)previewBar.children{[}PY{[}i{]}{]}.children{[}PX{[}i{]}-

3{]}.className="MB";

for(i=0;i\textless4;i++)\{

if(getClass(BX{[}i{]},BY{[}i{]})!="BK")\{

alert("Game Over!");

window.clearInterval(mTimer);

Play.disabled=false;

gameState=0;

return;

\}

\}

for(i=0;i\textless4;i++)setClass(BX{[}i{]},BY{[}i{]},"MB");

\}

function unSel()

\{

document.execCommand("Unselect");

window.setTimeout("unSel()",10);

\}

unSel();

Game Programming -- Penn Wu

229

\protect\hypertarget{index_split_011.htmlux5cux23p230}{}{}\includegraphics{index-230_1.png}

window.onunload=rel;

function rel()

\{

location.reload();

return false;

\}

\textless/script\textgreater{}

\textless BODY bgcolor=white onLoad="init()" onkeydown="return
keyControl();"\textgreater{}

\textless div id="area1"
style="position:absolute;left:10px;top:10px;"\textgreater\textless/div\textgreater{}

\textless div id="area2"
style="position:absolute;left:300px;top:10px;"\textgreater\textless/div\textgreater{}

\textless button id="Play"
style="position:absolute;left:10px;top:400px;width:100px"

onclick="beginGame();"\textgreater Play\textless/button\textgreater{}

\textless/BODY\textgreater{}

\textless/HTML\textgreater{}

3. Test the program. A sample output looks:

\textbf{Submittal}

Upon completing all the learning activities,

1. Upload all files you created in this lab to your remote web server.

2. Log in to to Blackboard, launch Assignment 12, and then scroll down
to question 11.

3. Copy and paste the URLs to the textbox. For example,

•

http://www.geocities.com/cis261/lab12\_1.htm

•

http://www.geocities.com/cis261/lab12\_2.htm

•

http://www.geocities.com/cis261/lab12\_3.htm

•

http://www.geocities.com/cis261/lab12\_4.htm

•

http://www.geocities.com/cis261/lab12\_5.htm

No credit is given to broken link(s).

Game Programming -- Penn Wu

230

\protect\hypertarget{index_split_011.htmlux5cux23p231}{}{}

Lecture \#13

Code Optimization

\protect\hypertarget{index_split_012.html}{}{}

\hypertarget{index_split_012.htmlux5cux23calibre_pb_11}{%
\subsection{Introduction}\label{index_split_012.htmlux5cux23calibre_pb_11}}

In computing, optimization is the process of modifying a system to make
some aspect of it work more efficiently or use fewer resources. A
computer game may be optimized so that it executes more rapidly, or is
capable of operating with less memory storage or other resources, or
draw less power.

Sometimes the biggest problem with a program is that it requires simply
too many resources, and these resources (memory, CPU cycles, network
bandwidth, or a combination) may be limited.

Code fragments that occur multiple times throughout a program are likely
to be size-sensitive, while code with many execution iterations may be
speed-sensitive.

Using a better algorithm probably will yield a bigger performance
increase than any amount of low-level optimizations and is more likely
to deliver an improvement under all execution conditions. As a rule,
high-level optimizations should be considered before doing low-level
optimizations. However, before optimizing, you should carefully consider
whether you need to optimize at all. According to the pioneering
computer scientist Donald Knuth, ``Premature optimization is the root of
all evil.''

Modularization If the program contains the same or similar blocks of
statements or it is required to process the same function several times,
you can avoid redundancy by using modularization techniques. By
modularizing the game programs you make them easy to read and improve
their structure.

Modularized programs are also easier to maintain and to update. Consider
the following code, in which all the objects (c1, c2, \ldots, c7) have
some common properties, such as position, width, and height.

\textless html\textgreater{}

\textless span id="c1" style=" \textbf{position:relative; width:30;
height:30;}
background-Color:red"\textgreater\textless/span\textgreater{}

\textless span id="c2" style=" \textbf{position:relative; width:30;
height:30;}
background-Color:green"\textgreater\textless/span\textgreater{}

\textless span id="c3" style=" \textbf{position:relative; width:30;
height:30;}
backgroundColor:orange"\textgreater\textless/span\textgreater{}

\textless span id="c4" style=" \textbf{position:relative; width:30;
height:30;}
background-Color:blue"\textgreater\textless/span\textgreater{}

\textless span id="c5" style=" \textbf{position:relative; width:30;
height:30;}
backgroundColor:yellow"\textgreater\textless/span\textgreater{}

\textless span id="c6" style=" \textbf{position:relative; width:30;
height:30;}
background-Color:pink"\textgreater\textless/span\textgreater{}

\textless span id="c7" style=" \textbf{position:relative; width:30;
height:30;}
background-Color:black"\textgreater\textless/span\textgreater{}

\textless/html\textgreater{}

Since all these properties have exactly the same values, you can extract
out these common properties and make them an individual code block that
can be access by each object. For example,

\textless html\textgreater{}

\textbf{\textless style\textgreater{}}

\textbf{span \{ position:relative; width:30; height:30; \}}

\textbf{\textless/style\textgreater{}}

Game Programming -- Penn Wu

231

\protect\hypertarget{index_split_012.htmlux5cux23p232}{}{}

\textless span id="c1"
style="background-Color:red"\textgreater\textless/span\textgreater{}

\textless span id="c2"
style="background-Color:green"\textgreater\textless/span\textgreater{}

\textless span id="c3"
style="background-Color:orange"\textgreater\textless/span\textgreater{}

\textless span id="c4"
style="background-Color:blue"\textgreater\textless/span\textgreater{}

\textless span id="c5"
style="background-Color:yellow"\textgreater\textless/span\textgreater{}

\textless span id="c6"
style="background-Color:pink"\textgreater\textless/span\textgreater{}

\textless span id="c7"
style="background-Color:black"\textgreater\textless/span\textgreater{}

\textless/html\textgreater{}

This shareable code block can be shared by as many object as you deem it
necessary. It is a good way to keep the file size small. For example,

\textless html\textgreater{}

\textless style\textgreater{}

span \{ position:relative; width:30; height:30; \}

\textless/style\textgreater{}

\textless script\textgreater{}

var code = "";

function init()\{

for (i=0; i\textless=255; i++) \{

code += "\textless span style=\textquotesingle background-

Color:rgb("+i+","+i+","+i+")\textquotesingle\textgreater\textless/span\textgreater";

\}

area.innerHTML = code;

\}

\textless/script\textgreater{}

\textless body onLoad="init()"\textgreater{}

\textless div id="area"\textgreater\textless/div\textgreater{}

\textless/body\textgreater{}

\textless/html\textgreater{}

Obviously creating reusable codes is a good way of modularization.

Use repetition

Consider the following code segment. By properly using a repetition
structure, such as the ``\emph{for}''

structure to

loop, in a game script, the length of the code in a game script is
significantly shortened. Thus, the generate codes

client machine downloads a size-reduced game script file, uses its
processing power to generate and display images on the screen without
losing any game functions.

var k=1;

code="";

function init() \{

for (i=1; i\textless=20; i++) \{

if (i\%5==1) \{

code += "\textless span id=\textquotesingle s"+k+"\textquotesingle{}
style=\textquotesingle top:"+(k-1)*40+"\textquotesingle\textgreater";
code += "\textless img id=\textquotesingle b"+i+"\textquotesingle{}
src=\textquotesingle bee.gif\textquotesingle\textgreater"; k++;

\}

else if (i\%5==0) \{ code += "\textless img
id=\textquotesingle b"+i+"\textquotesingle{}

src=\textquotesingle bee.gif\textquotesingle\textgreater\textless/span\textgreater\textless br\textgreater";
\}

else \{ code += "\textless img
id=\textquotesingle b"+i+"\textquotesingle{}
src=\textquotesingle bee.gif\textquotesingle\textgreater"; \}

\}

area.innerHTML= code;

Game Programming -- Penn Wu

232

\protect\hypertarget{index_split_012.htmlux5cux23p233}{}{}\includegraphics{index-233_1.png}

\includegraphics{index-233_2.png}

Notice that the number of images (e.g. 20) to be generated is
technically of no limit, and such code generation is done completely at
the client side. It programmatically solves the problem of file size
augmentation.

Since this method is a combination of repetition structure with
automatic code generation, it has a potential to allow players to decide
how many objects the game has each time he/she plays the game.

Function

``Lossless'' is a term used in Computer Sciences to refer to a
compression method in which no data lossless code

is lost. A code optimization method is said to be lossless if it retains
the functions without optimization

degrading any part of it.

In game programming, a game program frequently has to generate codes by
itself as response to the player's inputs. The term ``\textbf{automatic
code generation}'' (ACG) refers to a technique programmers can use to
tell computers how they generate game codes automatically according to
the patterns and structures the programmer built inside the program.

Consider the following example. When a user enters a value (e.g. 7) to
\emph{x} (which in this case is a textbox), the variable \emph{y} will
be assigned a value of \emph{x}* \emph{x} (x squared). The value of
\emph{y} (e.g. 49) is then used as the terminal value of the \emph{for}
loop. The \emph{for} loop in turn create a \emph{x} × \emph{x} table as
shown in the figure below.

\textless script\textgreater{}

function init() \{

var x = eval(frm.x.value);

var y = x*x;

frm.style.display="none";

var code1="";

for (i=1; i\textless=y; i++) \{

if (i\%x==0) \{

code1 += "\textless span id=c"+i+"
class=\textquotesingle cell\textquotesingle\textgreater"+i+"\textless/span\textgreater\textless br\textgreater";\}

else \{ code1 += "\textless span id=c"+i+"
class=\textquotesingle cell\textquotesingle\textgreater"+i+"\textless/span\textgreater";\}

\}

area.innerHTML = code1;

\}

\textless/script\textgreater{}

Inside the \emph{for} loop is a predefined code pattern and structure,
as shown below. Notice that \emph{i} is a value automatically assigned
by the \emph{for} loop based on whatever value \emph{i} currently is,
and its value Game Programming -- Penn Wu

233

\protect\hypertarget{index_split_012.htmlux5cux23p234}{}{}is in a range
from 1 to \emph{y}.

\textless span id=c \emph{i}
class=\textquotesingle cell\textquotesingle\textgreater{}
\emph{i}\textless/span\textgreater{}

When \emph{i} is 1, the program automatically generates the following
line (and that builds the first cell):

\textless span id=c \emph{1}
class=\textquotesingle cell\textquotesingle\textgreater{}
\emph{1}\textless/span\textgreater{}

When \emph{i} is 2, the program automatically generates the following
line (and that builds the second cell):

\textless span id=c \emph{2}
class=\textquotesingle cell\textquotesingle\textgreater{}
\emph{2}\textless/span\textgreater{}

When \emph{i} is 3, 4, 5, \ldots, \emph{n}, the program automatically
generates the following lines:

\textless span id=c \emph{3}
class=\textquotesingle cell\textquotesingle\textgreater{}
\emph{3}\textless/span\textgreater{}

\textless span id=c \emph{4}
class=\textquotesingle cell\textquotesingle\textgreater{}
\emph{4}\textless/span\textgreater{}

\textless span id=c \emph{5}
class=\textquotesingle cell\textquotesingle\textgreater{}
\emph{5}\textless/span\textgreater{}

......

\textless span id=c \emph{n}
class=\textquotesingle cell\textquotesingle\textgreater{}
\emph{n}\textless/span\textgreater{}

The following \emph{if} statement contains a condition \textbf{i\%
\emph{x}==0} which defines another pattern. The \%

sign is not the percentage sign, it represent the \textbf{modulus}
operator. Modulus is a lesser known arithmetic operator used to find the
remainder of division between two numbers. For example, 6 ÷

2 = 3 with a remainder 0 (the abbreviation is 6/2=3r0) and 7/5 = 1r2.
You can use the modulus operator just to get the remainder and ignore
the quotient, for example, 6\%2=0 and 7\%5=2. In other words, when a
value \emph{i}\% \emph{x}==0, then I must be a multiple of \emph{x}.

if (i\%x==0) \{

code1 += "\textless span id=c"+i+"
class=\textquotesingle cell\textquotesingle\textgreater"+i+"\textless/span\textgreater\textless br\textgreater";\}

else \{ code1 += "\textless span id=c"+i+"
class=\textquotesingle cell\textquotesingle\textgreater"+i+"\textless/span\textgreater";\}

\}

The above code use the condition \emph{i}\% \emph{x}==0 to decide if the
current value of \emph{i} is a multiple of \emph{x}. If \emph{i} is a
multiple of \emph{x} add an \textbf{\textless br\textgreater{}} tag as
shown below:

\textless span id=c \emph{1}
class=\textquotesingle cell\textquotesingle\textgreater{}
\emph{1}\textless/span\textgreater{} \textbf{\textless br\textgreater{}}
When \emph{x}=7, then each time when \emph{i} is a multiple of 7 (e.g.
7, 14, 21, 28, \ldots49), the \textless br\textgreater{} tag will break
a line, so the next number starts a new line. Namely, 1 through 7 make
the first line, 8

through 13 make the second line, 14 through 20 make the third line, and
so on.

.......

\textless span id=c \emph{6}
class=\textquotesingle cell\textquotesingle\textgreater{}
\emph{6}\textless/span\textgreater{}

\textless span id=c \emph{7}
class=\textquotesingle cell\textquotesingle\textgreater{}
\emph{7}\textless/span\textgreater{} \textbf{\textless br\textgreater{}}

.......

\textless span id=c \emph{13}
class=\textquotesingle cell\textquotesingle\textgreater{}
\emph{13}\textless/span\textgreater{}

\textless span id=c \emph{14}
class=\textquotesingle cell\textquotesingle\textgreater{}
\emph{14}\textless/span\textgreater{}
\textbf{\textless br\textgreater{}}

.......

\textless span id=c \emph{20}
class=\textquotesingle cell\textquotesingle\textgreater{}
\emph{20}\textless/span\textgreater{}

\textless span id=c \emph{21}
class=\textquotesingle cell\textquotesingle\textgreater{}
\emph{21}\textless/span\textgreater{}
\textbf{\textless br\textgreater{}}

\textbf{........}

ACG (automatic code generation) is also commonly used to automatically
insert new objects to a game. Consider the following code, which
continuously adding color bricks (in red, green, or blue) randomly to
the screen.

\textless html\textgreater{}

Game Programming -- Penn Wu

234

\protect\hypertarget{index_split_012.htmlux5cux23p235}{}{}\textless style\textgreater{}

span \{ position: absolute; width:25;height:25 \}

\textless/style\textgreater{}

\textless script\textgreater{}

var i=0;

var s,t, u, v, x, y;

var code=""

function init() \{

s = Math.floor(Math.random()*document.body.clientWidth/25);

t = Math.floor(Math.random()*document.body.clientHeight/25);

u = Math.floor(Math.random()*document.body.clientWidth/25);

v = Math.floor(Math.random()*document.body.clientHeight/25);

x = Math.floor(Math.random()*document.body.clientWidth/25);

y = Math.floor(Math.random()*document.body.clientHeight/25);

code="\textless span id=r"+i+"
style=\textquotesingle left:"+s*25+";top:"+t*25+";backgroundColor:red;\textquotesingle\textgreater\textless/span\textgreater"

code+="\textless span id=g"+i+"
style=\textquotesingle left:"+u*25+";top:"+v*25+";backgroundColor:green;\textquotesingle\textgreater\textless/span\textgreater"

code+="\textless span id=b"+i+"
style=\textquotesingle left:"+x*25+";top:"+y*25+";backgroundColor:blue;\textquotesingle\textgreater\textless/span\textgreater"

bd.innerHTML += code;

i++;

setTimeout("init()", 5);

\}

\textless/script\textgreater{}

\textless body onLoad="init()"\textgreater{}

\textless div id="bd"\textgreater\textless/div\textgreater{}

\textless/body\textgreater{}

\textless/html\textgreater{}

The current width of web browser is represented by
\textbf{document.body.clientWidth}, while the current height of the web
browser is represented by \textbf{document.body.clientHeight}. Since the
width and height of each brick is set to be 25 × 25 (pixels).

document.body.clientWdith

document.body.clientHeight

The following will determine the maximum number of brick per row and per
column allows.

document.body.clientWidth/25

document.body.clientHeight/25

The following will randomly specify a value that represents the
\emph{x}-coordinate of a brick.

Math.floor(Math.random()*document.body.clientWidth/25);

Similarly, the following will randomly specify a value that represents
the \emph{y}-coordinate of a brick.

Game Programming -- Penn Wu

235

\protect\hypertarget{index_split_012.htmlux5cux23p236}{}{}\includegraphics{index-236_1.png}

Math.floor(Math.random()*document.body.clientHeight/25);

You then use the randomly generate ( \emph{x}, \emph{y}) values to one
red, one green, and one blue brick each time when the \textbf{init()}
functions is called. This part of code is: code="\textless span
id=r"+i+"
style=\textquotesingle left:"+s*25+";top:"+t*25+";backgroundColor:red;\textquotesingle\textgreater\textless/span\textgreater"

code+="\textless span id=g"+i+"
style=\textquotesingle left:"+u*25+";top:"+v*25+";backgroundColor:green;\textquotesingle\textgreater\textless/span\textgreater"

code+="\textless span id=b"+i+"
style=\textquotesingle left:"+x*25+";top:"+y*25+";backgroundColor:blue;\textquotesingle\textgreater\textless/span\textgreater"

The innerHTML property will then insert the newly generated codes a
division with an ID \textbf{bd}.

bd.innerHTML += code;

The \textbf{i++} means to add 1 to the current value of \emph{i}, so the
value of \emph{i} is incremented by 1 each time when this statement is
read by computer. The setTimeout() method force the computer to call the
init() function every 5 milliseconds

setTimeout("init()", 5);

A sample output looks:

Colorful bricks will be randomly inserted to blank space.

Clearing

Many game codes create objects that have limited life span (duration).
For example a bullet that physical

the player uses to shoot the enemy. When the bullet is fired, its life
span should soon end if the memory

bullet miss or hit the target. Since each object in the game takes up
certain memory space, it is always a good practice to clear unneeded,
unnecessary or unwanted object from the physical memory.

Consider the following code, which does not delete the unneeded object
from memory.

\textless html\textgreater{}

\textless style\textgreater{}

span \{position:absolute;\}

\textless/style\textgreater{}

\textless script\textgreater{}

function init() \{

p1.innerHTML = "\textless span
id=\textquotesingle c\textquotesingle\textgreater o\textless/span\textgreater";

move();

\}

Game Programming -- Penn Wu

236

\protect\hypertarget{index_split_012.htmlux5cux23p237}{}{}function
move() \{

if (c.style.pixelLeft \textgreater=200) \{

clearTimeout(s1); p1.innerText = \textquotesingle\textquotesingle;\}

else \{ c.style.pixelLeft++;

s1=setTimeout(\textquotesingle move()\textquotesingle, 10);\}

\}

\textless/script\textgreater{}

\textless body onKeyDown="init();"\textgreater{}

\textless b id="p1"\textgreater\textless/b\textgreater{}

\textless/body\textgreater{}

\textless/html\textgreater{}

This game code functions well, but the only problem is. Each time when
the player press a key, the ``c'' object is created again without
deleting the previously created ``c'' object. Consequently, after
pressing a key many times, the player will encounted some problem caused
by memory overuse.

Applying object-oriented programming technique, in this case, helps to
remove unneeded objects.

JavaScript support object-oriented programming, so you can create an
object, use the object for a period of time, and delete the object
later.

\textless html\textgreater{}

\textless style\textgreater{}

span \{position:absolute;\}

\textless/style\textgreater{}

\textless script\textgreater{}

var i = 1;

function init() \{

\textbf{obj = new Object();}

obj.n="\textless span
id=\textquotesingle c"+i+"\textquotesingle\textgreater o\textless/span\textgreater";
p1.innerHTML = obj.n;

move();

\}

function move() \{

code = "if (c"+i+".style.pixelLeft \textgreater=200) \{ "; code += "
clearTimeout(s1); \textbf{obj = \textquotesingle\textquotesingle{}}
;p1.innerText = \textquotesingle\textquotesingle; i++;\}"; code += "else
\{ c"+i+".style.pixelLeft++; status=i; "; code += "
s1=setTimeout(\textquotesingle move()\textquotesingle, 10);\}";

eval(code);

\}

\textless/script\textgreater{}

\textless body onKeyDown="init();"\textgreater{}

\textless b id="p1"\textgreater\textless/b\textgreater{}

\textless/body\textgreater{}

\textless/html\textgreater{}

To create an object, assign an object ID (e.g. obj) to the Object()
method.

\textbf{obj = new Object();}

Game Programming -- Penn Wu

237

\protect\hypertarget{index_split_012.htmlux5cux23p238}{}{}You can then,
create properties for the \textbf{obj} object. In the following example,
``text'' is a new property of the obj object with a string value ``Test
the code''.

obj. \textbf{test} = "Test the code";

To delete the object, simply assign the object ID with null value;
\textbf{obj = \textquotesingle\textquotesingle{}} ;

Oncethe object is deleted, the game codes that control it are no longer
kept in physical meory.

Execution

In a previous lecture, you tried the UFO game, whose source code looks:
issues and

speed

\textless html\textgreater{}

\textless style\textgreater{}

.dots \{position:absolute; color:white\}

\textless/style\textgreater{}

\textless script\textgreater{}

code1 ="";

function draw\_star() \{

for (i=1; i\textless=600; i++) \{

x = Math.round(Math.random()*screen.width);

y = Math.round(Math.random()*screen.height);

code1 += "\textless span class=dots id=dot"+i;

code1 += " style=\textquotesingle left:" + x + "; top:" + y
+"\textquotesingle\textgreater.\textless/span\textgreater";

\}

bar.innerHTML = code1;

\}

function ufo\_fly() \{

ufo.style.left = Math.round(Math.random()*screen.width);

ufo.style.top = Math.round(Math.random()*screen.height);

setTimeout("ufo\_fly()", 500);

\}

\textless/script\textgreater{}

\textless body bgcolor=black
onLoad="draw\_star();ufo\_fly()"\textgreater{}

\textless div id="bar"
style="z-index:0"\textgreater\textless/div\textgreater{}

\textless img src="ufo.gif" id="ufo"
style="position:absolute;z-index:3;"\textgreater{}

\textless/body\textgreater{}

\textless/html\textgreater{}

If you modify just the \textbf{draw\_star()} function to the following,
the loading time will be much longer, and the execution will be slower.

\textless script\textgreater{}

function draw\_star() \{

for (i=1; i\textless=300; i++) \{

x = Math.round(Math.random()*screen.width);

y = Math.round(Math.random()*screen.height);

code1 = "\textless span class=dots id=dot"+i;

code1 += " style=\textquotesingle left:" + x + "; top:" + y
+"\textquotesingle\textgreater.\textless/span\textgreater";
bar.innerHTML += code1;

\}

\}

Game Programming -- Penn Wu

238

\protect\hypertarget{index_split_012.htmlux5cux23p239}{}{}..................

\textless/script\textgreater{}

Try take a closer look at these two code segments:

\textbf{code1 ="";}

function draw\_star() \{

for (i=1; i\textless=600; i++) \{

x = Math.round(Math.random()*screen.width);

y = Math.round(Math.random()*screen.height);

code1 += "\textless span class=dots id=dot"+i;

code1 += " style=\textquotesingle left:" + x + "; top:" + y
+"\textquotesingle\textgreater.\textless/span\textgreater";

\}

\textbf{bar.innerHTML = code1;} // outside the for loop \textbf{}

\}

function draw\_star() \{

for (i=1; i\textless=300; i++) \{

x = Math.round(Math.random()*screen.width);

y = Math.round(Math.random()*screen.height);

code1 = "\textless span class=dots id=dot"+i;

code1 += " style=\textquotesingle left:" + x + "; top:" + y
+"\textquotesingle\textgreater.\textless/span\textgreater";
\textbf{bar.innerHTML += code1;} // inside the for loop \textbf{}

\}

\}

The trick is that in the upper one, the following line is placed outside
the \emph{for} loop. But, in the lower one, it is place insider the
\emph{for} loop.

bar.innerHTML = code1;

When being executed, the one outside the \emph{for} loop, does not have
to insert the HTML code till the entire \emph{for} loop completes. Each
time the a new line, similar to the following, is produced by the for
loop, they are temporarily stored in the computer's memory (buffer) as
additional one to the previous ones.

\textless span class=dots id=dot \emph{n} style=\textquotesingle left:
\emph{x}; top:
\emph{y}\textquotesingle\textgreater.\textless/span\textgreater{} Where
\emph{n} is one of the number in \{1, 2, 3, \ldots, 300\}, while
\emph{x}, \emph{y} and randomly generated by the draw\_star() function.

At the end of the for loop, the computer memory accumulates the
following new HTML codes, and insert all them to the area defined by
\textless div id="bar"
style="z-index:0"\textgreater\textless/div\textgreater.

\textless span class=dots id=dot1 style=\textquotesingle left:
\emph{x1}; top:
\emph{y1}\textquotesingle\textgreater.\textless/span\textgreater{}

\textless span class=dots id=dot2 style=\textquotesingle left:
\emph{x2}; top:
\emph{y2}\textquotesingle\textgreater.\textless/span\textgreater{}

\textless span class=dots id=dot3 style=\textquotesingle left:
\emph{x3}; top:
\emph{y3}\textquotesingle\textgreater.\textless/span\textgreater{}

......

\textless span class=dots id=dot300 style=\textquotesingle left:
\emph{x300}; top:
\emph{y300}\textquotesingle\textgreater.\textless/span\textgreater{} It
is faster to insert all the above lines to the web page, then inserting
then one by one, which is exactly what the second code segement has to
do. When you place the following line inside the for loop, the
processing mechanism becomes redundant.

bar.innerHTML += code1;

First of all, each time when a for loop starts, the computer needs to go
to the \textless div id=''bar''

Game Programming -- Penn Wu

239

\protect\hypertarget{index_split_012.htmlux5cux23p240}{}{}\ldots\textgreater\textless/div\textgreater{}
area to read the currently available lines of code similar to following,
and then insert a new one to it:

\textless span class=dots id=dot \emph{n} style=\textquotesingle left:
\emph{x}; top:
\emph{y}\textquotesingle\textgreater.\textless/span\textgreater{} It
takes a significantly longer time to find the spot, read the contents,
and add new contents to it.

Plus, the computer has to do it 300 times. No wonder why the execution
time is longer, and the speed is significantly slower.

Review

1. \_\_ is the process of modifying a system to make some aspect of it
work more efficiently or use Questions

fewer resources.

A. Optimization

B. File size reduction

C. Performance tuning

D. Finetuning

2. If the program contains the same or similar blocks of statements or
it is required to process the same function several times, you can avoid
redundancy by using \_\_ techniques.

A. reengineering

B. collaboration

C. modularization

D. packaging

3. The following is an example of \_\_.

\textless style\textgreater span \{ position:relative; width:30;
height:30; backgroundColor:red\}

\textless/style\textgreater{}

\textless span id="c1"\textgreater\textless/span\textgreater{}

\textless span id="c2"\textgreater\textless/span\textgreater{}

\textless span id="c3"\textgreater\textless/span\textgreater{}

\textless span id="c4"\textgreater\textless/span\textgreater{}

\textless span id="c5"\textgreater\textless/span\textgreater{}

\textless span id="c6"\textgreater\textless/span\textgreater{}

\textless span id="c7"\textgreater\textless/span\textgreater{}

A. automatic code generation

B. modularization

C. object specialization

D. object generalization

4. By properly using a repetition structure, such as the ``for'' loop,
in a game script, \_\_.

A. the length of the code in a game script is significantly shortened.

B. the client machine downloads a size-reduced game script file.

C. the client machine uses its processing power to generate and display
images on the screen without losing any game functions.

D. All of the above.

5. The following is an example of \_\_.

\textless script\textgreater{}

function init() \{

var x = eval(frm.x.value);

var y = x*x;

frm.style.display="none";

Game Programming -- Penn Wu

240

\protect\hypertarget{index_split_012.htmlux5cux23p241}{}{}

var code1="";

for (i=1; i\textless=y; i++) \{

if (i\%x==0) \{

code1 += "\textless span id=c"+i+"
class=\textquotesingle cell\textquotesingle\textgreater"+i+"\textless/span\textgreater\textless br\textgreater";\}

else \{ code1 += "\textless span id=c"+i+"
class=\textquotesingle cell\textquotesingle\textgreater"+i+"\textless/span\textgreater";\}

\}

area.innerHTML = code1;

\}

\textless/script\textgreater{}

A. automatic code generation

B. modularization

C. object specialization

D. object generalization

6. In the following code, the condition i\%x==0 \_\_.

if (i\%x==0) \{ ...... \}

A. decide if i is a percentage of x

B. determine if i equals x

C. decide if the current value of i is a multiple of x

D. determine if i equals 0

7. Given the following code segment, what happens when i=14 and x=7?

if (i\%x==0) \{

code1 += "\textless span id=c"+i+"
class=\textquotesingle cell\textquotesingle\textgreater"+i+"\textless/span\textgreater\textless br\textgreater";\}

else \{ code1 += "\textless span id=c"+i+"
class=\textquotesingle cell\textquotesingle\textgreater"+i+"\textless/span\textgreater";\}

\}

A. Since i is a multiple of x, it the output is \textless span id=c14
class=\textquotesingle cell\textquotesingle\textgreater14\textless/span\textgreater\textless br\textgreater.

B. Since i is a multiple of x, it the output is \textless span id=c14
class=\textquotesingle cell\textquotesingle\textgreater14\textless/span\textgreater.

C. Since i is a multiple of x, it the output is \textless span id=c7
class=\textquotesingle cell\textquotesingle\textgreater7\textless/span\textgreater\textless br\textgreater.

D. Since i is a multiple of x, it the output is \textless span id=c7
class=\textquotesingle cell\textquotesingle\textgreater7\textless/span\textgreater.

8. The following line \_\_.

m1 = new Object();

A. assign the variable m1 with a new value "Object()".

B. create a new object with an ID m1.

C. for the computer to ignore the m1 object.

D. delete the m1 object from the current program.

9. In JavaScript, you can delete an object by \_\_.

A. assigning a new value to it.

B. re-declaring the object.

C. assining a null value to it.

D. converting the object to a variable.

10. Given the following incomplete code segemetn, where will you insert
the \textbf{bar.innerHTML =}

\textbf{code;} statement for the best execution performance?

.....

code = "";

function init() \{

// area1

Game Programming -- Penn Wu

241

\protect\hypertarget{index_split_012.htmlux5cux23p242}{}{} for (i=1;
i\textless{} 10; i++) \{

code += "\textless hr size="+i+" width=100\%\textgreater";

// area2

\}

// area3

\}

// area4

.....

\textless div id="bar"\textgreater\textless/div\textgreater{}

.....

A. area1

B. area2

C. area3

D. area4

Game Programming -- Penn Wu

242

\protect\hypertarget{index_split_012.htmlux5cux23p243}{}{}\includegraphics{index-243_1.png}

Lab \#13

Code Optimization

Learning Activity \#1:

1. Use Notepad to create a new file named lab13\_1.htm with the
following contents:

\textless html\textgreater{}

\textless style\textgreater{}

span \{position:absolute;\}

\textless/style\textgreater{}

\textless script\textgreater{}

var i = 1;

function init() \{

\textbf{obj = new Object();}

obj.n="\textless span
id=\textquotesingle c"+i+"\textquotesingle\textgreater o\textless/span\textgreater";
p1.innerHTML = obj.n;

move();

\}

function move() \{

code = "if (c"+i+".style.pixelLeft \textgreater=200) \{ "; code += "
clearTimeout(s1); \textbf{obj = \textquotesingle\textquotesingle{}}
;p1.innerText = \textquotesingle\textquotesingle; i++;\}"; code += "else
\{ c"+i+".style.pixelLeft++; status=i; "; code += "
s1=setTimeout(\textquotesingle move()\textquotesingle, 10);\}";

eval(code);

\}

\textless/script\textgreater{}

\textless body onKeyDown="init();"\textgreater{}

\textless b id="p1"\textgreater\textless/b\textgreater{}

\textless/body\textgreater{}

\textless/html\textgreater{}

2. Use Internet Explorer to test the program. There will be a
continuously changing number displayed in the status bar.

Learning Activity \#2:

1. Use Notepad to create a new file named lab13\_2.htm with the
following contents:

\textless html\textgreater{}

\textless style\textgreater{}

.cell \{

Game Programming -- Penn Wu

243

\protect\hypertarget{index_split_012.htmlux5cux23p244}{}{}\includegraphics{index-244_1.png}

\includegraphics{index-244_2.png}

border:solid 1 black; width:30;

height:30;background-Color:white;

text-align:center;text-valign:middle\}

\textless/style\textgreater{}

\textless script\textgreater{}

function init() \{

var x = eval(frm.x.value);

var y = x*x;

frm.style.display="none";

var code1="";

for (i=1; i\textless=y; i++) \{

if (i\%x==0) \{

code1 += "\textless span id=c"+i+"
class=\textquotesingle cell\textquotesingle\textgreater"+i+"\textless/span\textgreater\textless br\textgreater";\}

else \{ code1 += "\textless span id=c"+i+"
class=\textquotesingle cell\textquotesingle\textgreater"+i+"\textless/span\textgreater";\}

\}

area.innerHTML = code1;

\}

\textless/script\textgreater{}

\textless body\textgreater{}

\textless form id="frm"\textgreater{}

\textless i\textgreater x\textless/i\textgreater: \textless input
type="text" id="x" size=3\textgreater\textless br /\textgreater{}

\textless input type="button" value="Go" onClick="init()"\textgreater{}

\textless/form\textgreater{}

\textless div id="area"\textgreater\textless/div\textgreater{}

\textless/body\textgreater{}

\textless/html\textgreater{}

2. Use Internet Explorer to test the program. Enter a number (e.g. 12)
and then click Go.

Accept User Input

Result of automatic code generation

Learning Activity \#3:

1. Use Notepad to create a new file named lab13\_3.htm with the
following contents:

\textless html\textgreater{}

\textless style\textgreater{}

span \{ position: absolute; width:20 \}

\textless/style\textgreater{}

\textless script\textgreater{}

var i=0;

var s,t, x, y;

var code=""

function init() \{

s = Math.floor(Math.random()*document.body.clientWidth);

Game Programming -- Penn Wu

244

\protect\hypertarget{index_split_012.htmlux5cux23p245}{}{}\includegraphics{index-245_1.png}

t = Math.floor(Math.random()*document.body.clientHeight);

x = Math.floor(Math.random()*document.body.clientWidth);

y = Math.floor(Math.random()*document.body.clientHeight);

code="\textless span id=r"+i+"
style=\textquotesingle left:"+s+";top:"+t+";background-Color:red;\textquotesingle\textgreater\textless/span\textgreater"

code+="\textless span id=b"+i+"
style=\textquotesingle left:"+x+";top:"+y+";background-Color:blue;\textquotesingle\textgreater\textless/span\textgreater"

bd.innerHTML += code;

i++;

setTimeout("init()", 5);

\}

\textless/script\textgreater{}

\textless body onLoad="init()"\textgreater{}

\textless div id="bd"\textgreater\textless/div\textgreater{}

\textless/body\textgreater{}

\textless/html\textgreater{}

2. Use Internet Explorer to test the program. A sample output looks:
Learning Activity \#4:

1. Use Internet Explorer to go to
http://business.cypresscollege.edu/\textasciitilde pwu/cis261/files/bee.gif
and download the bee.gif file.

2. Use Notepad to create a new file named lab13\_4.htm with the
following contents:

\textless html\textgreater{}

\textless style\textgreater{}

img \{ position:relative; \}

span \{position:absolute;\}

\textless/style\textgreater{}

\textless script\textgreater{}

var k=1;

code="";

function init() \{

for (i=1; i\textless=20; i++) \{

if (i\%5==1) \{ code += "\textless span
id=\textquotesingle s"+k+"\textquotesingle{}
style=\textquotesingle top:"+(k-1)*40+"\textquotesingle\textgreater\textless img
id=\textquotesingle b"+i+"\textquotesingle{}
src=\textquotesingle bee.gif\textquotesingle\textgreater"; k++; \}

else if (i\%5==0) \{ code += "\textless img
id=\textquotesingle b"+i+"\textquotesingle{}
src=\textquotesingle bee.gif\textquotesingle\textgreater\textless/span\textgreater\textless br\textgreater";
\}

else \{ code += "\textless img
id=\textquotesingle b"+i+"\textquotesingle{}
src=\textquotesingle bee.gif\textquotesingle\textgreater"; \}

\}

area.innerHTML= code;

for (j=1; j\textless=4; j++) \{

eval("s"+j+".style.pixelWidth = 250;");

\}

Game Programming -- Penn Wu

245

\protect\hypertarget{index_split_012.htmlux5cux23p246}{}{}\includegraphics{index-246_1.png}

movebee();

\}

function movebee() \{

if (s1.style.pixelLeft + s1.style.pixelWidth \textgreater=
document.body.clientWidth) \{

s1.style.pixelLeft = 0

\}

else \{ s1.style.pixelLeft += 5; \}

if (s2.style.pixelLeft + s2.style.pixelWidth \textgreater=
document.body.clientWidth) \{

s2.style.pixelLeft = 0

\}

else \{

if (s1.style.pixelLeft\textless=40) \{ s2.style.pixelLeft =0; \}

else \{ s2.style.pixelLeft += 5; \}

\}

if (s3.style.pixelLeft + s3.style.pixelWidth \textgreater=
document.body.clientWidth) \{

s3.style.pixelLeft = 0

\}

else \{ s3.style.pixelLeft += 5; \}

if (s4.style.pixelLeft + s4.style.pixelWidth \textgreater=
document.body.clientWidth) \{

s4.style.pixelLeft = 0

\}

else \{

if (s3.style.pixelLeft\textless=40) \{ s4.style.pixelLeft =0; \}

else \{ s4.style.pixelLeft += 5; \}

\}

setTimeout("movebee()", 100);

\}

\textless/script\textgreater{}

\textless body onLoad="init()"\textgreater{}

\textless div id="area"
style="position:absolute"\textgreater\textless/div\textgreater{}

\textless/body\textgreater{}

\textless/html\textgreater{}

3. Use Internet Explorer to test the program.

Learning Activity \#5:

Game Programming -- Penn Wu

246

\protect\hypertarget{index_split_012.htmlux5cux23p247}{}{}Note: This
code is a fully functional game code, but if will function better if you
apply the object-oriented programming techniques (as dicussed in the
lecture) to it. Try modify the code with object-oriented programming.

1. Use Notepad to create a new file named lab13\_5.htm with the
following contents:

\textless html\textgreater{}

\textless script\textgreater{}

var i=0;

function init() \{

if (red.style.pixelLeft \textgreater= document.body.clientWidth) \{

red.style.pixelLeft=0; red.style.pixelTop+=20;

\}

else if (red.style.pixelTop \textgreater= document.body.clientHeight-20)
\{

red.style.pixelTop=0;

\}

red.style.pixelLeft+=5;

for (k=1; k\textless=i; k++) \{

eval("b"+i+".style.pixelTop+=5");

msg1="if (b"+i+".style.pixelTop \textgreater=
document.body.clientHeight-100)"; msg1+="
\{b"+i+".style.display=\textquotesingle none\textquotesingle;\}";

eval(msg1);

msg2="if (b"+i+".style.pixelLeft\textgreater=red.style.pixelLeft \&\&";
msg2+=" b"+i+".style.pixelLeft\textless=red.style.pixelLeft+20 \&\&";
msg2+=" b"+i+".style.pixelTop+20\textgreater=red.style.pixelTop \&\&";
msg2+=" b"+i+".style.pixelTop\textless=red.style.pixelTop+20) ";
msg2+="\{
red.style.display=\textquotesingle none\textquotesingle;p1.innerText=\textquotesingle Game
Over!\textquotesingle;\}"; eval(msg2);

\}

setTimeout("init()", 20);

\}

function fire() \{

i=i+1;

b\_top=Math.floor(Math.random()*document.body.clientWidth);

code="\textless span id=\textquotesingle b"+i+"\textquotesingle{}
style=\textquotesingle position:absolute; "; code+="
background-Color:blue; width:20; ";

code+=" height:20; top:0;
left:"+b\_top+"\textquotesingle\textgreater\textless/span\textgreater{}
"; bd.innerHTML+=code;

eval("b"+(i-1)+".style.display=\textquotesingle none\textquotesingle;");

\}

\textless/script\textgreater{}

\textless body id="bd" onLoad="init()" onKeyDown="var f = new fire()"

style="overflow: hidden"\textgreater{}

\textless span id="red" style="position:absolute; background-Color:red;
width:20; height:20; top:100"\textgreater\textless/span\textgreater{}

\textless p id="p1"\textgreater\textless/p\textgreater{}

\textless/body\textgreater{}

\textless/html\textgreater{}

2. Use Internet Explorer to test the program.

\textbf{Submittal}

Upon completing all the learning activities,

Game Programming -- Penn Wu

247

\protect\hypertarget{index_split_012.htmlux5cux23p248}{}{}1. Upload all
files you created in this lab to your remote web server.

2. Log in to to Blackboard, launch Assignment 13, and then scroll down
to question 11.

3. Copy and paste the URLs to the textbox. For example,

•

http://www.geocities.com/cis261/lab13\_1.htm

•

http://www.geocities.com/cis261/lab13\_2.htm

•

http://www.geocities.com/cis261/lab13\_3.htm

•

http://www.geocities.com/cis261/lab13\_4.htm

•

http://www.geocities.com/cis261/lab13\_5.htm

No credit is given to broken link(s).

Game Programming -- Penn Wu

248

\protect\hypertarget{index_split_012.htmlux5cux23p249}{}{}

Lecture \#14

Using Canvas and ExplorerCanvas

\protect\hypertarget{index_split_013.html}{}{}

\hypertarget{index_split_013.htmlux5cux23calibre_pb_12}{%
\subsection{Introduction}\label{index_split_013.htmlux5cux23calibre_pb_12}}

The \textbf{canvas} element is part of HTML5 and allows for dynamic
scriptable rendering of bitmap images. Canvas consists of a drawable
region defined in HTML code with height and width attributes. JavaScript
code may access the area through a full set of drawing functions similar
to other common 2D APIs, thus allowing for dynamically generated
graphics. Some anticipated uses of the canvas include building graphs,
animations, games, and image composition.

Firefox, Safari and Opera 9 support the canvas tag by default, but
Internet Explorer does not.

Thanks to Google's efforts in creating the ExplorerCanvas, which seems
to solve this problem to some extent.

ExplorerCanvas brings the same functionality to Internet Explorer. To
use, web developers only need to include a single script tag in their
existing web pages.

Good news is that Google distributes ExplorerCanvas with the Open Source
\textbf{BSD License} and the \textbf{}

\textbf{Apache License}. This is free license that allows use of the
source code for the development of free and open source software as well
as proprietary software.

Sample codes in this lecture note that use ExplorerCanvas respect
Google's copyright by adding the following line at the beginning of the
HTML codes. \textbf{}

\textless!-\/-This code uses ExplorerCanvas, developed by Google
Inc.-\/-\textgreater{} Using

Simply include the ExplorerCanvas tag in the same directory as your HTML
files, and ExplorerCanva

add the following code to your page, preferably in the
\textless head\textgreater{} tag.

s

\textless!-\/-{[}if IE{]}\textgreater{}

\textless script type="text/javascript"
src="excanvas.js"\textgreater\textless/script\textgreater{}

\textless!{[}endif{]}-\/-\textgreater{}

For example,

\textless!-\/-This code uses ExplorerCanvas, developed by Google
Inc.-\/-\textgreater{}

\textless html\textgreater\textless head\textgreater{}

\textbf{\textless!-\/-{[}if IE{]}\textgreater\textless script
type="text/javascript"}

\textbf{src="excanvas.js"\textgreater\textless/script\textgreater\textless!{[}endif{]}-\/-\textgreater{}}

\textless script type="text/javascript"\textgreater{}

function init() \{

var sid = document.getElementById("sqr");

if (sid.getContext) \{

var obj = sid.getContext("2d");

\textbf{obj.fillStyle = "rgb(255, 0, 0)";}

\textbf{obj.fillRect (10, 10, 100, 100);}

\}

\}

\textless/script\textgreater\textless/head\textgreater{}

\textless body onload="init();"\textgreater{}

\textless canvas id="sqr" width="150"
height="150"\textgreater\textless/canvas\textgreater{}

\textless/body\textgreater\textless/html\textgreater{}

Game Programming -- Penn Wu

249

\protect\hypertarget{index_split_013.htmlux5cux23p250}{}{}

The \textless canvas\textgreater{}

The \textless canvas\textgreater{} element creates a fixed size drawing
surface that can render graphics. It has only element

three attributes, \textbf{id}, \textbf{width} and \textbf{height}. These
are both optional and can also be set using DOM

properties. When no width and height attributes are specified, the
canvas will initially be 300

pixels wide and 150 pixels high. To avoid compatibility issues, be sure
to have an end tag

\textless/canvas\textgreater.

\textless canvas id="sqr" width="150"
height="150"\textgreater\textless/canvas\textgreater{} The \textbf{id}
attribute is used to identify the \textless canvas\textgreater{} tag so
it can be used as an object to work with the Document Object Model (DOM)
API.

The \textless canvas\textgreater{} element also works with CSS, so it
can be styled with margin, border, background color, etc. For example,

\textless style type="text/css"\textgreater{}

canvas \{ border: 1px solid black; \}

\textless/style\textgreater{}

The \textbf{getContext()} method is a DOM method that retrieves the
current active document type and document ID. It works with the
\textbf{getElementById} method to identify the object that will serve as
the drawing ground. For example,

var sid = document.getElementById(\textquotesingle sqr\textquotesingle);

..............

var obj = sid.getContext(\textquotesingle2d\textquotesingle);

..............

\textless canvas id="sqr" width="150"
height="150"\textgreater\textless/canvas\textgreater{} In the above
example, ``sqr'' is the ID of the \textless canvas\textgreater, which is
the drawing ground. The \textbf{getElementById} method retrieve the ID
(sqr) and transfers it a variable \textbf{sid}. The sid variable is then
used as a object to retrieve drawing context that will be specified by
getContext() method.

You can combine the following two lines,

var sid = document.getElementById(\textquotesingle sqr\textquotesingle);

var obj = sid.getContext(\textquotesingle2d\textquotesingle);

to

var obj =
document.getElementById(\textquotesingle sqr\textquotesingle).getContext(\textquotesingle2d\textquotesingle);

There are still some browser that do not support the
\textless canvas\textgreater{} element. You may want to display some
message notifying the viewer such incompatibility. A simple
if..then..else statement can do the work. For example,

var sid = document.getElementById(\textquotesingle sqr\textquotesingle);

if (sid.getContext)\{

var obj = canvas.getContext(\textquotesingle2d\textquotesingle);

// drawing code here

\} else \{

// message saying canvas unsupported here

\}

The drawing

The drawing of the graphics is done by drawing functions. Commonly used
drawing functions are: codes

Table: Drawing functions

Function

Description

Game Programming -- Penn Wu

250

\protect\hypertarget{index_split_013.htmlux5cux23p251}{}{}fillRect(x,y,width,height)

Draws a filled rectangle

strokeRect(x,y,width,height) Draws a rectangular outline

clearRect(x,y,width,height)

Clears the specified area and makes it fully

transparent

rect(x, y, width, height)

Adds a rectangular path to the path list

where \emph{x} and \emph{y} specify the position on the canvas (relative
to the origin) of the top-left corner of the rectangle. \emph{width} and
\emph{height} are pretty obvious.

Drawing function only handler the sketching. If you want to apply colors
to a shape, there are two important properties to use:
\textbf{fillStyle} and \textbf{strokeStyle}. Basically strokeStyle is
used for setting the shape outline color and fillStyle is for the fill
color. Color codes can be a string representing a CSS color value, a
gradient object, or a pattern object. We\textquotesingle ll look at
gradient and pattern objects later. By default, the stroke and fill
color are set to black (CSS color value \#000000). For example,

obj.fillStyle = "orange";

obj.fillStyle = "\#FFA500";

obj.fillStyle = "rgb(255,165,0)";

obj.fillStyle = "rgba(255,165,0,1)";

All the above generates exactly the same color.

The rgb() function has syntax of:

rgb(redvalue, greeValue, blueValue);

Each color value is between 0 and 255.

The \textbf{rgba()} function is similar to the \textbf{rgb()} function
but it has one extra parameter.

rgba(redvalue, greeValue, blueValue, transparencyValue);

The last parameter of rgba() sets the transparency value of this
particular color. The valid range is from 0.0 (fully transparent) to 1.0
(fully opaque).

To draw an rectangle with red color, use:

obj.fillStyle = "rgb(255, 0, 0)"; // define color

obj.fillRect (10, 10, 100, 100); // sketching

Currently canvas only supports one primitive shape -
\textbf{rectangles}. All other shapes must be created by combining one
or more paths.

The drawing

Shapes, such as polygon, are sketched using drawing path with some of
the following functions.

path

•

\textbf{beginPath()}: creates a path

•

\textbf{closePath()}: close the shape by drawing a straight line from
the current point to the start

•

\textbf{stroke()}: draw an outlined shape

•

\textbf{fill()}: paint a solid shape

When calling the fill() method any open shapes will be closed
automatically and it isn't necessary to use the closePath() method. The
following code, for example, will not formed a closed rectangle.

obj.beginPath();

obj.moveTo(10,50);

obj.lineTo(10,10);

Game Programming -- Penn Wu

251

\protect\hypertarget{index_split_013.htmlux5cux23p252}{}{}\includegraphics{index-252_1.png}

obj.lineTo(110,10);

obj.lineTo(110,50);

obj.stroke();

The output looks like:

The following code draws a triangle with filled color in red.

\textless!-\/-This code uses ExplorerCanvas, developed by Google
Inc.-\/-\textgreater{}

\textless html\textgreater\textless head\textgreater{}

\textless!-\/-{[}if IE{]}\textgreater\textless script
type="text/javascript"

src="excanvas.js"\textgreater\textless/script\textgreater\textless!{[}endif{]}-\/-\textgreater{}

\textless script type="text/javascript"\textgreater{}

function init() \{

var sid = document.getElementById("sqr");

if (sid.getContext) \{

var obj = sid.getContext("2d");

obj.fillStyle = "rgba(255, 0, 0)";

obj.beginPath();

obj.moveTo(100,10);

obj.lineTo(50,150);

obj.lineTo(100,200);

obj.fill();

\}

\}

\textless/script\textgreater\textless/head\textgreater{}

\textless body onload="init();"\textgreater{}

\textless canvas id="sqr" width="600"
height="200"\textgreater\textless/canvas\textgreater{}

\textless/body\textgreater\textless/html\textgreater{}

One important thing to know is that the stroke() function always display
the line in black, so the \textbf{fillStyle} project in the following
code will not display the lines in red.

obj.fillStyle = "rgb(255, 0, 0)";

obj.beginPath();

obj.moveTo(100,10);

obj.lineTo(50,150);

obj.lineTo(100,200);

obj.stroke();

When the canvas is initialized or the beginPath method is called, the
starting point is defaulted to the coordinate (0,0). The
\textbf{moveTo()} function defines the starting point. You can probably
best think of this as lifting a pen or pencil from one spot on a piece
of paper and placing it on the next.

The syntax is:

moveTo(x, y)

where x and y are the coordinates of the new starting point.

The lineTo method draws straight lines, and the syntax is:

Game Programming -- Penn Wu

252

\protect\hypertarget{index_split_013.htmlux5cux23p253}{}{}\includegraphics{index-253_1.png}

\includegraphics{index-253_2.png}

lineTo(x, y)

where x and y are the coordinates of the line\textquotesingle s end
point.

The following draws two triangles.

obj.beginPath();

obj.moveTo(100,10);

obj.lineTo(50,150);

obj.lineTo(150,50);

obj.lineTo(100,200);

obj.fill();

The \textbf{arc()} method draws arcs or circles. The syntax is:

arc(x, y, radius, startAngle, endAngle, anticlockwise);

where x and y are the coordinates of the circle\textquotesingle s
center. Radius is self explanatory. The startAngle and endAngle
parameters define the start and end points of the arc in radians. The
starting and closing angle are measured from the x axis. The
anticlockwise parameter is a boolean value which when true draws the arc
anticlockwise, otherwise in a clockwise direction.

For example, to draw an arc with a center (100, 100) and a radius of 50,
use:

\textless!-\/-This code uses ExplorerCanvas, developed by Google
Inc.-\/-\textgreater{}

\textless html\textgreater\textless head\textgreater{}

\textless!-\/-{[}if IE{]}\textgreater\textless script
type="text/javascript"

src="excanvas.js"\textgreater\textless/script\textgreater\textless!{[}endif{]}-\/-\textgreater{}

\textless script type="text/javascript"\textgreater{}

var i = 0;

function init() \{

var sid = document.getElementById("sqr");

if (sid.getContext) \{

var obj = sid.getContext("2d");

\textbf{obj.arc(100, 100, 50, 2 , Math.PI+(Math.PI*i)/2, true);}

obj.stroke();

\}

i+=0.1;

\}

\textless/script\textgreater\textless/head\textgreater{}

\textless body onload="init();" onKeyDown="init()"\textgreater{}

\textless canvas id="sqr" width="600"

height="600"\textgreater\textless/canvas\textgreater{}

\textless/body\textgreater\textless/html\textgreater{}

Angles in the arc function are measured in radians, not degrees. To
convert degrees to radians you can use the following JavaScript
expression:

var radians = (Math.PI/180)*degrees

Game Programming -- Penn Wu

253

\protect\hypertarget{index_split_013.htmlux5cux23p254}{}{}

Review

1. Which statement is correct?

Questions

A. The canvas element is part of HTML5 and allows for dynamic scriptable
rendering of bitmap images.

B. The canvas element consists of a drawable region defined in HTML code
with height and width attributes.

C. Internet Explorer 6 and 7 do not support the canvas element.

D. All of the above.

2. Google provides the \_\_ library that brings the canvas element to
Internet Explorer.

A. iCanvas

B. J3DJS

C. JSON

D. AJAX

3. Given the following code, which statement is correct?

obj.fillRect(10, 10, 100, 100);

A. The top-left corner of the rectangle is (100, 100).

B. The top-left corner of the rectangle is (10, 10).

C. It generates a square that has width and height of 10 pixels each.

D. It generates a rectangle that has a width of 10 pixels and height of
100 pixels.

4. Which defines a drawing ground that has a width of 400 pixels and
height of 300 pixels?

A. \textless canvas id="sqr"\textgreater\textless td width="400"
height="300"\textgreater\textless/td\textgreater\textless/canvas\textgreater{}
B. \textless canvas id="sqr"
size="300*400"\textgreater\textless/canvas\textgreater{} C.
\textless canvas id="sqr" width="400"
height="300"\textgreater\textless/canvas\textgreater{} D.
\textless canvas id="sqr" w="400"
h="300"\textgreater\textless/canvas\textgreater{} 5. Which function
draws a rectangular outline?

A. fillRect()

B. strokeRect()

C. clearRect()

D. rect()

6. Given the following code with error in it, how will you correct the
error?

obj.fillStyle = "rgba(255,105,10,2)";

A. Change rgba to rgb and keep the rest.

B. remove 2 and the comma before it; keep the rest.

C. change 2 to any value in the range from 0.0 to 1.0.

D. add the \# sign before every number.

7. Given the following code segment, the color of line is \_\_.

obj.fillStyle = "rgb(255, 0, 0)";

obj.beginPath();

obj.moveTo(100,10);

obj.lineTo(50,150);

obj.stroke();

A. red

B. black

C. while

D. orange

Game Programming -- Penn Wu

254

\protect\hypertarget{index_split_013.htmlux5cux23p255}{}{}

8. Given the following code, \_\_.

obj.beginPath();

obj.moveTo(10,50);

obj.lineTo(10,10);

obj.lineTo(110,10);

obj.lineTo(110,50);

obj.stroke();

A. It creates a closed rectangle.

B. It creates a closed triangle.

C. It creates a closed square.

D. None of the above.

9. Given the following code, \_\_.

obj.arc(100, 100, 50, 2 , Math.PI+(Math.PI*i)/2, true);

A. the center of the arc is (50, 2).

B. the radius is 50.

C. the starting point is (100, 100).

D. the ending point is (100, 50).

10. To convert degrees to radians you can use the following JavaScript
expression: A. var radians = (Math.PI/180)*degrees

B. var radians = Math.PI+(Math.PI*180)/2

C. var radians = (Math.PI/360)*degrees

D. var radians = Math.PI+(Math.PI*360)/2

Game Programming -- Penn Wu

255

\protect\hypertarget{index_split_013.htmlux5cux23p256}{}{}

Lab \#14

Using Canvas and ExplorerCanvas

Preparation \#1:

1. Create a new directory c:\textbackslash cis261 if it does not exist.

2. Use Internet Explorer to go to
http://business.cypresscollege.edu/\textasciitilde pwu/cis261/files/lab14.zip
to download lab14.zip to the c:\textbackslash cis261 directory. Extract
the excanvas.js file to C:\textbackslash cis261 directory.

Learning Activity \#1:

1. Change to the C:\textbackslash cis261 directory.

2. Use Notepad to create a new file named
C:\textbackslash cis261\textbackslash lab14\_1.htm, with the following:

\textless!-\/- This code uses ExplorerCanvas, developed by Google Inc.
-\/-\textgreater{}

\textless html\textgreater\textless head\textgreater{}

\textless!-\/-{[}if IE{]}\textgreater\textless script
type="text/javascript"
src="excanvas.js"\textgreater\textless/script\textgreater\textless!{[}endif{]}-\/-\textgreater{}

\textless script type="text/javascript"\textgreater{}

var x=100;

function init() \{

var sid = document.getElementById("sqr");

if (sid.getContext) \{

var obj = sid.getContext("2d");

obj.fillStyle = "rgb(255, 0, 0)";

obj.beginPath();

obj.moveTo(100,10);

obj.lineTo(x,150);

obj.lineTo(100,200);

obj.fill();

\}

if (x\textless=10) \{ window.location=window.location; \}

x-\/-;

setTimeout("init()", 10);

\}

\textless/script\textgreater\textless/head\textgreater{}

\textless body onload="init();"\textgreater{}

\textless canvas id="sqr" width="600"
height="200"\textgreater\textless/canvas\textgreater{}

\textless/body\textgreater\textless/html\textgreater{}

3. Test the program. A triangle will grow in size.

Learning Activity \#2

1. Use Notepad to create a new file named
C:\textbackslash cis261\textbackslash lab14\_2.htm, with the following:

\textless!-\/- This code uses ExplorerCanvas, developed by Google Inc.
-\/-\textgreater{}

\textless html\textgreater\textless head\textgreater{}

\textless!-\/-{[}if IE{]}\textgreater\textless script
type="text/javascript"
src="excanvas.js"\textgreater\textless/script\textgreater\textless!{[}endif{]}-\/-\textgreater{}

\textless script type="text/javascript"\textgreater{}

Game Programming -- Penn Wu

256

\protect\hypertarget{index_split_013.htmlux5cux23p257}{}{}\includegraphics{index-257_1.png}

var y=200;

function init() \{

var sid = document.getElementById("sqr");

if (sid.getContext) \{

var obj = sid.getContext("2d");

obj.beginPath();

obj.moveTo(10,200);

obj.lineTo(10,10);

obj.lineTo(110,10);

obj.lineTo(110,200);

obj.stroke();

obj.beginPath();

obj.moveTo(10,y);

obj.lineTo(110,y);

obj.stroke();

\}

if (y \textless= 10) \{ y=10; clearTimeout(s1);\}

else \{ y-\/-; \}

s1 = setTimeout("init()", 10);

\}

\textless/script\textgreater\textless/head\textgreater{}

\textless body onload="init();"\textgreater{}

\textless canvas id="sqr" width="600"
height="200"\textgreater\textless/canvas\textgreater{}

\textless/body\textgreater\textless/html\textgreater{}

2. Test the program. A sample output looks:

Learning Activity \#3

1. Use Notepad to create a new file named
C:\textbackslash cis261\textbackslash lab14\_3.htm, with the following:

\textless!-\/- This code uses ExplorerCanvas, developed by Google Inc.
-\/-\textgreater{}

\textless html\textgreater\textless head\textgreater{}

\textless!-\/-{[}if IE{]}\textgreater\textless script
type="text/javascript"
src="excanvas.js"\textgreater\textless/script\textgreater\textless!{[}endif{]}-\/-\textgreater{}

\textless script type="text/javascript"\textgreater{}

var i = 1;

function init() \{

var sid = document.getElementById("sqr");

if (sid.getContext) \{

var obj = sid.getContext("2d");

Game Programming -- Penn Wu

257

\protect\hypertarget{index_split_013.htmlux5cux23p258}{}{} obj.arc(100,
100, 50, 0 , (Math.PI/180)*i, false); obj.stroke();

\}

if (i\textgreater=360) \{ window.location=window.location; \}

i++;

setTimeout("init()", 10);

\}

\textless/script\textgreater\textless/head\textgreater{}

\textless body onload="init();"\textgreater{}

\textless canvas id="sqr" width="600"
height="600"\textgreater\textless/canvas\textgreater{}

\textless/body\textgreater\textless/html\textgreater{}

2. Test the program. Watch a small arc grow into a circle.

Learning Activity \#4

1. Use Notepad to create a new file named
C:\textbackslash cis261\textbackslash lab14\_4.htm, with the following:

\textless!-\/- This code uses ExplorerCanvas, developed by Google Inc.
-\/-\textgreater{}

\textless html\textgreater\textless head\textgreater{}

\textless!-\/-{[}if IE{]}\textgreater\textless script
type="text/javascript"
src="excanvas.js"\textgreater\textless/script\textgreater\textless!{[}endif{]}-\/-\textgreater{}

\textless script type="text/javascript"\textgreater{}

function init() \{

var sid = document.getElementById("sqr");

if (sid.getContext) \{

var obj = sid.getContext("2d");

obj.fillStyle = "rgb(255, 0, 0)";

for (i=0;i\textless4;i++)\{

for(j=0;j\textless3;j++)\{

obj.beginPath();

var x = 25+j*50; // x coordinate

var y = 25+i*50; // y coordinate

var radius = 20; // Arc radius

var startAngle = 0; // Starting point on circle var endAngle =
Math.PI+(Math.PI*j)/2; // End point on circle var anticlockwise =
i\%2==0 ? false : true; // clockwise or anticlockwise
obj.arc(x,y,radius,startAngle,endAngle, anticlockwise);

if (i\textgreater1)\{

obj.fill();

\} else \{

obj.stroke();

\}

\}

\}

\}

\}

\textless/script\textgreater\textless/head\textgreater{}

\textless body onload="init();"\textgreater{}

\textless canvas id="sqr" width="600"
height="200"\textgreater\textless/canvas\textgreater{} Game Programming
-- Penn Wu

258

\protect\hypertarget{index_split_013.htmlux5cux23p259}{}{}\includegraphics{index-259_1.png}

\textless/body\textgreater\textless/html\textgreater{}

2. Test the program. A sample output looks:

Learning Activity \#5:

1. Use Notepad to create a new file named
C:\textbackslash cis261\textbackslash lab14\_5.htm, with the following:

\textless!-\/- This code uses ExplorerCanvas, developed by Google Inc.
-\/-\textgreater{}

\textless html\textgreater{}

\textless meta http-equiv="refresh" content="1"\textgreater{}

\textless head\textgreater{}

\textless!-\/-{[}if IE{]}\textgreater\textless script
type="text/javascript"
src="excanvas.js"\textgreater\textless/script\textgreater\textless!{[}endif{]}-\/-\textgreater{}

\textless script type="text/javascript"\textgreater{}

var k = Math.floor(Math.random()*50);

function init() \{

var obj =
document.getElementById(\textquotesingle sqr\textquotesingle).getContext(\textquotesingle2d\textquotesingle);

obj.translate(75,75);

for (i=1;i\textless8;i++)\{ // Loop through rings (from inside to out)
obj.save();

obj.fillStyle =
\textquotesingle rgb(\textquotesingle+(k*i)+\textquotesingle,\textquotesingle+(255-k*i)+\textquotesingle,\textquotesingle+(255-k*i)+\textquotesingle)\textquotesingle;

for (j=0;j\textless i*6;j++)\{ // draw individual dots

obj.rotate(Math.PI*2/(i*6));

obj.beginPath();

obj.arc(0,i*9.5,5,0,Math.PI*2,true);

obj.fill();

\}

obj.restore();

\}

\}

\textless/script\textgreater\textless/head\textgreater{}

\textless body onload="init();" onKeyDown="init()"\textgreater{}

\textless canvas id="sqr"\textgreater\textless/canvas\textgreater{}

\textless/body\textgreater\textless/html\textgreater{}

2. Test the program. A sample output looks:

Game Programming -- Penn Wu

259

\protect\hypertarget{index_split_013.htmlux5cux23p260}{}{}\includegraphics{index-260_1.png}

\textbf{}

\textbf{Submittal}

Upon completing all the learning activities,

1. Upload all files you created in this lab to your remote web server.

2. Log in to to Blackboard, launch Assignment 14, and then scroll down
to question 11.

3. Copy and paste the URLs to the textbox. For example,

•

http://www.geocities.com/cis261/lab14\_1.htm

•

http://www.geocities.com/cis261/lab14\_2.htm

•

http://www.geocities.com/cis261/lab14\_3.htm

•

http://www.geocities.com/cis261/lab14\_4.htm

•

http://www.geocities.com/cis261/lab14\_5.htm

No credit is given to broken link(s).

Game Programming -- Penn Wu

260

\protect\hypertarget{index_split_013.htmlux5cux23p261}{}{}\includegraphics{index-261_1.png}

Lecture \#15

Multiple Player Games

\protect\hypertarget{index_split_014.html}{}{}

\hypertarget{index_split_014.htmlux5cux23calibre_pb_13}{%
\subsection{Introduction}\label{index_split_014.htmlux5cux23calibre_pb_13}}

A multiplayer game is a game which is played by two or more players. The
players might be independent opponents, formed into teams or be just a
single team pitted against the game.

JavaScript is not an ideal language for multiplayer games because its
code is downloaded locally to the player's computer. To make sure each
player can see other player's move, the game code should be written with
a server-side language (such as PHP or ASP.NET) that collect and process
data from each player's computer and present the output on each player's
browser.

So you make your move, and your HTML/JavaScript sends form data to your
server script. The script records this move and sends back a web page
containing updated display. At this point, if the other player refreshes
his screen, they will see the same web page as you. Alternatively you
can send back a page that automatically refreshes using metarefresh or
JavaScript timer.

If you just use JavaScript, the results are only visible on your
computer - there must be a central point that both of you can look at
and the only way is a server web page.

Although this lecture chooses to use PHP and MySQL to explain how a
networked game is programmed and played through the Web, PHP and SQL
(using MySQL) programming is beyond the scope of this course.

Standalone

A standalone multiplayer game does not require a computer network.
Players can download the multiplayer

game codes and play it locally with multiple sources of user inputs.
This lecture discusses two game

popular modes:

•

One player at a time.

•

Simultaneous gameplay.

In the simultaneous gameplay mode, two or more players play
simultaneously by sending user inputs to the same signal receiver.
Consider the following example, which allows two players to race the
horseback riding. A screen (which is a browser window) contains two
panels, each supports an individual player. Each player presses a
particular key once to start the running of the horse.

The coding mechanism is simple.

function keyed() \{

if(event.ctrlKey) \{

if(i==0) \{ horse1(); i++; \}

\}

if(event.keyCode==39) \{

if(j==0) \{ horse2(); j++; \}

\}

Game Programming -- Penn Wu

261

\protect\hypertarget{index_split_014.htmlux5cux23p262}{}{}

status = hrs1.style.pixelWidth;

\}

..............

\textless body onLoad="init()" onKeyDown="keyed()"\textgreater{} When
any of the two players presses the designated key (Shift or the → arrow
key), the \textbf{keyed()} function will calls the \textbf{horse1()} or
\textbf{horse2()} functions accordingly. The horse1() function, for
example, will move the first horse (with an ID ``\textbf{hrs1}'') at a
speed of either 1 pixel or 2 pixels per 50 milliseconds towards the
right. The movement is terminated when hrs1 hits the right border of the
panel it belongs to.

function horse1() \{

if ( hrs1.style.pixelLeft + hrs1.style.pixelWidth \textgreater=

player1.style.pixelWidth) \{

clearTimeout(s2); // terminate hrs2

clearTimeout(s1); // terminate hrs1

\}

else \{ hrs1.style.pixelLeft += Math.floor(Math.random()*2)+1;

s1 = setTimeout("horse1()", 50); \}

\}

Notice that there are two clearTimeout() methods used in the above code.
One is used to terminate hrs1, while the other terminates hrs2.

In the one-player-at-a-time mode, the programmer's job is to find a way
to count the player's turns. The programming trick is technically
simple. Declare a variable that counts how many times the game has been
restarted. In a two-player situation, the number is either an odd or
even number. When the number is odd, it is the first player's turn, when
the number is even, it is the second player's turn.

Consider the following code sample,

...............

var player\_turn;

...............

function player() \{

player\_turn++;

ball.style.display = "inline"

ball.style.pixelTop = 0;

ball.style.pixelLeft = Math.floor(Math.random()*250);

moveBall();

if (player\_turn\%2 == 0) \{

p1.innerText = 2;

area.style.backgroundColor="yellow";

ball.style.backgroundColor="blue";

\}

else \{

p1.innerText = 1;

area.style.backgroundColor="\#abcdef";

ball.style.backgroundColor="red";

\}

\}

................

\textless button onClick="player()"

style="position:absolute;top:400"\textgreater Change
Hand\textless/button\textgreater{}

................

Game Programming -- Penn Wu

262

\protect\hypertarget{index_split_014.htmlux5cux23p263}{}{}\includegraphics{index-263_1.png}

\includegraphics{index-263_2.png}

In this code, the variable \textbf{player\_turn} is incremented by 1
each time when the \textbf{player()} function is called by clicking the
button. The following statement checks if the current value is an even
number.

if (player\_turn\%2 == 0) \{

If so, the following statements will indicate that it is the second
player's turn, and the background color is changed to yellow, while the
ball is changed to blue.

p1.innerText = 2;

area.style.backgroundColor="yellow";

ball.style.backgroundColor="blue";

If the current number is an odd number, the computer will know it is the
first player's turn and carry out the following code segement.

p1.innerText = 1;

area.style.backgroundColor="\#abcdef";

ball.style.backgroundColor="red";

The following outputs will thus display alternatively on the screen.

Player 1

Player 2

Online

Networked games have existed for the last decade but their underlying
software architectures (Networked)

have been poorly designed with little in the way of reusability. A
networked game is a software multiplayer

system in which multiple users interact with each other in real-time,
even though those users may game

be physically located around the world.

Typically, each user accesses his/her own computer workstation or
console, using it to provide a user interface to the game environment.
These environments usually aim to provide users with a sense of realism
by incorporating realistic 3D graphics, spatial sound \& other
modalities to create an immersive experience.

Usually there is a server that synchronizes the game data among users.
One can try to understand this model through the analogy of a database
model used in the airport. In an airport terminal, crew members work on
different workstations to assign seat numbers to airplane passengers.

When one crew member confirms and secures a seat for a passenger through
a workstation, the workstation will immediately transmits the data to
the server. The server will simultaneously notify each every other
workstation that a seat is taken, and consequently each seat is assigned
to one and the only one passenger.

Networked game typically must integrate with database systems that store
persistent information about the game environment. These databases
include, for example, detailed information about the environment's
terrain elevation, the location of buildings and other static items in
the environment, and the initial Networked game configuration.

Game Programming -- Penn Wu

263

\protect\hypertarget{index_split_014.htmlux5cux23p264}{}{}

In a networked multiplayer game, each player makes movements and
actively responds to the game themes. There will be many data (each
player generates some of them) to be exchanged, processed, and
calculated by the centralized control server and each participant
computer.

Server

Client1

Client2

Client3

\ldots\ldots\ldots{}

Client \emph{n}

.

In this model, clients (workstation) must maintain a live (active)
connection with the server for bilateral (two-way) communication. Data
can then transmit from the client to the server and vice versa
simultaneously.

A networked multiplayer game is a game that accepts inputs from multiple
sources. Consider the following game code, in which one can use the
Shift and Control keys to control the Up and Down direction of an object
(e.g. a red square), the other uses and ↑ and ↓ arrow keys to controls
the second object (e.g. a blue square). Although the input device is the
same-\/-keyboard, there are two sources of user input.

\textless script\textgreater{}

function move() \{

document.body.focus();

// red

if (event.shiftKey) \{ red.style.pixelTop -\/-; \}

else if (event.ctrlKey) \{ red.style.pixelTop ++; \}

// blue

switch (event.keyCode) \{

case 38:

blue.style.pixelTop -\/-; break;

case 40:

blue.style.pixelTop ++; break;

\}

\}

.........

\textless/script\textgreater{}

Two players can simultaneously use two set of keys to control two
individual objects. However, both players must look at the same screen
and use the same keyboard. This is a physical limitation of JavaScript,
as a client-side scripting language.

A popular solution to this physical boundary is the so-called AJAX
(\textbf{A}synchronous \textbf{J}avaScript \textbf{A}nd \textbf{X}ML),
which combines the technologies of HTML, CSS, JavaScript, DHTML, and DOM

(Document Object Model) with an ability to work with server-side
scripting language. The core technology is the implementation of
\textbf{XMLHttpRequest} object allows web programmers to retrieve data
from the web server as a background activity.

When used in game programming, AJAX has four main components. JavaScript
defines game rules and program flow. The Document Object Model and
Cascading Style Sheets set the appearance in response to data fetched in
the background from the server by the XMLHttpRequest object.

Consider the following code segment,

Game Programming -- Penn Wu

264

\protect\hypertarget{index_split_014.htmlux5cux23p265}{}{}

\textless script type="text/javascript"\textgreater{}

var xmlhttp=false;

if (!xmlhttp \&\& typeof
XMLHttpRequest!=\textquotesingle undefined\textquotesingle) \{

xmlhttp = new XMLHttpRequest();

\}

function ajax\_call() \{

xmlhttp.open("GET",
\textquotesingle http://cis261.comyr.com/sqrgame.php?x1=\textquotesingle{}
+

document.getElementById(\textquotesingle red\textquotesingle).style.pixelLeft
+

\textquotesingle\&y1=\textquotesingle{} +
document.getElementById(\textquotesingle red\textquotesingle).style.pixelTop
+

\textquotesingle\&x2=\textquotesingle{} +
document.getElementById(\textquotesingle blue\textquotesingle).style.pixelLeft
+

\textquotesingle\&y2=\textquotesingle{} +
document.getElementById(\textquotesingle blue\textquotesingle).style.pixelTop
, true); xmlhttp.onreadystatechange=function() \{

if (xmlhttp.readyState==4) \{

document.getElementById(\textquotesingle bt\textquotesingle).innerHTML =
xmlhttp.responseText;

\}

\}

xmlhttp.send(null)

return false;

\}

.............

\textless/script\textgreater{}

.............

\textless p id="bt"\textgreater\textless/p\textgreater{}

The following is a valid URL that links the instructor's demo site.

http://cis261.comyr.com/sqrgame.php

In the above example, when the ajax\_call() function is called, the
client browser will attach four parameters (x1, y1, x2, and y2) to the
URL, as shown below. Each parameter represents the current coordinate
pair of the red and blue square.

http://cis261.comyr.com/sqrgame.php?x1=21\&y1=74\&x2=159\&y2=367

The GET method, specified in the xmlhttp.open() method delivers these
four parameter and their associated values to the web server (which is
the host ``cis261.comyr.com'' in this case), and then particularly to
the ``sqrgame.php'' server-side script. The source code of
``sqrgame.php'', written in PHP, looks:

\textless?php

\$x1=\$\_GET{[}"x1"{]};

\$y1=\$\_GET{[}"y1"{]};

\$x2=\$\_GET{[}"x2"{]};

\$y2=\$\_GET{[}"y2"{]};

\$con = mysql\_connect("localhost"," \emph{\textbf{MyUserID}}","
\emph{\textbf{MyPassWord}}"); if( mysql\_select\_db("
\emph{\textbf{DatabaseName}}") )

\{

\$sql = "UPDATE Coords set x1=\textquotesingle\$x1\textquotesingle,
y1=\textquotesingle\$y1\textquotesingle,
x2=\textquotesingle\$x2\textquotesingle,
y2=\textquotesingle\$y2\textquotesingle{} where
cid=\textquotesingle c1\textquotesingle";

\$query = mysql\_query(\$sql);

\$sql2 = mysql\_query("select * from Coords where
cid=\textquotesingle c1\textquotesingle"); if(\$sql2) \{

while(\$row = mysql\_fetch\_array(\$sql2))\{

Game Programming -- Penn Wu

265

\protect\hypertarget{index_split_014.htmlux5cux23p266}{}{} echo
\$row{[}\textquotesingle x1\textquotesingle{]}
.":".\$row{[}\textquotesingle y1\textquotesingle{]}.":".\$row{[}\textquotesingle x2\textquotesingle{]}.":".\$row{[}\textquotesingle y2\textquotesingle{]}.":";

\}

\}

mysql\_close();

\}

else

\{

die( "Error! Could not connect the database: " . mysql\_error() );

\}

?\textgreater{}

The following reads the HTTPREQUEST value sent by the client browser,
which is

``http://cis261.comyr.com/sqrgame.php?x1=21\&y1=74\&x2=159\&y2=367''

\$x1=\$\_GET{[}"x1"{]};

\$y1=\$\_GET{[}"y1"{]};

\$x2=\$\_GET{[}"x2"{]};

\$y2=\$\_GET{[}"y2"{]};

Consequently, 21 is assigned to the PHP variable \$x1, 74 to \$y1, 159
to \$x2, and 367 to \$y2. The following SQL-embedded PHP statement, at
the server side, will replace the variables with correct values.

UPDATE Coords set x1=\textquotesingle\$x1\textquotesingle,
y1=\textquotesingle\$y1\textquotesingle,
x2=\textquotesingle\$x2\textquotesingle,
y2=\textquotesingle\$y2\textquotesingle{}

where cid=\textquotesingle c1\textquotesingle{}

In other words, the PHP interpreter converts the above line to the
following, and passes the completed SQL statement to MySQL server.

UPDATE Coords set x1=\textbf{21}, y1=\textbf{74}, x2=\textbf{159},
y2=\textbf{367} where cid=\textquotesingle c1\textquotesingle{}

The MySQL server then stores these data into a table named ``Coords''. A
visualization of the table is:

mysql\textgreater{} select * from Coords where
cid=\textquotesingle c1\textquotesingle;

+-\/-\/-\/-\/-+-\/-\/-\/-+-\/-\/-\/-+-\/-\/-\/-\/-+-\/-\/-\/-\/-+

\textbar{} cid \textbar{} x1 \textbar{} y1 \textbar{} x2 \textbar{} y2
\textbar{}

+-\/-\/-\/-\/-+-\/-\/-\/-+-\/-\/-\/-+-\/-\/-\/-\/-+-\/-\/-\/-\/-+

\textbar{} \textbf{c1 \textbar{} 21 \textbar{} 74 \textbar{} 159
\textbar{} 367} \textbar{}

+-\/-\/-\/-\/-+-\/-\/-\/-+-\/-\/-\/-+-\/-\/-\/-\/-+-\/-\/-\/-\/-+

2 rows in set (0.00 sec)

The following SQL-embedded PHP statements will retrieve the newly added
values of x1, y2, x2, and y2, and output them to the receiver. The
receiver, in this case, is the \textbf{ajax\_call()} method on the
client side.

\$sql2 = mysql\_query("select * from Coords where
cid=\textquotesingle c1\textquotesingle"); if(\$sql3) \{

while(\$row = mysql\_fetch\_array(\$sql3))\{

\textbf{echo
\$row{[}\textquotesingle x1\textquotesingle{]}.":".\$row{[}\textquotesingle y1\textquotesingle{]}.":".\$row{[}\textquotesingle x2\textquotesingle{]}.":".\$row{[}\textquotesingle y2\textquotesingle{]}.
":";}

\}

\}

The echo command outputs the values to the xmlhttp.
\textbf{responseText} property (of the ajax\_call() method) in a pattern
similar to the following. In other words, the new value of
xmlhttp.responseText is:

21:74:159:367:

Game Programming -- Penn Wu

266

\protect\hypertarget{index_split_014.htmlux5cux23p267}{}{}The programmer
can then write JavaScript code to separate the string value
``\textbf{21:74:159:367:}''

into an array. For example,

var str = xmlhttp.responseText;

var crd = str.split(":");

The \textbf{split()} method is used to split a string into an array of
strings. The syntax is: \emph{ObjectID}.split( \emph{separator});

After going through the splitting, the string value
``\textbf{21:74:159:367:}'' is broken down to four elements: 21, 74,
159, and 367. The following statement

var crd = str.split(":");

is equivalent to

var crd= new Array(21, 74, 159, 367);

To retrieve the values of each element, use \textbf{crd{[}0{]}},
\textbf{crd{[}1{]}}, \textbf{crd{[}2{]}}, and \textbf{crd{[}3{]}}.
Consequently, the xmlhttp. \textbf{onreadystatechange} property will,
based on the following DHTML statements, assign the initial value of the
(x, y) coordinates to the red and blue squares and send them to the
client browser, each time when the player access the game.

function init() \{

...........

xmlhttp.onreadystatechange=function() \{

...........

var str = xmlhttp.responseText;

...........

code = "\textless span id=\textquotesingle red\textquotesingle{}
style=\textquotesingle background-

Color:red;left:"+\textbf{crd{[}0{]}}+";top:"+\textbf{crd{[}1{]}}+"\textquotesingle\textgreater\textless/span\textgreater";
code += "\textless span id=\textquotesingle blue\textquotesingle{}
style=\textquotesingle backgroundColor:blue;left:"+\textbf{crd{[}2{]}}+";top:"+\textbf{crd{[}3{]}}+"\textquotesingle\textgreater\textless/span\textgreater";
bd.innerHTML = code;

.......

\}

\}

.......

\textless body id="bd" onLoad=" \textbf{init()}"
onKeyDown="move()"\textgreater{} To build a new MySQL database with a
new table using PHP with embedded SQL statements, use:

\textless?php

\$con = mysql\_connect("localhost","sqr\_game","cis261"); if (!\$con)

\{

die(\textquotesingle Could not connect: \textquotesingle{} .
mysql\_error());

\}// Create database

if (mysql\_query("CREATE DATABASE sqr\_game",\$con))

\{

echo "Database created";

\}

else

\{

echo "Error creating database: " . mysql\_error();

\}

Game Programming -- Penn Wu

267

\protect\hypertarget{index_split_014.htmlux5cux23p268}{}{}// Create
table

mysql\_select\_db("a6732376\_game", \$con);

\$sql = "CREATE TABLE Coords

(

cid varchar(15) NOT NULL,

x1 int NOT NULL,

y1 int NOT NULL,

x2 int NOT NULL,

y2 int NOT NULL

)";

// Execute query

mysql\_query(\$sql,\$con);

mysql\_close(\$con); // destroy the connection to database

?\textgreater{}

When the game is involved in database, the network-based delays can be
very significant. These network delays are particularly difficult to
handle when multiple users or components interact with each other
directly.

For example, a network game must support accurate collision detection,
agreement, and resolution among participants. Accurate collision
detection is difficult because at any given point in time, no user has
accurate information about the other users' current positions; as we
have seen, the network delay means that all received information is
out-of-date.

It is possible, therefore, that one user might conclude, based on stale
information, which a collision occurred, while, in fact, the other user
actually moved to avoid the collision during the network delay period.

Game Programming -- Penn Wu

268

\protect\hypertarget{index_split_014.htmlux5cux23p269}{}{}

Lab \#15

Multiple Player Games

Learning Activity \#1: A simple two-player game

1. Use Notepad to create a new text file named lab15\_1.htm, with the
following contents:

\textless html\textgreater{}

\textless style\textgreater{}

div \{ height:200; border:solid 1 black; background-Color:\#abcdef;
position: absolute;\}

img \{ position:relative; \}

\textless/style\textgreater{}

\textless script\textgreater{}

var i=0; j=0;

function init() \{

player1.style.pixelWidth = 400;

player2.style.pixelWidth = 400;

hrs1.style.pixelWidth = 30;

hrs2.style.pixelWidth = 30;

\}

function keyed() \{

if(event.ctrlKey) \{

if(i==0) \{ horse1(); i++; \}

\}

if(event.keyCode==39) \{

if(j==0) \{ horse2(); j++; \}

\}

status = hrs1.style.pixelWidth;

\}

function horse1() \{

if ( hrs1.style.pixelLeft + hrs1.style.pixelWidth \textgreater=
player1.style.pixelWidth) \{

clearTimeout(s2);

clearTimeout(s1);

\}

else \{ hrs1.style.pixelLeft += Math.floor(Math.random()*2)+1;

s1 = setTimeout("horse1()", 50); \}

\}

function horse2() \{

if ( hrs2.style.pixelLeft + hrs2.style.pixelWidth \textgreater=
player2.style.pixelWidth) \{

clearTimeout(s1);

clearTimeout(s2);

\}

else \{ hrs2.style.pixelLeft += Math.floor(Math.random()*2)+1;

s2 = setTimeout("horse2()", 50); \}

\}

\textless/script\textgreater{}

Game Programming -- Penn Wu

269

\protect\hypertarget{index_split_014.htmlux5cux23p270}{}{}\includegraphics{index-270_1.png}

\textless body onLoad="init()" onKeyDown="keyed()"\textgreater{}

\textless table border=0 width="100\%"\textgreater{}

\textless tr\textgreater{}

\textless td\textgreater Player
1\textless/td\textgreater\textless td\textgreater Player
2\textless/td\textgreater{}

\textless/tr\textgreater{}

\textless tr\textgreater{}

\textless td width="50\%"\textgreater\textless div
id="player1"\textgreater{}

\textless img id="hrs1" src="hr.gif"\textgreater{}

\textless/div\textgreater\textless/td\textgreater{}

\textless td width="50\%"\textgreater\textless div
id="player2"\textgreater{}

\textless img id="hrs2" src="hr.gif"\textgreater{}

\textless/div\textgreater\textless/td\textgreater{}

\textless/tr\textgreater{}

\textless/table\textgreater{}

\textless/body\textgreater{}

\textless/html\textgreater{}

2. Test the program with Internet Explorer. Player 1 uses the ``Ctrl''
key and player 2 use the → arrow key to start the game simultaneously. A
sample output looks:

Learning Activity \#2: Two-player game -- one player at a time

1. Use Notepad to create a new text file named lab15\_2.htm, with the
following contents:

\textless html\textgreater{}

\textless style\textgreater{}

div \{ position:absolute; width:300; height:300; border:solid 1 black;
left:10; \}

span \{ position:relative; width:20; height:20; background-Color: red\}

hr \{ position:absolute; top:280\}

\textless/style\textgreater{}

\textless script\textgreater{}

var code="";

var player\_turn;

function init() \{

var i = Math.floor(Math.random()*250);

var j = Math.floor(Math.random()*280);

code = "\textless span id=\textquotesingle ball\textquotesingle{}
style=\textquotesingle left:"+j+"\textquotesingle\textgreater\textless/span\textgreater";
code += "\textless hr id=\textquotesingle bar\textquotesingle{}
size=\textquotesingle3\textquotesingle{}
width=\textquotesingle50\textquotesingle{}
color=\textquotesingle black\textquotesingle{}
style=\textquotesingle left:"+i+"\textquotesingle\textgreater";
area.innerHTML = code;

Game Programming -- Penn Wu

270

\protect\hypertarget{index_split_014.htmlux5cux23p271}{}{}
area.style.backgroundColor="\#abcdef"; player\_turn = 1;

p1.innerText = player\_turn;

moveBall();

\}

function moveBall() \{

if (ball.style.pixelTop \textgreater= 275) \{

clearTimeout(s1);

ball.style.display = "none"

\}

else if ((ball.style.pixelTop + ball.style.pixelHeight \textgreater=
260) \&\& (ball.style.pixelLeft \textgreater= bar.style.pixelLeft) \&\&

(ball.style.pixelLeft + 20 \textless= bar.style.pixelLeft + 50))

\{

ball.style.pixelTop = 0;

ball.style.pixelLeft = Math.floor(Math.random()*250);

\}

else \{ ball.style.pixelTop += 2; \}

s1 = setTimeout("moveBall()", 20);

\}

function player() \{

player\_turn++;

ball.style.display = "inline"

ball.style.pixelTop = 0;

ball.style.pixelLeft = Math.floor(Math.random()*250);

moveBall();

if (player\_turn\%2 == 0) \{

p1.innerText = 2;

area.style.backgroundColor="yellow";

ball.style.backgroundColor="blue";

\}

else \{

p1.innerText = 1;

area.style.backgroundColor="\#abcdef";

ball.style.backgroundColor="red";

\}

\}

function moveBar() \{

switch(event.keyCode) \{

case 37:

if (bar.style.pixelLeft\textless=10) \{ bar.style.pixelLeft = 10; \}

else \{ bar.style.pixelLeft-=2; \}

break;

case 39:

if (bar.style.pixelLeft \textgreater= 250) \{ bar.style.pixelLeft = 250;
\}

else \{ bar.style.pixelLeft+=2; \}

break;

\}

\}

\textless/script\textgreater{}

\textless body onLoad="init()" onKeyDown="moveBar()"\textgreater{} Game
Programming -- Penn Wu

271

\protect\hypertarget{index_split_014.htmlux5cux23p272}{}{}\includegraphics{index-272_1.png}

\includegraphics{index-272_2.png}

\textless div id="area"\textgreater{}

\textless/div\textgreater{}

\textless p style="position:absolute;top:350"\textgreater Current
Player: \textless b
id="p1"\textgreater\textless/b\textgreater\textless/p\textgreater{}

\textless button onClick="player()"
style="position:absolute;top:400"\textgreater Change
Hand\textless/button\textgreater{}

\textless/body\textgreater{}

\textless/html\textgreater{}

2. Test the program with Internet Explorer. A sample output looks:
Player 1

Player 2

Learning Activity \#3:

1. Use Notepad to create a new text file named lab15\_3.htm, with the
following contents:

\textless html\textgreater{}

\textless style\textgreater{}

span \{ position:absolute; width:20px; height:20px; \}

\textless/style\textgreater{}

\textless script\textgreater{}

function move() \{

document.body.focus();

// red

if (event.shiftKey) \{ red.style.pixelTop -\/-; \}

else if (event.ctrlKey) \{ red.style.pixelTop ++; \}

// blue

switch (event.keyCode) \{

case 38:

blue.style.pixelTop -\/-; break;

case 40:

blue.style.pixelTop ++; break;

\}

\}

function init() \{

x1 = Math.floor(Math.random()*90) + 10;

y1 = Math.floor(Math.random()*140) + 10;

x2 = Math.floor(Math.random()*90) + 10;

y2 = Math.floor(Math.random()*140) + 10;

x3 = Math.floor(Math.random()*90) + 10;

y3 = Math.floor(Math.random()*140) + 10;

Game Programming -- Penn Wu

272

\protect\hypertarget{index_split_014.htmlux5cux23p273}{}{}\includegraphics{index-273_1.png}

red.style.pixelLeft = x1;

red.style.pixelTop = y1;

blue.style.pixelLeft = x2;

blue.style.pixelTop = y2;

green.style.pixelLeft = x3;

green.style.pixelTop = y3;

action();

\}

var i=1;

var j=1;

var k=2;

function action() \{

if (red.style.pixelTop \textless=10 \textbar\textbar{}
red.style.pixelTop \textgreater=150) \{

i = i * -1;

\}

if (blue.style.pixelTop \textless=10 \textbar\textbar{}
blue.style.pixelTop \textgreater=150) \{

j = j * -1;

\}

if (green.style.pixelLeft \textless=10 \textbar\textbar{}
green.style.pixelLeft \textgreater=100) \{

k = k * -1;

\}

red.style.pixelTop += i;

blue.style.pixelTop += j;

green.style.pixelLeft += k;

setTimeout("action()", 50);

\}

\textless/script\textgreater{}

\textless body onLoad="init()" onKeyDown="move();"\textgreater{}

\textless span id="red"
style="background-Color:red;"\textgreater\textless/span\textgreater{}

\textless span id="blue"
style="background-Color:blue;"\textgreater\textless/span\textgreater{}

\textless span id="green" style="border: solid 1
green;"\textgreater\textless/span\textgreater{}

\textless/body\textgreater{}

\textless/html\textgreater{}

2. Test the program with Internet Explorer. A sample output looks:
Learning Activity \#4: Networked Two-Player Game

1. Use Notepad to create a new text file named lab15\_4.htm, with the
following contents: Game Programming -- Penn Wu

273

\protect\hypertarget{index_split_014.htmlux5cux23p274}{}{}

\textless html\textgreater{}

\textless style\textgreater{}

span \{position:absolute; width:20; height:20\}

\textless/style\textgreater{}

\textless script type="text/javascript"\textgreater{}

//var req=false;

var xmlhttp=false;

if (!xmlhttp \&\& typeof
XMLHttpRequest!=\textquotesingle undefined\textquotesingle) \{

xmlhttp = new XMLHttpRequest();

\}

function ajax\_call() \{

xmlhttp.open("GET",
\textquotesingle http://cis261.comyr.com/sqrgame.php?x1=\textquotesingle{}
+

document.getElementById(\textquotesingle red\textquotesingle).style.pixelLeft
+

\textquotesingle\&y1=\textquotesingle{} +
document.getElementById(\textquotesingle red\textquotesingle).style.pixelTop
+

\textquotesingle\&x2=\textquotesingle{} +
document.getElementById(\textquotesingle blue\textquotesingle).style.pixelLeft
+

\textquotesingle\&y2=\textquotesingle{} +
document.getElementById(\textquotesingle blue\textquotesingle).style.pixelTop
, true); xmlhttp.setRequestHeader("Content-Type", "text/html");
xmlhttp.onreadystatechange=function() \{

if (xmlhttp.readyState==4) \{

status = xmlhttp.responseText;

var str = xmlhttp.responseText;

var crd = str.split(":");

status = "red:("+crd{[}0{]}+", "+crd{[}1{]}+") blue:("+crd{[}2{]}+",
"+crd{[}3{]}+")";

\}

\}

xmlhttp.send(null)

return false;

setTimeout("ajax\_call()",100);

\}

function init() \{

xmlhttp.open("GET", "http://cis261.comyr.com/sqrinit.php" , true);
xmlhttp.setRequestHeader("Content-Type", "text/html");
xmlhttp.onreadystatechange=function() \{

if (xmlhttp.readyState==4) \{

//status = xmlhttp.responseText;

var str = xmlhttp.responseText;

var crd = str.split(":");

status = "red:("+crd{[}0{]}+", "+crd{[}1{]}+") blue:("+crd{[}2{]}+",
"+crd{[}3{]}+")"; var code="";

code = "\textless span id=\textquotesingle red\textquotesingle{}
style=\textquotesingle background-

Color:red;left:"+crd{[}0{]}+";top:"+crd{[}1{]}+"\textquotesingle\textgreater\textless/span\textgreater";
code += "\textless span id=\textquotesingle blue\textquotesingle{}
style=\textquotesingle backgroundColor:blue;left:"+crd{[}2{]}+";top:"+crd{[}3{]}+"\textquotesingle\textgreater\textless/span\textgreater";
bd.innerHTML = code;

\}

\}

Game Programming -- Penn Wu

274

\protect\hypertarget{index_split_014.htmlux5cux23p275}{}{}\includegraphics{index-275_1.png}

xmlhttp.send(null)

return false;

\}

function move() \{

if(event.shiftKey) \{

switch(event.keyCode) \{

case 37:

red.style.pixelLeft -\/-; break;

case 39:

red.style.pixelLeft ++; break;

case 38:

red.style.pixelTop -\/-; break;

case 40:

red.style.pixelTop ++; break;

\}

\}

if(event.ctrlKey) \{

switch(event.keyCode) \{

case 37:

blue.style.pixelLeft -\/-; break;

case 39:

blue.style.pixelLeft ++; break;

case 38:

blue.style.pixelTop -\/-; break;

case 40:

blue.style.pixelTop ++; break;

\}

\}

ajax\_call();

\}

\textless/script\textgreater{}

\textless body id="bd" onLoad="init()" onKeyDown="move()"\textgreater{}

\textless/body\textgreater{}

\textless/html\textgreater{}

2. Test the program with Internet Explorer. Hold the Shift key and press
→, ↓, ↑, or → arrow keys to move the red square. Hold the Ctrl key and
press →, ↓, ↑, or → arrow keys to move the blue square. A sample output
looks: Game Programming -- Penn Wu

275

\protect\hypertarget{index_split_014.htmlux5cux23p276}{}{}Learning
Activity \#5:

1. Use Notepad to create a new text file named lab15\_5.htm, with the
following contents:

\textless html\textgreater{}

\textless style\textgreater{}

span \{ background-Color: white; border:solid 1 black;

width:50; height:50; text-align:center;

\}

\textless/style\textgreater{}

\textless script\textgreater{}

var req;

var code = "";

var mymark;

var cell = new Array();

function reset() \{

for (k=1; k\textless=9; k++) \{

eval("c"+k+".innerText=\textquotesingle{} \textquotesingle");

\}

save\_mark();

\}

function init() \{

for (i=1; i\textless=9; i++) \{

code += "\textless span id=\textquotesingle c"+i+"\textquotesingle{}
onClick=\textquotesingle set\_mark()\textquotesingle\textgreater\textless/span\textgreater";
if (i\%3==0) \{ code += "\textless br\textgreater"; \}

\}

area.innerHTML = code;

init\_request() ;

\}

function init\_request() \{

if (window.XMLHttpRequest) \{

req = new XMLHttpRequest();

\}

else if (window.ActiveXObject) \{

req = new ActiveXObject("Microsoft.XMLHTTP");

\}

req.open(\textquotesingle GET\textquotesingle, "tictoe.php", true);

req.setRequestHeader("Content-Type", "text/html");
req.onreadystatechange = processResponse;

req.send(null);

\}

function player\_mark() \{

frm.innerText = "You chose : " + mymark;

\}

function set\_mark() \{

var keyID = event.srcElement.id;

document.getElementById(keyID).innerHTML = "\textless b\textgreater" +
mymark + "\textless/b\textgreater"; Game Programming -- Penn Wu

276

\protect\hypertarget{index_split_014.htmlux5cux23p277}{}{}
cell{[}keyID{]} = mymark;

save\_mark();

\}

function save\_mark() \{

if (window.XMLHttpRequest) \{

req = new XMLHttpRequest();

\}

else if (window.ActiveXObject) \{

req = new ActiveXObject("Microsoft.XMLHTTP");

\}

var url = "tictoe.php" +

"?c1=" + cell{[}\textquotesingle c1\textquotesingle{]} +

"\&c2=" + cell{[}\textquotesingle c2\textquotesingle{]} +

"\&c3=" + cell{[}\textquotesingle c3\textquotesingle{]} +

"\&c4=" + cell{[}\textquotesingle c4\textquotesingle{]} +

"\&c5=" + cell{[}\textquotesingle c5\textquotesingle{]} +

"\&c6=" + cell{[}\textquotesingle c6\textquotesingle{]} +

"\&c7=" + cell{[}\textquotesingle c7\textquotesingle{]} +

"\&c8=" + cell{[}\textquotesingle c8\textquotesingle{]} +

"\&c9=" + cell{[}\textquotesingle c9\textquotesingle{]};

req.open(\textquotesingle GET\textquotesingle, url, true);

req.setRequestHeader("Content-Type", "text/html");
req.onreadystatechange = processResponse;

req.send(null);

\}

function processResponse() \{

if(req.readyState == 4) \{

var str = req.responseText;

status = str;

var crd = str.split(":");

c1.innnerText = crd{[}0{]};

c2.innnerText = crd{[}1{]};

c3.innnerText = crd{[}2{]};

c4.innnerText = crd{[}3{]};

c5.innnerText = crd{[}4{]};

c6.innnerText = crd{[}5{]};

c7.innnerText = crd{[}6{]};

c8.innnerText = crd{[}7{]};

c9.innnerText = crd{[}8{]};

var crd = str.split(":");

\}

//setTimeout("save\_mark()",100);

\}

\textless/script\textgreater{}

\textless body onLoad="init()"\textgreater{}

\textless form id="frm"\textgreater Choose your mark:

Game Programming -- Penn Wu

277

\protect\hypertarget{index_split_014.htmlux5cux23p278}{}{}\includegraphics{index-278_1.png}

\includegraphics{index-278_2.png}

\textless input type="radio" name="mark"
onClick="mymark=\textquotesingle O\textquotesingle;
player\_mark()"\textgreater O

\textless input type="radio" name="mark"
onClick="mymark=\textquotesingle X\textquotesingle;
player\_mark()"\textgreater X

\textless/form\textgreater{}

\textless div id="area"\textgreater{}

\textless/div\textgreater{}

\textless p\textgreater\textless button
onClick="reset()"\textgreater Reset\textless/button\textgreater\textless/p\textgreater{}

\textless/body\textgreater{}

\textless/html\textgreater{}

2. Test the program with Internet Explorer. A sample output looks: Game
Programming -- Penn Wu

278

\protect\hypertarget{index_split_014.htmlux5cux23p279}{}{}

Lecture \#14

Using Yahoo! User Interface (YUI) Library

\protect\hypertarget{index_split_015.html}{}{}

\hypertarget{index_split_015.htmlux5cux23calibre_pb_14}{%
\subsection{Introduction}\label{index_split_015.htmlux5cux23calibre_pb_14}}

The Yahoo! User Interface (YUI) Library is a set of utilities and
controls, written in JavaScript, for building richly interactive web
applications using techniques such as DOM

(Document Object Model) scripting, DHTML and AJAX. The YUI Library also
includes several core CSS resources. All components in the YUI Library
have been released as open source under a BSD license and are free for
all uses.

The YUI Event

The YUI Event Utility is designed for creating event-driven
applications. Yahoo says it can Utility

make in-browser programming easier because there is no full-featured
JavaScript or CSS

library that can work in every browser. Yahoo believes their YUI can
support the vast majority of browsers, and thus make your web
programming journey easier.

All the events provided by DOM (Document Object Model) are in the
yahoo-dom-event.js library file. DOM is an application programming
interface (API). It defines the logical structure of documents and the
way a document is accessed and manipulated. With the DOM

programmers can build documents, navigate their structure, and add,
modify, or delete elements and content.

Technically speaking, the YUI Event Utility is actually a set of script
files. Yahoo currently allows programmers to externally access this file
and includes it in HTML files. Simply add the following bold-faced lines
next to the \textless HTML\textgreater{} tag.

\textless html\textgreater{}

\textbf{\textless script type="text/javascript"}

\textbf{src="http://yui.yahooapis.com/2.5.2/build/yahoo-dom-}

\textbf{event/yahoo-dom-event.js"\textgreater\textless/script\textgreater{}}

To illustrate event handling syntax, the instructor creates an area
using \textless div\textgreater{} and \textless/div\textgreater{} tags.

\textless div id="btn"\textgreater{} Click Me\textless/div\textgreater{}

The following CSS style code defines the appearance of this area.

\textless style type="text/css"\textgreater{}

\#btn \{background-color:\#abcdef; border: solid 1 black;

width:100; cursor:pointer; text-align:center\}

\textless/style\textgreater{}

Next, create a function that receives a single argument, the event
object (e), and pops up an alert which says "Welcome to CIS262!":

var msg = function(e) \{

alert("Welcome to CIS262!");

\}

This area (with and ID btn) must be associated with the function that
process the event. You need to use the Event Utility's
\textbf{addListener} method to tie the \textbf{msg} function with the
\textbf{btn} area, so it will serves as a handler for the \textbf{click}
event. Add the following line: YAHOO.util.Event.addListener("btn",
"click", msg); The complete source code now looks:

\textless html\textgreater{}

Game Programming -- Penn Wu

279

\protect\hypertarget{index_split_015.htmlux5cux23p280}{}{}\includegraphics{index-280_1.png}

\includegraphics{index-280_2.png}

\textless script type="text/javascript"

src="http://yui.yahooapis.com/2.5.2/build/yahoo-dom-

event/yahoo-dom-event.js"\textgreater\textless/script\textgreater{}

\textless script\textgreater{}

(function() \{

var msg = function(e) \{

alert("Welcome to CIS262!");

\}

YAHOO.util.Event. \textbf{addListener}("btn", "click", msg);

\})();

\textless/script\textgreater{}

\textless style type="text/css"\textgreater{}

\#btn \{background-color:\#abcdef; border: solid 1 black;

width:100; cursor:pointer; text-align:center\}

\textless/style\textgreater{}

\textless body\textgreater{}

\textless div id="btn"\textgreater{} Click Me\textless/div\textgreater{}

\textless/body\textgreater{}

\textless/html\textgreater{}

When testing the program, you first see the following:

Clicking on it, The Yahoo Event utility fires a pop-up window that
looks: Notice that the \textbf{YAHOO.util.Event.on} is an alias for
addListener. In the above example, you can use the following line
instead.

YAHOO.util.Event. \textbf{on}("btn", "click", msg); A complete list of
event utility is available at

http://developer.yahoo.com/yui/docs/YAHOO.util.Event.html.

Using the YUI

The Dom Collection comprises a family of convenience methods that
simplify common DOM-Dom Collection

scripting tasks, including element positioning and CSS style management,
while normalizing for cross-browser inconsistencies. It contains these
subsections:

•

Positioning HTML Elements

•

Getting and Setting Styles

•

Getting the Viewport Size

•

Managing Class Names

Consider the following example. It uses the following two methods of the
YUI Dom Collection

•

\textbf{getXY}: get an element's position relative to the document.

•

\textbf{setXY}: position an element relative to the document.

\textless html\textgreater{}

Game Programming -- Penn Wu

280

\protect\hypertarget{index_split_015.htmlux5cux23p281}{}{}\textless script
type="text/javascript"

src="http://yui.yahooapis.com/2.5.2/build/yahoo-dom-event/yahoo-dom-event.js"\textgreater\textless/script\textgreater{}

\textless script\textgreater{}

(function() \{

var move = function(e) \{

\textbf{YAHOO.util.Dom.setXY(\textquotesingle square\textquotesingle,
YAHOO.util.Event.getXY(e));}

\};

YAHOO.util.Event.on(document, "click", move);

\})();

\textless/script\textgreater{}

\textless style type="text/css"\textgreater{}

\#square \{width:20px; height:20px;background-color:\#00f;\}

\textless/style\textgreater{}

\textless span id="square"\textgreater\textless/span\textgreater{}

\textless/html\textgreater{}

The following line uses the \textbf{getXY()} method to retrieve the
\emph{x}- and \emph{y}-coordinates of the mouse cursor when the user
click the mouse relative to the HTML document in the browser. It then
uses the setXY() method to force the ``square'' object to move to that
position from wherever it is.

\textbf{YAHOO.util.Dom.setXY(\textquotesingle square\textquotesingle,
YAHOO.util.Event.getXY(e));}

Using the Drag \&

The term ``\textbf{drag and drop}'' is defined as moving an element from
one location to another by Drop Utility

dragging it with the mouse. It is a technique to directly manipulate an
object by moving it and placing it somewhere else using mouse.

The \textbf{Drag \& Drop Utility} allows you to create a draggable
interface efficiently, buffering you from browser-level abnormalities
and enabling you to focus on the interesting logic surrounding your
particular implementation. This component enables you to create a
variety of standard draggable objects with just a few lines of code and
then, using its extensive API, add your own specific implementation
logic.

This Drag \& Drop Utility is developed based on the following logic: The
basics are straightforward: the user holds down the mouse

button while the mouse hovers over a page element, and the user then
moves the mouse with the button still depressed. The

element follows the mouse around until the user finally

releases it.

Consider the following object and it styles. It is a red square that has
width and height of 50

pixels and a black border. Its id is ``red''.

\textless style type="text/css"\textgreater{}

span \{ width:50; height:50; cursor:default; border:solid 1

black\}

\textless/style\textgreater{}

......

\textless span id="red"
style="background-Color:red"\textgreater\textless/span\textgreater{} You
need to include the following library file in your web page with the
script tag:

\textless script type="text/javascript"

Game Programming -- Penn Wu

281

\protect\hypertarget{index_split_015.htmlux5cux23p282}{}{}src="http://yui.yahooapis.com/2.5.2/build/yahoo-dom-event/\textbf{yahoo-dom-event.js}"\textgreater\textless/script\textgreater{}

To use Drag and Drop utilities, include the following library file in
your web page with the script tag:

\textless script type="text/javascript"

src="http://yui.yahooapis.com/2.5.2/build/dragdrop/\textbf{dragdrop-min.js}"\textgreater\textless/script\textgreater{}

To enable a simple drag and drop of any DOM element, create a new
instance of \textbf{YAHOO.util.DD}, providing the constructor with the
HTML id of the element you want to make draggable. Optionally, you can
provide an element reference instead of an id. For example,

var r = new YAHOO.util.DD("red");

Alternatively you can use an element reference. For example,

var r1=YAHOO.util.Dom.get("red");

var r = new YAHOO.util.DD(r1);

In this example, a new instance of \textbf{YAHOO.util.DD} is created for
an element whose HTML id is ``red''; this enables ``red'' to be dragged
and dropped.

When dragging an object, the \textbf{onDOMReady} event handler is fired
automatically. You can then use it create a visual effect of moving the
object. A sample way to create such function is:

\textless script type="text/javascript"\textgreater{}

(function() \{

var r;

YAHOO.util.Event.onDOMReady(function() \{

r = new YAHOO.util.DD("red");

\});

\})();

\textless/script\textgreater{}

Note that while the element can now be moved visually on the page, this
visual motion is accomplished by changing the element\textquotesingle s
style properties. Its actual location in the DOM itself remains
unchanged.

The basic approach used for drag-and-drop is as follows:

•

Event handlers inspect the incoming event to determine the element being
dragged.

•

A \textbf{mousedown} handler saves the starting co-ordinates, sets the
zIndex so that the element appears in front during the drag, changes
some other style settings to indicate a drag has begun, and performs
other initialisation.

•

A \textbf{mousemove} handler inspects the mouse\textquotesingle s
co-ordinates and moves the element accordingly using its left and top
properties. Here\textquotesingle s where cross-browser support gets
nasty - mouse co-ordinates in the event object are seriously
platform-specific.

•

A \textbf{mouseup} handler restores normal style settings and performs
any other cleaning up.

Using the

The Animation Utility enables the rapid prototyping and implementation
of animations Animation Utility

involving size, opacity, color, position, and other visual
characteristics. \textbf{YAHOO.util.Anim} is the base animation class
that provides the interface for building animated effects. Its syntax
is: YAHOO.util.Anim ( el , attributes , duration , method );

Its parameters are:

•

\emph{el} \textless String \textbar{} HTMLElement\textgreater{}
Reference to the element that will be animated Game Programming -- Penn
Wu

282

\protect\hypertarget{index_split_015.htmlux5cux23p283}{}{}•

\emph{attributes} \textless Object\textgreater{} The attribute(s) to be
animated. Each attribute is an object with at minimum a "to" or "by"
member defined. Additional optional members are "from"

(defaults to current value), "units" (defaults to "px"). All attribute
names use camelCase.

•

\emph{duration} \textless Number\textgreater{} (optional, defaults to 1
second) Length of animation (frames or seconds), defaults to time-based

•

\emph{method} \textless Function\textgreater{} (optional, defaults to
YAHOO.util.Easing.easeNone) Computes the values that are applied to the
attributes per frame (generally a YAHOO.util.Easing method)

To use the Animation utility, include the following source file:

\textless script

src="http://yui.yahooapis.com/2.5.2/build/animation/animation-min.js"
type="text/javascript"\textgreater\textless/script\textgreater{}
Consider the following example. In it, the instructor creates an
instance of \textbf{YAHOO.util.Anim} to animate an object.

\textless html\textgreater{}

\textless script type="text/javascript"

src="http://yui.yahooapis.com/2.5.2/build/yahoo-dom-

event/yahoo-dom-event.js"\textgreater\textless/script\textgreater{}

\textless script type="text/javascript"

src="http://yui.yahooapis.com/2.5.2/build/animation/animation-min.js"\textgreater\textless/script\textgreater{}

\textless button
id="btn"\textgreater Vanish\textless/button\textgreater{}

\textless div id="bar" style="background-Color:red;
width:70\%"\textgreater\textless/div\textgreater{}

\textless script type="text/javascript"\textgreater{}

(function() \{

var attributes = \{

width: \{ to: 0 \}

\};

var anim = new YAHOO.util.Anim(\textquotesingle bar\textquotesingle,
attributes);

YAHOO.util.Event.on(\textquotesingle btn\textquotesingle,
\textquotesingle click\textquotesingle, function() \{

anim.animate();

\});

\})();

\textless/script\textgreater{}

\textless/html\textgreater{}

The object to be animated is a red bar defined by:

\textless div id="bar" style="background-Color:red;
width:100\%"\textgreater\textless/div\textgreater{} A button is used to
fire the event defined in YAHOO.util.Event.on method:

\textless button
id="btn"\textgreater Vanish\textless/button\textgreater{}

The final step is to call the animate method on our instance to start
the animation. The button will be the trigger that begins the animation
sequence:

YAHOO.util.Event.on(\textquotesingle btn\textquotesingle,
\textquotesingle click\textquotesingle, function() \{

anim.animate();

Game Programming -- Penn Wu

283

\protect\hypertarget{index_split_015.htmlux5cux23p284}{}{}\includegraphics{index-284_1.png}

\includegraphics{index-284_2.png}

\});

When the ``Vanish'' button is clicked, the width of the red bar decrease
at a constant speed till it disappears.

to

The YUI Animation Utility includes an Easing feature, which allows you
to customize how the animation behaves. For example, add the duration
and easing arguments are optional.

var anim = new YAHOO.util.Anim(\textquotesingle bar\textquotesingle,
attributes, \textbf{2},

\textbf{YAHOO.util.Easing.easeOut});

In this case, the value 2 is the duration of animation. The
\textbf{easeOut} property will slow the animation gradually as it nears
the end.

Similarly, you can change \textbf{YAHOO.util.Easing.easeOut} to any of
the following for different visual effects:

•

YAHOO.util.Easing.backBoth

•

YAHOO.util.Easing.backIn

•

YAHOO.util.Easing.backOut

•

YAHOO.util.Easing.bounceBoth

•

YAHOO.util.Easing.bounceIn

•

YAHOO.util.Easing.bounceOut

•

YAHOO.util.Easing.easeBoth

•

YAHOO.util.Easing.easeBothStrong

•

YAHOO.util.Easing.easeIn

•

YAHOO.util.Easing.easeInStrong

•

YAHOO.util.Easing.easeNone

•

YAHOO.util.Easing.easeOut

•

YAHOO.util.Easing.easeOutStrong

•

YAHOO.util.Easing.elasticBoth

•

YAHOO.util.Easing.elasticIn

•

YAHOO.util.Easing.elasticOut

The Animation Utility also allows you to animate the motion of an
\textbf{HTMLElement} along a curved path using control points.

Try create a demo object and a button that will move the object:

\textless button id="btn"\textgreater Move\textless/button\textgreater{}
\textless br/\textgreater{}

\textless span id="square"\textgreater\textless/span\textgreater{}

Use CSS style to define how this object will look:

\textless style type="text/css"\textgreater{}

\#square \{

background-Color:red; height:20px; width:20px;

\}

\textless/style\textgreater{}

Next, create an instance of \textbf{YAHOO.util.Motion}, passing it the
element we wish to animate, and the points attribute (an array of {[}x,
y{]} positions), with the point we are animating to, and Game
Programming -- Penn Wu

284

\protect\hypertarget{index_split_015.htmlux5cux23p285}{}{}the control
points that will influence the path:

\textless script type="text/javascript"\textgreater{}

var attributes = \{

points: \{ to: {[}10,50{]}, control: {[}{[}150,400{]}, {[}300,100{]},

{[}50,800{]}, {[}450,10{]} {]} \}

\};

var anim = new
YAHOO.util.Motion(\textquotesingle square\textquotesingle, attributes);

\textless/script\textgreater{}

\textbf{{[}10,50{]}} is the finish point, namely it is last point the
red square will move to. The array,

{[}150,400{]}, {[}300,100{]}, {[}50,800{]}, and {[}450,10{]}, defines
the path of the movement. In each set

{[} \emph{x}, \emph{y}{]}, the value represents \emph{x}- and
\emph{y}-coordinates on the HTML document.

The syntax of \textbf{YAHOO.util.Motion} is:

YAHOO.util.Motion ( \emph{el} , \emph{attributes} , \emph{duration} ,
\emph{method} ); The parameters are:

•

\emph{el} \textless String \textbar{} HTMLElement\textgreater{}
Reference to the element that will be animated

•

\emph{attributes} \textless Object\textgreater{} The attribute(s) to be
animated. Each attribute is an object with at minimum a "to" or "by"
member defined. Additional optional members are "from"

(defaults to current value), "units" (defaults to "px"). All attribute
names use camelCase.

•

\emph{duration} \textless Number\textgreater{} (optional, defaults to 1
second) Length of animation (frames or seconds), defaults to time-based

•

\emph{method} \textless Function\textgreater{} (optional, defaults to
YAHOO.util.Easing.easeNone) Computes the values that are applied to the
attributes per frame (generally a YAHOO.util.Easing method)

The final step is to call the
\textless code\textgreater animate\textless/code\textgreater{} method on
our instance to start the animation. The button will be the trigger that
begins the animation sequence.

\textless script type="text/javascript"\textgreater{}

YAHOO.util.Event.on(\textquotesingle btn\textquotesingle,
\textquotesingle click\textquotesingle, function() \{

anim.animate();

\});

\textless/script\textgreater{}

The complete code will look:

\textless html\textgreater{}

\textless script type="text/javascript"

src="http://yui.yahooapis.com/2.5.2/build/yahoo-dom-

event/yahoo-dom-event.js"\textgreater\textless/script\textgreater{}

\textless script type="text/javascript"

src="http://yui.yahooapis.com/2.5.2/build/animation/animation-min.js"\textgreater\textless/script\textgreater{}

\textless style type="text/css"\textgreater{}

\#square \{

background-Color:red; height:20px; width:20px; left:10;

top:50\}

\}

\textless/style\textgreater{}

\textless button
id="btn"\textgreater Move\textless/button\textgreater\textless br/\textgreater{}

\textless span id="square"\textgreater\textless/span\textgreater{}

\textless script type="text/javascript"\textgreater{}

(function() \{

Game Programming -- Penn Wu

285

\protect\hypertarget{index_split_015.htmlux5cux23p286}{}{}\includegraphics{index-286_1.png}

var attributes = \{

points: \{ to: {[}10,50{]}, control: {[}{[}150,400{]}, {[}300,100{]},

{[}50,800{]}, {[}450,10{]} {]} \}

\};

var anim = new
YAHOO.util.Motion(\textquotesingle square\textquotesingle, attributes);

YAHOO.util.Event.on(\textquotesingle btn\textquotesingle,
\textquotesingle click\textquotesingle, function() \{

anim.animate();

\});

\})();

\textless/script\textgreater{}

\textless/html\textgreater{}

Coping with the

The Portable Network Graphic (PNG) image format was developed by the
World Wide Web IE transparent

Consortium as a better GIF than GIF many years ago. It is the trend in
the Web developing issue

industry to use .png instead of .gif graphics. However, Internet
Explorer in Windows XP (or early) platform supports PNG transparency
only up to the 256-color palette. When the PNG

transparent image uses a better palette, IE will make transparent parts
opaque. For example, Web developers have long complained about this
problems. Microsoft admits the problem
(http://support.microsoft.com/?scid=kb;en-us;265221) and offers a
workaround that\textquotesingle s not entirely satisfactory.

Microsoft's AlphaImageLoader uses DirectX components to produce the same
transparency effect (http://msdn.microsoft.com/workshop/author/

filter/reference/filters/AlphaImageLoader.asp ). It\textquotesingle s a
filter you can apply to any other HTML

element, even a block of text. Here it\textquotesingle s applied to DIV.

\textless DIV ID="oDiv" STYLE="position:absolute; left:140px;
height:400; width:400;
filter:progid:DXImageTransform.Microsoft.AlphaImageLoader
(src=\textquotesingle image.png\textquotesingle, sizing
Method=\textquotesingle scale\textquotesingle);" \textgreater{}

\textless/DIV\textgreater{}

This may not seem that painful, but if you have several PNGs on your
page, you need to repeat this for each one. Obviously this workaround is
a lot more complicated to use than a simple

\textless img\textgreater{} tag.

Review Questions

1. Which statement is true about the Yahoo! User Interface?

A. It is a set of utilities and controls, written in C++, for building
application.

B. It is written in Java using techniques such as CSS and DHTML.

C. It is a set of toolkit Yahoo! developed to sell for profits.

D. None of the above.

D

2. Which statement is not true?

Game Programming -- Penn Wu

286

\protect\hypertarget{index_split_015.htmlux5cux23p287}{}{}A. The YUI
Event Utility is designed for creating event-driven applications.

B. DOM is short for Data Object Mode.

C. There is no full-featured JavaScript or CSS library that can work in
every browser.

D. All the events provided by DOM are in the yahoo-dom-event.js library
file B

3. Given the following code segment, which statement is correct?

var msg = function(e) \{

alert("Welcome to CIS262!");

\}

A. var is the name of function.

B. msg is the name of function.

C. msg is a variable that holds the value "Welcome to CIS262!".

D. alert is a variable that holds the value "Welcome to CIS262!".

B

4. Which has exactly the same functionality as the following:

YAHOO.util.Event.addListener("btn", "click", msg); A.
YAHOO.util.Event.on("btn", "click", msg); B.
YAHOO.util.Event.getListener("btn", "click", msg); C.
YAHOO.util.Event.loadListener("btn", "click", msg); D.
YAHOO.util.Event.get("btn", "click", msg); A

5. Which method of the YAHOO.util.Dom class can position an element
relative to the document?

A. setXY

B. getXY

C. readXY

D. writeXY

A

6. Given the following code segment, which statement is correct?

YAHOO.util.Dom.setXY(\textquotesingle square\textquotesingle,
YAHOO.util.Event.getXY(e));

A. It uses the getXY() method to retrieve the x- and y-coordinates of
the object e.

B. It uses the setXY() method to retrieve the x- and y-coordinates of
the object e.

C. It uses the getXY() method to force the ``square'' object to move to
a new position from wherever it is.

D. All of the above.

A

7. The term ``drag and drop'' is defined as \_\_.

A. clicking a button to change its location.

B. clicking an icon to change its location.

C. moving an element from one location to another by dragging it with
the mouse.

Game Programming -- Penn Wu

287

\protect\hypertarget{index_split_015.htmlux5cux23p288}{}{}D. moving an
element from one location to another by holding the Shift key.

C

8. Which class is the one that support "drag and drop"?

A. YAHOO.util.Dom

B. YAHOO.util.Event

C. YAHOO.util.DD

D. YAHOO.util.Anim

C

9. Given the following code segment, which is the name of an animated
object?

var anim = new YAHOO.util.Anim(\textquotesingle bar\textquotesingle,
attributes, 2, YAHOO.util.Easing.easeOut); A. anim

B. bar

C. 2

D. easeOut

B

10. Given the following code segment, which is the finish point of the
movement?

points: \{ to: {[}10,50{]}, control: {[}{[}150,400{]}, {[}300,100{]},
{[}50,800{]}{]} \}

A. (10, 50)

B. (150, 400)

C. (300, 100)

D. (50, 800)

A

Game Programming -- Penn Wu

288

\protect\hypertarget{index_split_015.htmlux5cux23p289}{}{}\includegraphics{index-289_1.png}

\includegraphics{index-289_2.png}

Lab \#14

Using Yahoo! User Interface (YUI) Library

\textbf{Learning Activity \#1: Using YUI Event Utility}

1. Use Notepad to create a new file named \textbf{lab14\_1.htm} with the
following contents:

\textless html\textgreater{}

\textless script type="text/javascript"

src="http://yui.yahooapis.com/2.5.2/build/yahoo-dom-event/yahoo-dom-event.js"\textgreater{}

\textless/script\textgreater{}

\textless script\textgreater{}

(function() \{

var msg = function(e) \{

alert("Welcome to CIS262!");

\}

YAHOO.util.Event. \textbf{addListener}("btn", "click", msg);

\})();

\textless/script\textgreater{}

\textless style type="text/css"\textgreater{}

\#btn \{background-color:\#abcdef; border: solid 1 black; width:100;
cursor:pointer; text-align:center\}

\textless/style\textgreater{}

\textless body\textgreater{}

\textless div id="btn"\textgreater{} Click Me\textless/div\textgreater{}

\textless/body\textgreater{}

\textless/html\textgreater{}

2. Use Internet Explorer to test the code. Testing the program, and
click on the following button.

3. Clicking on it, The Yahoo Event utility fires a pop-up window that
looks: \textbf{Learning Activity \#2: Using Drag-\&-Drop Library}

1. Use Notepad to create a new file named \textbf{lab14\_2.htm} with the
following contents:

\textless html\textgreater{}

\textless head\textgreater{}

\textless script type="text/javascript"

src="http://yui.yahooapis.com/2.5.2/build/yahoo-dom-event/yahoo-dom-event.js"\textgreater{}

\textless/script\textgreater{}

\textless script type="text/javascript"

src="http://yui.yahooapis.com/2.5.2/build/dragdrop/dragdrop-min.js"\textgreater{}

\textless/script\textgreater{}

\textless style type="text/css"\textgreater{}

Game Programming -- Penn Wu

289

\protect\hypertarget{index_split_015.htmlux5cux23p290}{}{}\includegraphics{index-290_1.png}

span \{ width:50; height:50; cursor:default; border:solid 1 black\}

\textless/style\textgreater{}

\textless script type="text/javascript"\textgreater{}

(function() \{

var r, g, b;

YAHOO.util.Event.onDOMReady(function() \{

r = new YAHOO.util.DD("red");

g = new YAHOO.util.DD("green");

b = new YAHOO.util.DD("blue");

\});

\})();

\textless/script\textgreater{}

\textless/head\textgreater{}

\textless body\textgreater{}

\textless span id="red"
style="background-Color:red"\textgreater\textless/span\textgreater{}

\textless span id="green"
style="background-Color:green"\textgreater\textless/span\textgreater{}

\textless span id="blue"
style="background-Color:blue"\textgreater\textless/span\textgreater{}

\textless/body\textgreater{}

\textless/html\textgreater{}

2. Use Internet Explorer to test the code.

\textbf{Learning Activity \#3: Animation with different visual effects}
1. Use Notepad to create a new file named \textbf{lab14\_3.htm} with the
following contents:

\textless html\textgreater{}

\textless script type="text/javascript"

src="http://yui.yahooapis.com/2.5.2/build/yahoo-dom-event/yahoo-dom-event.js"\textgreater{}

\textless/script\textgreater{}

\textless script type="text/javascript"

src="http://yui.yahooapis.com/2.5.2/build/animation/animation-min.js"\textgreater\textless/script\textgreater{}

\textless button
id="btn"\textgreater Vanish\textless/button\textgreater{}

\textless button id="btn\_e"\textgreater Vanish with
easeOut\textless/button\textgreater{}

\textless button id="btn\_b"\textgreater Vanish with
backIn\textless/button\textgreater\textless p\textgreater{}

\textless div id="red" style="background-Color:red;
width:90\%"\textgreater\textless/div\textgreater\textless br
/\textgreater{}

\textless div id="blue" style="background-Color:blue;
width:90\%"\textgreater\textless/div\textgreater\textless br
/\textgreater{}

\textless div id="yellow" style="background-Color:yellow;
width:90\%"\textgreater\textless/div\textgreater\textless br
/\textgreater{}

\textless script type="text/javascript"\textgreater{}

(function() \{

var attributes = \{

width: \{ to: 0 \}

\};

Game Programming -- Penn Wu

290

\protect\hypertarget{index_split_015.htmlux5cux23p291}{}{}\includegraphics{index-291_1.png}

var anim = new YAHOO.util.Anim(\textquotesingle red\textquotesingle,
attributes);

var anim\_e = new YAHOO.util.Anim(\textquotesingle blue\textquotesingle,
attributes, 2,

YAHOO.util.Easing.easeOut);

var anim\_b = new
YAHOO.util.Anim(\textquotesingle yellow\textquotesingle, attributes, 2,

YAHOO.util.Easing.backIn);

YAHOO.util.Event.on(\textquotesingle btn\textquotesingle,
\textquotesingle click\textquotesingle, function() \{

anim.animate();

\});

YAHOO.util.Event.on(\textquotesingle btn\_e\textquotesingle,
\textquotesingle click\textquotesingle, function() \{

anim\_e.animate();

\});

YAHOO.util.Event.on(\textquotesingle btn\_b\textquotesingle,
\textquotesingle click\textquotesingle, function() \{

anim\_b.animate();

\});

\})();

\textless/script\textgreater{}

\textless/html\textgreater{}

2. Use Internet Explorer to test the code. Click the buttons to see the
different visual effects.

\textbf{Learning Activity \#4: An online puzzle game}

Note: If you are using a browser that supports .png transparency, you
can use the shortened code in Appendix A.

\textbf{}

1. Use Notepad to create a new file named \textbf{lab14\_4.htm} with the
following contents:

\textless html\textgreater{}

\textless head\textgreater{}

\textless script type="text/javascript"

src="http://yui.yahooapis.com/2.5.2/build/yahoo-dom-event/yahoo-dom-event.js"\textgreater{}

\textless/script\textgreater{}

\textless script type="text/javascript"

src="http://yui.yahooapis.com/2.5.2/build/dragdrop/dragdrop-min.js"\textgreater{}

\textless/script\textgreater{}

\textless style type="text/css"\textgreater{}

img \{ cursor:default; position:absolute\}

\textless/style\textgreater{}

\textless script type="text/javascript"\textgreater{}

function init() \{

m1.style.pixelLeft =
Math.floor(Math.random()*(document.body.clientWidth-200));
m1.style.pixelTop =
Math.floor(Math.random()*(document.body.clientHeight-200));
m2.style.pixelLeft =
Math.floor(Math.random()*(document.body.clientWidth-200));
m2.style.pixelTop =
Math.floor(Math.random()*(document.body.clientHeight-200));
m3.style.pixelLeft =
Math.floor(Math.random()*(document.body.clientWidth-200));
m3.style.pixelTop =
Math.floor(Math.random()*(document.body.clientHeight-200));
m4.style.pixelLeft =
Math.floor(Math.random()*(document.body.clientWidth-200)); Game
Programming -- Penn Wu

291

\protect\hypertarget{index_split_015.htmlux5cux23p292}{}{}m4.style.pixelTop
= Math.floor(Math.random()*(document.body.clientHeight-200));
m5.style.pixelLeft =
Math.floor(Math.random()*(document.body.clientWidth-200));
m5.style.pixelTop =
Math.floor(Math.random()*(document.body.clientHeight-200));
m6.style.pixelLeft =
Math.floor(Math.random()*(document.body.clientWidth-200));
m6.style.pixelTop =
Math.floor(Math.random()*(document.body.clientHeight-200));

\}

(function() \{

YAHOO.util.Event.onDOMReady(function() \{

var m1 = new YAHOO.util.DD("m1");

var m2 = new YAHOO.util.DD("m2");

var m3 = new YAHOO.util.DD("m3");

var m4 = new YAHOO.util.DD("m4");

var m5 = new YAHOO.util.DD("m5");

var m6 = new YAHOO.util.DD("m6");

\});

\})();

\textless/script\textgreater{}

\textless/head\textgreater{}

\textless body onLoad="init()"\textgreater{}

\textless DIV ID="m1" STYLE="position:absolute; width:128; height:97;
filter:progid:DXImageTransform.Microsoft.AlphaImageLoader(

src=\textquotesingle http://business.cypresscollege.edu/\textasciitilde pwu/cis262/1.png\textquotesingle,

sizingMethod=\textquotesingle scale\textquotesingle);"\textgreater{}

\textless/DIV\textgreater{}

\textless DIV ID="m2" STYLE="position:absolute; width:127; height:153;
filter:progid:DXImageTransform.Microsoft.AlphaImageLoader(

src=\textquotesingle http://business.cypresscollege.edu/\textasciitilde pwu/cis262/2.png\textquotesingle,

sizingMethod=\textquotesingle scale\textquotesingle);"\textgreater{}

\textless/DIV\textgreater{}

\textless DIV ID="m3" STYLE="position:absolute; width:126; height:113;
filter:progid:DXImageTransform.Microsoft.AlphaImageLoader(

src=\textquotesingle http://business.cypresscollege.edu/\textasciitilde pwu/cis262/3.png\textquotesingle,

sizingMethod=\textquotesingle scale\textquotesingle);"\textgreater{}

\textless/DIV\textgreater{}

\textless DIV ID="m4" STYLE="position:absolute; width:95; height:132;
filter:progid:DXImageTransform.Microsoft.AlphaImageLoader(

src=\textquotesingle http://business.cypresscollege.edu/\textasciitilde pwu/cis262/4.png\textquotesingle,

sizingMethod=\textquotesingle scale\textquotesingle);"\textgreater{}

\textless/DIV\textgreater{}

\textless DIV ID="m5" STYLE="position:absolute; width:130; height:138;
filter:progid:DXImageTransform.Microsoft.AlphaImageLoader(

src=\textquotesingle http://business.cypresscollege.edu/\textasciitilde pwu/cis262/5.png\textquotesingle,

sizingMethod=\textquotesingle scale\textquotesingle);"\textgreater{}

\textless/DIV\textgreater{}

\textless DIV ID="m6" STYLE="position:absolute; width:109; height:113;
filter:progid:DXImageTransform.Microsoft.AlphaImageLoader(

src=\textquotesingle http://business.cypresscollege.edu/\textasciitilde pwu/cis262/6.png\textquotesingle,

sizingMethod=\textquotesingle scale\textquotesingle);"\textgreater{}

\textless/DIV\textgreater{}

\textless/body\textgreater{}

\textless/html\textgreater{}

Game Programming -- Penn Wu

292

\protect\hypertarget{index_split_015.htmlux5cux23p293}{}{}2. Use
Internet Explorer to test the code.

\textbf{Learning Activity \#5: Moving along a curve}

1. Use Notepad to create a new file named \textbf{lab14\_5.htm} with the
following contents:

\textless html\textgreater{}

\textless script type="text/javascript"
src="http://yui.yahooapis.com/2.5.2/build/yahoo-dom-event/yahoo-dom-event.js"\textgreater\textless/script\textgreater{}

\textless script type="text/javascript"

src="http://yui.yahooapis.com/2.5.2/build/animation/animation-min.js"\textgreater\textless/script\textgreater{}

\textless style type="text/css"\textgreater{}

\#square \{

background-Color:red; height:20px; width:20px; left:10; top:50\}

\}

\textless/style\textgreater{}

\textless button
id="btn"\textgreater Move\textless/button\textgreater\textless br/\textgreater{}

\textless span id="square"\textgreater\textless/span\textgreater{}

\textless script type="text/javascript"\textgreater{}

(function() \{

var attributes = \{

points: \{ to: {[}10,50{]}, control: {[}{[}150,400{]}, {[}300,100{]},
{[}50,800{]}, {[}450,10{]} {]} \}

\};

var anim = new
YAHOO.util.Motion(\textquotesingle square\textquotesingle, attributes);

YAHOO.util.Event.on(\textquotesingle btn\textquotesingle,
\textquotesingle click\textquotesingle, function() \{

anim.animate();

\});

\})();

\textless/script\textgreater{}

\textless/html\textgreater{}

2. Use Internet Explorer to test the code. Click the button to move the
object along a curve.

\textbf{Submittal}

1. Complete all the 5 learning activities in this lab.

2. Create a .zip file named lab4.zip containing ONLY the following
self-executable files.

•

Lab14\_1.htm

•

Lab14\_2.htm

•

Lab14\_3.htm

•

Lab14\_4.htm

•

Lab14\_5.htm

3. Log in to Blackboard, and enter the course site.

4. Upload the zipped file to question 11 of Assignment 04 as response.

Game Programming -- Penn Wu

293

\protect\hypertarget{index_split_015.htmlux5cux23p294}{}{}Appendix A:

\textless html\textgreater{}

\textless head\textgreater{}

\textless script type="text/javascript"

src="http://yui.yahooapis.com/2.5.2/build/yahoo-dom-event/yahoo-dom-event.js"\textgreater{}

\textless/script\textgreater{}

\textless script type="text/javascript"

src="http://yui.yahooapis.com/2.5.2/build/dragdrop/dragdrop-min.js"\textgreater{}

\textless/script\textgreater{}

\textless style type="text/css"\textgreater{}

img \{ cursor:default; position:absolute\}

\textless/style\textgreater{}

\textless script type="text/javascript"\textgreater{}

function init() \{

m1.style.pixelLeft =
Math.floor(Math.random()*(document.body.clientWidth-200));
m1.style.pixelTop =
Math.floor(Math.random()*(document.body.clientHeight-200));
m2.style.pixelLeft =
Math.floor(Math.random()*(document.body.clientWidth-200));
m2.style.pixelTop =
Math.floor(Math.random()*(document.body.clientHeight-200));
m3.style.pixelLeft =
Math.floor(Math.random()*(document.body.clientWidth-200));
m3.style.pixelTop =
Math.floor(Math.random()*(document.body.clientHeight-200));
m4.style.pixelLeft =
Math.floor(Math.random()*(document.body.clientWidth-200));
m4.style.pixelTop =
Math.floor(Math.random()*(document.body.clientHeight-200));
m5.style.pixelLeft =
Math.floor(Math.random()*(document.body.clientWidth-200));
m5.style.pixelTop =
Math.floor(Math.random()*(document.body.clientHeight-200));
m6.style.pixelLeft =
Math.floor(Math.random()*(document.body.clientWidth-200));
m6.style.pixelTop =
Math.floor(Math.random()*(document.body.clientHeight-200));

\}

(function() \{

YAHOO.util.Event.onDOMReady(function() \{

var m1 = new YAHOO.util.DD("m1");

var m2 = new YAHOO.util.DD("m2");

var m3 = new YAHOO.util.DD("m3");

var m4 = new YAHOO.util.DD("m4");

var m5 = new YAHOO.util.DD("m5");

var m6 = new YAHOO.util.DD("m6");

\});

\})();

\textless/script\textgreater{}

\textless/head\textgreater{}

\textless body onLoad="init()"\textgreater{}

\textbf{\textless img id="m1"
src="http://business.cypresscollege.edu/\textasciitilde pwu/cis262/1.png"\textgreater{}}

\textbf{\textless img id="m2"
src="http://business.cypresscollege.edu/\textasciitilde pwu/cis262/2.png"\textgreater{}}

\textbf{\textless img id="m3"
src="http://business.cypresscollege.edu/\textasciitilde pwu/cis262/3.png"\textgreater{}}

\textbf{\textless img id="m4"
src="http://business.cypresscollege.edu/\textasciitilde pwu/cis262/4.png"\textgreater{}}

\textbf{\textless img id="m5"
src="http://business.cypresscollege.edu/\textasciitilde pwu/cis262/5.png"\textgreater{}}

\textbf{\textless img id="m6"
src="http://business.cypresscollege.edu/\textasciitilde pwu/cis262/6.png"\textgreater{}}

\textless/body\textgreater{}

\textless/html\textgreater{}

Game Programming -- Penn Wu

294

\protect\hypertarget{index_split_015.htmlux5cux23p295}{}{}

Lecture \#14

Basics of 2D/3D Graphics

Using graphics

In computer graphics, graphics software or image editing software is a
program or collection software

of programs that enable a person to manipulate visual images on a
computer. They are very useful in producing high-quality graphics and
image files. Many games use such pre-produced graphics and images to
enhance their multimedia performance.

Most graphics programs have the ability to import and export one or more
graphics file formats. Several graphics programs support animation, or
digital video. Vector graphics animation can be described as a series of
mathematical transformations that are applied in sequence to one or more
shapes in a scene. Raster graphics animation works in a similar fashion
to film-based animation, where a series of still images produces the
illusion of continuous movement.

Graphics and

Graphics images created for use in the World Wide Web have the file name
extensions: \textbf{.gif} Images types

(Graphics Interchange Format), \textbf{.jpg} or jpeg (pronounced as
jaypeg) or \textbf{.png} (stands for portable network graphics and is
pronounced as \textquotesingle ping\textquotesingle). There are many
more less popular graphics formats used on the world wide web. You can
discover their formats by viewing their file extensions. The file
extension is the three letters that come after the file name.

The .gif, .jpg (jpeg) and .png file formats allow image data to be
stored in a compressed form that maximizes the amount of space used but
describes the colors, dimensions, and excepting jpg format, any
transparent background in the image.

The gif and png formats permit transparent backgrounds, making them
suitable for logos (like the cockerel), web buttons etc. When you
don\textquotesingle t want to have a colored background around the
image, like this, but want the background page color to show through.
Gifs are used for animated graphics. Both gif and png are best used when
the image contains only a few colors and are excellent for line
drawings. PNG format graphics may not be visible in all browsers.

The jpg (jpeg) format is good for images that contain a range of colors
and shades, e.g.

photographs but isn\textquotesingle t much good for any images with
sharp edges, including lettering or line drawings.

The png format is expected to eventually replace gif. It supports
transparency and is better than gif at displaying color, although image
sizes tend to be larger.

A general rule is that the images have to be of a fairly small size
otherwise they take to long to load when people are looking at web
pages.

Animated Gifs

The so-called " \textbf{Animated GIF}" is an animation created by
combining multiple GIF images in one file. The result is multiple
images, displayed one after another, that give the appearance of
movement. You can use a GIF animation tool to create sequences of images
to simulate animation and allows for transparent background colors.
Animated GIFs can generate higher response rates than static banners.

Creating Animated You can start with a check list of all the things you
need to create animated gifs.

Gifs

•

An imaging software such as paint shop pro or color works,

•

A gif assembling software such as Gif animator, Animation Shop, Giffy,

•

Creativity, and

•

A lot of patience

A number of software programs are available either for free or shareware
on the Internet. We Game Programming -- Penn Wu

295

\protect\hypertarget{index_split_015.htmlux5cux23p296}{}{}\includegraphics{index-296_1.png}

\includegraphics{index-296_2.png}

\includegraphics{index-296_3.png}

\includegraphics{index-296_4.png}

\includegraphics{index-296_5.png}

\includegraphics{index-296_6.png}

\includegraphics{index-296_7.png}

\includegraphics{index-296_8.png}

\includegraphics{index-296_9.png}

\includegraphics{index-296_10.png}

\includegraphics{index-296_11.png}

recommend Paint Shop Pro (JASC. Inc.) as the imaging software, and Gif
Animator (Ulead).

You may e familiar with other software programs. A complete list of
software programs available can be found here.

Creating animated gifs is really simple. Let us start off with an
example. Below is an image of a ball.

The aim of this first exercise is to make the ball move from left to
right and then back.

In the imaging software often a new work area 450 pixels wide and height
equal to the original image (in this case it is 49 pixels). Copy the
ball and paste it into the working area at the far left.

Create a new working area same height and width. Paste the ball at a
position more at the right than the previous frame. Repeat this
procedure until the ball is completely at the right. At this point you
should have a number of frames like these shown below.

File 1

File 2

File 3

File 4

File 5

File 6

File 7

File 8

File 9

Now import the individual images in the gif assembler program in
sequential order, and then in the reverse order. Save the animation. The
end result should look something like this: You can use Internet
Explorer to visit

http://business.cypresscollege.edu/\textasciitilde pwu/cis261/files/rolling\_ball.gif
to see how this gif file is animated.

Transparency

A major problem which you might encounter when creating animated gifs is
that the background of the image does not blend so well with the
background you are using. This can be solved in two ways:

•

Change the background color of the animation to the one in your webpage,
or else

•

Create a transparent background

The use of transparent backgrounds allows us to blend any kind of
animation (or still gifs) to any background being used. Consider the
following image.

Game Programming -- Penn Wu

296

\protect\hypertarget{index_split_015.htmlux5cux23p297}{}{}\includegraphics{index-297_1.png}

\includegraphics{index-297_2.png}

\includegraphics{index-297_3.png}

\includegraphics{index-297_4.png}

This image blends well with the white background which this page has.
The reason is that the background color of the animation is also white.

Now change the background color. Notice that the image will not blend so
well as above.

Now change the white background to a transparent one. The result is the
following.

transparent

Transparent backgrounds may seem the best way how to blend animations to
the background.

At times however the end result will not be the expected one. Let us
consider the following image:

As you can clearly see the different frames of the animation remained
visible. This is a common problem when using transparencies on moving
objects. This can be easily solved.

One of the options of every gif animation program is the way how the
image (frame will be removed from the screen. Usually the current
options are available (notice that the name of option varies from
software to software):

•

Do Not Remove

•

To Background Color

•

To Previous Image

In the above example the animation was set to Do Not Remove. On changing
the settings so that each frame is removed \textbf{To background Color},
so the results looks: Game Programming -- Penn Wu

297

\protect\hypertarget{index_split_015.htmlux5cux23p298}{}{}\includegraphics{index-298_1.png}

\includegraphics{index-298_2.png}

\includegraphics{index-298_3.png}

\includegraphics{index-298_4.png}

\includegraphics{index-298_5.png}

\includegraphics{index-298_6.png}

Reducing the file

File size is a very important thing which needs to be kept in mind when
creating animated size

gifs. The smaller they are in size the faster they will load on the web
page. It is not the first time that animations produced have a file size
over 20K. This will mean that they will take far longer to download.

Well software producers have done something to help us by creating the
so called gif reducers. What is the function of these programs? These
programs will act in 3 different ways:

•

\textbf{Palette reducers} - This feature will reduce the number of
colors within the animation from the original number (e.g. 256) to a
user defined or predefined number (e.g. 64).

Obviously the image will sometimes loose in quality and become dithered.
In this case one can change the number of resulting colors to an
acceptable level.

•

Removal of Redundant pixels - This is a very cool feature. Basically the
software will compare the different frames and will remove/crop any
pixels which do not change from one image to an other. Consider the
following panther. It is composed of 2 frames. In the initial frame the
panther has the eyes open. In the second frame the tiger has the eyes
closed. The software cropped off the pixels which do not change.

+

Removal of redundant pixels

=

This process has reduced the file size by 5,279 bytes which is
equivalent to 63\% of the original file size.

•

Removal of Comment Blocks - Comment blocks are text annotations added to
the images which do not appear in the animation. These annotations
usually state who created the animated gif, and which program was used
in creating them. Although the amount of bytes added to the file size
may be few, some programs remove these annotations as well.

These features are shared in the majority of software packages
available.

Game Programming -- Penn Wu

298

\protect\hypertarget{index_split_015.htmlux5cux23p299}{}{}\includegraphics{index-299_1.png}

\includegraphics{index-299_2.png}

3D graphics

3D computer graphics are works of graphic art created with the aid of
digital computers and 3D software. The term may also refer to the
process of creating such graphics, or the field of study of computer
graphic techniques and related technology. An example of 3D graphics is:
Each 3D object takes up space in three directions, defined in 3D
parlance as the \textbf{X}, \textbf{Y}, and \textbf{Z}

axes. Many computer languages refer the three axes are \textbf{width},
\textbf{height}, and \textbf{depth}.

•

\textbf{width}: the object's left-to-right dimension

•

\textbf{height}: the object's top-to-bottom dimension

•

\textbf{depth}: the object's front-to-back dimension

depth

height

width

Take a close look at the above illustration, which the instructor
created for this class. 3D

computer graphics are different from 2D computer graphics in that a
three-dimensional representation of geometric data is stored in the
computer for the purposes of performing calculations and rendering 2D
images. Such images may be for later display or for real-time viewing.

3D modeling is the process of preparing geometric data for 3D computer
graphics, and is akin to sculpting, whereas the art of 2D graphics is
analogous to photography. Despite these differences, 3D computer
graphics rely on many of the same algorithms as 2D computer graphics.

In computer graphics software, the distinction between 2D and 3D is
occasionally blurred; 2D

applications may use 3D techniques to achieve effects such as lighting,
and primarily 3D may use 2D rendering techniques.

Although this class is not designed for you to learn 3D graphics, do not
hesitate to download some 3D modeling software and try for yourself. Use
Internet Explorer to visit
http://www.freeserifsoftware.com/software/3dPlus/default.asp, and
download a copy of 3DPlus 2 (which is a completely FREE 3D animation and
modeling software).

By the way, currently, OpenGL and Direct3D are two popular APIs for
generation of real-time imagery. Real-time means that image generation
occurs in \textquotesingle real time\textquotesingle, or
\textquotesingle on the fly\textquotesingle, and may be highly
user-interactive. Many modern graphics cards provide some degree of Game
Programming -- Penn Wu

299

\protect\hypertarget{index_split_015.htmlux5cux23p300}{}{}hardware
acceleration based on these APIs, frequently enabling display of complex
3D

graphics in real-time.

Review Questions

8. Which graphics formats permit transparent backgrounds?

A. .jpg

B. .wma

C. .png

D. .bmp

9. To allow an image to blends wells with the background of a given
area, you need to make sure the image file is \_\_.

A. intangible

B. impermeable

C. transparent

D. invisible

10. Which is not a common dimension of a 3D object?

A. width

B. height

C. depth

D. breadth

Game Programming -- Penn Wu

300

\protect\hypertarget{index_split_015.htmlux5cux23p301}{}{}\includegraphics{index-301_1.png}

\includegraphics{index-301_2.png}

\includegraphics{index-301_3.png}

\includegraphics{index-301_4.png}

Lab \#15

Game Graphics

\textbf{}

\textbf{Preparation \#1:}

3. Create a new directory named \textbf{C:\textbackslash games}.

4. Use Internet Explorer to visit
http://business.cypresscollege.edu/\textasciitilde pwu/cis261/files/lab2.zip
and download the zipped file. Extract the files to
C:\textbackslash games directory. Make sure the C:\textbackslash games
directory contains:

•

star1.gif

•

star2.gif

•

logo1.gif

•

logo2.gif

•

logo3.gif

5. Use Internet Explorer to visit
http://www.freeserifsoftware.com/software/PhotoPlus/, and download a
copy of PhotoPlus 6 (which is a completely \textbf{FREE} image and photo
editing software) to your C:\textbackslash{} drive.

6. Install the PhotoPlus 6 software. An installation tip is available at
http://www.freeserifsoftware.com/software/PhotoPlus/getting\_started.asp.

\textbf{Learning Activity \#1: Using Microsoft Paint to create 2D image
file} 1. Click Start, Programs, Accessories, and then Paint to open the
Microsoft Paint.

2. On the toolbar, click Image, and then Attributes..

3. Set the image width to 300 pixels and height to 300 pixels, as shown
below.

4. Click OK.

5. On the toolbox, click the Line icon.

.

6. Change the color to red.

7. Create a labyrinth. For example,

Game Programming -- Penn Wu

301

\protect\hypertarget{index_split_015.htmlux5cux23p302}{}{}\includegraphics{index-302_1.png}

\includegraphics{index-302_2.png}

\includegraphics{index-302_3.png}

\includegraphics{index-302_4.png}

\includegraphics{index-302_5.png}

8. Add a background color if you wish. For example,

9. Save the file as
\textbf{C:\textbackslash games\textbackslash lab2\_1.gif} for later use.
(You will use this image file in a later lecture.) \textbf{Learning
Activity \#2: Creating animated Gif file} (using PhotoPlus 6.0 or your
preferred tool) \textbf{}

1. Click Start, (All) Programs, and then Serif PhotoPlus6 to launch the
program.

2. Click the Create New Animation option.

3. Set the Width to be \textbf{50}, and Height to be \textbf{50}. Click
OK.

4. The screen now looks:

5. Click File, and Open, and then locate the star1.gif file to open it.

Game Programming -- Penn Wu

302

\protect\hypertarget{index_split_015.htmlux5cux23p303}{}{}\includegraphics{index-303_1.png}

\includegraphics{index-303_2.png}

\includegraphics{index-303_3.png}

\includegraphics{index-303_4.png}

\includegraphics{index-303_5.png}

\includegraphics{index-303_6.png}

6. Open star2.gif, too.

7. Click on \textbf{star1.gif}, click Edit, and then Copy.

8. Click the \textbf{Untitled \emph{n}} \textbf{} (where \emph{n} is a
number), click Edit, Paste, and then As New Layer.

9. The Animation Tab now looks:

10. Right click on the \textbf{Frame 1 of 1} picture, as shown below,
and then click New\ldots{}

11. Click on \textbf{star2.gif}, click Edit, and then Copy.

12. Click the \textbf{Untitled \emph{n}} \textbf{} (where \emph{n} is a
number), click Edit, Paste, and then As New Layer.

13. The Animation Tab now looks:

14. Click the

button to test the animation. The star should now rotate.

15. To export the artwork, click File, and the Export....

Game Programming -- Penn Wu

303

\protect\hypertarget{index_split_015.htmlux5cux23p304}{}{}\includegraphics{index-304_1.png}

16. Save the file as
\textbf{C:\textbackslash games\textbackslash lab2\_2.gif}.

Game Programming -- Penn Wu

304

\protect\hypertarget{index_split_015.htmlux5cux23p305}{}{}\textbf{Misc.
functions}

typeof null;

Returns string ``object'' (or whatever the

thing is).

num.toString(radix);

Returns num var as a string var.

parseInt(``3 blind mice'');

Returns 3.

parseInt(``0xFF'');

Returns 255.

parseInt(``ff'', 16);

Returns 255 also.

parseInt(``eleven'');

Returns NaN.

javascript:alert(``Hi there'');

Execute JS as a URL.

eval(``3 + 12'');

Evaluates a text string as code.

escape(str);

Returns str in web-compliant form. E.g.

``hi ho'' = ``hi\%20ho''.

unescape(str);

Returns str from web-compliant form.

Undoes escape().

\textbf{Window functions}

alert(``Are you sure about that?'');

Prompts with OK box.

confirm(``Continue loading page?'');

Prompts with OK / CANCEL message

box.

prompt(``Enter your name, please.'');

Prompts with text box.

close();

Close the window.

find(), home(), print(), stop();

Duplicates of buttons. Not in IE4.

focus(), blur();

Cause us focus, or lose focus. Not in

IE3.

moveBy(), moveTo();

Move the window.

resizeBy(), resizeTo();

Resize the window.

scrollBy(), scrollTo();

Scroll the document displayed in the

window.

var intervalID = setInterval(``bounce()'', 10000);

Set func to be repeatedly called w/

delay. Call OUT of func.

clearInterval(intervalID);

Cancel function to be repeatedly

invoked with delay.

setTimeout(``display\_time()'', 10000);

Call display\_time() in 1 sec. Put in

display\_time() to loop.

clearTimeout();

Cancel function invoked \emph{once} after

delay.

var w = window.open(``smallwin.html'', ``SmallWin'',

Opens another window, loading

smallwin.html, with name

``width=400,height=300,status,resizeable,menubar'');

SmallWin, and given dimensions.

\textbf{Window properties}

closed

Returns true if window has been closed

(useful for open()).

defaultStatus, status

Sets default status line, current status

line (in status bar).

document

Refers to the current document object

(the HTML page).

frames{[}{]}

Refers to frames, if any.

history

A reference to the history object,

represents browsing hist.

history.back();

Go back a link.

history.forward();

Go forward a link.

history.go();

Goes to a link, buggy in NS2\&3, weird in

IE3 -- best avoid.

innerHeight, innerWidth, outerHeight, outerWidth

Inner and outer dimensions of the

window; not in IE4.

locationbar, menubar, personalbar, scrollbars, statusbar, toolbar
References to visibility of parts. Not in

IE4.

name

Name of current window. Useful for \textless A

TARGET\textgreater, for ex.

opener

Reference to Window opened this, or

null if opened by user.

\textbf{Document functions}

document.write(``\textless h2\textgreater Table of
Factorials\textless/h2\textgreater'');

Outputs to current doc, writes HTML.

document.writeln(``Hi there.'');

Outputs with a \textless CR\textgreater{} appended at end.

document.forms{[}i{]}.elements{[}j++{]};

Access forms and form elements via

array scripting.

Game Programming -- Penn Wu

305

\protect\hypertarget{index_split_015.htmlux5cux23p306}{}{}document.close();

Closes this window-if opened by JS, in

later browsers.

\textbf{Document properties}

document.location

Represents URL document displayed.

Set to load new doc.

document.forms{[}0{]}, document.myform

Refer to document forms.

document.alinkColor

Color of hyperlink while clicked on

(same as \textless BODY\textgreater{} tag).

document.anchors{[}{]}

Hyperlink array.

document.applets{[}{]}

Applet array.

document.bgColor = ``\#040404'';

Background color of document.

document.cookie

Allows JS to read / write cookies. == ``\,'' if

not set.

document.embeds{[}{]}

Embedded array.

document.fgColor = ``blue'';

Text color of document (same as

\textless BODY\textgreater{} tag).

document.images{[}{]}

Images array.

document.lastModified

Returns string of the date we were last

modified.

document.linkColor

Color of unclicked links. Same as LINK

attr. in \textless BODY\textgreater.

document.links{[}{]}

Links array.

document.referrer

URL of doc that ref'd us, if any.

document.title

The title (\textless TITLE\textgreater) of this document.

document.URL

URL we were loaded from, same as

location.href.

document.vlinkColor

Visited link color. Same as VLINK in

\textless BODY\textgreater.

document.domain

Returns the name of the domain you're

currently at.

Game Programming -- Penn Wu

306

\protect\hypertarget{index_split_015.htmlux5cux23p307}{}{}

\textbf{Navigator properties}

navigator.appName

The simple name of the web browser.

navigator.appVersion

The version number and/or other

version info about browser.

navigator.userAgent

appName and appVersion combined,

usually.

navigator.appCodeName

The code name of the browser. E.g.,

``Mozilla.''

navigator.platform

Platform they're running on as of JS1.2.

navigator.language

Language of browser. ``en'' (English).

NS4+, not IE.

navigator.userLanguage, navigator.systemLanguage

IE4+ version of navigator.language

property.

navigator.javaEnabled()

Returns true if Java supported and

enabled on this browser.

\textbf{Math functions}

Math.round(x/15);

Rounds to the nearest integer.

Math.pow(x,y);

Returns xy.

Math.sqrt(x*x + y*y);

Returns sqare root of argument.

Math.random();

Returns random between 0.0 -- 0.1.

Math.max(i, j);

Returns greater of two numbers.

Math.min(i, j);

Returns lesser of two numbers.

Math.floor(j);

Rounds j down.

Math.ceil(j);

Rounds j up.

\textbf{String functions}

str.length;

Returns a string's character length.

str.charAt(str.length - 1);

Returns the last character of a string.

SEE NOTES BELOW.

str.substring(1, 4);

Returns str{[}1{]} through str{[}3{]}.

str.indexOf(`a'), str.lastIndexOf(`` ``);

Returns position of first ``a'' / last `` `` in

string str, -1 if none.

str.anchor(name), str.big(), str.blink(), str.bold(), str.fixed(),
str.italics(), str.link(href); Return str with certain formatting

imposed upon it.

str.small(), str.strike(), str.sub(), str.sup()

anchor=\textless A NAME=name\textgreater, link=\textless A

HREF\textgreater, rest obvious.

str.fontcolor(``\#090909''), str.fontsize(1-7 \textbar{} ``+2'');

Set the string's font color / size.

str.match(), str.replace(), str.search();

Regexp / string match / replace

functions.

str.slice(2{[}, 6{]});

Returns str{[}2{]} through str{[}5{]}, neg. args

start from end.

str = ``1,2,3,4,5''; arr = str.split(``,'');

Returns array of substrings, split by

delimiter ``,''.

str.substr(5,2);

Returns str{[}5{]} through str{[}5+2{]}.

str.toUpperCase(), str.toLowerCase();

Convert a string's case.

\textbf{Vital notes}

• · \textbf{Semi-colons} are optional, but \textbf{recommended}.

• · JavaScript is \textbf{case-sensitive}; HTML embedded names (such as
onClick) are \textbf{not}.

• · Always declare variables with \textbf{var}. Variables not declared
with var are global automatically. Keep vars declared on top for
clarity. Ex: var ind = 0;

• · Fun with strings: \textbf{``Hi there''} and \textbf{`Hi there'} are
both legal string definitions.

• · \textbf{Octal} number definitions begin with a 0. \textbf{Hex}
begins with 0x (or 0X). Ex: 026, 0xAF, 0377, 0xff...

• · JavaScript represents all numbers as floating point. \textbf{Numbers
can be extremely large}, like: -999 tril \textless-\textgreater{} +999
tril.

• · \textbf{String indexing}, like str{[}str.length - 1{]}, is supported
by Nav4+, not IE4 though (IE5?).

• · \textbf{null} is a special value in JS. It is \textbf{not}
equivelant to 0. It represents the lack of an object, number, string,
etc. Sometimes, converted to 0, though.

• · \textbf{Functions} can be \textbf{nested} since JS1.2.

• · \textbf{undefined} can be tested for by making an uninitialized
variable: var undefined; if (myform{[}``checkbox'' + ind{]} ==
undefined) \ldots{}

\textbf{Useful code tidbits}

Game Programming -- Penn Wu

307

\protect\hypertarget{index_split_015.htmlux5cux23p308}{}{}\textless input
type = ``button'' ... onClick = ``alert(`You clicked
me!')''\textgreater{} When user clicks button, execute

``onClick'' portion.

var square = new Function(``x'', ``return x*x;'');

Function literal -- variable holds function

definition.

var square = function(x) \{ return x*x; \}

Function literal. square(144) would

execute it.

image.width \textless-\textgreater{} image{[}``width''{]}

Two ways to access object properties.

var pattern = new RegExp(``\textbackslash bjava\textbackslash b'',
``i'');

Creates RegExp object (regular-

expression).

var o = new Object();

Makes a general object... you can make

up properties.

var point = \{ x:2.3, y:-1.2 \};

Object literal -- general object with init'd

properties.

var sq = \{ upleft: \{ x:point.x, y:point.y \}, lowright: \{
x:(point.x+side), y:(point.y+side) \}\}; Object literal, with
sq.upleft.x == point.x,

for example.

document.images{[}i{]}.width;

Way to access images as array of

document object.

var a = new Array(); a{[}0{]} = 1.2; a{[}1{]} = ``JavaScript''; a{[}2{]}
= true; a{[}3{]} = \{ x:1, y:3 \}; Creates an array. Once made, indexed

elems. added easy.

var a = {[}1.2, ``JavaScript'', true, \{ x:1, y:3 \}{]};

Alternate way since JS1.2.

var matrix = {[}{[}1,2,3{]}, {[}4,5,6{]}, {[}7,8,9{]}{]};

Nested array definition.

var sparseArray = {[}1,,,,5{]};

Makes array with some undefined

elements.

for (var i in obj);

The for/in loop loops through the

properties of an object.

Circle.prototype.pi = 3.14159;

Sets a pi val for all Circle objects.

\textless body bgcolor = ``\&\{favorite\_color();\};''\textgreater{}

\&\{ JS-statements; \}; used in NS3+, !IE4,

embed JS in HTML.

\textbf{Notable constants}

Number.MAX\_VALUE

Largest representable number.

Number.MIN\_VALUE

Most negative representable number.

Number.NaN

Special Not-a-number value.

Number.POSITIVE\_INFINITY

Special value to represent infinity

Number.NEGATIVE\_INFINITY

Special value to represent negative

infinity.

Game Programming -- Penn Wu

308

\protect\hypertarget{index_split_015.htmlux5cux23p309}{}{}

\textbf{Object-based browser detection}

Document object

Browser that supports it

document.images

NS3+, IE4+

!document.images

NS2, IE3

document.layers

NS4+

document.all

IE4+

document.layers \textbar\textbar{} document.all

NS4+, IE4+

if (document.images) document.images{[}0{]}.src =
``/images/myimg1.jpg''; Example usage of browser object

detection.

\textbf{Useful events}

Handler

Triggered when

Supported by

OnAbort

Loading interrupted.

Image

OnBlur

Element loses input focus.

Text elms., Window, all other elms.

OnChange

User changes an elm., moves on.

Select, text input elements

OnClick

User single-click. Ret. false = cancel.

Link, button elements

OnError

Error occurs while loading an image.

Image

OnFocus

Element given input focus.

Text elms., Window, all other elms.

OnLoad

Document or image finishes loading.

Window, Image

OnMouseOut

Mouse moves off element.

Link

OnMouseOver

Mouse moves over elm.

Link

OnReset

Form reset request, false = no reset.

Form

OnSubmit

Form submit, false = no submit.

Form

OnUnload

Document is unloaded.

Window

\textbf{Cookie stuff}

document.cookie = ``version='' + escape(document.lastModified) +

Sets persistent cookie. escape()

converts to web form,

``; path=/; expires='' + mydate.toGMTString();

unescape() undoes from web form.

javascript:alert(document.cookie)

Type this in your browser to see the

cookie set for site!

document.cookie = ``Name=Joe Bob; path=/'';

Minimal cookie setting.

\textbf{Variable arguments}

function add\_all\_together() \{

The \textbf{arguments} method stores the

arguments

for (i = 0; i \textless{} add\_all\_together.arguments.length; i++)

themselves, and the number of

arguments

total += add\_all\_together.arguments{[}i{]};

passed to each function, as we

demonstrate

\}

here.

\textbf{Pre-load and update images}

myimg1 = new Image();

Make a new image object,

myimg1.src = ``/images/image01.jpg'';

set the image src to preload it.

function imgfilter(imgobj, newimg) \{

if (document.images)

If the browser supports the

document.images method,

document.images{[}imgobj{]} = newimg.src;

set a new src for the imgobj image

object argument.

\}

\textbf{Date stuff}

var now = new Date();

Date obj representing current date and

time.

var xmas = new Date(97, 11, 25);

Date obj for 25-Dec-97, note months

index from 0!

now.toLocaleString();

Returns string of date and time.

xmas.toGMTString();

Returns string of date and time in GMT

time.

\textbf{Creating a Plain-Text Document}

var w = window.open(``\,'', ``console'',
``width=600,height=300,resizeable''); We specify the optional
{[}window.{]}open

because there's

Game Programming -- Penn Wu

309

\protect\hypertarget{index_split_015.htmlux5cux23p310}{}{}w.document.open(``text/plain'');

a document.open() function too.

w.document.writeln(msg);

\textbf{To insure a child window you want to update is still open}

This will close the window declared

above if it's still open,

if (!w.closed) w.close();

only if that window hasn't been closed

by the user.

\textbf{To generate a random number between X and Y}

function generate( lbound, ubound ) \{

return Math.floor( ( ubound - lbound + 1 ) * Math.random() + lbound );

\}

// This one I got somewhere. It doesn't work. It returns only odd
numbers, and goes 1 beyond the set range sometimes???

function generate( x, y ) \{

var range = y-x+1;

var i = ( "" + range ).length;

var num = ( Math.floor( Math.random() * Math.pow( 10, I ) ) \% range ) +

parseInt( x );

return num;

\}

\textbf{Detecting Shockwave}

\textless SCRIPT LANGUAGE="JavaScript"\textgreater{}

\textless!---hiding contents of script from old browsers, just in case

//If this browser understands the mimeTypes property and recognizes the
MIME Type //"application/futuresplash"...

if (navigator.mimeTypes \&\&
navigator.mimeTypes{[}"application/x-shockwave-flash"{]})\{

//...write out the following \textless EMBED\textgreater{} tag into the
document.

document.write(\textquotesingle\textless EMBED SRC="flash\_movie.swf"
WIDTH="220" HEIGHT="110" LOOP="true"
QUALITY="high"\textgreater\textquotesingle);

\}

//Otherwise,...

else \{

//...write out the following \textless IMG\textgreater{} tag into the
document. The image need

//not be the same size as the Flash movie, but it may help you lay out
the

/page if you can predict the size of the object reliably.

document.write(\textquotesingle\textless IMG SRC="welcome.gif"
WIDTH="220" HEIGHT="110" ALT="Non-Shockwave
Welcome"\textgreater\textquotesingle);

\}

//Done hiding from old browsers. -\/-\textgreater{}

\textless/SCRIPT\textgreater{}

\textbf{Using JS to Write to Frames}

Just give the frames names, then access them by name:

\textless frameset cols="*,*"\textgreater{}

\textless frame name=left src="a.html"\textgreater{}

\textless frame name=right src="b.html"\textgreater{}

\textless/frameset\textgreater{}

top . right . document . open ();

top . right . document . writeln ("Hello.");

top . right . document . close ();

top.document.location = ``newpage.htm'';

Breaks from frames, goes to new page.

Game Programming -- Penn Wu

310

\protect\hypertarget{index_split_015.htmlux5cux23p311}{}{}

1

2

3

4

5

6

7

8

9

10

11

12

13

14

1

D

D

D

C

B

C

D

2

B

B

B

D

B

A

B

3

B

B

C

C

B

D

D

4

C

B

A

A

A

B

B

5

B

C

B

A

B

A

A

6

A

A

A

D

A

A

B

7

A

A

D

B

C

A

C

8

C

A

D

D

D

D

C

9

D

B

B

A

B

D

C

10

D

C

A

A

A

A

D

Game Programming -- Penn Wu

311

\protect\hypertarget{index_split_015.htmlux5cux23p312}{}{}

Game Programming -- Penn Wu

312

\end{document}
